\begin{document}

\def\title{Worksheet 10}

\newcommand{\qitem}{\qpart\item}

\renewcommand{\labelenumi}{(\alph{enumi})} % change default enum format to (a)
\renewcommand{\theenumi}{(\alph{enumi})} % fix reference format accordingly.
\renewcommand{\labelenumii}{\roman{enumii}.} % second level labels.
\renewcommand{\theenumii}{\roman{enumii}.}
    \maketitle

    \vspace{0.5em}

    \subsection*{Linearization}
    Many systems in the real world are nonlinear, which means that we cannot represent them with a matrix-vector equation.
    We would like to use linear tools (such as diagonalization) to solve these equations, so we typically linearize nonlinear systems. \\
    \newline
    \textbf{When is a system linear?} \\
    \newline
    A system is linear if it follows the \textbf{scaling} and \textbf{additivity} properties:
    \begin{enumerate}
        \item \textbf{Scaling}: For a linear system $f$, $\boxed{f(ax) = af(x)}$ for every $a$ and for every $x$.
        \item \textbf{Additivity}: For a linear system $f$, $\boxed{f(x + y) = f(x) + f(y)}$ for every $x, y$.
    \end{enumerate}
    \textit{Note: Equations of the form $f(x) = ax + b$ with nonzero $b$ are not linear; they are considered to be \textbf{affine}}. \\
    \newline
    \textbf{Linearizing a nonlinear system} \\
    \newline
    We convert a nonlinear system into a linear system by using a first-order Taylor approximation:
    $$\boxed{f(x) \approx f(x^*) + \frac{df}{dx} \bigg\rvert_{x = x^*} (x - x^*)}$$
    Or, for an equation with an input, $u$:
    $$\boxed{f(x, u) \approx f(x^*, u^*) + \frac{df}{dx} \bigg\rvert_{x = x^*} (x - x^*) + \frac{df}{du} \bigg\rvert_{u = u^*} (u - u^*)}$$
    In these equations, we linearize the system around an \textbf{equilibrium point} or \textbf{operating point}, $(x^*, u^*)$.
    This is a point that we choose when making our approximation, typically so that
    $$f(x^*, u^*) = 0$$
    \newline
    \textbf{Linearizing a system of nonlinear equations} \\
    \newline
    Say we have a system of nonlinear functions, typically represented by
    \begin{align*}
        \vec{f}(\vec{x}) = \begin{bmatrix}
            f_1(x_1, \dots, x_n) \\
            \vdots \\
            f_m(x_1, \dots, x_n)
        \end{bmatrix}
    \end{align*}
    For these systems, you can linearize each equation using the partial derivative of the function with respect to each state variable:
    \begin{center}
        \begin{align*}
            f_1(\vec{x}) \approx f_1(\vec{x}^*) + \frac{\partial f_1}{\partial x_1} \bigg\rvert_{x_1 = x_1^*} (x_1 - x_1^*) + \cdots + \frac{\partial f_1}{\partial x_n} \bigg\rvert_{x_n = x_n^*} (x_n - x_n^*) \\
            \vdots \\
            f_m(\vec{x}) \approx f_m(\vec{x}^*) + \frac{\partial f_m}{\partial x_1} \bigg\rvert_{x_1 = x_1^*} (x_1 - x_1^*) + \cdots + \frac{\partial f_m}{\partial x_n} \bigg\rvert_{x_n = x_n^*} (x_n - x_n^*)
        \end{align*}
    \end{center}
    In this case, we choose an $\vec{x}^*$ and $\vec{u}^*$ such that $\vec{f}(\vec{x}^*, \vec{u}^*) = \vec{0}$. \\
    \newline
    We can express this linearization as a matrix-vector equation:
    \begin{align*}
        \vec{f}(\vec{x}) = \begin{bmatrix}
            \frac{\partial f_1}{\partial x_1} \bigg\rvert_{x_1^*} & \cdots & \frac{\partial f_1}{\partial x_n} \bigg\rvert_{x_n^*} \\
            \vdots & \ddots & \vdots \\
            \frac{\partial f_m}{\partial x_1} \bigg\rvert_{x_1^*} & \cdots & \frac{\partial f_m}{\partial x_n} \bigg\rvert_{x_n^*}
        \end{bmatrix} \begin{bmatrix}
            (x_1 - x_1^*) \\
            \vdots \\
            (x_n - x_n^*)
        \end{bmatrix} = \boxed{J_{\vec{x}} \vec{\delta x}}
    \end{align*}
    Where $J_{\vec{x}}$ is the \textbf{Jacobian matrix} of $\vec{f}$ with respect to $\vec{x}$ and $\vec{\delta x}$ is the distance of $\vec{x}$ from the equilibrium point. \\
    \newline
    If we are looking at a system of equations with a vector input, the linearization will be as follows
    \begin{align*}
        \vec{f}(\vec{x}, \vec{u}) = \begin{bmatrix}
            \frac{\partial f_1}{\partial x_1} \bigg\rvert_{x_1^*} & \cdots & \frac{\partial f_1}{\partial x_n} \bigg\rvert_{x_n^*} \\
            \vdots & \ddots & \vdots \\
            \frac{\partial f_m}{\partial x_1} \bigg\rvert_{x_1^*} & \cdots & \frac{\partial f_m}{\partial x_n} \bigg\rvert_{x_n^*}
        \end{bmatrix} \begin{bmatrix}
            (x_1 - x_1^*) \\
            \vdots \\
            (x_n - x_n^*)
        \end{bmatrix} + \begin{bmatrix}
            \frac{\partial f_1}{\partial u_1} \bigg\rvert_{u_1^*} & \cdots & \frac{\partial f_1}{\partial u_k} \bigg\rvert_{u_k^*} \\
            \vdots & \ddots & \vdots \\
            \frac{\partial f_m}{\partial u_1} \bigg\rvert_{u_1^*} & \cdots & \frac{\partial f_m}{\partial u_k} \bigg\rvert_{u_k^*}
        \end{bmatrix} \begin{bmatrix}
            (u_1 - u_1^*) \\
            \vdots \\
            (u_k - u_k^*)
        \end{bmatrix} = \boxed{J_{\vec{x}} \vec{\delta x} + J_{\vec{u}} \vec{\delta u}}
    \end{align*}


\clearpage

\begin{qunlist}
% Author: Taejin Hwang
% Emails: taejin@berkeley.edu

\qns{An Introduction to Linearization}

The majority of the examples seen in 16B so far have used functions that are \textbf{linear}.
A linear function $f(x)$ is a function that satisfies the two following properties:
\begin{align*}
  &\textbf{Scaling:} \ \ f(\alpha x) = \alpha f(x) \ \ \text{for any scalar $\alpha.$} \\
  &\textbf{Superposition:} \ \ f(x_{1} + x_{2}) = f(x_{1}) + f(x_{2}) \ \ \text{for any $x_{1}, x_{2}$ in the domain of $f.$}
\end{align*}

\begin{enumerate}
  \qitem \textbf{Is the function $f(x) = 5x$ linear? Prove why or why not.}

  \ws {
    \vspace{75px}
  }
  \sol {
    We can show that $f(x)$ satisfies both the properties of scaling and superposition.
    \begin{itemize}
      \item Scaling: $f(\alpha x) = 5(\alpha x) = \alpha (5x) = \alpha f(x).$
      \item Superposition: $f(x_{1} + x_{2}) = 5(x_{1} + x_{2}) = 5 x_{1} + 5 x_{2} = f(x_{1}) + f(x_{2}).$
    \end{itemize}
  }

  \qitem \textbf{Is the function $f(x) = e^{x}$ linear? Prove why or why not.}

  \ws {
    \vspace{75px}
  }

  \sol {
    This function satisfies neither the properties of scaling nor superposition.
    Here are some counterexamples:
    \begin{itemize}
      \item Scaling: $f(0 \cdot x) = e^{0 \cdot x} = 1 \neq 0 \cdot e^{x}.$
      \item Superposition: $f(x_{1} + x_{2}) = e^{x_{1} + x_{2}} = e^{x_{1}} e^{x_{2}} \neq e^{x_{1}} + e^{x_{2}}.$
    \end{itemize}
  }
\end{enumerate}

Every matrix $A$ can be seen as a linear function $f(\vec{x}) = A \vec{x}$ since $f$ will satisfy both properties above.

In this question however, we will turn our attention to functions that are \textbf{nonlinear}, meaning it will not satisfy both properties above.
Some common examples of nonlinear functions are: $\sin(x), x^{2},$ or even $x - 4.$
Linear functions are very convenient to work with, and we already have a large tool-kit on how to handle linear systems.
So ideally, we would like every system to be linear.

If we can construct a linear approximation for a nonlinear function, then we could use all of our techniques for linear systems to approximate the behavior of a nonlinear system!
The process of creating a linear approximation for a nonlinear function is called \textbf{linearization.}

The main ideas for linearization will come from Taylor Expansion, in which we'll give a first order approximation for a function around a point $x^{*}$ as
\begin{equation} \label{eqn:ta}
  f(x) \approx f(x^{*}) + \frac{f'(x^{*})}{1!} (x - x^{*}) = f(x^{*}) + f'(x^{*}) (x - x^{*})
\end{equation}

This is a first order approximation because we stop the taylor series approximation after the first term.
We could easily have a nth order approximation by taking more derivatives.

\meta {
  We like to pick operating points $x^{*}$ in which $f(x^{*}) = 0$ since it will simplify equation 1 to $f(x) = f'(x^{*}) (x - x^{*}).$
}

Let's now take a look at the differential equation:
\begin{equation} \label{eqn:nl}
  \ddt{}{t} x(t) = - \sin(x)
\end{equation}
With initial condition, $x(0) = x_{0},$ since $\sin(x)$ is a nonlinear function, this differential equation will be difficult to solve.
Therefore, we will linearize this system by taking an \textbf{equilibrium point} $x^{*}$ and taking a first order Taylor approximation.

\begin{enumerate}[resume]
  \qitem \textbf{What is the first order Taylor approximation of $f(x) = - \sin(x)$ around $x^{*} = 0$?}

  \ws {
    \vspace{75px}
  }
  \sol {
    Using the first order Taylor expansion formula given in \eqref{eqn:ta}, we get:
    $$f(x) \approx f(x^{*}) + f'(x^{*}) (x - x^{*})$$
    The derivative of $\sin(x)$ is $\cos(x)$ so plugging in for $x^{*} = 0,$ we get:
    $$f(x) \approx -\sin(0) - \cos(0) (x - 0) = -x$$
  }

  \qitem We will now look at a small window of $x(t)$ around $x^{*}$ and let $x(t) \approx x^{*} + \delta_{x}(t)$ where $\delta_{x}(t)$ is small enough so the error of the Taylor approximation is small.
  \textbf{Plug in this $x(t)$ into the original differential equation \eqref{eqn:nl} and use the first order Taylor approximation derived in the previous part to come up with a new differential equation in terms of $\delta_{x}(t).$}

  \ws {
    \vspace{100px}
  }

  \sol {
    Plugging in $x(t) \approx x^{*} + \delta_{x}(t),$ we'll get:
    $$f(x) = -x = -x^{*} - \delta_{x}(t) = -\delta_{x}(t)$$
    Differentiating $x(t),$ and noticing that $x^{*}$ is a constant (zero in this case), we get:
    $$\ddt{}{t} x(t) = \ddt{}{t} \delta_{x}(t)$$
    Therefore, the linearized differential equation represented by \eqref{eqn:nl} will be:
    \begin{equation}
      \ddt{}{t} \delta_{x}(t) = -\delta_{x}(t)
    \end{equation}
  }

  \qitem \textbf{What is the solution to the new differential equation derived in the previous part, in terms of $\delta_{x}(t)$?}

  \ws {
    \vspace{75px}
  }
  \meta {
    It is important to emphasize here that this $\delta_{x}(t)$ is no longer a function of $x$ rather, it is a function of the distance to $x.$
    This however, does not matter since when we are doing system analysis, we rarely care about the actual solution to this system, rather we care about its behavior.
  }

  \sol {
    This is a first order homogenous differential equation, with zero input.
    Therefore, the solution will be:
    \begin{equation}
      \delta_{x}(t) = \delta_{x}(0) e^{-t}
    \end{equation}
  }

  \qitem \textbf{Using the solution for $\delta_{x}(t),$ what is the new linearized approximation for $x(t)?$ Is this approximation valid for all values of $t?$}

  \ws {
    \vspace{75px}
  }

  \meta {
    Remember that linearization is an approximation to a system as well, and some linearizations are considered stable, meaning that it will be accurate for all values of $t$ whereas some will be unstable, meaning they only work in a small neighborhood around $x^{*}.$ The smallness assumption says that $\delta_{x}(t)$ should be small enough such that the approximation $f(x) \approx f(x^{*}) + f'(x^{*}) \delta_{x}(t)$ is valid.
  }

  \sol {
    The solution derived in the previous part was $\delta_{x}(t) = \delta_{x}(0) e^{-t}.$
    Converting back to $x(t),$ we see that the new linearized approximation is $x(t) \approx x^{*} + \delta_{x}(t) = \delta_{x}(t) = \delta_{x}(0) e^{-t}.$
    This solution is stable since $\delta_{x}(t) \leq \delta_{x}(0)$ for all values of $t \geq 0.$
    If the linearization satisfies the smallness assumption, meaning $\delta_{x}(0)$ is small enough that $f(x) \approx f(x^{*}) + f'(x^{*}) \delta_{x}(t),$ then the linearization will be valid for all values of $t \geq 0.$
  }
\end{enumerate}

\clearpage
% Author: Naomi Sagan
% Emails: naomi.sagan@berkeley.edu

\qns{Jacobian Intuition}

In this problem, we will build the intuition behind constructing the Jacobian matrix used for linearizing systems of nonlinear functions. \\
\newline
For this problem, consider the following system of functions in terms of $\vec{x} = \begin{bmatrix} x_1 \\ x_2 \end{bmatrix}$:
\begin{align*}
    f_1(x_1, x_2) = x_1^3 + \sin(x_2) \\
    f_2(x_1, x_2) = e^{2x_1} - x_2 - 1
\end{align*}
Let the operating point about which you linearize the system be
\begin{align*}
    \vec{x}^{*} = \begin{bmatrix} 0 \\ 0 \end{bmatrix}
\end{align*}

\begin{enumerate}
    \qitem Find the linear approximation of $f_1$ about $\vec{x}^{*}$.
    \begin{align*}
        f_1(x_1, x_2) \approx f_1(x_1^{*}, x_2^{*}) + \frac{\partial f_1}{\partial x_1} \bigg\rvert_{\vec{x*}} (x_1 - x_1^{*}) + \frac{\partial f_1}{\partial x_2} \bigg\rvert_{\vec{x*}} (x_2 - x_2^{*})
    \end{align*}
    \begin{enumerate}[label = (\roman*)]
        \item First, take $x_2$ as constant. What is the linear approximation of $f_1(x_1, x_2^{*})$? \\
        \ws{
        \vspace{100px}
        }
        \sol{
            If we take $x_2$ as constant, the approximation of $f_1$ is as follows:
            \begin{align*}
                f_1(x_1, x_2^{*}) \approx f_1(x_1^{*}, x_2^{*}) + \frac{\partial f_1}{\partial x_1} \bigg\rvert_{\vec{x*}} (x_1 - x_1^{*})
            \end{align*}
            Notice that $f_1(\vec{x}^{*}) = 0$, so
            \begin{align*}
                f_1(x_1, x_2^{*}) \approx \frac{\partial f_1}{\partial x_1} \bigg\rvert_{\vec{x*}} (x_1 - x_1^{*})
            \end{align*}
            Taking the partial derivative of $f_1$ with respect to $x_1$:
            \begin{align*}
                f_1(x_1, x_2^{*}) \approx 3(x_1^{*})^2(x_1 - x_1^{*})
            \end{align*}
            Thus, we get
            \begin{align*}
                f_1(x_1, x_2^{*}) \approx 0
            \end{align*}
        }
        \item Now, take $x_1$ as constant. What is the linear approximation of $f_1(x_1^{*}, x_2)$? \\
        \ws{
        \vspace{100px}
        }
        \sol{
            If we take $x_1$ as constant, the approximation of $f_1$ is as follows:
            \begin{align*}
                f_1(x_1^{*}, x_2) \approx f_1(x_1^{*}, x_2^{*}) + \frac{\partial f_1}{\partial x_2} \bigg\rvert_{\vec{x*}} (x_2 - x_2^{*})
            \end{align*}
            Plugging in $\vec{0}$ for $\vec{x}^{*}$:
            \begin{align*}
                f_1(x_1^{*}, x_2) \approx \frac{\partial f_1}{\partial x_2} \bigg\rvert_{\vec{0}} \cdot x_2
            \end{align*}
            Taking the partial derivative of $f_1$ with respect to $x_2$:
            \begin{align*}
                f_1(x_1^{*}, x_2) \approx \cos(x_2^{*}) \cdot x_2 = x_2
            \end{align*}
        }
        \item Now, what is $f_1(x_1, x_2)$ linearized around $\vec{x}^{*}$? \\
        \ws{
        \vspace{100px}
        }
        \sol{
            To find the linear approximation of $f_1$ with respect to both $x_1$ and $x_2$, add the approximation of $f_1$ that depends on $\frac{\partial f_1}{\partial x_1}$ to the approximation that depends on $\frac{\partial f_1}{\partial x_2}$.
            \begin{align*}
                f_1(x_1, x_2) \approx f_1(x_1^{*}, x_2^{*}) + \frac{\partial f_1}{\partial x_1} \bigg\rvert_{\vec{x*}} (x_1 - x_1^{*}) + \frac{\partial f_1}{\partial x_2} \bigg\rvert_{\vec{x*}} (x_2 - x_2^{*})
            \end{align*}
            Remembering that $f_1(x_1^{*}, x_2^{*}) = 0$ and plugging in the results from parts (i) and (ii), we have
            \begin{align*}
                f_1(x_1, x_2) \approx 0 + 0(x_1) + x_2 = x_2
            \end{align*}
        }
    \end{enumerate}

    \qitem Using the same process as in part (a), find the linear approximation of $f_2$ about $\vec{x}^{*}$. \\
    \ws{
    \vspace{100px}
    }
    \sol {
        First, find the partial derivative of $f_2$ with respect to $x_1$, evaluated at the operating point $\vec{x} = \vec{0}$:
        \begin{align*}
            \frac{\partial f_2}{\partial x_1} \bigg\rvert_{\vec{x*}} = 2e^{2x_1^{*}} = 2e^0 = 2
        \end{align*}
        Then, find the partial derivative of $f_2$ with respect to $x_2$ at the operating point:
        \begin{align*}
            \frac{\partial f_2}{\partial x_2} \bigg\rvert_{\vec{x*}} = -1
        \end{align*}
        Then, the linear approximation of $f_2$ is:
        \begin{align*}
            f_2(x_1, x_2) \approx f_2(x_1^{*}, x_2^{*}) + \frac{\partial f_2}{\partial x_1} \bigg\rvert_{\vec{x*}} (x_1 - x_1^{*}) + \frac{\partial f_2}{\partial x_2} \bigg\rvert_{\vec{x*}} (x_2 - x_2^{*}) \\
            f_2(x_1, x_2) \approx 0 + 2(x_1 - 0) - 1(x_2 - 0) = 2x_1 - x_2
        \end{align*}
    }

    \qitem Now, we want to approximate $\vec{f}(\vec{x}) = \begin{bmatrix} f_1(\vec{x}) \\ f_2(\vec{x}) \end{bmatrix}$ around $\vec{x} = \vec{0}$ as a matrix-vector equation.
    \begin{align*}
        \vec{f}(\vec{x}) \approx \vec{f}({\vec{x}^{*}}) + J_{\vec{x}}(\vec{x} - \vec{x}^{*})
    \end{align*}
    Fill in the Jacobian, $J_{\vec{x}}$, by finding the constants $a, b, c, d$ that complete the following equation:
    \begin{align*}
        \begin{bmatrix}
            f_1(\vec{x}) \\
            f_2(\vec{x})
        \end{bmatrix} \approx
        \begin{bmatrix}
            f_1(\vec{x}^{*}) \\
            f_2(\vec{x}^{*})
        \end{bmatrix} +
        \begin{bmatrix}
            a & b \\
            c & d
        \end{bmatrix}
        \begin{bmatrix}
            x_1 - x_1^{*} \\
            x_2 - x_2^{*}
        \end{bmatrix}
    \end{align*}
    \ws{
    \vspace{100px}
    }

    \sol{
        $a$ is the component of the approximation of $f_1$ that multiplies $x_1$:
        \begin{align*}
            a = \frac{\partial f_1}{\partial x_1} \bigg\rvert_{\vec{x*}} = 0
        \end{align*}
        $b$ is the component of the approximation of $f_1$ that multiplies $x_2$:
        \begin{align*}
            b = \frac{\partial f_1}{\partial x_2} \bigg\rvert_{\vec{x*}} = 1
        \end{align*}
        $c$ is the component of the approximation of $f_2$ that multiplies $x_1$:
        \begin{align*}
            c = \frac{\partial f_2}{\partial x_1} \bigg\rvert_{\vec{x*}} = 2
        \end{align*}
        $d$ is the component of the approximation of $f_2$ that multiplies $x_2$:
        \begin{align*}
            d = \frac{\partial f_2}{\partial x_2} \bigg\rvert_{\vec{x*}} = -1
        \end{align*}

        So, the linear approximation of the system is as follows:
        \begin{align*}
        \begin{bmatrix}
            f_1(\vec{x}) \\
            f_2(\vec{x})
        \end{bmatrix} \approx
        \begin{bmatrix}
            f_1(\vec{0}) \\
            f_2(\vec{0})
        \end{bmatrix} +
        \begin{bmatrix}
            0 & 1 \\
            2 & -1
        \end{bmatrix}
        \begin{bmatrix}
            x_1 - 0 \\
            x_2 - 0
        \end{bmatrix} =
        \begin{bmatrix}
            0 & 1 \\
            2 & -1
        \end{bmatrix}
        \begin{bmatrix}
            x_1 \\
            x_2
        \end{bmatrix}
    \end{align*}
    }
\end{enumerate}

\clearpage
% Author: Naomi Sagan
% Emails: naomi.sagan@berkeley.edu

\qns{Linearization Practice}

Remember that we linearize nonlinear systems using a first-order Taylor approximation.
\begin{align}
f(x) \approx f(x*) + f'(x*)(x - x*)
\end{align}
We linearize the function around $x*$, also known as the \textbf{operating point} of the approximation.
We typically choose the operating point such that $f(x*) = 0$, so that the Taylor approximation is
\begin{align}
f(x) \approx f'(x*)(x - x*)
\end{align}

\begin{enumerate}
    \qitem Consider $f(x) = \cos(x)$
    \begin{enumerate}[label = (\roman*)]
        \item Find all operating points such that $f(x*) = 0$.
        \sol {
            We know that the cosine function is 0 at odd multiples of $\pi/2$ (i.e. $\dots, -\pi/2, \pi/2, 3\pi/2, \dots$). So,
            \begin{align*}
                x* = (2k + 1) \frac{\pi}{2}
            \end{align*}
            where k is any integer.
        }

        \item Choosing one of those operation points, find $f'(x*)$.
        \sol {
            For our solution, we're choosing $x* = \pi/2$, but you can use any valid operating point.
            \begin{align*}
                f'(x) = \frac{d}{dx} \cos(x) = -\sin(x) \\
                f'(x*) = f'(\pi/2) = -\sin(\pi/2) = -1
            \end{align*}
        }

        \item Linearize the function around the same operating point you used for part (ii).
        \sol {
            Again, we're using $x* = \pi/2$ as our operating point. We chose the operating point such that $f(x*) = 0$, so we can approximate $f$ as follows:
            \begin{align*}
                f(x) \approx f'(x*)(x - x*) \\
                f(x) \approx -(x - \pi/2)
            \end{align*}
        }
    \end{enumerate}

    \qitem Repeat the same process for $f(x) = e^x - 1$
    \begin{enumerate}[label = (\roman*)]
        \item Find all operating points such that $f(x*) = 0$.
        \sol {
           To find the operating point, we set $e^{x*} - 1$ equal to 0, which is the same as setting $e^{x*}$ equal to 1.
           $e^x$ is 1 only when $x = 0$, so we can define our operating point as $x* = 0$.
        }

        \item Choosing one of those operation points, find $f'(x*)$.
        \sol {
            Plugging $x* = 0$ into the first derivative of f, we get:
            \begin{align*}
                f'(x) = \frac{d}{dx} (e^x - 1) = e^x \\
                f'(x*) = f'(0) = e^0 = 1
            \end{align*}
        }

        \item Linearize the function around the same operating point you used for part (ii).
        \sol {
            As before, we can approximate $f$ as follows:
            \begin{align*}
                f(x) \approx f'(x*)(x - x*) \\
                f(x) \approx (x - 0) = x
            \end{align*}
        }
    \end{enumerate}

    \qitem [OPTIONAL] Repeat the same process for $f(x) = x - x^2$
    \begin{enumerate}[label = (\roman*)]
        \item Find all operating points such that $f(x*) = 0$.
        \sol {
           To find the operating point, we set $x* - x*^2$ to 0:
           \begin{align*}
                x* - x*^2 = x*(1 - x*) = 0 \\
                x* = 0, 1
           \end{align*}
        }

        \item Choosing one of those operation points, find $f'(x*)$.
        \sol {
            We will choose to linearize around $x* = 0$:
            \begin{align*}
                f'(x) = \frac{d}{dx} (x - x^2) = 1 - 2x \\
                f'(x*) = f'(0) = 1 - 0 = 1
            \end{align*}
        }

        \item Linearize the function around the same operating point you used for part (ii).
        \sol {
            As before, we can approximate $f$ as follows (using $x* = 0$, as in part (ii)):
            \begin{align*}
                f(x) \approx f'(x*)(x - x*) \\
                f(x) \approx (x - 0) = x
            \end{align*}
        }
    \end{enumerate}
\end{enumerate}
\clearpage
\qns{Nonlinear Pendulum}

In this question, we'll take a look at a nonlinear system that cannot be written in state-space, and linearize it around an equilibrium point to create a linear approximation.

Consider the damped pendulum shown below, where the pendulum has length $L$ and makes an angle $\theta$ with the vertical axis:
\begin{center}
\includegraphics[scale=0.3]{\bank/linearization/figures/pendulum.png}
\end{center}
The kinetics of this pendulum over time can be represented by the following differential equation:
\begin{equation}
  \frac{d^{2} \theta}{dt^{2}} = -\frac{g}{L} \sin \theta - \frac{b}{m} \ddt{\theta}{t} + \frac{\cos \theta}{mL} u
\end{equation}
Where $m$ is the mass of the pendulum, $g$ is gravitational acceleration, $b$ is a damping coefficient due to air resistance, and $u(t)$ is an input force on the pendulum. We will also define the following state-variable $\vec{x}$ 
\begin{equation*}
  \vec{x} = \begin{bmatrix} x_{1} \\ x_{2} \end{bmatrix} = \begin{bmatrix} \theta \\ \omega \end{bmatrix} = \begin{bmatrix} \theta \\ \ddt{\theta}{t} \end{bmatrix}
\end{equation*}

If we try putting this system into state-space form we will see that it is impossible since the system is nonlinear.
Therefore, we will use our knowledge of linearization to linearize this system.

Recall that a nonlinear system $\ddt{}{t} \vec{x}(t) = f(\vec{x}, u)$ can be linearized as:
\begin{equation}
  \ddt{}{t} \vec{\delta}_{\vec{x}}(t) = J_{\vec{x}} \ \vec{\delta}_{\vec{x}}(t) + J_{u} \ \delta_{u}(t)
\end{equation}
where $J_{\vec{x}}$ and $J_{u}$ are the respective Jacobians of $f$ with respect to $\vec{x}$ and $u.$

\begin{enumerate}
  \qitem \textbf{How can you write $\ddt{}{t} \vec{x}(t) = f(\vec{x}, u)$ as a vector differential equation?}
  
  \ws {
    \vspace{75px}
  }

  \sol {
    $$\ddt{}{t} \vec{x} 
    = \ddt{}{t} \begin{bmatrix} \theta \\ \ddt{\theta}{t} \end{bmatrix} 
    = \begin{bmatrix} f_{1}(\vec{x}, u) \\ f_{2}(\vec{x}, u) \end{bmatrix}
    = \begin{bmatrix} \ddt{\theta}{t} \\ -\frac{g}{L} \sin \theta - \frac{b}{m} \ddt{\theta}{t} \frac{\cos \theta}{m \cdot L} u(t) \end{bmatrix}$$
  }

  \newpage
  \qitem \textbf{What are $J_{\vec{x}}$ and $J_{u}$ evaluated the equilibrium points $\vec{x}^{*} = \begin{bmatrix} 0 \\ 0 \end{bmatrix}, u^{*} = 0$?}

  \ws {
    \vspace{150px}
  }

  \meta {
    You may want to discuss the shape of the Jacobians, and why the Jacobian for $\vec{x}$ is a $2 \times 2$ matrix while the Jacobian for $u$ is a $2 \times 1$ vector.
  }

  \sol {
    The individual functions $f_{1}, f_{2}$ in terms of the state variables $x_1, x_2$ and the input $u$ are
    \begin{align*} 
      &f_{1}(x_{1}, x_{2}, u) = x_{2} \\
      &f_{2}(x_{1}, x_{2}, u) = -\frac{g}{L} \sin x_{1} - \frac{b}{m} x_{2} + \frac{\cos{x_1}}{mL} u
    \end{align*}

    The Jacobian for $\vec{x}$ is:
    $$J_{\vec{x}} =
    \begin{bmatrix} \frac{\partial f_{1}}{\partial x_{1}} & \frac{\partial f_{1}}{\partial x_{2}} \\
    \frac{\partial f_{2}}{\partial x_{1}} & \frac{\partial f_{2}}{\partial x_{2}}
    \end{bmatrix} = 
    \begin{bmatrix} 0 & 1 \\
    -\frac{g}{L} \cos x_{1} & -\frac{b}{m} 
    \end{bmatrix} = 
    \begin{bmatrix} 0 & 1 \\
    -\frac{g}{L} \cos x_1 - \frac{\sin{x_1}}{mL} u & -\frac{b}{m} 
    \end{bmatrix}
    $$
    The Jacobian for $u$ is:
    $$J_{u} =
    \begin{bmatrix} \frac{\partial f_{1}}{\partial u} \\ \frac{\partial f_{2}}{\partial u} \end{bmatrix} =
    \begin{bmatrix} 0 \\ \frac{\cos x_{1}}{mL} \end{bmatrix} = 
    \begin{bmatrix} 0 \\ \frac{\cos \theta}{mL} \end{bmatrix}
    $$
    Evaluating at the operating point, $\vec{x}^{*}, u^{*}$ we get:
    $$J_{\vec{x}} |_{\vec{x} = \vec{x}^{*}} =
    \begin{bmatrix} 0 & 1 \\
    -\frac{g}{L} & -\frac{b}{m} 
    \end{bmatrix} \hspace{0.3 cm}
    J_{u} |_{u = u^{*}} =
    \begin{bmatrix} 0 \\ \frac{1}{mL} \end{bmatrix}
    $$
  }

  \qitem \textbf{Using these Jacobian matrices, construct the new linearized system in state-space form.}
  \ws {
    \vspace{125px}
  }
  \sol {
    The state-space form will be:
    $$\ddt{}{t} \delta_{x}(t) = 
    \begin{bmatrix} 0 & 1 \\
    -\frac{g}{L} & -\frac{b}{m} 
    \end{bmatrix} \vec{\delta}_{\vec{x}}(t) + 
    \begin{bmatrix} 0 \\ \frac{1}{mL} \end{bmatrix} \vec{\delta}_{u}(t)
    $$
  }

  \qitem \textbf{What are the eigenvalues of this system?}

  \ws {
    \vspace{75px}
  }

  \sol {
    The eigenvalues of this system, are the eigenvalues of the $J_{\vec{x}}$ matrix:
    $$\begin{bmatrix} 0 & 1 \\
    -\frac{g}{L} & -\frac{b}{m} 
    \end{bmatrix}$$
    The determinant of $J_{\vec{x}} - \lambda I$ is:
    $$\text{det} \bigg( \begin{bmatrix} -\lambda & 1 \\
    -\frac{g}{L} & -\frac{b}{m} -\lambda
    \end{bmatrix} \bigg) = (-\lambda)(-\frac{b}{m} - \lambda) + \frac{g}{L} = \lambda^{2} + \frac{b}{m} \lambda + \frac{g}{L}$$
    The eigenvalues are the roots of this characteristic polynomial:
    $$\lambda = -\frac{b}{2m} \pm \frac{1}{2} \sqrt{\big( \frac{b}{m} \big)^{2} - \frac{4g}{L}}$$
  }

  % \qitem Under what conditions is this system stable? Is is stable when $b = 2, \ m = 2, \ g = 10, \ L = \frac{1}{2}?$

  % \meta {
  %   Note that $g$ and $L$ do not affect stability, and will only affect the damping of the system.
  %   They may cause the eigenvalues to be complex, but the imaginary part of eigenvalues do not affect stability.
  % }

  % \sol {
  %   Remember that a system is stable, when $\mathfrak{Re}(\lambda_{i}) < 0$ for all $i.$
  %   For this system, $\mathfrak{Re}(\lambda_{1}) = \mathfrak{Re}(\lambda_{2}) = -\frac{b}{2m}$ meaning the system will be stable when $\frac{b}{2m} > 0.$
  %   Therefore, the system is indeed stable, when $b = 2$ and $m = 2.$ 
  % }

  % \qitem Is this system controllable?

  % \meta {
  %   You are allowed to claim this system is controllable since it is in Controllable Canonical Form, but we'll do the derivation through the controllability matrix $\mathcal{C}$ anyway.
  % }

  % \sol {
  %   We can claim that this system is controllable, since it is in Controllable Canonical Form. 
  %   Alternatively we can also compute the controllability matrix as:
  %   $$\mathcal{C} = 
  %   \begin{bmatrix} 
  %   0 & \frac{1}{mL} \\
  %   \frac{1}{mL} & -\frac{b}{m^{2} L}
  %   \end{bmatrix}$$
  %   Since $\frac{1}{mL}$ cannot be zero, the second column cannot be a scalar multiple of the first. 
  %   Therefore, the controllability matrix is full rank, and the system is controllable.
  % }

  % \qitem Now suppose that there is a strong wind pushing our pendulum which changes the dynamics so that $b = -2.$ 
  % Show that this system is unstable.

  % \meta {
  %   Remember that linearization is an approximation to a system at a given operating point.
  %   Since the linearization is defined in terms of $\delta_{\vec{x}}(t),$ the distance from the operating point, 
  %   if a linearization is unstable, it means that the error of the approximation will be unbounded as you get farther away from the operating point.
  % }

  % \sol {
  %   Since $b = -2, \mathfrak{Re}(\lambda_{1}) = \mathfrak{Re}(\lambda_{2}) = \frac{1}{2} \geq 0$ which implies the system is unstable.
  % }

  % \qitem Show that the system is in Controllable Canonical Form for the given physical values: $b = -2, \ m = 2, \ g = 10, \ L = \frac{1}{2}.$

  % \sol {
  %   Plugging these values in, we see that the system is indeed in Controllable Canonical Form:
  %   $$\ddt{}{t} \delta_{x}(t) = 
  %   \begin{bmatrix} 
  %   0 & 1 \\
  %   -20 & 1
  %   \end{bmatrix} \vec{\delta}_{\vec{x}}(t) + 
  %   \begin{bmatrix} 0 \\ 1 \end{bmatrix} \vec{\delta}_{u}(t)
  %   $$
  % }

  % \qitem  We will use the fact that this system is in Controllable Canonical Form to stabilize the system. 
  % Design a feedback controller so that the eigenvalues of the system will be $\lambda_{1} = -2, \ \lambda_{2} = -3.$

  % \sol {
  %   Setting $u = K \vec{\delta}_{\vec{x}}(t)$ as the input, where $K = \begin{bmatrix} k_{1} & k_{2} \end{bmatrix},$
  %   the closed-loop matrix will be:
  %   $$A_{cl} = \begin{bmatrix}
  %   0 & 1 \\
  %   -20 + k_{1} & 1 + k_{2}
  %   \end{bmatrix}$$
  %   We know that the characteristic polynomial of this matrix will be $\lambda^{2} - (1 + k_{2}) \lambda - (-20 + k_{1}).$
  %   To make the eigenvalues $-2$ and $-3,$ we want our characteristic polynomial to be $\lambda^{2} + 5 \lambda + 6$ so picking $k_{1} = 14$ and $k_{2} = -6$ will suffice.
  % }

\end{enumerate}
\end{qunlist}
\end{document}
