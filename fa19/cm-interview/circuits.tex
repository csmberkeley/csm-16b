% {\Large \textbf{Mechanical:}}
\qns{Circuits}
\qcontributor{Ramsey Mardini}



\begin{enumerate}

\qitem \textbf{Conceptual Question:} When can you use phasor transforms to solve a circuit?

\begin{circuitikz}
  \draw
  (0, 0) node[op amp] (opamp) {}
  (-3.6, 0.1) node[label=$v_\text{in}$] {} 
  (2.125, -0.4) node[label=$v_\text{out}$] {}
  (-3.25, 0.5) to[short,o-] (-3, 0.5)
  (opamp.-) to[R=$R_1$] (-3, 0.5)
  (opamp.-) to[short,*-] ++(0,1.5) coordinate (leftC)
            to[R=$R_2$]           (leftC -| opamp.out)
            to[short,-*]    (opamp.out)
            to[short,-o] ++ (0.5,0) coordinate (out)     
  (leftC)   to[short,*-] ++ (0,1.5)  coordinate (leftR) 
            to[C=$C$]           (leftR -| opamp.out)
            to[short,-*]    (leftC -| opamp.out)
   (opamp.+) -- ++ (0,-0.5) node[ground] {};
 \end{circuitikz}

\qitem The circuit above is a first order low-pass filter.

	\begin{enumerate}
		\item What is its transfer function?
		\item What is its DC gain?
		\item Show that the 3-dB frequency is $\omega = 1 / R_2 C$
	\end{enumerate}

\end{enumerate}