\documentclass[11pt]{article}
\usepackage{../ee16}
\usepackage{../markup}
\usepackage{tikz}
\usetikzlibrary{calc}
\usepackage{color,hyperref,listings,enumitem}
\usepackage{algorithm}
\usepackage{algpseudocode}
\usepackage{tkz-euclide}
\usepackage{physics}
\usepackage{multicol}
\usepackage{pgfplots}
\usepackage{adjustbox}
\usepackage{cancel}
\usepackage{commath}
\usepackage{graphicx}
\usepackage{mathdots}
\usepackage[american,siunitx]{circuitikz}
\sisetup{per-mode=fraction}
\sisetup{quotient-mode=fraction}
\DeclareSIUnit \decade {dec}
\lstset{basicstyle=\ttfamily}
\newcommand{\fillin}[1]{\underline{\hskip #1}}
\newcommand{\doublehrule}{\hrule \vskip 0.02in \hrule}
\newcommand*\circled[1]{\tikz[baseline=(char.base)]{
  \node[shape=circle,draw,inner sep=2pt] (char) {#1};}}

\definecolor{blueish}{rgb}{0.7,0.1,.7}
\newcommand{\sol}[1]{{\color{blueish} \textbf{Solution: } #1}} % solutions in pink

\newcommand{\meta}[1]{} % no meta in solutions

% eg, ie, etc, wrt, viz newcommands
\newcommand{\eg}{{\it e.g.\@}} % \@ for normal spacing after. use: \eg\/ or \eg,
\newcommand{\ie}{{\it i.e.\@}} 
\newcommand{\wrt}{{\it wrt}}
\newcommand{\viz}{{\it viz.\@}}
\newcommand{\vs}{{\it vs.\@}}
\newcommand{\aka}{{\it aka}}
\newcommand{\apriori}{{\it a priori}}
\newcommand{\etc}{\textit{etc.\@}}    
\newcommand{\bank}{../../questionBank}                                            
% etc defined so that there is only a single period after it if it is followed 
% by a period
% see: http://en.wikibooks.org/wiki/TeX/def                                      
% see: http://www.tug.org/pipermail/texhax/2004-July/002449.html                 
% see: http://en.wikibooks.org/wiki/TeX/if                                       
%\ifx \etc \undefined                                                             
%  %\def\myarg{#1}                                                                
%  %\def\myfullstop{.}                                                            
%  %\def\etc{#1}{\textit{etc} arg is ``\myarg'' \ifx\myarg\myfullstop\then.\else.#1\fi\/}
%  \def\etc#1{\textit{etc}\if#1..\else.#1\fi}                                     
%\fi                                                                              
%                        

% this should be moved to a new newcommands file
\newcommand{\ignore}[1]{}

% label and ref newcommands
% now mostly deprecated (but redefined and kept for backward compatibility) - move to \cref from the cleveref package
%\crefname{algocf}{alg.}{algs.}
%\Crefname{algocf}{Algorithm}{Algorithms}
%\crefname{equation}{}{}
%\Crefname{equation}{}{}

% this really should be in a newcommands-annotation.tex
\newcommand{\redHL}[1]{{\color{red}\bf#1}}

%DONE: xyzref{} should be able to handle multiple arguments so that you can
%reference multiple figures/equations at the same time -AG
%Now fixed with some package/command CREF.
\newcommand{\corrlabel}[1]{\label{corr:#1}}
\newcommand{\corrref}[1]{Corollary \ref{corr:#1}}
%\newcommand{\assref}[1]{Assumption \ref{ass:#1}}
\newcommand{\asslabel}[1]{\label{ass:#1}}
%\newcommand{\eqnref}[1]{(\ref{eq:#1})}
%\newcommand{\eqnlabel}[1]{\label{eq:#1}}
\newcommand{\eqnlabel}[1]{\label{#1}}
\newcommand{\eqnref}[1]{\cref{#1}}
%\newcommand{\figlabel}[1]{\label{fig:#1}}
%\newcommand{\figref}[1]{Fig.~\ref{fig:#1}}
\newcommand{\lstlabel}[1]{\label{lst:#1}}
\newcommand{\lstref}[1]{Listing~\ref{lst:#1}}
\newcommand{\figlabel}[1]{\label{#1}}
%\newcommand{\figref}[1]{\cref{#1}}
\newcommand{\chaplabel}[1]{\label{chap:#1}}
\newcommand{\seclabel}[1]{\label{sec:#1}}
%\newcommand{\secref}[1]{Sec.~\ref{sec:#1}}
%\newcommand{\chapref}[1]{Chapter~\ref{chap:#1}}
\newcommand{\exampleref}[1]{Example~\ref{ex:#1}}
\newcommand{\applabel}[1]{\label{app:#1}}
%\newcommand{\appref}[1]{Appendix~\ref{app:#1}}
\newcommand{\tablabel}[1]{\label{#1}}
%\newcommand{\tabref}[1]{\cref{#1}}
\newcommand{\thlabel}[1]{\label{th:#1}}
\newcommand{\thref}[1]{Theorem~\ref{th:#1}}
\newcommand{\thmlabel}[1]{\label{thm:#1}}
%\newcommand{\thmref}[1]{Theorem \ref{thm:#1}}
\newcommand{\deflabel}[1]{\label{def:#1}}
\newcommand{\defref}[1]{Definition~\ref{def:#1}}
\newcommand{\lemlabel}[1]{\label{lem:#1}}
%\newcommand{\lemref}[1]{Lemma \ref{lem:#1}}
\newcommand{\examplelabel}[1]{\label{ex:#1}}
\newcommand{\labexample}[1]{\label{example:#1}}
\newcommand{\refexample}[1]{Example{ }\ref{example:#1}}
\newcommand{\httpref}[1]{}
\let\el=\eqnlabel
\let\er=\eqnref

%%%%%% cross references (moved here from ee16.sty)
% \newcommand{\chapref}[1]{Chapter~\ref{#1}}
% \newcommand{\secref}[1]{Section~\ref{#1}}
% \newcommand{\figref}[1]{Figure~\ref{#1}}
% \newcommand{\tabref}[1]{Table~\ref{#1}}
% \newcommand{\exref}[1]{Exercise~\ref{#1}}
% %\newcommand{\eqref}[1]{Equation~(\ref{#1})}
% \newcommand{\partref}[1]{Part~\ref{#1}}
% \newcommand{\appref}[1]{Appendix~\ref{#1}}
% \newcommand{\sideref}[1]{the sidebar titled #1}
% \newcommand{\pgref}[1]{page~\pageref{#1}}


%\newcommand{\ddt}[2]{{\frac{d}{d #2}\left. #1 \right.}}
\newcommand{\ddt}[2]{{\frac{d #1}{d #2}}}
\newcommand{\dddt}[2]{{\frac{d^2 #1}{d #2 ^2}}}
\newcommand{\ddx}[2]{\frac{\partial{#1}}{\partial{#2}}}
\newcommand{\Ddx}[2]{{\frac{D}{d{#2}}\left[ #1 \right]}}
\newcommand{\qhat}{\hat{q}}
\newcommand{\Qhat}{\hat{Q}}
\newcommand{\fhat}{\hat{f}}
\newcommand{\Fhat}{\hat{F}}
\newcommand{\deq}{\triangleq}
%\newcommand{\I}{I\left[\right]}
\newcommand{\ELL}[1]{L\left[#1\right]}
\newcommand{\w}{\omega}
\newcommand{\calD}{{\cal D}}
\renewcommand{\jmath}{j}
\newcommand{\Pinvt}{{P(t)}^{-1}}
\newcommand{\halmos}{\hskip\textwidth minus\textwidth \rule{6pt}{6pt}}
\newcommand{\calA}{{\cal A}}
\newcommand{\calAT}{{\cal A}^T}
\newcommand{\calL}{{\cal L}}
\newcommand{\calLadj}{\calL^\dagger}
\newcommand{\calR}{{\cal R}}
\newcommand{\Reals}{\Bbb{R}}
\newcommand{\Real}{\Bbb{R}}
\newcommand{\Safe}{\ensuremath{{\bf S}}}
\newcommand{\Pade}{{Pad\'e}}
\newcommand{\Complexes}{\Bbb{C}}
\newcommand{\HvecqLanc}{\Hvec^{(L)}_q}
\newcommand{\HvecqArn}{\Hvec^{(A)}_q}
\newcommand{\Fvec}{\vec{F}}
\newcommand{\calDT}{{\cal D}^T}
\newcommand{\calCtil}{\tilde{\cal C}}
\newcommand{\calJtil}{\tilde{\cal J}}
\newcommand{\calDtil}{\tilde{\cal D}}
\newcommand{\JstarDphi}{J_{\Delta \phi}^*}
\newcommand{\Jstarphi}{J_{\phi}^*}
\newcommand{\vecbphi}{\vec b_{\phi}}
\newcommand{\vecbext}{\vec b_{\text{ext}}}
\newcommand{\phistarIdeal}{\phi_{\text{ideal}}^*}
\newcommand{\apdx}[1]{\appref{#1}} 
\newcommand{\delbydel}[2]{\ensuremath{\frac{\partial {#1}}{\partial {#2}}}}
\newcommand{\delbydeln}[3]{\ensuremath{\frac{\partial^{{#1}}
{#2}}{\partial{#3}^{{#1}}}}}
\newcommand{\delbydelat}[3]{\ensuremath{\left. \frac{\partial
{#1}}{\partial {#2}} \right|_{{#3}}}}
\newcommand{\dee}{\ensuremath{\mathrm{d}}}
\newcommand{\dbyd}[2]{\ensuremath{\frac{\mathrm{d} {#1}}{\mathrm{d}
{#2}}}}
\newcommand{\dbydn}[3]{\ensuremath{\frac{\mathrm{d}^{{#1}}{#2}}
{\mathrm{d}{#3}^{{#1}}}}}
\newcommand {\dist}[1] {{{\rm dist}\left({#1}\right)}}
\def\Matrix#1{\begin{pmatrix} #1 \end{pmatrix}}
\newcommand{\textdiag}{{\text{diag}}}
\newcommand{\ihat}{\hat{\imath}}
\newcommand{\jhat}{\hat{\jmath}}
\newcommand{\khat}{\hat{k}}
%\newcommand{\del}{\nabla}
\newcommand{\Rxx}{R_{\vec x \vec x}}
\newcommand{\Sxx}{S_{\vec x \vec x}}
\newcommand{\Snn}{S_{\vec n \vec n}}
\newcommand{\vecphi}{\vec \phi}
\newcommand{\Dphi}{\Delta \phi}
\newcommand{\vecDphi}{\vv{\Delta \phi}}
\newcommand{\dphi}{\delta \phi}
\newcommand{\vecdphi}{\vv{\delta \phi}}
\newcommand{\ah}{\alpha_g} % g for group
%\newcommand{\tDagger}{{\tilde t}}
\newcommand{\tDagger}{{t^{\!\dagger}}}
%\newcommand{\norm}[1]{\left\lVert#1\right\rVert}
%\newcommand{\abs}[1]{\left\lvert#1\right\rvert}
\newcommand{\texthyphen}{\text{--}}

%The substack command can be used to produce a multi-line subscripts or superscripts:
%\sum_{\substack{0\le i\le m\\ 0<j<n}} P(i,j) produces a two-line subscript underneath the sum.
%A slightly more general form is the subarray environment which allows you to specify that each line should be left-aligned instead of centered, as here:
% \sum_{\begin{subarray}{l} i\in\Lambda\\ 0<j<n \end{subarray}} P(i,j) 

%\newcommand{\be}[1]{\begin{equation}\eqnlabel{#1}} % now moved to cleveref
\newcommand{\be}[1]{\begin{equation}\label{#1}}
\def\ee{\end{equation}}

\def\endproofmark{$\Box$}
\newenvironment{proof}{\par{\bf Proof}:}{\endproofmark}
\newenvironment{proofof}{\par{\bf Proof }}{\endproofmark}


\newcounter{axiomnumber}
\setcounter{axiomnumber}{0}
\renewcommand\theaxiomnumber{\the\lecturenumber.\arabic{axiomnumber}}
\newenvironment{axiom}[1]{\par\refstepcounter{axiomnumber}
{\bf Axiom \theaxiomnumber\ (#1)}:
\begingroup}%
{\endgroup}

\newcounter{defnnumber}
\setcounter{defnnumber}{0}
\renewcommand\thedefnnumber{\the\lecturenumber.\arabic{defnnumber}}
\newenvironment{defn}[1]{\par\refstepcounter{defnnumber}
{\bf Definition \thedefnnumber\ (#1)}:
\begingroup}%
{\endgroup}

\newcounter{theoremnumber}
\setcounter{theoremnumber}{0}
\renewcommand\thetheoremnumber{\the\lecturenumber.\arabic{theoremnumber}}
\newenvironment{theorem}{\par\refstepcounter{theoremnumber}
{\bf Theorem \thetheoremnumber}:
\begingroup}%
{\endgroup}

\newcounter{lemmanumber}
\setcounter{lemmanumber}{0}
\renewcommand\thelemmanumber{\the\lecturenumber.\arabic{lemmanumber}}
\newenvironment{lemma}{\par\refstepcounter{lemmanumber}
{\bf Lemma \thelemmanumber}:
\begingroup}%
{\endgroup}

\newenvironment{corollary}{\par\refstepcounter{theoremnumber}
{\bf Corollary \thetheoremnumber}:
\begingroup}%
{\endgroup}

%%%%%% additional symbols or names for them

%\newcommand{\implies}{\:\;{\Rightarrow}\:\;}
\newcommand{\impliessymbol}{\Rightarrow}
\newcommand{\entails}{\models}
\newcommand{\lequiv}{\;\;{\Leftrightarrow}\;\;}
\newcommand{\lequivsymbol}{\Leftrightarrow}
\newcommand{\xor}{\not\lequiv}
\newcommand{\All}[1]{\forall\,#1\;\;}
\newcommand{\Exi}[1]{\exists\,#1\;\;}
\newcommand{\Exii}[1]{\exists!\,#1\;\;}% -pnorvig
\newcommand{\Iot}[2]{\iota\,#1\,#2}
\newcommand{\Lam}[2]{\lambda #1\;#2}
\newcommand{\Qua}[3]{[#1\,#2\;#3]}

\def\<{\langle}
\def\>{\rangle}

\newcommand{\union}{{\,{\cup}\,}}
\newcommand{\elt}{{\,{\in}\,}}  %%%cuts down on spacing
\newcommand{\eq}{{\,{=}\,}}     %%%cuts down on spacing
\def\stimes{{\,\times\,}}       %%%cuts down on spacing

%\def\ceil#1{\lceil #1 \rceil}
%\def\floor#1{\lfloor #1 \rfloor}
\DeclarePairedDelimiter\ceil{\lceil}{\rceil}
\DeclarePairedDelimiter\floor{\lfloor}{\rfloor}
\DeclarePairedDelimiterX{\innp}[2]{\langle}{\rangle}{#1, #2}
%\DeclareMathOperator*{\argmin}{argmin}
\newcommand*{\argmin}{\operatornamewithlimits{argmin}\limits} % moved to ee16.sty
\newcommand*{\argmax}{\operatornamewithlimits{argmax}\limits} % moved to ee16.sty

%\def\cents{\mbox{c}}
%\AtBeginDocument{\def\cents{\hbox{\rm\rlap{\hspace{.07ex}$/$}c}}}
\newcommand{\cents}{\text{\textcent{}}}
\def\cons{{}\bullet{}}

%%%%%% bold font in math mode; this sucks but is simplest for now
\newcommand{\mbf}[1]{\mbox{{\bfseries #1}}}
\newcommand{\smbf}[1]{\mbox{{\scriptsize\bfseries #1}}}

\def\N{\mathbb{N}}
\def\R{\mathbb{R}}
\def\C{\mathbb{C}}
\def\X{\mbf{X}}
\def\x{\mbf{x}}
\def\sx{\smbf{x}}
\def\Y{\mbf{Y}}
\def\y{\mbf{y}}
\def\sy{\smbf{y}}
\def\E{\mbf{E}}
\def\e{\mbf{e}}
\def\T{\mbf{T}}
\def\O{\textrm{O}}  % repeated below because it gets redefined by some package?
\def\Q{\mathbb{Q}}
\def\se{\smbf{e}}
\def\Z{\mathbb{Z}}
\def\z{\mbf{z}}
\def\sz{\smbf{z}}
\def\F{\mathbb{F}}
\def\f{\mbf{f}}
\def\A{\mbf{A}}
\def\P{\mbf{P}}
\def\B{\mbf{B}}
\def\b{\mbf{b}}
\def\m{\mbf{m}}
\def\I{\mbf{I}}
\def\ones{\mbf{1}}
\def\ev{\mbf{ev}}
\def\fv{\mbf{ev}}
\def\sv{\mbf{sv}}
\def\e{\mathop{\mathrm{e}}\nolimits}  % for e = 2.718...

\def\third{{\textstyle{1\over 3}}}
\def\half{{\textstyle{1\over 2}}}
\def\quarter{{\textstyle{1\over 4}}}
\def\VarOmega{\mathchar"10A }
\def\Ex#1{{\mathbb E}(#1)}
\def\Var#1{{\rm Var}(#1)}
\def\Aset{{\cal A}}
\def\Bset{{\cal B}}

\def\Bin{\text{Bin}}
\def\Geom{\text{Geom}}
\def\Poiss{\text{Poiss}}
\def\Exp{\text{Exp}}
\def\Norm{N}

\newcommand{\mymod}{\textup{mod}}

\def\mdw@dots#1{\ensuremath{\mathpalette\mdw@dots@i{#1}}}
\def\mdw@dots@i#1#2{%
  \setbox\z@\hbox{$#1\mskip1.8mu$}%
  \dimen@\wd\z@%
  \setbox\z@\hbox{$#1.$}%
  #2%
}
\def\Ddots{%
  \mdw@dots{\mathinner{%
    \mkern1mu%
    \raise\dimen@\vbox{\kern7\dimen@\copy\z@}%
    \mkern2mu%
    \raise4\dimen@\copy\z@%
    \mkern2mu%
    \raise7\dimen@\box\z@%
    \mkern1mu%
  }}%
}




\usetikzlibrary{positioning}

\begin{document}

\def\title{Worksheet 4}

\newcommand{\qitem}{\qpart\item}

\renewcommand{\labelenumi}{(\alph{enumi})} % change default enum format to (a)
\renewcommand{\theenumi}{(\alph{enumi})} % fix reference format accordingly.
\renewcommand{\labelenumii}{\roman{enumii}.} % second level labels.
\renewcommand{\theenumii}{\roman{enumii}.}

\maketitle

\vspace{0.5em}

\begin{qunlist}

\qns{Transfer Function}

Create a Bode plot of the following transfer function:

$$H(\omega) = \frac{1}{10} \frac{((j\omega)^2 + 110j\omega + 1000)(j\omega + 10000)}
	{(j\omega+1000)((j\omega)^2 + 101j\omega + 100)}$$

\sol{

First of all, we decompose the second-order terms:
$$H(\omega) = \frac{1}{10} \frac{((j\omega)+100)(j\omega+10))(j\omega + 10000)}
	{(j\omega+1000)((j\omega)+100)(j\omega+1))}  =
		 \frac{1}{10} \frac{(j\omega+10)(j\omega + 10000)}
	{(j\omega+1000)(j\omega+1)}$$
Then, we convert it to the normal form:
$$H(\omega) =
		 \frac{10}{1} \frac{(\frac{j\omega}{10}+1)(\frac{j\omega}{10000} + 1)}
	{(\frac{j\omega}{1000}+1)(j\omega+1)}$$
	
\begin{figure}[H]
\centering
\scalebox{0.9}{\includegraphics[scale=0.6]{\bank/transfer/figures/q_complicated_fixed.png}}
\end{figure}

}

\newpage
\qcontributor{Saavan Patel}
\qcontributor{Siddharth Iyer}

\section{Transfer Function}

When we write the transfer function of an arbitrary circuit, it always takes the following form. This is called a ``rational transfer function.'' We also like to factor the numerator and denominator, so that they become easier to work with and plot:

\begin{align*}
H(\omega) &= \frac{z(\omega)}{p(\omega)} =\frac{(j\omega)^{N_{z0}}}{(j\omega)^{N_{p0}}} \left( \frac{(j \omega)^n \alpha_n + (j\omega)^{n-1} \alpha_{n-1} + \cdots + j\omega \alpha_1 + \alpha_0}{(j \omega)^m \beta_m + (j\omega)^{m-1} \beta_{m-1} + \cdots + j\omega \beta_1 + \beta_0} \right) \\
&= K  \frac{(j\omega)^{N_{z0}} \left(1 + j\frac{\omega}{\omega_{z1}}\right)\left(1 + j\frac{\omega}{\omega_{z2}}\right) \cdots \left(1 + j\frac{\omega}{\omega_{zn}}\right)}{(j\omega)^{N_{p0}} \left(1 + j\frac{\omega}{\omega_{p1}}\right)\left(1 + j\frac{\omega}{\omega_{p2}}\right) \cdots \left(1 + j\frac{\omega}{\omega_{pm}}\right)}
\end{align*}

Here, we define the constants $\omega_{z}$ as ``zeros'' and $\omega_{p}$ as ``poles'', and $N_{z0}$, $N_{p0}$ are the number of zeros and poles at $\omega = 0$ 

\section{Bode Plots}
Bode plots provide us with a simple and easy tool to plot these transfer functions by hand. \textbf{Always remember that Bode plots are an approximation}; if you want the precisely correct plots, you need to use numerical methods (like solving using MATLAB or IPython).

When we make Bode plots, we plot the frequency and magintude on a logarithmic scale
and the angle in either degrees or radians.We use the log scale because it allows us to break up complex transfer functions into its constituent components. 

When making the Bode plot (and plotting using a logarithmic unit), we treat each individual pole and zero independently, and then add them back together at the end. We can use the Bode plot rules to help us plot each of the individual poles and zeros.

\[
|H(\omega)|  =  \log_{10}\left({\frac{V_{out}}{V_{in}}}  \right)
\]

\newpage
\subsection{Algorithm}
Given a frequency response $H(\omega)$,
\begin{enumerate}
\item
  Break $H(\omega)$ into a product of poles and zeros.
  Appropriately divide terms to reduce $H(\omega)$ into one of the given forms. 
\item
  Draw out the Bode plot for each pole and zero in the product above.
\item
  Add the resulting plots to get the final Bode plot.
\end{enumerate}


\newpage
% Author: Taejin Hwang
% Email: taejin@berkeley.edu

\qns{An Introduction to Systems}

In the next couple of worksheets, we will be examining systems. 
Many physical systems such as the motion of a car, can be modeled using a system.
Often times, when we are describing a system, we will have a \textbf{state variable $\vec{x},$}
that will often be a multivariable function. 
For a given system, we can often write a differential equation describing its change over time as
\begin{align}
\ddt{\vec{x}(t)}{t} = f(\vec{x}(t))
\end{align}

In this problem, we will examine a specific form of systems that can be put in \textbf{State-space representation.} \vspace{0.5 cm}
For linear systems (we will define what it means to be linear later) the state-space representation is:
\begin{align}
\ddt{\vec{x}(t)}{t} = A \vec{x}(t) + B \vec{u}(t)
\end{align}

Where $A$ is the $n \times n$ state matrix, $\vec{x}$ is a state vector in $\mathbb{R}^n$, $B$ is a $n \times d$ input matrix, and $\vec{u}$ is an input vector in $\mathbb{R}^d$. We will usually consider a $B$ as a vector in $\mathbb{R}^n$ and $u(t)$ will be a scalar input.

Given the following system:
\begin{align*}
    \ddt{}{t}x_{1}(t) &= 3 x_{1}(t) - 2 x_{2}(t) + 4 \\
    \ddt{}{t}x_{2}(t) &= - x_{1}(t) + 5 x_{2}(t) + 2
\end{align*}

With initial conditions $x_{1}(0) = 2, \ x_{2}(0) = 3,$

\begin{enumerate}
    \qitem What is an appropriate state vector for this system?

    \sol{
        We have to variables $x_1$ and $x_2$ therefore we define our state vector as:
        $$\vec{x} = \begin{bmatrix} x_{1} \\ x_{2} \end{bmatrix}$$
    }

    \qitem What is the initial condition of this system?

    \sol{
        We have the individual initial conditions for $x_1$ and $x_2$ but we must also a define an initial condition for our state vector.
        $$\vec{x}(0) = \begin{bmatrix} x_{1}(0) \\ x_{2}(0) \end{bmatrix} = \begin{bmatrix} 2 \\ 3 \end{bmatrix} $$
    }

    \qitem Write out the system of differential equations as in state-space form.
    
    \sol{
        $$\ddt{}{t}\vec{x}(t) = \begin{bmatrix} 3 & -2 \\ -1 & 5 \end{bmatrix} \begin{bmatrix} x_{1}(t) \\ x_{2}(t) \end{bmatrix} 
        = \begin{bmatrix} 3 & -2 \\ -1 & 5 \end{bmatrix} \vec{x}(t) + \begin{bmatrix} 4 \\ 2 \end{bmatrix}$$
    }
\end{enumerate}
\end{qunlist}

\end{document}

