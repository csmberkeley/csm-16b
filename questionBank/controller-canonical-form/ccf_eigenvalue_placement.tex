\qns{Controllable Canonical Form Practice}
% \qcontributor{Yen-Sheng Ho}

Now, consider the specific system
\begin{align} \label{eq:sys}
\vec{x}(t+1) = A \vec{x}(t) + \vec{b} u(t) = \begin{bmatrix}
-2 & 0 \\
-3 & -1
\end{bmatrix} \vec{x}(t) + 
\begin{bmatrix}
\frac{1}{2} \\
\frac{1}{2}
\end{bmatrix} u(t)
\end{align}

\begin{enumerate}
\qitem Show that the system \eqref{eq:sys} is controllable.

\sol{
\begin{align*}
C = \begin{bmatrix}
\vec{b} & A\vec{b}
\end{bmatrix}
 = \begin{bmatrix}
 \frac{1}{2} & -1 \\
 \frac{1}{2} & -2
 \end{bmatrix}
 \mbox{ (full rank)}
\end{align*}
}
\end{enumerate}

Since the system is controllable,
there exists a transformation $\vec{z} = T\vec{x}$ such that
\begin{align} \label{eq:canonical}
\vec{z}(t+1) = \widetilde{A}\vec{z}(t) + \widetilde{\vec{b}} u(t) = \begin{bmatrix}
0 & 1  \\
 -a_0 & -a_1 \\
\end{bmatrix} 
\vec{z}(t) +
\begin{bmatrix}
0  \\ 1
\end{bmatrix}
u(t)
\end{align}

\begin{enumerate}[resume]
\qitem Compute the matrix $\widetilde{A}$. 

\sol{
\begin{align*}
\mbox{det}(\lambda I - A) = (\lambda+2)(\lambda+1) = \lambda^2 +3 \lambda + 2 = \lambda^2 + a_1 \lambda + a_0
\end{align*}
By inspection, we have $a_1 = 3$ and $a_0 = 2$. Thus,
\begin{align*}
\widetilde{A} = \begin{bmatrix}
0 & 1 \\
-2 & -3
\end{bmatrix}
\end{align*}
}

\qitem Compute the controllability matrices $C = \begin{bmatrix}
\vec{b} & A\vec{b}
\end{bmatrix}$
and 
$\widetilde{C} = \begin{bmatrix}
\widetilde{\vec{b}} & \widetilde{A}\widetilde{\vec{b}} 
\end{bmatrix}$.

\sol{
\begin{align*}
C 
 = \begin{bmatrix}
 \frac{1}{2} & -1 \\
 \frac{1}{2} & -2
 \end{bmatrix}
\end{align*}
\begin{align*}
\widetilde{C} = \begin{bmatrix}
0 & 1 \\ 
1 & -3
\end{bmatrix}
\end{align*}
}

\qitem Compute the transformation matrix $T = \widetilde{C}C^{-1}$.

\sol{

First, we compute
\begin{align*}
C^{-1} = 
\frac{1}{\frac{1}{2}(-2)-\frac{1}{2}(-1)}\begin{bmatrix}
-2 & 1 \\
-\frac{1}{2} & \frac{1}{2}
\end{bmatrix}
=
\begin{bmatrix}
4 & -2 \\
1 & -1
\end{bmatrix}
\end{align*}

Then we have
\begin{align*}
T = \begin{bmatrix}
0 & 1 \\ 
1 & -3
\end{bmatrix} \begin{bmatrix}
4 & -2 \\
1 & -1
\end{bmatrix} =
\begin{bmatrix}
1 & -1 \\
1 & 1
\end{bmatrix}
\end{align*}
}

\qitem Show that the system (\ref{eq:sys}) is \textit{unstable} in open-loop.

\sol{
From the characteristic polynomial $\lambda^2 +3 \lambda + 2$,
we know it is unstable since the eigenvalues are $\lambda = -1 \mbox{ and }-2$.
}
\end{enumerate}

Now, we want to make the system \textit{stable} by applying state feedback for the system in the canonical form (\ref{eq:canonical}).
That is, let $u(t)$ be $u(t) = -\vec{k}^T\vec{z}(t) = \begin{bmatrix}
-k_0 & -k_1
\end{bmatrix}
\vec{z}(t)$.
After applying state feedback, the systems (\ref{eq:sys}) and (\ref{eq:canonical}) have the form
\begin{align*}
\vec{x}(t+1) &= A_{cl} \vec{x}(t) \\
\vec{z}(t+1) &= \widetilde{A}_{cl} \vec{z}(t)
\end{align*}

\begin{enumerate}[resume]

\qitem Compute $A_{cl}$ and $\widetilde{A}_{cl}$ in terms of $k_0$ and $k_1$.

\sol{

For the original system, we have (note that $u(t) = -\vec{k}^T\vec{z}(t) = -\vec{k}^T T\vec{x}(t)$)
\begin{align*}
\vec{x}(t+1) &= A\vec{x}(t) + \vec{b}(-\vec{k}^T T\vec{x}(t))\\
&= (A - \vec{b}\vec{k}^T T) \vec{x}(t)
\end{align*}
Thus,
\begin{align*}
A_{cl} &= A + \vec{b}(-\vec{k}^T)T 
= \begin{bmatrix}
-2 & 0 \\
-3 & -1
\end{bmatrix}
 + \frac{1}{2}\begin{bmatrix}
 -k_0 -k_1 & k_0 - k_1 \\
 -k_0 - k_1 & k_0 - k_1
 \end{bmatrix}
 = \begin{bmatrix}
 -2 -\frac{1}{2}k_0 - \frac{1}{2} k_1 & 0 + \frac{1}{2}k_0 - \frac{1}{2} k_1\\
  -3 -\frac{1}{2}k_0 - \frac{1}{2} k_1 & -1 + \frac{1}{2}k_0 - \frac{1}{2} k_1
 \end{bmatrix}
 \end{align*}

For the canonical form, we have 
\begin{align*}
\vec{z}(t+1) &= \widetilde{A}\vec{z}(t) + \widetilde{\vec{b}}(-\vec{k}^T \vec{z}(t))\\
&= (\widetilde{A} - \widetilde{\vec{b}}\vec{k}^T )\vec{z}(t)
\end{align*}

Thus,
 \begin{align*}
 \widetilde{A}_{cl} &= \widetilde{A} + \widetilde{\vec{b}}(-\vec{k}^T) 
= \begin{bmatrix}
0 & 1 \\
-2 & -3
\end{bmatrix}
 + \begin{bmatrix}
 0 & 0 \\
 -k_0 & -k_1
 \end{bmatrix}
 = \begin{bmatrix}
 0 & 1 \\
 -2 - k_0 & -3 - k_1
 \end{bmatrix}
\end{align*}
}

\qitem Compute $\vec{k}$ so that $\widetilde{A}_{cl}$ has eigenvalues $\lambda = \pm \frac{1}{2}$ (\textit{Hint: use Formula (\ref{eq:poly})}).

\sol{

First, for the given eigenvalues we have
\begin{align*}
(\lambda+\frac{1}{2}) (\lambda-\frac{1}{2}) = \lambda^2 -\frac{1}{4}
\end{align*}

Second, by comparison
\begin{align*}
-a_1 &=  0 = -3 - k_1 \\
-a_0 &= -(-\frac{1}{4}) = -2 - k_0
\end{align*}
We have
\begin{align*}
k_1 &= -3 \\
k_0 &= -\frac{9}{4}
\end{align*}
}

\qitem Using the $\vec{k}$ you derived in the previous part, show that
$A_{cl}$ also has eigenvalues $\lambda = \pm \frac{1}{2}$ by explicit calculation.

\meta {
  This part emphasizes the idea doing feedback with the $K$ matrix is much easier in the Controller basis as opposed to the standard basis.
}

\sol{
\begin{align*}
A_{cl} = 
\begin{bmatrix}
 -2 -\frac{1}{2}k_0 - \frac{1}{2} k_1 & 0 + \frac{1}{2}k_0 - \frac{1}{2} k_1\\
  -3 -\frac{1}{2}k_0 - \frac{1}{2} k_1 & -1 + \frac{1}{2}k_0 - \frac{1}{2} k_1
 \end{bmatrix}
= \begin{bmatrix}
\frac{5}{8} & \frac{3}{8} \\
\frac{-3}{8} & \frac{-5}{8}
\end{bmatrix}
\end{align*}

The characteristic polynomial is given by
\begin{align*}
(\frac{5}{8} - \lambda) (\frac{-5}{8}-\lambda) + \frac{9}{64} = \lambda^2 - \frac{16}{64} = \lambda^2 - \frac{1}{4}
\end{align*}

Thus, $\lambda = \pm \frac{1}{2}$.
}

\end{enumerate}
