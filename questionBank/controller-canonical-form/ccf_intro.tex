% Author: Yen-Sheng Ho

\qns{Controller Canonical Form Introduction}

\meta {
  Note that the proof for CCF is unconventional in that we assume there exists a transformation $T$ from the standard to controller basis, and then we derive what the actual transformation is, and show that it can only exist when the system is controllable.
}

Consider a linear discrete time system below ($\vec{x} \in
\mathbb{R}^n$, $u \in \mathbb{R}$, and $\vec{b} \in \mathbb{R}^n$).
\begin{align*}
\vec{x}(t+1) = A \vec{x}(t) + \vec{b} u(t)
\end{align*}

If the system is \textit{controllable},
then there exists a change of coordinates $\vec{z} = T\vec{x}$ (where $T$ is an invertible $n\times n$ matrix)
such that in the transformed coordinates, the system is in
\textit{controller canonical form} (CCF), which is given by
\begin{align*}
\vec{z}(t+1) = \widetilde{A}\vec{z}(t) + \widetilde{\vec{b}} u(t) = \begin{bmatrix}
0 & 1 & 0 & \cdots & 0 \\
0 & 0 & 1 & \cdots & 0 \\
\vdots & \vdots & \vdots  & \ddots & \vdots \\
 0 & 0 & 0 & \cdots & 1 \\
 a_0 & a_{1} & a_{2} & \ldots & a_{n-1} \\
\end{bmatrix} 
\vec{z}(t) +
\begin{bmatrix}
0 \\ 0 \\ 0 \\ \vdots \\ 1
\end{bmatrix}
u(t)
\end{align*}

The characteristic polynomials of the matrices $A$ and $\widetilde{A}$ are the same and given by
\begin{align*}\label{eq:poly}
\boxed{\mbox{det}(A - \lambda I) = \mbox{det}(\widetilde{A} - \lambda I)
= \lambda^n  - a_{n-1} \lambda^{n-1} - a_{n-2} \lambda^{n-2} - \ldots - a_0}
\end{align*}

\begin{enumerate}
\qitem \textbf{Show that $\widetilde{A} = TAT^{-1}$ and $\widetilde{\vec{b}} = T\vec{b}$.}

\ws{\vspace{150px}}
\sol{

Starting from the original system, we have 
\begin{align*}
\vec{x}(t+1) &= A \vec{x}(t) + \vec{b} u(t) \\
\Rightarrow T\vec{x}(t+1) &= TA \vec{x}(t) + T\vec{b} u(t) \\
\Rightarrow T\vec{x}(t+1) &= TA (T^{-1}T) \vec{x}(t) + T\vec{b} u(t) \\
\Rightarrow \vec{z}(t+1) &= TAT^{-1}\vec{z}(t) + T\vec{b} u(t) 
\end{align*}

Comparing to the canonical form, we have
\begin{align*}
\widetilde{A} = TAT^{-1} \mbox{, and }  \widetilde{\vec{b}} = T\vec{b}
\end{align*}
}

\qitem \textbf{Show that $A$ and $\widetilde{A}$ have the same eigenvalues.}
\textit{Hint: let $\vec{v}$ be an eigenvector of $A$. Use $T\vec{v}$ as an eigenvector for $\widetilde{A}$.}

\ws{\vspace{150px}}
\sol{

Let $\lambda$ and $\vec{v}$ be an eigenvalue and its corresponding eigenvector of $A$. That is, $A\vec{v} = \lambda \vec{v}$.
Next we consider $\widetilde{A}T\vec{v}$:
\begin{align*}
\widetilde{A}(T\vec{v}) &=  TAT^{-1} T\vec{v} \\
&= TA\vec{v}\\
&= T(\lambda\vec{v}) \\
&= \lambda(T\vec{v})
\end{align*}
We have shown that $\lambda$ is also an eigenvalue of $\widetilde{A}$ and its corresponding eigenvector is $T\vec{v}$.
}

\qitem Let's show that the characteristic polynomial has matching coefficients to the bottom row of $\widetilde{A}$ for a 2D system:
$$\vec{z}(t + 1) = \begin{bmatrix} 0 & 1 \\ a_0 & a_1 \end{bmatrix} \vec{z}(t) + \begin{bmatrix} 0 \\ 1 \end{bmatrix}u(t)$$

\textbf{Find the characteristic polynomial of this system in the CCF basis $\operatorname{det}(\widetilde{A} - \lambda I)$. Show that it matches the boxed formula above.}

\ws{\vspace{100px}}
\sol{
  \begin{align*}
    \operatorname{det}(\widetilde{A} - \lambda I) &= (-\lambda)(a_1 - \lambda) - 1(a_0) \\
    &= \lambda^2 - a_1 \lambda - a_0
  \end{align*}

  This matches the formula, where $n = 2$.
}

\qitem Since we've shown that the eigenvalues of $\widetilde{A}$ in the new basis are equal to the original eigenvalues of the system, \textit{if we use system feedback to do eigenvalue placement in the CCF basis, then we'll have accomplished our goal.}

The CCF basis provides a nice way to perform system feedback, since the characteristic polynomial is conveniently determined by the bottom row of the $\widetilde{A}$ matrix and our $\vec{b}$ vector allows us to arbitrarily modify the last row.

Let the controllability matrix for the \textit{original basis} be:
$$\mathcal{C} = \begin{bmatrix} \vec{b} & A\vec{b} & \cdots & A^{n-1}\vec{b} \end{bmatrix}$$

Let the controllability matrix in the \textit{transformed CCF basis} be:
$$\widetilde{\mathcal{C}} = \begin{bmatrix}
\widetilde{\vec{b}} & \widetilde{A}\widetilde{\vec{b}} & \cdots & \widetilde{A}^{n-1}\widetilde{\vec{b}}
\end{bmatrix}$$

\textbf{Show that $\widetilde{\mathcal{C}} = T\mathcal{C}$, which is equivalent to $T = \widetilde{\mathcal{C}}\mathcal{C}^{-1}$.}

\sol{
\begin{align*}
\widetilde{\mathcal{C}} &= \begin{bmatrix}
\widetilde{\vec{b}} & \widetilde{A}\widetilde{\vec{b}} & \cdots & \widetilde{A}^{n-1}\widetilde{\vec{b}}
\end{bmatrix} \\
&= \begin{bmatrix}
T{\vec{b}} & (T{A}T^{-1})(T{\vec{b}}) & \cdots & (T{A}^{n-1}T^{-1})(T{\vec{b}})
\end{bmatrix} \\
&= \begin{bmatrix}
T{\vec{b}} & T{A}{\vec{b}} & \cdots & T{A}^{n-1}{\vec{b}}
\end{bmatrix} \\
&= T\mathcal{C} \\
\implies T &= \widetilde{\mathcal{C}}\mathcal{C}^{-1}
\end{align*}

Notice that this transformation $T$ is only valid when $\mathcal{C}$ is invertible $\iff$ $\mathcal{C}$ is full rank (i.e. the system is controllable).
}

\end{enumerate}
