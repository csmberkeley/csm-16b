% Author: Dun-Ming Huang
% Email: dunmingbrandonhuang@berkeley.edu
% CSM16A Fall 2022
\qns{We Null The Mechanics and We're Gonna Play It}

\textbf{Leaning Goal}: Learn the mechanical linear algebra for finding nullspaces. \\
Find the nullspace for the following matrices:

\begin{tasks}(3)
    \task {
        $
            \begin{bmatrix}
                0 & 0 \\
                0 & 1
            \end{bmatrix}
        $
    }
    \task {
        $
            \begin{bmatrix}
                2 & 1 \\
                4 & 2
            \end{bmatrix}
        $
    }
    \task {
        $
            \begin{bmatrix}
                1 & 3 & 2 \\
                5 & 15 & 10 \\
                2 & 6 & 4 \\
            \end{bmatrix}
        $
    }
    \task {
        $
            \begin{bmatrix}
                1 & 3 & 2 \\
                1 & 1 & 0 \\
                0 & 2 & 2 \\
            \end{bmatrix}
        $
    }
    \task {
        $
            \begin{bmatrix}
                1 & 0 & -5 \\
                0 & -2 & 4 \\
                -3 & 1 & 1 \\
            \end{bmatrix}
        $
    }
    \task {
        $
            \begin{bmatrix}
                2 & 1 & -3\\
                1 & -2 & -4\\
                -2 & 2 & 6
            \end{bmatrix}
        $
    }
\end{tasks}
Do you notice something specifically about the matrix of part (e)?

\meta{
    \textbf{Introduce the general approach of finding nullspaces} (an algorithmic one, to be specific) before starting the question!
    
}
\ans{
    Review: To find the nullspace of such a matrix $A$, we would set up a system:
            \[A \vec{x} = \vec{0}\]
    and attempt to find the set of all possible $\vec{x}$. \\
    \begin{enumerate}
        \item[a)] {
            Let us construct a system:
            \begin{align*}
                \begin{sysmatrix}{cc|c}
                    0 & 0 & 0 \\
                    0 & 1 & 0
                \end{sysmatrix}
            \end{align*}
            At this point, for a solution $\begin{bmatrix} x_1 & x_2 \end{bmatrix}^T$, while it must be $x_2 = 0$ so a contradiction does not occur, $x_1$ can be any value. \\
            Therefore:
            \[
                N \bigg(
                    \begin{bmatrix}
                        0 & 0 \\
                        0 & 1
                    \end{bmatrix}
                \bigg)
                =
                \bigg\{
                    \alpha
                    \begin{bmatrix}
                        1 \\ 0
                    \end{bmatrix}
                    \ :\ \alpha \in \R
                \bigg\}
            \]
        }
        \item[b)] {
            Let us construct a system:
            \begin{align*}
                \begin{sysmatrix}{cc|c}
                    2 & 1 & 0 \\
                    4 & 2 & 0
                \end{sysmatrix}
                &\!\begin{aligned}
                    &\ro{R_1 \rightarrow R_1} \\
                    &\ro{R_2 \rightarrow R_2 - 2R_1}
                \end{aligned}
                \begin{sysmatrix}{cc|c}
                    2 & 1 & 0 \\
                    0 & 0 & 0
                \end{sysmatrix}
            \end{align*}
            At this point, for a solution $\begin{bmatrix} x_1 & x_2 \end{bmatrix}^T$, we can configure the relationship that $x_2 = -2x_1$. Therefore:
            \[
                N \bigg(
                    \begin{bmatrix}
                        2 & 1 \\
                        4 & 2
                    \end{bmatrix}
                \bigg)
                =
                \bigg\{
                    \alpha
                    \begin{bmatrix}
                        -2 \\ 1
                    \end{bmatrix}
                    \ :\ \alpha \in \R
                \bigg\}
            \]
        }
        \item[c)] {
            Let us construct a system:
            \begin{align*}
                \begin{sysmatrix}{ccc|c}
                    1 & 3 & 2 & 0 \\
                    5 & 15 & 10 & 0 \\
                    2 & 6 & 4 & 0 \\
                \end{sysmatrix}
                &\!\begin{aligned}
                    &\ro{R_1 \rightarrow R_1} \\
                    &\ro{R_2 \rightarrow R_2 - 5R_1} \\
                    &\ro{R_3 \rightarrow R_3 - 2R_1}
                \end{aligned}
                \begin{sysmatrix}{ccc|c}
                    1 & 3 & 2 & 0 \\
                    0 & 0 & 0 & 0 \\
                    0 & 0 & 0 & 0 \\
                \end{sysmatrix}
            \end{align*}
            At this point, for a solution $\begin{bmatrix} x_1 & x_2 & x_3 \end{bmatrix}^T$, we can configure the relationship that $x_1 = -3x_2 - 2x_3$, for free variables $x_2$ and $x_3$. Therefore:
            \[
                N \Bigg(
                    \begin{bmatrix}
                        1 & 3 & 2 \\
                        5 & 15 & 10 \\
                        2 & 6 & 4 \\
                    \end{bmatrix}
                \Bigg)
                =
                \Bigg\{
                    \alpha
                    \begin{bmatrix}
                        -3 \\ 1 \\ 0
                    \end{bmatrix} +
                    \beta
                    \begin{bmatrix}
                        -2 \\ 0 \\ 1
                    \end{bmatrix}
                    \ :\ \alpha, \beta \in \R
                \Bigg\}
            \]
        }
        \item[d)] {
            Let us construct a system:
            \begin{align*}
                \begin{sysmatrix}{ccc|c}
                    1 & 3 & 2 & 0 \\
                    1 & 1 & 0 & 0 \\
                    0 & 2 & 2 & 0 \\
                \end{sysmatrix}
                &\!\begin{aligned}
                    &\ro{R_1 \rightarrow R_1} \\
                    &\ro{R_2 \rightarrow R_2 - R_1} \\
                    &\ro{R_3 \rightarrow R_3}
                \end{aligned}
                \begin{sysmatrix}{ccc|c}
                    1 & 3 & 2 & 0 \\
                    0 & 2 & 2 & 0 \\
                    0 & 2 & 2 & 0 \\
                \end{sysmatrix} \\
                &\!\begin{aligned}
                    &\ro{R_1 \rightarrow R_1} \\
                    &\ro{R_2 \rightarrow \frac{1}{2} R_2} \\
                    &\ro{R_3 \rightarrow R_2 - R_3}
                \end{aligned}
                \begin{sysmatrix}{ccc|c}
                    1 & 3 & 2 & 0 \\
                    0 & 1 & 1 & 0 \\
                    0 & 0 & 0 & 0 \\
                \end{sysmatrix}
            \end{align*}
            At this point, for a solution $\begin{bmatrix} x_1 & x_2 & x_3 \end{bmatrix}^T$, we can configure the relationship that:
            \[
                \begin{cases}
                    x_1 = -3x_2 - 2x_3 \\
                    x_2 = -x_3 \\
                    \rightarrow x_1 = x_3
                \end{cases}
            \]
            Therefore:
            \[
                N \Bigg(
                    \begin{bmatrix}
                        1 & 3 & 2 \\
                        1 & 1 & 0 \\
                        0 & 2 & 2 \\
                    \end{bmatrix}
                \Bigg)
                =
                \Bigg\{
                    \alpha
                    \begin{bmatrix}
                        1 \\ -1 \\ 1
                    \end{bmatrix}
                    \ :\ \alpha \in \R
                \Bigg\}
            \]
        }
        \item[e)] {
            Let us construct a system:
            \begin{align*}
                \begin{sysmatrix}{ccc|c}
                    1 & 0 & -5 & 0 \\
                    0 & -2 & 4 & 0 \\
                    -3 & 1 & 1 & 0 \\
                \end{sysmatrix}
                &\!\begin{aligned}
                    &\ro{R_1 \rightarrow R_1} \\
                    &\ro{R_2 \rightarrow -\frac{1}{2} R_2} \\
                    &\ro{R_3 \rightarrow 3R_1 + R_3}
                \end{aligned}
                \begin{sysmatrix}{ccc|c}
                    1 & 0 & -5 & 0 \\
                    0 & 1 & -2 & 0 \\
                    0 & 1 & -14 & 0 \\
                \end{sysmatrix} \\
                &\!\begin{aligned}
                    &\ro{R_1 \rightarrow R_1} \\
                    &\ro{R_2 \rightarrow R_2} \\
                    &\ro{R_3 \rightarrow R_2 - R_3}
                \end{aligned}
                \begin{sysmatrix}{ccc|c}
                    1 & 0 & -5 & 0 \\
                    0 & 1 & -2 & 0 \\
                    0 & 0 & 12 & 0 \\
                \end{sysmatrix}
            \end{align*}
            The system provides a trivial solution: $\vec{x} = \begin{bmatrix} 0 & 0 & 0 \end{bmatrix}^T$. \\
            Therefore, the nullspace is trivial:
            \[
                N \Bigg(
                    \begin{bmatrix}
                        1 & 0 & -5 \\
                        0 & -2 & 4 \\
                        -3 & 1 & 1 \\
                    \end{bmatrix}
                \Bigg)
                =
                \Bigg\{
                    \begin{bmatrix}
                        0 \\ 0 \\ 0
                    \end{bmatrix}
                \Bigg\}
            \]
        }
        \item[f)] {
            Let us construct a system:
            \begin{align*}
                \begin{sysmatrix}{ccc|c}
                    2 & 1 & -3 & 0 \\
                    1 & -2 & -4 & 0 \\
                    -2 & 2 & 6 & 0
                \end{sysmatrix}
                &\!\begin{aligned}
                    &\ro{R_1 \rightarrow R_1} \\
                    &\ro{R_2 \rightarrow R_1 - 2R_2} \\
                    &\ro{R_3 \rightarrow R_1 + R_3}
                \end{aligned}
                \begin{sysmatrix}{ccc|c}
                    2 & 1 & -3 & 0 \\
                    0 & 5 & 5 & 0 \\
                    0 & 3 & 3 & 0
                \end{sysmatrix} \\
                &\!\begin{aligned}
                    &\ro{R_1 \rightarrow R_1} \\
                    &\ro{R_2 \rightarrow \frac{1}{5} R_2} \\
                    &\ro{R_3 \rightarrow R_2 - \frac{5}{3} R_3}
                \end{aligned}
                \begin{sysmatrix}{ccc|c}
                    2 & 1 & -3 & 0 \\
                    0 & 1 & 1 & 0 \\
                    0 & 0 & 0 & 0
                \end{sysmatrix}
            \end{align*}
            At this point, for a solution $\begin{bmatrix} x_1 & x_2 & x_3 \end{bmatrix}^T$, we can configure the relationship that:
            \[
                \begin{cases}
                    2x_1 = -x_2  + 3x_3 \\
                    x_2 = -x_3 \\
                    \rightarrow x_1 = 2x_3
                \end{cases}
            \]
            Therefore:
            \[
                N \Bigg(
                    \begin{bmatrix}
                        2 & 1 & -3 \\
                        1 & -2 & -4 \\
                        -2 & 2 & 6
                    \end{bmatrix}
                \Bigg)
                =
                \Bigg\{
                    \alpha
                    \begin{bmatrix}
                        2 \\ -1 \\ 1
                    \end{bmatrix}
                    \ :\ \alpha \in \R
                \Bigg\}
            \]
        }
    \end{enumerate}
}
