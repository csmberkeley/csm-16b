% Author: Dun-Ming Huang
% Email: dunmingbrandonhuang@berkeley.edu
% CSM16A Spring 2024

\qns{Cheby Cheby Chapa Chapa}

\textbf{Learning Goals:}
\begin{bindenum}
    \item Learn how the Chebyshev's Inequality was derived from observation, and its interpretation.
    \item Apply Chebyshev Inequality on realistic use-cases.
    \item See Chebyshev Inequality's other origin from another famous inequality (Markov's Inequality).
\end{bindenum}

Chapa did not study anything before the midterm, but somehow scored $90\%$.
You find this to be suspicious, and decides to use what you have learned in EECS 16A to expose Chapa once and for all\dots

From Edstem page of your course, you found the following details:
\begin{bindenum}
    \item The average of this exam was $66\%$.
    \item The standard deviation of this exam was $8\%$.
    \item In the end, $500$ students took the exam.
\end{bindenum}

\begin{enumerate}
    \item {
        Suppose that out of $n=500$ students, only $k=250$ students received a score higher than $a\%$. \\
        Let all students' score be represented as a variable $x_i$, and the vector of all students' score to be $\vec{x} = \begin{bmatrix} x_1 & \cdots & x_{500} \end{bmatrix}$.
        What is the upper bound of $\frac{n}{k}$ in terms of $a$ and $x_i$?
    }
    \meta {
        This question corresponds to the RMS form of Chebyshev's Inequality definition.
    }
    \ans {
        Let the norm of $\vec{x}$ be represented as:
        \begin{align*}
            \| \vec{x} \|^2 &= \sum_i x_i^2 \\
            &= x_1^2 + x_2^2 + \cdots + x_{500}^2
        \end{align*}
        We notice that there will be $250$ elements within the $500$ that will be higher than $a\%$.
        In such manner, we derive that:
        \begin{align*}
            \| \vec{x} \|^2 
            &= x_1^2 + x_2^2 + \cdots + x_{500}^2 \geq k a^2 \\
            \frac{\| \vec{x} \|^2}{a^2} &\geq k \\
            \frac{\| \vec{x} \|^2 / n}{a^2} &\geq \frac{k}{n}
        \end{align*}
        To simplify this expression further, we can use the definition of $rms$: $rms(\vec{x}) = \sqrt{\frac{1}{n} \sum_{i=1}^n x_i^2}$, such that:
        \[
            {\left(\frac{{\rm rms}(\vec{x})}{a}\right)}^2 \geq \frac{k}{n}
        \]

    }

    \item {
        Now, instead, let us define a separate variable $\vec{y}$ such that $\vec{y} = \vec{x} - \bar{\vec{x}}$. \\
        Suppose that only $k$ out of $n$ elements $y_i$ are larger than some value $c$, use the conclusion of part (a) to derive an upper bound of $\frac{k}{n}$ using the variance of $x$.
    }
    \meta {
        This question corresponds to the variance form of Chebyshev's Inequality definition.
    }
    \ans {
        From the previous part, we were able to conclude that for some variable $X$, the phenomenon that $k$ of $n$ elements of $X$ are larger than some value $b$ may be expressed within the inequality:
        \[
            \frac{\| X \|^2 / n}{b^2} \geq \frac{k}{n}
        \]
        Now, substituting $X$ with $\vec{y} = \vec{x} - \bar{\vec{x}}$, and using the prompt's provided values, we instead arrive at:
        \[
            \frac{\| \vec{x} - \bar{\vec{x}} \|^2 / n}{c^2} \geq \frac{k}{n}
        \]
        However, note that $var(\vec{x}) = \frac{\| \vec{x} - \bar{\vec{x}} \|^2}{n}$. Therefore, in turn, we find that:
        \[
            \frac{var(x)}{c^2} \geq \frac{k}{n}
        \]
    }
\end{enumerate}
From the two previous subparts, we arrive at an interesting inequality called Chebyshev's Inequality, which basically states that:
\begin{ln-define}{Chebyshev's Inequality}{}
    For a random variable $X$ with finite expectation $\mathbb{E}[X] = \bar{X} = \mu$, for any positive constant $c$:
    \[
        \mathbb{P}[|X - \mu| \geq c] \leq \frac{Var(X)}{c^2}
    \]
    \tcblower
    Specifically, we obtained this conclusion via subpart (b). \\
    In subpart (b), assuming all variables were obtained at uniform (equal) probability, then the probability at which our variable value $y_i = x_i - \bar{\vec{x}}$ is equal to the proportion of values $y_i$ that are larger than $c$, which is $k/n$.
\end{ln-define}
In other words, now we can find the probability at which one person's exam score subtracted by the average exam score is larger than a specific amount.

\begin{enumerate}[resume]
    \item {
        What is the probability that a randomly chosen student from the course obtained a $90\%$ on the exam? \\
        Remember:
        \begin{bindenum}
            \item The average of this exam was $66\%$.
            \item The standard deviation of this exam was $8\%$.
            \item In the end, $500$ students took the exam.
        \end{bindenum}
    }
    
    \ans {
        Using Chebyshev's Inequality,
        \begin{align*}
            \mathbb{P}[|X - \mu| \geq c] &\leq \frac{Var(X)}{c^2} \\
            \mathbb{P}[X - 0.66 \geq 0.24] &\leq \mathbb{P}[|X - 0.66| \geq 0.24]\\
            &\leq \frac{0.08}{{0.24}^2} = \frac{1}{9}
        \end{align*}
    }

    \item {
        Since Chapa did not study at all, let us assume that Chapa guessed all the answers on the exam. \\
        You did some simulation and found that, across $100000$ simulations, randomly guessing every answer on the exam gives the following distribution for exam scores:
        \begin{bindenum}
            \item The average of this simulation was $25\%$.
            \item The standard deviation of this simulation was $5\%$.
        \end{bindenum}
        Judging by this, argue whether Chapa did something other than randomly guessing on the exam. Justify your reasoning using Chebyshev's Inequality.
    }
    
    \ans {
        Based on Chebyshev's Inequality, if Chapa really randomly guessed every answer on the exam, then the probability at which he arrives at his grade, $90\%$, would be:
        \begin{align*}
            \mathbb{P}[|X - \mu| \geq c] &\leq \frac{Var(X)}{c^2} \\
            \mathbb{P}[X - 0.25 \geq 0.65] &\leq \mathbb{P}[|X - 0.25| \geq 0.65]\\
            &\leq \frac{0.05}{{0.65}^2} = \frac{1}{169}
        \end{align*}
        There is an extremely low probability (less than $0.01$) that Chapa actually obtains his score if he only randomly guessed answers on the exam.
        Therefore, we can suspect that Chapa did something other than randomly guessing on the exam.
    }
\end{enumerate}