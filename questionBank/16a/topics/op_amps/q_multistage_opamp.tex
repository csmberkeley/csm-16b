% Author: Emily Gosti, Yannan Tuo, Sukrit Arora
% Email: egosti@berkeley.edu, ytuo@berkeley.edu, sukrit.arora@berkeley.edu

\qns{Multi-stage Amplifier}

\textbf{Learning Goal:} The objective of this problem is to understand how multiple stages of op-amp circuits can be used to achieve a specific circuit gain.

\textbf{Relevant Notes:} \notes{Note 19 Section 19.5} goes over inverting and non-inverting amplifiers.

\meta{
Highlight that it is not necessary to have a buffer between two inverting amplifiers. Although resistors $R_3$ and $R_4$ are drawing current from the first amplifier output, the first output $V_{o,1}$ is not going to change. This is because the output of an ideal op-amp can supply any current to the load without changing the output voltage. So in this case there will be no loading effect.

Mention that loading effect will NOT occur if the second amplifier is not drawing any input current OR the first amplifier output voltage does not change.
}


\begin{enumerate}

\item{What is the range of values that we can scale $V_{in}$ by when using a non-inverting op amp? (What are possible values for the gain?)
}

\ans{
Recall that when using a non-inverting op amp, the equation for $V_{out}$  is given by
$$V_{out}=V_{in}(1 + \frac{R_2}{R_1})$$ 
where $R_2$ is the upper resistor and $R_1$ is the lower one (connected to ground).
The circuit gain $G$ is represented by 
$$G=\frac{V_{out}}{V_{in}}=1 + \frac{R_2}{R_1}$$
We can choose any values for resistance from [0, $\infty$); then, the minimum gain would be 1 if we chose $R_2 = 0$; the maximum gain approaches infinity as we choose some very large $R_2$ and/or a very small $R_1$. Hence, the range of gains is from 1 to infinity, and our range of values for $V_{out}$ is [$V_{in}$, $\infty$).
}

\item{
What is the range of values that we can scale $V_{in}$ by when using an inverting op amp? (What are the possible values for the gain?)
}

\ans{
Recall that when using an inverting op amp, the equation for $V_{out}$ is  given by 
$$V_{out}=-V_{in}\frac{R_2}{R_1}$$. 
The circuit gain is represented by 
$$G=-\frac{R_2}{R_1}$$. 

Again, we can choose any values for resistance from [0, $\infty$); then, the minimum absolute value scaling would be 0 if we chose $R_2 = 0$; the maximum absolute value scaling approaches infinity as we choose some very large $R_2$ and/or a very small $R_1$. Hence, the range of gains is from 0 to -$\infty$, and our range of values for $V_{out}$ is also ($-\infty$, 0].

}

\item{
Can you design a single inverting/ non-inverting amplifier with circuit gain $G=0.5$? If not, what range of gain values is not reachable using a single inverting op amp or a single non-inverting op amp? 
}

\ans{
From part (a), we found that a non-inverting op amp can only reach gain values from [1, $\infty$). From part (b), we found that an inverting op amp can only reach gain values from (-$\infty$, 0]. So using only a single op-amp, we cannot reach values between (0, 1), specifically $G = 0.5$.
}

\item{
How would you construct a circuit using inverting/ non-inverting amplifiers so that the overall circuit gain is $G=0.5$?
}


\ans{
We can use two inverting op-amps to achieve overall gain $G = 0.5$. Say the first op-amp has gain $G_1$ and the second op-amp $G_2$, so $G = G_1G_2$. From part (b), these gains can take on values from (-$\infty$, 0], so multiplying them together helps reach $G = 0.5$.

 \begin{center}
    \begin{circuitikz} 
    \draw (0,0) 
    node[op amp] (opamp) {}
    (opamp.-) node[left] {}
    to[R, l = $R_1$, *-o]
    ++(-2,0)
    to[short, l = $V_{in}$] ++(0,0)
    (opamp.+) node[left] {} 
    to[short] node[ground] {} ++(0,-1)
    (opamp.out) node[right] {}
    to[short] ++(0, 1.5)
    to[R, l = $R_2$, i = $i$] ++(-2.4, 0)
    to[short] ++(0,-1)
    (opamp.out) node[right, *-o] {}
    to[short, *-o] ++(0.5, 0)
    to node[above] {$V_{o,1}$} ++(0,0)
    to[short] ++(1, 0);
    
    
    \draw (6,-0.5) 
    node[op amp] (opamp) {}
    (opamp.-) node[left] {}
    to[R, l = $R_3$, *-o]    ++(-2,0)
    to node[above] {$V_{in,2}$} ++(0,0)
    (opamp.+) node[left] {} 
    to[short] node[ground] {} ++(0,-1)
    (opamp.out) node[right] {}
    to[short] ++(0, 1.5)
    to[R, l = $R_4$, i = $i$] ++(-2.4, 0)
    to[short] ++(0,-1)
    (opamp.out) node[right, *-o] {}
    to[short, l = $V_{o}$, *-o] ++(0.5, 0);
    
    \end{circuitikz}
    \end{center}
    
    For the first inverting op-amp, we have $G_1 = -\frac{R_2}{R_1}$, and for the second, $G_2 = -\frac{R_4}{R_3}$. Plugging these into the equation for $G$:
    $$G = G_1G_2 = \frac{R_2R_4}{R_1R_3} = 0.5$$
    $$R_1R_3 = 2R_2R_4$$
    To reach out desired gain, we can pick any combination of $R_1$, $R_2$, $R_3$, and $R_4$ that satisfies this equation. One such solution is $R_1 = 200\si{\ohm}$ and $R_2 = R_3 = R_4 = 100\si{\ohm}$.
}


\end{enumerate}
