% Author: Jessica Lin
% Email: jessica.jx.lin@berkeley.edu
% CSM16A Fall 2022

% node[label={[font=\footnotesize]above:$u_1$}] {}

\qns{Introduction to Op-Amp Analysis}

\textbf{Learning Goal:} The goal of this question is to build an understanding of negative feedback and op amps.


\begin{enumerate}
\item 

\begin{center}
  \begin{circuitikz} [american voltages]
    \draw (0,0) node[op amp,yscale=-1] (opamp) { }
      (opamp.+) -- (-5.5, 0.5)
      (opamp.-) -- (-1.2, -2)
      (opamp.out) -- (2, 0) to [R, l=$R_1$] (2, -2) to (-1.2, -2);

    % Current source
    \draw (opamp.-) to (-2, -0.5);
    \draw (-4, -0.5) to (-2, -0.5) 
        to [R = $R_2$] (-2, -3.5)
        to (-4, -3.5)
        to [I, l = $I_s$] (-4, -0.5);
    \draw (-3, -3.5) to (-3, -4) node[ground] {};
    
    % Voltage source
    \draw (-5.5, -1.5) to [V=$V_{in}$,invert] (-5.5, 0.5);
    \draw (-5.5, -1.4) -- (-5.5, -1.5) node[ground] { };

    % Draw v_out
    \draw (2, 0) to [short, -o] (4, 0);
    \draw (4, -1) to [short, o-] (4, -1.2) node[ground] { };
    \node[draw=none] at (4.3, 0) {$+$};
    \node[draw=none] at (4.3, -1) {$-$};
    \node[draw=none] at (4.2, -0.5) {$V_{out}$};

    % label nodes
    \node[draw=none] at (-1.2, 0.8) {$u_+$};
    \node[draw=none] at (-1.2, -0.2) {$u_-$};
    
  \end{circuitikz}
\end{center}

\item Is this circuit in negative feedback?

\sol{

Yes, this circuit is in negative feedback. First, we zero out all the independent sources: the voltage source labeled $V_{in}$ is shorted and connected to ground, and the current source labeled $I_s$ becomes an open circuit. We can see that the path from $V_{out}$ through $R_1$ to $R_2$ to ground is effectively a voltage divider. We can wiggle the output and see how the behavior of the circuit changes. If we increase $V_{out}$, then the voltage at the negative terminal of the op-amp, $u_-$, increases. Since $V_{out} = A(u_+ - u_-)$, an increase in $u_-$ mesol that $V_{out}$ will decrease. As an increase in $V_{out}$ leads to the circuit adjusting to decrease $V_{out}$, this circuit is in negative feedback.

Below is a circuit diagram where the arrows indicate the wiggling and the response:

\begin{center}
\begin{circuitikz} [american voltages]
    \draw (0,0) node[op amp,yscale=-1] (opamp) { }
      (opamp.+) -- (-5.5, 0.5)
      (opamp.-) -- (-1.2, -2)
      (opamp.out) -- (2, 0) to [R, l=$R_1$] (2, -2) to (-1.2, -2);

    % Current source
    \draw (opamp.-) to (-2, -0.5);
    \draw (-4, -0.5) to (-2, -0.5) 
        to [R = $R_2$] (-2, -3.5)
        to (-4, -3.5)
        to [short, -o] (-4, -2.5) 
        (-4, -1.5) to [short, o-] (-4, -0.5);
    \draw (-3, -3.5) to (-3, -4) node[ground] {};
    
    % Voltage source
    \draw (-5.5, -1.5) to (-5.5, 0.5);
    \draw (-5.5, -1.4) -- (-5.5, -1.5) node[ground] { };

    % Draw v_out
    \draw (2, 0) to [short, -o] (4, 0);
    \draw (4, -1) to [short, o-] (4, -1.2) node[ground] { };
    \node[draw=none] at (4.3, 0) {$+$};
    \node[draw=none] at (4.3, -1) {$-$};
    \node[draw=none] at (4.2, -0.5) {$V_{out}$};

    % Draw the arrows
    \draw[->] [color = red] (1.3, -0.3) to (1.3, 0.3);
    \draw[->] [color = blue] (-1.6, -0.8) to (-1.6, -0.2);
    \draw[->] [color = blue] (1, 0.3) to (1, -0.3);
    
    % label nodes
    \node[draw=none] at (-1.2, 0.8) {$u_+$};
    \node[draw=none] at (-1.2, -0.2) {$u_-$};
  \end{circuitikz}
\end{center}
}

\item Solve for $V_{out}$ as a function of $V_{in}$, $I_s$, $R_1$, and $R_2$. 

\sol{

Let's first label some currents:

\begin{center}
  \begin{circuitikz} [american voltages]
    \draw (0,0) node[op amp,yscale=-1] (opamp) { }
      (opamp.+) -- (-6.5, 0.5)
      (opamp.-) -- (-2, -0.5) -- (-2, -2)
      (opamp.out) -- (2, 0) to [R, l=$R_1$, i<=$i_1$] (2, -2) to (-2, -2);

    % Current source
    \draw (opamp.-) to (-3, -0.5);
    \draw (-5, -0.5) to (-3, -0.5) 
        to [R = $R_2$, i = $i_2$] (-3, -3.5)
        to (-5, -3.5)
        to [I, l = $I_s$, i = $I_s$] (-5, -0.5);
    \draw (-4, -3.5) to (-4, -4) node[ground] {};
    
    % Voltage source
    \draw (-6.5, -1.5) to [V=$V_{in}$,invert] (-6.5, 0.5);
    \draw (-6.5, -1.4) to (-6.5, -1.5) node[ground] { };

    % Draw v_out
    \draw (2, 0) to [short, -o] (4, 0);
    \draw (4, -1) to [short, o-] (4, -1.2) node[ground] { };
    \node[draw=none] at (4.3, 0) {$+$};
    \node[draw=none] at (4.3, -1) {$-$};
    \node[draw=none] at (4.2, -0.5) {$V_{out}$};

    % draw currents
    \draw (-2, 0.5) to [short, i_=$i_+$] (opamp.+);
    \draw (-2, -0.5) to [short, i_=$i_-$] (opamp.-);
    
    % label nodes
    \node[draw=none] at (-1.2, 0.8) {$u_+$};
    \node[draw=none] at (-1.2, -0.2) {$u_-$};
    
  \end{circuitikz}
\end{center}

Since we know the op-amp is in negative feedback, we have the following golden rules:
\begin{align*}
    u_+ & = u_- \\
    i_+ & = i_- = 0 \\
\end{align*}
First, we write the KCL equation at the $u_-$ node: 
\[
    I_s = i_2 + (i_- + i_1) = i_1 + i_2 \text{ since $i_- = 0$} \\
\]
Then, we write Ohm's Laws equations for each resistor, noting that $u_- = u_+ = V_{in}$:
\begin{align*}
    i_1 & = \frac{u_- - V_{out}}{R_1} = \frac{V_{in} - V_{out}}{R_1} \\
    i_2 & = \frac{u_-}{R_2} = \frac{V_{in}}{R_2}
\end{align*}
Now, we can substitute these currents into our KCL equation:
\begin{align*}
    I_s & = i_1 + i_2 \\
    I_s & = \frac{V_{in} - V_{out}}{R_1} + \frac{V_{in}}{R_2} \\
    I_sR_1 & = V_{in} - V_{out} + \frac{V_{in}R_1}{R_2} \\
    V_{out} & = V_{in}(1 + \frac{R_1}{R_2}) - I_sR_1
\end{align*}
}

\end{enumerate}