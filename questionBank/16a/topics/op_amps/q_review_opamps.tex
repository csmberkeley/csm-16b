% Lydia Lee, Spring 2019, lydia.lee@berkeley.edu
% Urmita Sikder, Fall 2020, urmita@berkeley.edu
\qns{Operational Amplifiers}

\textbf{Learning Goal:} This problem will help to understand op-amps in negative feedback and the operation of an inverting amplifier.

\textbf{Relevant Notes:} \notes{Note 18} goes over op-amp properties and derivation of circuit responses.

\meta{
\begin{itemize}
\item Walk through the steps to check whether a circuit will have negative feedback.
\item Explain why the 2nd golden rule only applies to negative feedback. 
\item Make sure that students understand that we solve these problems like other circuits, but with the addition of the golden rules. Althought it isn't covered in this worksheet, we can use op amp topologies to solve for most op amp problems. 
\item Ask students if they are comfortable with applying to basic patterns of noninverting / inverting op amp configurations in problems.
\end{itemize}


}


Operational amplifiers (op amps for short) are typically drawn like the figure on the left, and their internal workings can be represented by the figure on the right.

\begin{minipage}{0.45\linewidth}
\begin{center}
  \begin{circuitikz}
    \draw (0,0) node[op amp,yscale=-1] (opamp) { }
      (-2.5, 0.5) to [short, i=$i_{+}$, o-] (opamp.+) 
      (-2.5, -0.5) to [short, i=$i_{-}$, o-] (opamp.-) 
      (opamp.out) node[right ] {$V_{out}$} 
      ;
    \node[draw=none,text=black] at (-2.8, 0.5) {$V_{+}$};
    \node[draw=none,text=black] at (-2.8, -0.5) {$V_{-}$};

    \draw (0, 0.5) to [short, -o] (0, 1.0);
    \node[draw=none,text=black] at (0, 1.5) {$V_{DD}$};
    \draw (0, -0.5) to [short, -o] (0, -1.0);
    \node[draw=none,text=black] at (0, -1.5) {$V_{SS}$};
  \end{circuitikz}
\end{center}
\end{minipage}
\begin{minipage}{.45\linewidth}
  \begin{center}
    % \textcolor{red}{This diagram may change slightly from semester to semester. We should stay consistent with the course notes to not confuse students.}
    \begin{circuitikz}
      % Nodes on the left
      \node[draw=none,text=black] at (-1, 0.8) {$V_{+}$};
      \node[draw=none,text=black] at (-1, -0.8) {$V_{-}$};
      \node[draw=none,text=black] at (-.8, 0.3) {$+$};
      \node[draw=none,text=black] at (-.8, -0.3) {$-$};
      \draw (-1.5, 0.5) to [short, -o] (-1, 0.5); 
      \draw (-1.5, -0.5) to [short, -o] (-1, -0.5); 
      
      % ground and variable voltage
      \draw (0.5,-1.5) to [V=$V_{SS}$] ++(0,-2)
		node[ground] () {};
%      \draw (0.5,-1.5) -- (0.5,-1.8) to [short, -o](0.5, -2);
%      \node[draw=none,text=black] at (1, -2) {$V_{SS}$};
      \draw (0.5, -1.5) -- (1, -1) -- (0.5, -0.5) -- (0, -1) -- cycle;
      \node[draw=none,text=black] at (0.5, -0.8) {$+$};
      \node[draw=none,text=black] at (0.5, -1.2) {$-$};
      \node[draw=none,text=black] at (1.8, -1) {$\frac{V_{DD}-V_{SS}}{2}$};
      \draw (0.5, -0.5) -- (0.5, 0) { };
      \draw (0.5, 0) -- (1, 0.5) -- (0.5, 1) -- (0, .5) -- cycle;
      \node[draw=none,text=black] at (0.5, 0.7) {$+$};
      \node[draw=none,text=black] at (0.5, 0.3) {$-$};
      \node[draw=none,text=black] at (2, 0.5) {$A(V_{+} - V_{-})$};
      
      % Upwards
      \draw (0.5, 1) -- (0.5, 1.5) to [short, -o] (3.5, 1.5);
      \draw (3.5,-1.5) to [short, o-] (3.5,-1.8) node[ground]{ };
      \node[draw=none,text=black] at (5.5, -0.3) {$V_{out}=A(V_+ - V_-) + \frac{V_{DD} +V_{SS} }{2}$};
      \node[draw=none,text=black] at (3.9, 1) {$+$};
      \node[draw=none,text=black] at (3.9, -1.6) {$-$};
    \end{circuitikz}
  \end{center} 
\end{minipage}

Here, $V_+$ and $V_-$ are input voltages, $V_{DD}$ and $V_{SS}$ are what we call the ``supply rails'', and $V_{out}$ is the output voltage. From the diagram and knowing that $V_{out}$ cannot exceed the supply rail voltages, we have a relationship between the outputs and the inputs:
\begin{equation*}
V_{out} =
\begin{cases} 
V_{SS} &  \textrm{,  if}
\hspace{1.4cm}A(V_+ - V_-) + \frac{V_{DD} +V_{SS} }{2} < V_{SS} \\
A(V_+ - V_-) + \frac{V_{DD} +V_{SS} }{2} &  \textrm{,  if}
\hspace{0.4cm}V_{SS}\leq A(V_+ - V_-) + \frac{V_{DD} +V_{SS} }{2} \leq V_{DD} \\
V_{DD} &  \textrm{,  if}
\hspace{0.4cm}V_{DD}<A(V_+ - V_-) + \frac{V_{DD} +V_{SS} }{2} \\
\end{cases}
\end{equation*}

\begin{center}
\begin{tikzpicture}
  \begin{axis}[
    xmin=-120, xmax=120,
    ymin=-120, ymax=120,
    axis lines=center,
    axis on top=true,
    grid style={dashed},
    xlabel={\emph{$u_{+} - u_{-}$}},
    ylabel={\emph{$V_{out}$}},
    yticklabels={,,},
    xticklabels={,,},
  ]
  \addplot [draw=blue,thick] coordinates {(5, 70) (100, 70)};
  \addplot [draw=blue,thick] coordinates {(-5, -20) (-100, -20)};
  \addplot [draw=blue,thick] coordinates {(-5, -20) (5, 70)};
  \addplot [
      mark=*, blue, solid
    ]
    coordinates {
            (-5, -20)
            (5, 70)};

    \node[anchor=north west] at (axis cs:-5,-20) {$(0,V_{SS})$};
    \node[anchor=south east] at (axis cs:5,70) {$(0,V_{DD})$};
  \end{axis}
\end{tikzpicture}
\end{center}

Typically gain $A$ is quite large, meaning the the sloped region in the center is quite narrow.
\begin{enumerate}
\item{
Much of EE16A will have our analysis be restricted to ideal op-amps. However, it is important to know non-ideal behaviors. What are some main differences between ideal and non-ideal op-amps?
}

\ans{
Ideal op-amps: Infinite gain $A = \frac{V_o}{V_i}$ (where $V_i = V_+ - V_-$). This means that the gain will be infinity if there was no limit on the rails, but it cannot reach infinity because it gets clipped at the positive or negative rails. infinite input resistance, zero input current. Non-ideal op amps: finite gain, finite input resistance, finite input current.
}
%\item
%The classic formulation of the negative feedback problem looks like so:
%\begin{figure}[H]
%	\centering
%	\tikzstyle{block} = [draw, fill=white, rectangle, 
%      minimum height=3em, minimum width=3em]
%	\tikzstyle{sum} = [draw, fill=white, circle, node distance=1cm]
%	\tikzstyle{input} = [coordinate]
%	\tikzstyle{output} = [coordinate]
%	\tikzstyle{midoutput} = [coordinate]
%	\tikzstyle{pinstyle} = [pin edge={to-,thin,black}]
%	\begin{tikzpicture}[auto, node distance=2cm,>=latex']
%    % We start by placing the blocks
%    \node [input, label=$s_\text{in}$] (input) {$s_\text{in}$};
%    \node [sum, right of=input] (sum) { };
%    \node [block, right of=sum] (gain) {$A$};
%    % \node [midoutput, right of=gain] (midoutput) {};
%    \node [output, right of=gain, label=$s_\text{out}$] (output) {$s_\text{out}$};
%    % \draw [->] (gain) -- node[name=u] (output) {output} ;
%    \node [block, below of=gain] (F) {$f$};
%      
%   % Once the nodes are placed, connecting them is easy. 
%    \draw [draw,->] (input) -- node[pos=0.85] {$+$} node [near end] {} (sum);
%   \draw [->] (sum) -- node {$s_\text{error}$} (gain);
%   \draw [->] (gain) -- node [name=y] {} (output);
%   \draw [->] (y) |- (F);
%   \draw [->] (F) -| node[pos=0.99] {$-$} node [near end] {$s_\text{fb}$} (sum);
%  \end{tikzpicture}
%	\label{fig:nfb_block_diagram}
%	\caption{Classic negative feedback block diagram}
%\end{figure}
%For the block diagram above, find the output in terms of the input.

%\ans{
%For the block diagram above, the output is related to the input by 
%\begin{align*}
%	s_\text{out} &= \frac{A}{1+Af}s_\text{in}\\
%		&\approx \frac{1}{f} \text{ when }A\to\infty
%\end{align*}
%
%}
\item
If the gain of an operational amplifier/op amp is given by $A\to\infty$, so we can make some assumptions known as the ``Golden Rules''. What are the ``Golden Rules'' and when are they applicable?

\ans{
\begin{itemize}
	\item Always true: $i_- = i_+ = 0\si{\ampere}$
	\item Negative feedback only: $u_+ = u_-$
\end{itemize}
}

\item

Determine if the following system is in negative feedback.
	\begin{center}
		\begin{circuitikz}
	\draw
	(0,0) node[op amp] (AMP) {}
	(AMP.-) to[short] ++(0,1) coordinate (topLeft)
		to[R,l=$R_f$] (topLeft -| AMP.out)
		to[short] (AMP.out)
		to[short,-o] ++(1,0)
		to[open,o-o,v^=$v_\text{out}$] ++(0,-2)
		node[ground] () {}
	(AMP.-) to[R,l_=$R_s$] ++(-2,0)
		to[sV,v_=$v_\text{in}$] ++(0,-2)
		node[ground] () {}
	(AMP.+) to[short] ++(0,-1)
		node[ground] () {};
\end{circuitikz}
	\end{center}
\ans{
	This is in negative feedback, and in fact is a fairly well-known schematic called the inverting amplifier.
	\begin{itemize}
		\item For our initial stimulus, we'll kick $v_\text{out}$ up
		\item How does $u_-$ respond? Well, we can't assume $u_+ = u_-$ (because that requires the system to be in negative feedback), so we'll look at the resistors. Note that this is a voltage divider with $u_-$ in the and, $v_\text{out}$ at the top, so when $v_\text{out}$ moves upward, $u_-$ travels upward too
		\item If $u_-$ goes up, that means $v_\text{out}$ goes down because of the op amp---the opposite direction as the inital stimulus!
	\end{itemize}
}

\item

Determine if the following system is in negative feedback.
	\begin{center}
		\begin{circuitikz}
	\draw
	(0,0) node[op amp] (AMP) {}
	(0,2) node[buffer,xscale=-1] (FB) {}
	(AMP.out) to[short] (AMP.out |- FB.in)
		to[short] (FB.in)
	(FB.out) to[short] (FB.out -| AMP.-)
		to[short] (AMP.-)
		to[R,l=$R$] ++(-2,0)
		to[sV,v_=$v_\text{in}$] ++(0,-2)
		node[ground] () {}
	(AMP.+) to[short] ++(-.5,0)
		node[ground] () {}
	(AMP.out) to[short] ++(1,0)
		to[open,o-o,v^=$v_\text{out}$] ++(0,-1)
		node[ground] () {}
	(FB) node[] () {-5};
\end{circuitikz}
	\end{center}
	\textbf{Note: } The triangular block with the label "-5" in the figure above represents an amplifier with a factor of -5; In other words, it takes $V_{\text{out}}$ as an input from the right side and outputs $-5V_{\text{out}}$ on the left side.
\ans{
	This is \textit{not} in negative feedback! Going through the process of checking:
	\begin{itemize}
		\item For our initial stimulus, we'll kick $v_\text{out}$ up
		\item With the $-5 \cdot v_{out}$, that means $u_-$ goes down
		\item If $u_-$ goes down, that means $v_\text{out}$ goes up---the same direction as the initial stimulus!
	\end{itemize}
}

\item

Find the expression of $v_{out}$ for the following circuit:
	\begin{center}
		\begin{circuitikz}
	\draw
	(0,0) node[op amp] (AMP) {}
	(AMP.-) to[short] ++(0,1) coordinate (topLeft)
		to[R,l=$R_f$] (topLeft -| AMP.out)
		to[short] (AMP.out)
		to[short,-o] ++(1,0)
		to[open,o-o,v^=$v_\text{out}$] ++(0,-2)
		node[ground] () {}
	(AMP.-) to[R,l_=$R_s$] ++(-2,0)
		to[sV,v_=$v_\text{in}$] ++(0,-2)
		node[ground] () {}
	(AMP.+) to[short] ++(0,-1)
		node[ground] () {};
		    \draw (0, 0.5) to [short, -o] (0, 1.0);
    \node[draw=none,text=black] at (0.5, 1) {$V_{DD}$};
    \draw (0, -0.5) to [short, -o] (0, -1.0);
    \node[draw=none,text=black] at (0, -1.5) {$V_{SS}$};
\end{circuitikz}
	\end{center}
	
\ans{
	\begin{center}
		\begin{circuitikz}
	\draw
	(0,0) node[op amp] (AMP) {}
	(AMP.-) to[short] ++(0,1) coordinate (topLeft)
		to[R,l=$R_f$, i=$I_f$] (topLeft -| AMP.out)
		to[short] (AMP.out)
		to[short,-o] ++(1,0)
		to[open,o-o,v^=$v_\text{out}$] ++(0,-2)
		node[ground] () {}
	(AMP.-) to[R,l_=$R_s$, i<=$I_{in}$] ++(-3,0)
		to[sV,v_=$v_\text{in}$] ++(0,-2)
		node[ground] () {}
	(AMP.+) to[short] ++(0,-1)
		node[ground] () {};
		\draw (AMP.-) node[below] {$u_{-}$} to[short,*-, i=$I_{-}$] ++(0.4, 0);
		\draw (AMP.+) node[xshift=-10] {$u_{+}$} to[short, i=$I_{+}$] ++(0.3, 0);
		    \draw (0, 0.5) to [short, -o] (0, 1.0);
    \node[draw=none,text=black] at (0.5, 1) {$V_{DD}$};
    \draw (0, -0.5) to [short, -o] (0, -1.0);
    \node[draw=none,text=black] at (0, -1.5) {$V_{SS}$};
\end{circuitikz}
	\end{center}
	
The first golden rule for op amps tell us:
\begin{center}
$I_+ = I_- = 0$
\end{center}
Since we verified that this circuit is in negative feedback in part (c), the second golden rule also applies:
\begin{center}
$u_+ = u_-$
\end{center}

Using KCL, we can write the following equations:
\begin{center}
$I_{in}$ = $I_f + I_-$
\end{center}
Since $I_- = 0$, we can write
\begin{center}
$I_{in}$ = $I_f$
\end{center}

Since the positive input terminal is connected to ground, $u_+=0$, i.e. $u_-=u_+=0$ .

Using the formula, V = IR: 
\begin{center}
$I_{in}$ = $\frac{v_{in} - u_-}{R_s}$ = $\frac{v_{in}}{R_s}$ 

$I_f$ = $\frac{u_- - v_{out}}{R_f}$ = $\frac{-v_{out}}{R_f}$ 
\end{center}

From the KCL equations, we know $I_{in}$ = $I_f$. Substituting the equations from above, 
\begin{center}
$\frac{v_{in}}{R_s}$ =$ \frac{-v_{out}}{R_f} $

$v_{out}$ = -$\frac{v_{in}R_f}{R_s}$
\end{center}


}

\itemPlot $v_{out}$ vs. time for the following $v_{in}$. Assume $V_{DD}=10\si{\volt}$, $V_{SS}=-10\si{\volt}$, $R_s=100\si{\ohm}$, $R_f=500\si{\ohm}$.
\begin{center}
	\begin{tikzpicture}
	\begin{axis}[
	xmin=-5, xmax=60,
	ymin=-1.5, ymax=1.5,
	axis lines=center,
	axis on top=true,
	grid style={dashed},
	xlabel={\emph{$t$}},
	ylabel={\color{blue}{\emph{$v_{in}(t)$}}},
	yticklabels={,,},
	xticklabels={,,},
	width=6in,
	height=3in,
	]
	\node [left,color=blue] at (axis cs:  0, 1) {$1V$}; 
	\node [left,color=blue] at (axis cs:  0, -1) {$-1V$}; 
		\node [below,color=blue] at (axis cs:  10, 0) {$1\si{\milli\second}$};
		\node [below,color=blue] at (axis cs:  20, 0) {$2\si{\milli\second}$};
	\node [below,color=blue] at (axis cs:  30, 0) {$3\si{\milli\second}$}; 
	\node [below,color=blue] at (axis cs:  40, 0) {$4\si{\milli\second}$};

	\addplot [draw=blue,thick] coordinates{ (0,-1) (10, 0) (20, 1)(30, 0)(40, -1)(50, 0)};
	\end{axis}
	\end{tikzpicture}
\end{center}

\meta{Make a distinction between op-amp gain $A$, which is ideally infinite and circuit gain, $G$, which is finite. }

\ans{

Using the formula for $v_{out}$ obtained in part (e) and the values for $R_f$ and $R_s$, we can find the circuit gain:
\begin{align*}
G=\frac{v_{out}}{v_{in}}= -\frac{R_f}{R_s}=-\frac{500}{100}=-5
\end{align*}
So the input will be multiplied by a factor of $-5$.
We can substitute to find $v_{out}$ along different times.

\begin{table}[ht]
\centering 
\begin{tabular}{c c c c} 
\hline\hline 
$v_{in}$ & Time & $v_{out}$ Calculation & $v_{out}$ \\ [0.5ex] 
\hline
-1 & 0 & $-\frac{(-1)(500)}{100}$  & 5  \\ [0.5ex] 
\hline
0 & 1 & $-\frac{(0)(500)}{100}$  & 0 \\[0.5ex] 
\hline
1 & 2 & $-\frac{(1)(500)}{100}$  & -5 \\[0.5ex] 
\hline
0 & 3 & $-\frac{(0)(500)}{100}$  & 0 \\[0.5ex] 
\hline
-1 & 4 & $-\frac{(-1)(500)}{100}$  & 5 \\ [0.5ex] 
\hline
0 & 5 & $-\frac{(0)(500)}{100}$  & 0 \\[0.5ex] 
\hline 
\end{tabular}
\label{table:nonlin} 
\end{table}

Using the $v_{out}$ values we calculated above in the table, we can plot the $v_{out}$ vs. time graph.

\begin{center}
	\begin{tikzpicture}
	\begin{axis}[
	xmin=-5, xmax=60,
	ymin=-1.5, ymax=1.5,
	axis lines=center,
	axis on top=true,
	grid style={dashed},
	xlabel={\emph{$t$}},
	ylabel={\color{red}{\emph{$v_{out}(t)$}}},
	yticklabels={,,},
	xticklabels={,,},
	width=6in,
	height=3in,
	]
	\node [left,color=red] at (axis cs:  0, 1) {$5V$}; 
	\node [left,color=red] at (axis cs:  0, -1) {$-5V$}; 
		\node [below,color=blue] at (axis cs:  10, 0) {$1\si{\milli\second}$};
		\node [below,color=blue] at (axis cs:  20, 0) {$2\si{\milli\second}$};
	\node [below,color=blue] at (axis cs:  30, 0) {$3\si{\milli\second}$}; 
	\node [below,color=blue] at (axis cs:  40, 0) {$4\si{\milli\second}$};
	
	\addplot [draw=red,thick] coordinates{ (0,1) (10, 0) (20, -1)(30, 0)(40, 1)(50, 0)};
	\end{axis}
	\end{tikzpicture}
\end{center}

}

\itemWhat happens if $R_f$ is changed to $R_f=2000\si{\ohm}$. Plot $v_{out}$ vs. time for the same $v_{in}$, where $V_{DD}=10\si{\volt}$, $V_{SS}=-10\si{\volt}$, and $R_s=100\si{\ohm}$. 

\ans{

Similar to part (f), using the formula for $v_{out}$ obtained in part (e) and the values for $R_f$ and $R_s$, we can find the circuit gain:
\begin{align*}
G=\frac{v_{out}}{v_{in}}= -\frac{R_f}{R_s}=-\frac{2000}{100}=-20
\end{align*}
So the input will be multiplied by a factor of $-20$.
We can substitute to find $v_{out}$ along different times.

\begin{table}[ht]
\centering 
\begin{tabular}{c c c c} 
\hline\hline 
$v_{in}$ & Time & $v_{out}$ Calculation & $v_{out}$ \\ [0.5ex] 
\hline
-1 & 0 & $-\frac{(-1)(2000)}{100}$  & 20  \\ [0.5ex] 
\hline
0 & 1 & $-\frac{(0)(2000)}{100}$  & 0 \\[0.5ex] 
\hline
1 & 2 & $-\frac{(1)(2000)}{100}$  & -20 \\[0.5ex] 
\hline
0 & 3 & $-\frac{(0)(2000)}{100}$  & 0 \\[0.5ex] 
\hline
-1 & 4 & $-\frac{(-1)(2000)}{100}$  & 20 \\ [0.5ex] 
\hline
0 & 5 & $-\frac{(0)(2000)}{100}$  & 0 \\[0.5ex] 
\hline 
\end{tabular}
\label{table:nonlin} 
\end{table}

Using the $v_{out}$ values we calculated above in the table, we can plot the $v_{out}$ vs. time graph. \newline \newline
Here, we have to also consider the values given for $V_{DD}$ and $V_{SS}$. The given bounds are $V_{DD}=10\si{\volt}$, $V_{SS}=-10\si{\volt}$, but the values we obtained for $V_{out}$ are between $-20\si{\volt}$ and $20\si{\volt}$. Therefore, we have to "clip" the graph at these points to not go past $-10\si{\volt}$ and $10\si{\volt}$, as shown in the graph below. 

\begin{center}
	\begin{tikzpicture}
	\begin{axis}[
	xmin=-5, xmax=60,
	ymin=-1.5, ymax=1.5,
	axis lines=center,
	axis on top=true,
	grid style={dashed},
	xlabel={\emph{$t$}},
	ylabel={\color{red}{\emph{$v_{out}(t)$}}},
	yticklabels={,,},
	xticklabels={,,},
	width=6in,
	height=3in,
	]
	\node [left,color=red] at (axis cs:  0, 1) {$20V$}; 
	\node [left,color=red] at (axis cs:  0, 0.5) {$10V$}; 
	\node [left,color=red] at (axis cs:  0, -1) {$-20V$}; 
	\node [left,color=red] at (axis cs:  0, -0.5) {$-10V$}; 
		\node [below,color=blue] at (axis cs:  10, 0) {$1\si{\milli\second}$};
		\node [below,color=blue] at (axis cs:  20, 0) {$2\si{\milli\second}$};
	\node [below,color=blue] at (axis cs:  30, 0) {$3\si{\milli\second}$}; 
	\node [below,color=blue] at (axis cs:  40, 0) {$4\si{\milli\second}$};
	\addplot [draw=red,thick] coordinates{ (0,0.5) (5, 0.5) (10, 0) (15, -0.5) (25, -0.5) (30, 0) (35, 0.5) (45, 0.5)(50, 0)};
	\addplot [draw=red,dashed] coordinates{ (0,1) (10, 0) (20, -1)(30, 0)(40, 1)(50, 0)};
	\end{axis}
	\end{tikzpicture}
\end{center}
}

\end{enumerate}
