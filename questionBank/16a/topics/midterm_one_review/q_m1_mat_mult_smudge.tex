% Author: Dun-Ming Brandon Huang
% bMail: dunmingbrandonhuang@berkeley.edu
% Question Source: Previous Exams
% Solution Source: Self

\qns{Singular Value De... compositions}

\textbf{Learning Topic}: Matrix Multiplications
Professor Courtade writes with a sharpie to accommodate the vision of as many people as possible.\\
Unfortunately, some characters get smudged, which makes them difficult to read. The following is a (hypothetical) passage from lecture notes, and the smudges are labeled \circled{1}, \circled{2}, \dots, \circled{10}. Your task is to identify correct expressions for each of the smudges.\\
Here is the hypothetical message:\\
\hrule
\hspace*{\fill}\begin{minipage}{\textwidth-10mm}
    Let $A\in\R^{n\times\circled{1}}$ be a matrix with rank r. It is always possible to write A in terms of its compact SVD:
    \[A=U\Sigma V^{T}\]
    where $\Sigma$ is a diagonal $r\times r$ matrix, and $U\in\R^{\circled{2}\times\circled{3}}$ and $V\in\R^{\circled{4}\times\circled{5}}$ have orthonormal columns.\\
    This means that $U^{T}U=I_{\circled{6}}$ and $V^{T}V=I_{\circled{7}}$, where we write $I_{m}$ to denote the $m\times\circled{8}$ identity matrix for an integer m.\\
    The columns of U form a basis for the range of A, which is defined as:
    \[range(A)={A\vec{x}=\vec{x}\in\R^{k}}\]
    Note that $range(A)$ is a subspace of $\R^{\circled{9}}$, which has dimension \circled{10}.
\end{minipage}

\begin{enumerate}
    \setlength\itemsep{4em}
    \item{
        Fill in the value each smudge should represent:
        \begin{center}
            \begin{tabular}{|c|c|c|c|c|}
                \hline
                Number & k & m & n & r \\
                \hline
                \circled{1} & $\bigcirc$ & $\bigcirc$ & $\bigcirc$ & $\bigcirc$ \\
                \hline
                \circled{2} & $\bigcirc$ & $\bigcirc$ & $\bigcirc$ & $\bigcirc$ \\
                \hline
                \circled{3} & $\bigcirc$ & $\bigcirc$ & $\bigcirc$ & $\bigcirc$ \\
                \hline
                \circled{4} & $\bigcirc$ & $\bigcirc$ & $\bigcirc$ & $\bigcirc$ \\
                \hline
                \circled{5} & $\bigcirc$ & $\bigcirc$ & $\bigcirc$ & $\bigcirc$ \\
                \hline
                \circled{6} & $\bigcirc$ & $\bigcirc$ & $\bigcirc$ & $\bigcirc$ \\
                \hline
                \circled{7} & $\bigcirc$ & $\bigcirc$ & $\bigcirc$ & $\bigcirc$ \\
                \hline
                \circled{8} & $\bigcirc$ & $\bigcirc$ & $\bigcirc$ & $\bigcirc$ \\
                \hline
                \circled{9} & $\bigcirc$ & $\bigcirc$ & $\bigcirc$ & $\bigcirc$ \\
                \hline
                \circled{10} & $\bigcirc$ & $\bigcirc$ & $\bigcirc$ & $\bigcirc$ \\
                \hline
            \end{tabular}
        \end{center}
    }
    \meta{
        This question comes from Q4 of Spring 2020’s Midterm 1.\\
        \begin{bindenum}
            \item There are multiple orders to solve this problem in. This solution utilizes an order that prioritizes on finding all known or relatively directly obvious information.
            \item It is encouraged that students utilize the property of this type of question to perform revisions. This also helps them practice debugging their solutions.
        \end{bindenum}
        
    }
    \ans{
        Let us start with the directly obvious items.\\
        Item \circled{2}: Since $A=U\Sigma V^{T}\in\R^{n\times\circled{1}}$, to make appropriate dimensions so that the matrix multiplication in A's factorization (compact SVD, or called a decomposition) holds, $U\in\R^{n\times\circled{3}}$. Therefore, $\circled{2}=n$.\\
        Item \circled{3}: Since $U\Sigma$ is a proper multiplication and $\Sigma\in\R^{r\times r}$, $\circled{3}=r$.\\
        Item \circled{5}: Since $\Sigma V^{T}$ is a proper multiplication and $\Sigma\in\R^{r\times r}$, we would get that $\circled{5}=r$.\\
        Item \circled{6}: Since $U\in\R^{n\times r}$, $U^{T}U\in\R^{r\times r}$. Therefore, $\circled{6}=r$.\\
        Item \circled{7}: Since $V^{T}\in\R^{r\times\circled{4}}$, $V^{T}V\in\R^{r\times r}$. Therefore, $\circled{7}=r$.\\
        Item \circled{8}: An identity matrix is square, therefore $\circled{8}=m$.\\
        Now, let's observe the items that might need help from what information we have deducted above:\\
        Item \circled{1}: It was stated in the definition of $range(A)$ that the matrix $A$ can be multiplied by vectors that can be expressed as k-by-1 matrices. Therefore, $A\in\R^{n\times k}$, leading to the conclusion that $\circled{1}=k$.\\
        Item \circled{4}: Item \circled{4} should be equal to item \circled{1} due to the way $A$ is multiplied as $U\Sigma V^{T}$. Therefore, $\circled{4}=k$.\\
        Item \circled{9}: Since $A\in\R^{n\times k}$ and $\vec{x}\in\R^{k\times 1}$, expression $A\vec{x}\in\R^{n\times 1}$. Therefore, $range(A)$ is a set of n-dimensional vectors. If $range(A)$ should also be a subset, it would be a subset of $\R^{n}$. Therefore, $\circled{9}=n$.\\
        Item \circled{10}: The range of A, $range(A)$, can also be paraphrased as $col(A)$, whose dimension is the rank of $A$. We, at the beginning of the hypothetical passage, were given that $rank(A)=r$. Therefore, $\circled{10}=r$.\\
        \\
        To express the above choices in the option table:\\
        \begin{center}
            \begin{tabular}{|c|c|c|c|c|}
                \hline
                Number & k & m & n & r \\
                \hline
                \circled{1} & $\bullet$ & $\bigcirc$ & $\bigcirc$ & $\bigcirc$ \\
                \hline
                \circled{2} & $\bigcirc$ & $\bigcirc$ & $\bullet$ & $\bigcirc$ \\
                \hline
                \circled{3} & $\bigcirc$ & $\bigcirc$ & $\bigcirc$ & $\bullet$ \\
                \hline
                \circled{4} & $\bullet$ & $\bigcirc$ & $\bigcirc$ & $\bigcirc$ \\
                \hline
                \circled{5} & $\bigcirc$ & $\bigcirc$ & $\bigcirc$ & $\bullet$ \\
                \hline
                \circled{6} & $\bigcirc$ & $\bigcirc$ & $\bigcirc$ & $\bullet$ \\
                \hline
                \circled{7} & $\bigcirc$ & $\bigcirc$ & $\bigcirc$ & $\bullet$ \\
                \hline
                \circled{8} & $\bigcirc$ & $\bullet$ & $\bigcirc$ & $\bigcirc$ \\
                \hline
                \circled{9} & $\bigcirc$ & $\bigcirc$ & $\bullet$ & $\bigcirc$ \\
                \hline
                \circled{10} & $\bigcirc$ & $\bigcirc$ & $\bigcirc$ & $\bullet$ \\
                \hline
            \end{tabular}
        \end{center}
        
    }
\end{enumerate}
    
