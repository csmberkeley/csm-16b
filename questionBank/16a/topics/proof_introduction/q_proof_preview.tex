% Author: Dylan Supencheck
% Email: dylansupencheck@berkeley.edu

% This question was written to gently introduce students to
% the idea of proofs before they have learned any linear algebra.
% As such, the proofs only require algebra. 
% This was written for CSM drop-in section.

\qns{Proof Preview}

A proof shows a statement is true based on a starting set of facts. They are often of the form: "Given \emph{X} show that \emph{Y}". In this case, \emph{X} is your starting set of facts, often a definition or theorem. \emph{Y} is the statement you are trying to make clear is true.

Let's walk through a couple proofs!

\begin{enumerate}

% Part A
\item Given two odd integers, \emph{m} and \emph{n}, which can be expressed as
\begin{align*}
    m = 2a + 1 \\
    n = 2b + 1
\end{align*}
with \emph{a} and \emph{b} being any arbitrary integers, show that the sum of \emph{m} and \emph{n} must be even.

\begin{enumerate}[label=(\roman*)]
% Part i
\item \textbf{What are we given?}

\ans {

We're given two odd integers, and their form:
\begin{align*}
    m = 2a + 1 \\
    n = 2b + 1
\end{align*}
}

% Part ii
\item \textbf{What do we need to show?}

\ans {

We want to show that $m + n = k$, where $k$ is an even integer.
$$ k = 2c $$
Assuming our c is an arbitrary integer.
}

% Part iii
\item \textbf{How to go from ``what we are given'' to ``what we need to show'' :}\\
Now manipulate the expression from (i) using mathematically logical steps to reach the expression from part (ii). 

\ans {

\begin{align*}
    m + n = (2a + 1) + (2b + 1) \\
    = 2a + 2b + 2 \\
    = 2(a + b + 1) \\
    = 2c\textnormal{, where $c = a + b + 1$ is an integer.}
\end{align*}

We've shown $m + n$ is $2c$, which is of the form of an even integer (note that $c$ is once again just some arbitrary integer). Thus, we can conclude the sum of $m$ and $n$ is even.  
}
\end{enumerate} % End parts (i) to (iii)

% Part B
\item Given two even integers, \emph{m} and \emph{n}, which can be expressed as
\begin{align*}
    m = 2a \\
    n = 2b
\end{align*}
with \emph{a} and \emph{b} being any arbitrary integers, show that the sum of \emph{m} and \emph{n} must be even.

\begin{enumerate}[label=(\roman*)]
% Part i
\item \textbf{What are we given?}

\ans {

We're given two even integers, and their form:
\begin{align*}
    m = 2a \\
    n = 2b
\end{align*}
}

% Part ii
\item \textbf{What do we need to show?}

\ans {

We want to show that $m + n = k$, where $k$ is an even integer.
$$ k = 2c $$
Assuming our c is an arbitrary integer.
}

% Part iii
\item \textbf{How to go from ``what we are given'' to ``what we need to show'' :}\\
Now manipulate the expression from (i) using mathematically logical steps to reach the expression from part (ii). 

\ans {

\begin{align*}
    m + n = 2a + 2b \\
    = 2(a + b) \\
    = 2k\textnormal{, where $k = a + b$ is an integer.}
\end{align*}

We've shown $m + n$ is $2k$, which is of the form of an even integer. Thus, we can conclude the sum of $m$ and $n$ is even.  
}
\end{enumerate} % End parts (i) to (iii)

\end{enumerate} % End parts A to B
