% Titan Yuan -- titan@berkeley.edu
%Summer 2017 MT2 problem
\qns{Taking the Super-L}

Consider the following circuit below:

\begin{center}
\begin{circuitikz}

\draw (0, 0) to [short] ++ (0, 1.5)
	to [open, v=$v_R$, o-o] ++ (3, 0)
	to [short] ++ (0, -1.5);
\draw (0, 0) to [R, l_=$\SI{5}{\ohm}$] ++ (3, 0)
	to [short] ++ (0, -3)
	to [R, l_=$\SI{20}{\ohm}$] ++ (-3, 0)
	to [V_=$\SI{5}{\volt}$, a^=$V_{s_1}$, invert] ++ (0, 3);
\draw (0, -3) to [short] ++ (0, -3)
	to [I, l=$\SI{2}{\ampere}$, a=$I_{s_2}$] ++ (0, -3)
	to [short] ++ (3, 0);
\draw (3, -6) to [short] ++ (3, 0)
	to [R, l=$\SI{20}{\ohm}$] ++ (0, -3)
	to [short] ++ (-3, 0)
	to [R, l_=$\SI{10}{\ohm}$] ++ (0, 3)
	to [I, l_=$\SI{1}{\ampere}$, a^=$I_{s_1}$] ++ (0, 3);
\draw (0, -3) to [short] ++ (3, -3);

\end{circuitikz}
\end{center}

In this problem, you will use \textit{superposition} to find the voltage $v_R$ across the $\SI{5}{\ohm}$ resistor.

\begin{enumerate}

\item
First, turn off all sources \textit{except} $V_{s_1}$.  Find $v_{R_1}$, the voltage across the $\SI{5}{\ohm}$ resistor, if all sources \textit{except} $V_{s_1}$ are turned off.

\ans{

We turn off all sources except $V_{s_1}$ to get the following circuit:

\begin{center}
\begin{circuitikz}

\draw (0, 0) to [short] ++ (0, 1.5)
	to [open, v=$v_{R_1}$, o-o] ++ (3, 0)
	to [short] ++ (0, -1.5);
\draw (0, 0) to [R, l_=$\SI{5}{\ohm}$] ++ (3, 0)
	to [short] ++ (0, -3)
	to [R, l_=$\SI{20}{\ohm}$] ++ (-3, 0)
	to [V_=$\SI{5}{\volt}$, a^=$V_{s_1}$, invert] ++ (0, 3);
\draw (0, -3) to [short] ++ (0, -3)
	to [short] ++ (0, -1)
	to [open, o-o] ++ (0, -1)
	to [short] ++ (0, -1)
	to [short] ++ (3, 0);
\draw (3, -6) to [short] ++ (3, 0)
	to [R, l=$\SI{20}{\ohm}$] ++ (0, -3)
	to [short] ++ (-3, 0)
	to [R, l_=$\SI{10}{\ohm}$] ++ (0, 3)
	to [short] ++ (0, 1)
	to [open, o-o] ++ (0, 1)
	to [short] ++ (0, 1);
\draw (0, -3) to [short] ++ (3, -3);

\end{circuitikz}
\end{center}

Since the $\SI{5}{\ohm}$ and the $\SI{20}{\ohm}$ resistors are in series, we can use the voltage divider formula to find the voltage $v_{R_1}$ across the $\SI{5}{\ohm}$ resistor.

\[v_{R_1} = \frac{\SI{5}{\ohm}}{\SI{5}{\ohm} + \SI{20}{\ohm}} \cdot \SI{5}{\volt} = \SI{1}{\volt}\]

}



\item
Now turn off all sources \textit{except} $I_{s_1}$.  Find $v_{R_2}$, the voltage across the $\SI{5}{\ohm}$ resistor, if all sources \textit{except} $I_{s_1}$ are turned off.

\ans{

We turn off all sources except $I_{s_1}$ to get the following circuit:

\begin{center}
\begin{circuitikz}

\draw (0, 0) to [short] ++ (0, 1.5)
	to [open, v=$v_{R_2}$, o-o] ++ (3, 0)
	to [short] ++ (0, -1.5);
\draw (0, 0) to [R, l_=$\SI{5}{\ohm}$] ++ (3, 0)
	to [short] ++ (0, -3)
	to [R, l_=$\SI{20}{\ohm}$] ++ (-3, 0)
	to [short] ++ (0, 3);
\draw (0, -3) to [short] ++ (0, -3)
	to [short] ++ (0, -1)
	to [open, o-o] ++ (0, -1)
	to [short] ++ (0, -1)
	to [short] ++ (3, 0);
\draw (3, -6) to [short] ++ (3, 0)
	to [R, l=$\SI{20}{\ohm}$] ++ (0, -3)
	to [short] ++ (-3, 0)
	to [R, l_=$\SI{10}{\ohm}$] ++ (0, 3)
	to [I, l_=$\SI{1}{\ampere}$, a^=$I_{s_1}$] ++ (0, 3);
\draw (0, -3) to [short] ++ (3, -3);

\end{circuitikz}
\end{center}

Since the $\SI{20}{\ohm}$ and the $\SI{5}{\ohm}$ resistors are in parallel, we can use the current divider formula to find the current $i_{R_2}$ through the $\SI{5}{\ohm}$ resistor.
\[i_{R_2} = -\frac{\SI{20}{\ohm}}{\SI{20}{\ohm} + \SI{5}{\ohm}} \cdot \SI{1}{\ampere} = -\SI{0.8}{\ampere}\]
\[v_{R_2} = -\SI{0.8}{\ampere} \cdot \SI{5}{\ohm} = -\SI{4}{\volt}\]

}



\item
Now turn off all sources \textit{except} $I_{s_2}$.  Find $v_{R_3}$, the voltage across the $\SI{5}{\ohm}$ resistor, if all sources \textit{except} $I_{s_2}$ are turned off.

\ans{

We turn off all sources except $I_{s_2}$ to get the following circuit:

\begin{center}
\begin{circuitikz}

\draw (0, 0) to [short] ++ (0, 1.5)
	to [open, v=$v_{R_3}$, o-o] ++ (3, 0)
	to [short] ++ (0, -1.5);
\draw (0, 0) to [R, l_=$\SI{5}{\ohm}$] ++ (3, 0)
	to [short] ++ (0, -3)
	to [R, l_=$\SI{20}{\ohm}$] ++ (-3, 0)
	to [short] ++ (0, 3);
\draw (0, -3) to [short] ++ (0, -3)
	to [I, l=$\SI{2}{\ampere}$, a=$I_{s_2}$] ++ (0, -3)
	to [short] ++ (3, 0);
\draw (3, -6) to [short] ++ (3, 0)
	to [R, l=$\SI{20}{\ohm}$] ++ (0, -3)
	to [short] ++ (-3, 0)
	to [R, l_=$\SI{10}{\ohm}$] ++ (0, 3)
	to [short] ++ (0, 1)
	to [open, o-o] ++ (0, 1)
	to [short] ++ (0, 1);
\draw (0, -3) to [short] ++ (3, -3);

\end{circuitikz}
\end{center}

There is no current flowing through the $\SI{5}{\ohm}$ resistor, so $v_{R_3} = \SI{0}{\volt}$.

}



\item
Now \textbf{find the voltage} $v_R$ across the $\SI{5}{\ohm}$ resistor if \textit{all sources} are on.

\ans{

We add up $v_{R_1}$, $v_{R_2}$, and $v_{R_3}$ to get:
\[v_R = \SI{1}{\volt} - \SI{4}{\volt} + \SI{0}{\volt} = -\SI{3}{\volt}\]

}

\end{enumerate}

