% Author: YOUR NAME
% Email: YOUR EMAIL
% CSM16A SEMESTER YEAR

\qns{Circuit Question: Negative Feedback}

Given the following circuit:
\begin{center}
    \begin{circuitikz}
        \draw
            (0, 0) node [op amp] (AMP) {}
            (AMP.out)
                to [short, i_=$i_o$] ++ (0.5, 0) coordinate (AMPTrueOut)
            (AMP.-)
                to [short] ++(-3, 0) coordinate (u-)
                to [short] ++(0, 1.5) coordinate (topLeft)
                to [R, l^=${R_1=3k\Omega}$] ++(2.5, 0) coordinate (r1)
                to [V=$V_1$, l_=$-5V$] (r1 -| AMPTrueOut)
                to [short] (AMPTrueOut)
            (u-)
                to [short] ++(-1, 0) coordinate(IsEnd)
                to [I, invert, l_=$I_S$, i<=$6mA$] ++(0, -2) node [ground] {}
            (AMP.+)
                to [short] ++(-1, 0)
                to [V=$V_S$, l_=$4V$] ++(0, -1.5) node [ground] {}
            (AMPTrueOut)
                to [short] ++(0.5, 0)
                to [R, l^=${R_2=1k\Omega}$] ++(0, -2) node [ground] {}
            (AMPTrueOut)
                to [short, -o] ++(1, 0) node [above] {$V_{o}$}
            ;
    \end{circuitikz}
\end{center}

\begin{enumerate}
    \item {
        Explain whether or not this op-amp circuit has negative feedback.
    }
    \ans {
        To show that a circuit has negative feedback, we show that provided a constant $u_+$, the increase of $V_{out}$ leads to the increase of $u_-$.
        This is because we would then be able to deduce that the increase of Vout leads to the decrease of $u_+ - u_-$, and yet, in an op-amp model, $V_{out} = A \times (u_+ - u_-)$.
        In the above context, a positive increase in the output leads to a decrease in the output, resulting in the phenomenon we know as negative feedback.
        To answer this question, we strive to identify the above process in this subpart of the question.
        
        The solution will use the same passive sign convention drawn below throughout the entire problem:
        \begin{center}
            \begin{circuitikz}
                \draw
                    (0, 0) node [op amp] (AMP) {}
                    (AMP.out)
                        to [short, i_=$i_o$] ++ (0.5, 0) coordinate (AMPTrueOut)
                    (AMP.-)
                        to [short] ++(-3, 0) coordinate (u-)
                        to [short] ++(0, 1.5) coordinate (topLeft)
                        to [R, l^=${R_1=3k\Omega}$, i_=$i_{R_1}$] ++(2.5, 0) coordinate (r1)
                        to [V=$V_1$, l_=$-5V$] (r1 -| AMPTrueOut)
                        to [short] (AMPTrueOut)
                    (u-)
                        to [short] ++(-1, 0) coordinate(IsEnd)
                        to [I, invert, l_=$I_S$, i<=$6mA$] ++(0, -2) node [ground] {}
                    (AMP.+)
                        to [short] ++(-1, 0)
                        to [V=$V_S$, l_=$4V$] ++(0, -1.5) node [ground] {}
                    (AMPTrueOut)
                        to [short] ++(0.5, 0)
                        to [R, l^=${R_2=1k\Omega}$, i_=$i_{R_2}$] ++(0, -2) node [ground] {}
                    (AMPTrueOut)
                        to [short, -o] ++(1, 0) node [above] {$V_{o}$}
                    ;
            \end{circuitikz}
        \end{center}
        The above circuit reveals the relatinoship that $u_- - R_1 i_{R_1} + 5V = V_o$.
        By increasing $V_o$, we will in turn end up increasing $u_-$, as the above equation needs to hold.
        Therefore, by the analysis we make at the beginning of the problem, we see negative feedback in this circuit.
        
    }

    \item {
        Find the output voltage, $V_o$.
    }
    \ans {
        \textbf{\textit{Approach 1: Using Superposition.}} \\
        To perform superposition, we will have to consider three alternative versions of the given circuit, each with only one source left (and all others zeroed).
        In this section, we will go through the analyses for these versions of circuit.
        
        \textit{Circuit A: where $I_s$ is not zeroed.} \\
        The circuit diagram is as portrayed below:
        \begin{center}
            \begin{circuitikz}
                \draw
                    (0, 0) node [op amp] (AMP) {}
                    (AMP.out)
                        to [short, i_=$i_o$] ++ (0.5, 0) coordinate (AMPTrueOut)
                    (AMP.-)
                        to [short] ++(-3, 0) coordinate (u-)
                        to [short] ++(0, 1.5) coordinate (topLeft)
                        to [R, l^=${R_1=3k\Omega}$, i_=$i_{R_1}$] ++(2.5, 0) coordinate (r1)
                        to [short] (r1 -| AMPTrueOut)
                        to [short] (AMPTrueOut)
                    (u-)
                        to [short] ++(-1, 0) coordinate(IsEnd)
                        to [I, invert, l_=$I_S$, i<=$6mA$] ++(0, -2) node [ground] {}
                    (AMP.+)
                        to [short] ++(-1, 0)
                        to [short] ++(0, -1.5) node [ground] {}
                    (AMPTrueOut)
                        to [short] ++(0.5, 0)
                        to [R, l^=${R_2=1k\Omega}$, i_=$i_{R_2}$] ++(0, -2) node [ground] {}
                    (AMPTrueOut)
                        to [short, -o] ++(1, 0) node [above] {$V_{o}$}
                    ;
            \end{circuitikz}
        \end{center}
        In the above circuit, we may observe the relationship $u_- - R_1 i_{R_1} = V_o$, such that the increase of $V_o$ leads to an increase in $u_-$ provided a constant $u_+$.
        We may also acknowledge via Kirchhoff's Current Law that, in circuit A, $i_{R_1} = 6mA$.
        This aspect of the circuit also lets us recognize that negative feedback occurs in circuit A, such that $u_- = u_+ = 0V$.
        Therefore, at circuit A, $V_o = -18V$.

        \textit{Circuit B: where $V_s$ is not zeroed.} \\
        The circuit diagram is as portrayed below:
        \begin{center}
            \begin{circuitikz}
                \draw
                    (0, 0) node [op amp] (AMP) {}
                    (AMP.out)
                        to [short, i_=$i_o$] ++ (0.5, 0) coordinate (AMPTrueOut)
                    (AMP.-)
                        to [short] ++(-3, 0) coordinate (u-)
                        to [short] ++(0, 1.5) coordinate (topLeft)
                        to [R, l^=${R_1=3k\Omega}$, i_=$i_{R_1}$] ++(2.5, 0) coordinate (r1)
                        to [short] (r1 -| AMPTrueOut)
                        to [short] (AMPTrueOut)
                    (u-)
                        to [short] ++(-1, 0) coordinate(IsEnd)
                        to [open, o-o] ++(0, -2) node [ground] {}
                    (AMP.+)
                        to [short] ++(-1, 0)
                        to [V=$V_S$, l_=$4V$] ++(0, -1.5) node [ground] {}
                    (AMPTrueOut)
                        to [short] ++(0.5, 0)
                        to [R, l^=${R_2=1k\Omega}$, i_=$i_{R_2}$] ++(0, -2) node [ground] {}
                    (AMPTrueOut)
                        to [short, -o] ++(1, 0) node [above] {$V_{o}$}
                    ;
            \end{circuitikz}
        \end{center}
        By the golden rule of ideal op-amps, we claim that there are no currents flowing into the negative and positive terminals of our op-amp.
        Therefore, no current is flowing through $R_1$.
        This indicates that $u_- - R_1 i_{R_1} = u_- - 0 = V_{out}$. Furthermore, from the negative feedback observed here, we may determine via golden rules of op-amp that $u_- = u_+$.
        Thus, $V_o = 4V$

        \textit{Circuit C: where $V_1$ is not zeroed.} \\
        The circuit diagram is as portrayed below:
        \begin{center}
            \begin{circuitikz}
                \draw
                    (0, 0) node [op amp] (AMP) {}
                    (AMP.out)
                        to [short, i_=$i_o$] ++ (0.5, 0) coordinate (AMPTrueOut)
                    (AMP.-)
                        to [short] ++(-3, 0) coordinate (u-)
                        to [short] ++(0, 1.5) coordinate (topLeft)
                        to [R, l^=${R_1=3k\Omega}$, i_=$i_{R_1}$] ++(2.5, 0) coordinate (r1)
                        to [V=$V_1$, l_=$-5V$] (r1 -| AMPTrueOut)
                        to [short] (AMPTrueOut)
                    (u-)
                        to [short] ++(-1, 0) coordinate(IsEnd)
                        to [open, o-o] ++(0, -2) node [ground] {}
                    (AMP.+)
                        to [short] ++(-1, 0)
                        to [short] ++(0, -1.5) node [ground] {}
                    (AMPTrueOut)
                        to [short] ++(0.5, 0)
                        to [R, l^=${R_2=1k\Omega}$, i_=$i_{R_2}$] ++(0, -2) node [ground] {}
                    (AMPTrueOut)
                        to [short, -o] ++(1, 0) node [above] {$V_{o}$}
                    ;
            \end{circuitikz}
        \end{center}
        We may recognize from this diagram that $u_- - R_1 i_{R_1} + 5V = V_o$, which makes circuit C involve a negative feedback loop.
        Because so, the golden rules of ideal op-amps allow us to deduce that $u_- = u_+ = 0V$.
        Furthermore, provided that no current may ever flow into the negative terminal of an ideal op-amp, we may recognize that there are no current flowing across $R_1$.
        Therefore, realistically, $V_o = 5V$ in this circuit.

        \textit{Summing up the Above Results.} \\
        Summing up the above results, via superposition, we observe that $V_o = -18V + 4V + 5V = -9V$.
        
        \textbf{\textit{Approach 2: Using NVA.}} \\
        We note from part 1 that negative feedback is observed in the overall circuit. Therefore, $u_+ = u_- = 4V$.
        Meanwhile, by Kirchhoff's Current Law and the property of ideal op-amps that currents do not flow into their negative terminals, we note that $i_{R_1} = 6mA$.
        Using the relationship found at part 1: $u_- - R_1 i_{R_1} + 5V = V_o$, we arrive at the conclusion that:
        \begin{align*}
            V_o &= u_- - R_1 i_{R_1} + 5V \\
            &= 4V - 18V + 5V = 9V
        \end{align*}
    }

    \item {
        Find the output current, $i_o$.
    }
    \ans {
        Via Kirchhoff's Current Law, we may find that $i_{R_1} + i_o = i_{R_2}$.
        We see that $i_{R_2} = -9mA$, and that $i_{R_1} = 6mA$.
        Therefore, it must be that $i_0 = -9mA - 6mA = -15mA$.
    }
\end{enumerate}
