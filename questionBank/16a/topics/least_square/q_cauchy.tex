% Zipeng Lin - yuslzp@berkeley.edu
\qns{Cauchy Schwarz and norm} 
\textbf{Learning goal: learn more proof about module 3}.

\meta{
\begin{enumerate}
    \item First review the Cauchy inequality.
    \item Review the definition of p-norm and how that relates to the terms in
        Cauchy Schwarz.
\end{enumerate}
}

a. Recall the Cauchy Schwarz inequality:

\[
    \abs{\langle \vec{x}, \vec{y}\rangle} \le{} \norm{\vec{x}} \norm{\vec{y}}
\]

prove that for an $n$ dimensional vector  $x$, we have

\[
    \norm{x}_1 \le{} \sqrt{n}  \norm{x}_2
\]



\ans{

    Suppose $\vec{x} = \left( x_1, x_2, \ldots{}, x_n \right)$, then denote another
    vector  $\vec{1} = (1, 1, \ldots{}, 1)$ also an n dimension vector. Then by Cauchy
    Schwarz inequality
    
    \[
        \abs{\langle \vec{x}, \vec{1} \rangle} \le{} \norm{\vec{x}} \norm{\vec{1}}
    \]

    notice that in the Cauchy Schwarz, the norm is the length of vector in the
    sense of two dimension, so the right side is just $\sqrt{n} \norm{x}_2 $,
    while th left side is just $\norm{x}_1$
    
}
