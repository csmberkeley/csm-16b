% Author: Christopher Duroiu
% Email: chduroiu@berkeley.edu
% CSM16A Spring 2024
\qns{Parallel Capacitors}

\textbf{Learning Goal: }
\begin{bindenum}
    \item Teach students how to apply capacitor voltage / current formulas
    \item Teach students how to deal with sinusoidal inputs
\end{bindenum}

\begin{center}
    \begin{circuitikz}[american]
    \draw (0,3) to[sV, v=$V_{in}$] (0,0);
    \draw (3,3) to[C, v=$C_1$] (3,0);
    \draw (6,3) to[C, v=$C_2$] (6,0);
    \draw (0,3) to (6,3);
    \draw (0,0) to (6,0);
    \draw (3,2.1) to[short, i=$I_1$] (3,2);
    \draw (6,2.1) to[short, i=$I_2$] (6,2);
    \end{circuitikz}
\end{center}

Suppose $V_{in}(t)$ = $2sin(x)$ and $C_{1} = 2F, C_{2} = 3F$

\begin{enumerate}
    \item {
        Find the current $I_1$ after $2\pi$ seconds
    }
    \meta{\\
        This question introduces a straightforward application of the capacitor-current equation
    }
    \ans{\\
        We know the formula for the current entering a capacitor is $i_{C}(t) = C \frac{dv(t)}{dt}$ \\
        \\
        Thus, the current enter capacitor $C_1$ after $2\pi$ seconds is: \\
        $i_{1}(t) = C_{1} \frac{dv(t)}{dt}$ \\
        $V_{in}$ = $2sin(x)$ \\
        $\frac{dv_{in}(t)}{dt}$ = $2cos(t)$ \\
        $i_{1}(2\pi) = C_{1}*2cos(2\pi)$ \\
        $i_{1}(2\pi) = C_{1}*2*1$ \\
        $i_{1}(2\pi) = C_{1}*2*1$ \\
        $i_{1}(2\pi) = 4 A.$ \\
    }

    \item {
        Find the current $I_2$ after $3\pi$ seconds
    
    }
    \meta{\\
        This question introduces a straightforward application of the capacitor-current equation

    }
    \ans{\\
        As in the previous question, we know \\
        $i_{2}(3\pi) = C_{2} \frac{dv(3\pi)}{dt}$ \\
        $\frac{dv_{in}(t)}{dt}$ = $2cos(t)$ \\
        Thus \\
        $i_{2}(3\pi) = C_{2}*2cos(3\pi)$ \\
        $i_{2}(3\pi) = C_{2}*2*(-1)$ \\
        $i_{2}(3\pi) = -6 A.$
    }
    \item {
        Derive an expression for the equivalent capacitance between $C_1$ and $C_2$ and evaluate the numerical value
    
    }
    \meta{\\
        This question asks students to derive the parallel capacitor formula, practicing derivation

    }
    \ans{\\
        $i_{1}(t) = C_{1} \frac{dv(t)}{dt}$ \\
        $i_{2}(t) = C_{2} \frac{dv(t)}{dt}$ \\
        $i_{t}(t) = i_{1}(t) + i_{2}(t)$ \\
        $i_{t}(t) = C_{1} \frac{dv(t)}{dt} + C_{2} \frac{dv(t)}{dt}$ \\
        $i_{t}(t) = (C_{1} + C_{2})\frac{dv(t)}{dt}$ \\
        Thus, the equivalent capacitance is: \\
        $C_{eq} = C_{1} + C_{2}$ \\
        $C_{eq} = 2 + 3 = 5 F.$
    }
    \item {
        Now, suppose our voltage source was replaced with a current source $i(t) = 3 A.$ as shown in the figure below \\
        \begin{center}
        \begin{circuitikz}[american]
        \draw (0,0) to[I, v=$I_{in}$] (0,3);
        \draw (3,3) to[C, v=$C_1$] (3,0);
        \draw (6,3) to[C, v=$C_2$] (6,0);
        \draw (0,3) to (6,3);
        \draw (0,0) to (6,0);
        \end{circuitikz}
        \end{center}
        \\
        What is the voltage across capacitor $C_1$ after 2 seconds. Assume both capacitors are initially uncharged
    }
    \meta{\\
        This question introduces the capacitor voltage formula for DC sources, again practicing derivation

    }
    \ans{\\
        Replacing the capacitors with the equivalent capacitor $C_{eq}$ we have\\
        $i_{t}(t) = C_{eq} \frac{dv(t)}{dt}$ \\
        $\frac{i_{t}(t)}{C_{eq}} = \frac{dv(t)}{dt}$ \\
        $\frac{i_{t}(t)dt}{C_{eq}} = dv(t)$ \\
        $\int \frac{i_{t}(t)dt}{C_{eq}} = \int dv(t)$ \\
        $\frac{i_{t}(t)}{C_{eq}}t = V_C(t) - V_0$ \\
        $V_C(t) = \frac{i_{t}(t)}{C_{eq}}t + V_0$ \\
        thus \\
        $V_C(2) = \frac{i_{2}(t)}{C_{eq}}t + V_0$ \\
        $V_C(2) = \frac{3(t)}{C_{eq}}t + 0$ \\
        $V_C(2) = \frac{3*2}{5} + 0$ = 1.25 V.
    }
    
\end{enumerate}
