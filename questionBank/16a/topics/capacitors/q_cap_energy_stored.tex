% Author: Anya Shrivastava
\qns{Energy Stored in Capacitors}

\textbf{Learning Goal:} The goal of this problem is to help students gain and understanding of the physical intuition behind capacitors. 

\textbf{Relevant Notes:} \notes{Note 16: Section 16.4} explains the physics of capacitors. Figures are sourced from this note.

\begin{enumerate}

\itemImagine you have two parallel plates of metal, and you set a voltage across the plates. What types of charges are accumulated on each of the plates connected to the positive and negative sides of the capacitor?

\begin{figure}[h]
\begin{center}
\includegraphics[scale=0.2]{../q_cap_energy_stored_figs/plates.png}
\end{center}

\end{figure}

\ans{

\begin{figure}[h]
\begin{center}
\includegraphics[scale=0.2]{../q_cap_energy_stored_figs/plates2.png}
\end{center}

\end{figure}

Positive charges are accumulated on the top surface of the plate connected to the positive side. Negative charges are accumulated on the bottom surface of the plate connected to the negative side. 
}

\itemWe have seen in class that $C = \epsilon \frac{A}{d}$, where $\epsilon$ is the permittivity of the substance between the two plates, $A$ is the surface area of the plate, and $d$ is the distance between the plates. A charged capacitor has a voltage across it. What else does this tell us about the charged capacitor? \\
\textit{Hint: If you have voltage across something, what other quantity does it store?}

\meta{ If students are having difficulty understanding the physical intuition behind capacitors, feel free to use the water bucket analogy referenced in \note{Note 16.4}. The note states: “ The more water (charge) you put in the bucket, the higher the water level (voltage) would be. The water level in the bucket is determined by its dimensions. Similarly the capacitance is set by the dimensions of the conductor plates and some properties of the material separating them.” }

\ans{

This means there is energy stored by the capacitor. Energy is stored because charges of the same type repel each other, and we need to provide energy whenever we make these charges close to each other. 
}

\itemWe have seen in class that $dE = V_c dQ$ where $dE$ is a small amount of energy, $dQ$ is a small amount of charge, and $V_c$ is the voltage across the capacitor. We have also seen that $dQ = C dV_c$. Using these two relationships, derive an expression for the energy stored by a capacitor with a capacitance $C$ and a voltage $V$ across it when it is fully charged.

\meta { Explain the significance of the answer, it shows that capacitors can be used as an energy storage element in circuits because the charges don't just vanish, instead they remain and the capacitor continues to store energy. For example, capacitors can be used in circuits to dissipate energy in a controlled, smooth manner.}

\ans{

Will be added later! See Note 16 equations 12-14.
}
    
\end{enumerate}