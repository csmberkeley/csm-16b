% Author: Aurelia Wang
% Author Email: aureliawang@berkeley.edu
% CSM16A Fall 2023

\qns{Properties and Proofs of Matrix Inverses} \\

\meta {
  \begin{itemize}
    \item Begin by reviewing the definition of a matrix inverse.
    \item Discuss the property that if a matrix \( A \) has an inverse \( A^{-1} \), then \( AA^{-1} = I \), where \( I \) is the identity matrix.
    \item It might be helpful to visualize the matrix inverse as "undoing" the transformation represented by the matrix.
    \item Make it clear that not all matrices have inverses. Matrices that do have inverses are called invertible or non-singular.
  \end{itemize} 
}

\begin{enumerate}
  \item State whether the following statements are true or false. Provide reasons or counterexamples.

  \begin{enumerate}
    \item If a matrix \( A \) has an inverse \( A^{-1} \), then \( A \) is square.
    
    \ans {
      True. Only square matrices can have inverses.
    }

    \item Two matrices \( A \) and \( B \) that are inverses of each other are necessarily square.
    
    \ans {
      True. Inverses only exist for square matrices.
    }

    \item If \( AB = I \), then \( A \) and \( B \) are inverses of each other.
    
    \ans {
      True. If the product of two matrices is the identity matrix, they are inverses of each other.
    }
  \end{enumerate}

  \item Prove the following about matrix inverses:

  \begin{enumerate}
    \item If \( A \) is invertible, then its inverse \( A^{-1} \) is unique.
    \ans {
      Let's assume there are two inverses, \( B \) and \( C \), such that \( AB = AC = I \). 
      Multiplying both sides of \( AB = I \) by \( C \) on the right, we get:
      \[ ABC = IC \]
      Since \( AC = I \) and \( IC = C \), it follows that \( ABC = C \).
      Now, multiplying \( AC = I \) by \( B \) on the left, we get:
      \[ BAC = BI \]
      Given \( AB = I \), we know that \( BA = I \). Thus, we see that \( BAC = B \).
      Combining these results, \( C = B \). Hence, the inverse is unique.
    }

    \item If \( A \) and \( B \) are both invertible, then their product \( AB \) is also invertible, and the inverse is \( B^{-1}A^{-1} \).
    \ans {
      To show \( AB \) is invertible with inverse \( B^{-1}A^{-1} \):
      \[ (AB)(B^{-1}A^{-1}) = A(BB^{-1})A^{-1} = AIA^{-1} = AA^{-1} = I \]
      And,
      \[ (B^{-1}A^{-1})(AB) = B^{-1}(A^{-1}A)B = B^{-1}IB = B^{-1}B = I \]
      Thus, \( AB \) is invertible with inverse \( B^{-1}A^{-1} \).
    }
  \end{enumerate}
\end{enumerate}