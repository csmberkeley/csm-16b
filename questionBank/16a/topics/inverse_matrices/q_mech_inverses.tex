% CSM16A Spring 2022
\allowdisplaybreaks
\qns{Mechanical Inverses}

\textbf{Learning Goal:} Practice calculating inverses, and identifying when they do not exist, for a given matrix.

\meta{
For this question, feel free to cover as many parts as your time in section allows. \\

The question gives two parts to cover in section and two left for practice, but more/less can be covered in section depending on time constraints and student familiarity with inverses. \\

The derivation of inverse formula for 2x2 would be useful to provide:
\begin{align*}
        \begin{sysmatrix}{cc|cc}
            a & b & 1 & 0 \\
            c & d & 0 & 1
        \end{sysmatrix}
        % Step 1: one-ing first elements of each row.
        &\!\begin{aligned}
            &\ro{\frac{1}{a}R_1 \rightarrow R_1} \\
            &\ro{\frac{1}{c}R_2 \rightarrow R_2}
        \end{aligned}
        \begin{sysmatrix}{cc|cc}
            1 & \frac{b}{a} & \frac{1}{a} & 0 \\
            1 & \frac{d}{c} & 0 & \frac{1}{c}
        \end{sysmatrix} \\
        % Step 2: zeroing first element of second row
        &\!\begin{aligned}
            &\ro{R_1 - R_2 \rightarrow R_2}
        \end{aligned}
        \begin{sysmatrix}{cc|cc}
            1 & \frac{b}{a} & \frac{1}{a} & 0 \\
            0 & \frac{bc-ad}{ac} & \frac{1}{a} & -\frac{1}{c}
        \end{sysmatrix} \\
        % Step 3: one-ing second element of second row
        &\!\begin{aligned}
            &\ro{\frac{ac}{bc-ad}R_2 \rightarrow R_2}
        \end{aligned}
        \begin{sysmatrix}{cc|cc}
            1 & \frac{b}{a} & \frac{1}{a} & 0 \\
            0 & 1 & \frac{c}{bc-ad} & -\frac{a}{bc-ad}
        \end{sysmatrix} \\
        % Step 4: zeroing second element of first row
        &\!\begin{aligned}
            &\ro{R_1 - \frac{a}{b}R_2 \rightarrow R_1}
        \end{aligned}
        \begin{sysmatrix}{cc|cc}
            1 & 0 & \frac{d}{ad-bc} & -\frac{b}{ad-bc} \\
            0 & 1 & -\frac{c}{ad-bc} & \frac{a}{ad-bc}
        \end{sysmatrix} \\
    \end{align*}
    The entire elimination process is shown as above.
    The derivation for the element $\frac{d}{ad-bc}$ might be slightly confusing, so here follows a breakdown of the algebraic steps taken"
    \begin{align*}
        \frac{1}{a} - \frac{b}{a}\frac{c}{bc-ad}
        &= \frac{1}{a}(1 + \frac{bc}{ad-bc}) \\
        &= \frac{1}{a}(\frac{ad-bc}{ad-bc} + \frac{bc}{ad-bc}) \\
        &= \frac{1}{a}(\frac{ad}{ad-bc}) \\
        &= \frac{d}{ad-bc}
    \end{align*}
    By the above efforts, we have successfully derived that for a matrix:
    \[A = 
        \begin{bmatrix}
            a & b \\
            c & d
        \end{bmatrix}
    \]
    Its inverse can be expressed in the formula:
    \[A^{-1} = \frac{1}{ad-bc}
        \begin{bmatrix}
            d & -b \\
            -c & a
        \end{bmatrix}
    \]
}

In each part, determine whether the inverse of $\mathbf{A}$ exists. If it exists, find it. 

\begin{enumerate}
\item $\mathbf{A} = \begin{bmatrix}
1 & 0\\
0 & 9
\end{bmatrix}$

\ans{

We use Gaussian elimination (also known as the Gauss-Jordan method):
\[
\left[\begin{array}{cc|cc}
1 & 0 & 1 & 0\\
0 & 9 & 0 & 1
\end{array}\right]
\overbrace{\Rightarrow}^{R_2 \leftarrow \frac{1}{9} R_2}
\left[\begin{array}{cc|cc}
1 & 0 & 1 & 0\\
0 & 1 & 0 & \frac{1}{9}
\end{array}\right].
\]

Therefore, we get $\mathbf{A}^{-1} = \begin{bmatrix}
1 & 0\\
0 & \frac{1}{9}
\end{bmatrix}$.

}

\meta{

\textbf{(5 min)}

}

\item$\mathbf{A} = \begin{bmatrix}
5 & 4\\
1 & 1
\end{bmatrix}$

\ans{

We use Gaussian elimination:
\begin{align*}
&\left[\begin{array}{cc|cc}
5 & 4 & 1 & 0\\
1 & 1 & 0 & 1
\end{array}\right]
&\overbrace{\Rightarrow}^{R_1 \leftarrow R_2}
&\left[\begin{array}{cc|cc}
1 & 1 & 0 & 1\\
5 & 4 & 1 & 0
\end{array}\right] \\
\overbrace{\Rightarrow}^{R_2 \leftarrow -5R_1+R_2}
&\left[\begin{array}{cc|cc}
1 & 1 & 0 & 1\\
0 & -1 & 1 & -5
\end{array}\right]
&\overbrace{\Rightarrow}^{R_1 \leftarrow R_1 + R_2}
&\left[\begin{array}{cc|cc}
1 & 0 & 1 & -4\\
0 & -1 & 1 & -5
\end{array}\right] \\
\overbrace{\Rightarrow}^{R_2 \leftarrow -R_2}
&\left[\begin{array}{cc|cc}
1 & 0 & 1 & -4\\
0 & 1 & -1 & 5
\end{array}\right]. 
\end{align*}

Therefore, we get $\mathbf{A}^{-1} = \begin{bmatrix}
1 & -4\\
-1 & 5
\end{bmatrix}$.

}

\meta{

\textbf{(2 min)}

}



\item(PRACTICE) \\ \\
$\mathbf{A} = \begin{bmatrix}
5 & 5 & 15\\
2 & 2 & 4\\
1 & 1 & 4
\end{bmatrix}$

\ans{

We use Gaussian elimination:
\begin{align*}
&\left[\begin{array}{ccc|ccc}
5 & 5 & 15 & 1 & 0 & 0\\
2 & 2 & 4 & 0 & 1 & 0\\
1 & 1 & 4 & 0 & 0 & 1
\end{array}\right]
&\overbrace{\Rightarrow}^{R_1 \leftarrow \frac{1}{5} R_1}
&\left[\begin{array}{ccc|ccc}
1 & 1 & 3 & \frac{1}{5} & 0 & 0\\
2 & 2 & 4 & 0 & 1 & 0\\
1 & 1 & 4 & 0 & 0 & 1
\end{array}\right]\\
\overbrace{\Rightarrow}^{R_2 \leftarrow \frac{1}{2} R_2}
&\left[\begin{array}{ccc|ccc}
1 & 1 & 3 & \frac{1}{5} & 0 & 0\\
1 & 1 & 2 & 0 & \frac{1}{2} & 0\\
1 & 1 & 4 & 0 & 0 & 1
\end{array}\right]
&\overbrace{\Rightarrow}^{R_2 \leftarrow R_2 - R_1}
&\left[\begin{array}{ccc|ccc}
1 & 1 & 3 & \frac{1}{5} & 0 & 0\\
0 & 0 & -1 & -\frac{1}{5} & \frac{1}{2} & 0\\
1 & 1 & 4 & 0 & 0 & 1
\end{array}\right]\\
\overbrace{\Rightarrow}^{R_3 \leftarrow R_3 - R_1}
&\left[\begin{array}{ccc|ccc}
1 & 1 & 3 & \frac{1}{5} & 0 & 0\\
0 & 0 & -1 & -\frac{1}{5} & \frac{1}{2} & 0\\
0 & 0 & 1 & -\frac{1}{5} & 0 & 1
\end{array}\right]
&\overbrace{\Rightarrow}^{R_3 \leftarrow R_3 + R_2}
&\left[\begin{array}{ccc|ccc}
1 & 1 & 3 & \frac{1}{5} & 0 & 0\\
0 & 0 & -1 & -\frac{1}{5} & \frac{1}{2} & 0\\
0 & 0 & 0 & -\frac{2}{5} & \frac{1}{2} & 1
\end{array}\right].
\end{align*}

While row-reducing, we notice that the second column doesn't have a pivot (and that there is also a row of zeros). Therefore, no inverse exists.

}

\meta{

\textbf{(8 min)}

}

\item(PRACTICE) \\ \\
$\mathbf{A} = \begin{bmatrix}
5 & 5 & 15\\
2 & 2 & 4\\
1 & 0 & 4
\end{bmatrix}$

\ans{

We use Gaussian elimination:
\begin{align*}
&\left[\begin{array}{ccc|ccc}
5 & 5 & 15 & 1 & 0 & 0\\
2 & 2 & 4 & 0 & 1 & 0\\
1 & 0 & 4 & 0 & 0 & 1
\end{array}\right]
&\overbrace{\Rightarrow}^{R_1 \leftarrow \frac{1}{5} R_1}
&\left[\begin{array}{ccc|ccc}
1 & 1 & 3 & \frac{1}{5} & 0 & 0\\
2 & 2 & 4 & 0 & 1 & 0\\
1 & 0 & 4 & 0 & 0 & 1
\end{array}\right]\\
\overbrace{\Rightarrow}^{R_2 \leftarrow \frac{1}{2} R_2}
&\left[\begin{array}{ccc|ccc}
1 & 1 & 3 & \frac{1}{5} & 0 & 0\\
1 & 1 & 2 & 0 & \frac{1}{2} & 0\\
1 & 0 & 4 & 0 & 0 & 1
\end{array}\right]
&\overbrace{\Rightarrow}^{R_2 \leftarrow R_2 - R_1}
&\left[\begin{array}{ccc|ccc}
1 & 1 & 3 & \frac{1}{5} & 0 & 0\\
0 & 0 & -1 & -\frac{1}{5} & \frac{1}{2} & 0\\
1 & 0 & 4 & 0 & 0 & 1
\end{array}\right]\\
\overbrace{\Rightarrow}^{R_3 \leftarrow R_3 - R_1}
&\left[\begin{array}{ccc|ccc}
1 & 1 & 3 & \frac{1}{5} & 0 & 0\\
0 & 0 & -1 & -\frac{1}{5} & \frac{1}{2} & 0\\
0 & -1 & 1 & -\frac{1}{5} & 0 & 1
\end{array}\right]
&\overbrace{\Rightarrow}^{R_2 \leftrightarrow R_3}
&\left[\begin{array}{ccc|ccc}
1 & 1 & 3 & \frac{1}{5} & 0 & 0\\
0 & -1 & 1 & -\frac{1}{5} & 0 & 1\\
0 & 0 & -1 & -\frac{1}{5} & \frac{1}{2} & 0
\end{array}\right] \\
\overbrace{\Rightarrow}^{R_2 \leftarrow -R_2}
&\left[\begin{array}{ccc|ccc}
1 & 1 & 3 & \frac{1}{5} & 0 & 0\\
0 & 1 & -1 & \frac{1}{5} & 0 & -1\\
0 & 0 & -1 & -\frac{1}{5} & \frac{1}{2} & 0
\end{array}\right]
&\overbrace{\Rightarrow}^{R_3 \leftarrow -R_3}
&\left[\begin{array}{ccc|ccc}
1 & 1 & 3 & \frac{1}{5} & 0 & 0\\
0 & 1 & -1 & \frac{1}{5} & 0 & -1\\
0 & 0 & 1 & \frac{1}{5} & -\frac{1}{2} & 0
\end{array}\right] \\
\overbrace{\Rightarrow}^{R_2 \leftarrow R_2+R_3}
&\left[\begin{array}{ccc|ccc}
1 & 1 & 3 & \frac{1}{5} & 0 & 0\\
0 & 1 & 0 & \frac{2}{5} & -\frac{1}{2} & -1\\
0 & 0 & 1 & \frac{1}{5} & -\frac{1}{2} & 0
\end{array}\right]
&\overbrace{\Rightarrow}^{R_1 \leftarrow R_1-3 R_3}
&\left[\begin{array}{ccc|ccc}
1 & 1 & 0 & -\frac{2}{5} & \frac{3}{2} & 0\\
0 & 1 & 0 & \frac{2}{5} & -\frac{1}{2} & -1\\
0 & 0 & 1 & \frac{1}{5} & -\frac{1}{2} & 0
\end{array}\right] \\
\overbrace{\Rightarrow}^{R_1 \leftarrow R_1-R_2}
&\left[\begin{array}{ccc|ccc}
1 & 0 & 0 & -\frac{4}{5} & 2 & 1\\
0 & 1 & 0 & \frac{2}{5} & -\frac{1}{2} & -1\\
0 & 0 & 1 & \frac{1}{5} & -\frac{1}{2} & 0
\end{array}\right].
\end{align*}

Therefore, we get $\mathbf{A}^{-1} = \begin{bmatrix}
-\frac{4}{5} & 2 & 1\\
\frac{2}{5} & -\frac{1}{2} & -1\\
\frac{1}{5} & -\frac{1}{2} & 0
\end{bmatrix}$.

}

\meta{

\textbf{(8 min)}

}

\end{enumerate}
