% Author: Anna Chou
% Email: menghuichou@berkeley.edu
% CSM16A Spring 2022
% Reference: Linear Algebra and its Application - Pearson Education by David C.

\qns{Linear Algebra Like a Pro}

\textbf{Learning Goal: } Understand and practice different aspects of linear algebra we learned so far.

Reference: \textit{Linear Algebra and its Application - Pearson Education by David C.} 

\meta{Make sure student know when to use inspections and when should not.}


    True and False.
        \begin{enumerate}
        %1.7 from David
            \item The columns of a matrix $A$ are linearly independent if the equation $Ax = 0$ has the trivial solution.
            \item The columns of any 4x5 matrix are linearly dependent.
            \item The columns of any 5x4 matrix are linearly independent.
            \item If x and y are linearly independent, and if {x, y, z} is linearly dependent, then z is in Span\{x, y\}.
            \item If $H$ = Span\{$b_1$ ... $b_n$\}, then \{$b_1$ ... $b_n$\} is a basis for $H$.
            \item Whenever a system has free variables, the solution set contains many solutions. %1.2
            \item If the equation $Ax=b$ is inconsistent, then b is not in the set spanned by the columns of $A$. %1.4
            \item If the augmented matrix$[A \text{ } b]$ has a pivot position in every row, then equation $Ax=b$ is consistent.%1.4
            \item If $A$ is an $m \text{x} n$ matrix whose columns do not span $\R^m$, then the equation $Ax=b$ is inconsistent for some $b$ in $\R^m$. %1.4
            \item If $A$ is invertible, then elementary row operations that reduce $A$ to the identity $I_n$ also reduce $A^{-1}$ to $I_n$. %2.2
        \end{enumerate}
            \ans{
            
                \begin{enumerate}
                    \item True.This is definition.
                    \item True. There is at least one free variable in the solution of $Ax = 0$, thus there is infinite numbers of solutions.
                    \item False. Shape does not guarantee (in)dependency. If one of the columns is linearly dependent, then the matrix is not linearly independent.
                    \item True. Since {x, y, z} is linearly dependent and x and y are linearly independent, z must be a linear combination of the other columns. Thus, z can be expressed with a linear combination of x and y, which denotes to Span\{x, y\}.
                    \item False. \{$b_1$ ... $b_n$\} is a basis for $H$ if and only if Span $H$ is a linearly independent set. Recall that you can put linearly dependent columns in Span but you can't do that in basis. Basis should be the simplest form of linearly independent column.
                    \item False. If the system is inconsistent then even if you have free variables in system, the system always has no solution.
                    \item True. This is definition!
                    \item False. If we have a pivot in the last column (augment), the system is inconsistent.
                    \item True. If $A$ cannot span entire space, it implies some $b$ in the space cannot be reached by the mapping of $A$.
                    \item True.
                \end{enumerate}
            }
            
        
