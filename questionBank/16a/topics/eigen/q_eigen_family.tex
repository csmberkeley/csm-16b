% Authors:  Viraj and Jainil
% CSM16A Spring 2022
\qns{CSM Eigen-question}

It's important to know what the eigenvalues of a matrix are. In 16B, you'll see how eigenvalues of a matrix determine the stability of a system, and how we can stabilize a system by changing the eigenvalues. We're going to attempt something similar here: we have a matrix, and we're going to try to change the eigenvalues.\\
Consider the following matrix:

$$A = \begin{bmatrix}
-1 & 2\\
2 & x
\end{bmatrix}$$

\begin{enumerate}

% Part A
\item What value of $x$ makes this matrix have linearly dependent columns?\\
\begin{enumerate}[label=(\roman*)]
    \item 0
    \item -1
    \item -4
    \item 4
\end{enumerate}

\ans{

-4: the second column becomes $-2$ times the first.\\
}

% Part B
\item Let $x = -4$. What are the eigenvalues and eigenvectors of this matrix?\\
\begin{enumerate}[label=(\roman*)]
    \item $\lambda_1 = 0, \lambda_2 = -5, \vec{v_1} = \begin{bmatrix}
    2\\
    1
    \end{bmatrix}, \vec{v_2} = \begin{bmatrix}
    2\\
    -4
    \end{bmatrix}$

    \item $\lambda_1 = 0, \lambda_2 = 5, \vec{v_1} = \begin{bmatrix}
    2\\
    1
    \end{bmatrix}, \vec{v_2} = \begin{bmatrix}
    1\\
    -2
    \end{bmatrix}$

    \item $\lambda_1 = 0, \lambda_2 = 0, \vec{v_1} = \begin{bmatrix}
    2\\
    1
    \end{bmatrix}, \vec{v_2} = \begin{bmatrix}
    2\\
    1
    \end{bmatrix}$

    \item $\lambda_1 = 0, \lambda_2 = -5, \vec{v_1} = \begin{bmatrix}
    0\\
    0
    \end{bmatrix}, \vec{v_2} = \begin{bmatrix}
    1\\
    -2
    \end{bmatrix}$

    \item $\lambda_1 = 0, \lambda_2 = 0, \vec{v_1} = \begin{bmatrix}
    0\\
    0
    \end{bmatrix}, \vec{v_2} = \begin{bmatrix}
    0\\
    0
    \end{bmatrix}$\\
\end{enumerate}

\ans{

Answer: The first option.

(i) $\lambda_1 = 0, \lambda_2 = -5, \vec{v_1} = \begin{bmatrix}
2\\
1
\end{bmatrix}, \vec{v_2} = \begin{bmatrix}
2\\
-4
\end{bmatrix}$\\

We arrive at this by finding the determinant of $ \begin{bmatrix}
-1 - \lambda & 2\\
2 & -4 - \lambda
\end{bmatrix}$\\

We get the eigenvalues by letting the determinant equal 0, and solving for lambda.\\
We then plug in the values of $\lambda = 0, -5$, and find the nullspace. The matrices we get are:\\
$$\begin{bmatrix}
-1 & 2\\
2 & -4
\end{bmatrix} \text{ and }\begin{bmatrix}
4 & 2\\
2 & 1
\end{bmatrix}$$

From there, we find the nullspace, by Gaussian Elimination, getting $\begin{bmatrix}
2\\
1
\end{bmatrix}$ and $\begin{bmatrix}
2\\
-4
\end{bmatrix}$ respectively. \\

Notice that the eigenvector can be scaled arbitrarily - the correct option does not have the second vector in its most simplified form!\\
}

% Part C
\item Now let $x = 2$. What are the eigenvalues of the matrix?\\
\begin{enumerate}[label=(\roman*)]
    \item $\lambda_1 = -3, \lambda_2 = 2$ 
    \item $\lambda_1 = 3, \lambda_2 = -2$
    \item $\lambda_1 = 1, \lambda_2 = -2$
    \item $\lambda_1 = \sqrt{5}, \lambda_2 = 2\sqrt{2}$ \\
\end{enumerate}

\ans{
$\lambda_1 = 3, \lambda_2 = -2$, by the same process of setting the determinant to zero, and solving for $\lambda$.\\
}

% Part D
\item (Challenge) Now, let's generalize: for what values of x does this matrix have both eigenvalues less than 0? 
\\Hint: solve for the eigenvalues in terms of x.\\
\begin{enumerate}[label=(\roman*)]
    \item $x < -4$
    \item $x > -4$
    \item $x < 4$
    \item $x < 0$
\end{enumerate}

\ans{
$x < -4$. Solving for the characteristic polynomial, we get $\lambda^2 + (1-x)\lambda - x - 4 = 0
$. Plugging this into the quadratic formula, we get: $\frac{-1 + x \pm \sqrt{(1-x)^2 + 4(x + 4)}}{2}$. The important thing to note here is that the larger of the two solutions is going to be the case where there is an addition as opposed to a subtraction, so we can just consider this case, and set it as less than zero.\\

Hence: 
$$\frac{-1 + x + \sqrt{(1-x)^2 + 4(x + 4)}}{2} < 0$$
$$\sqrt{(1-x)^2 + 4(x + 4)} < 1 - x$$
$$\sqrt{x^2 + 2x + 17} < 1 - x$$

Note: polynomial on LHS is always positive, so we can do:
$$x^2 + 2x + 17 < x^2 - 2x + 1$$
$$2x + 17 <- 2x + 1$$
$$4x < -16$$
$$x < -4.$$
}


% ------------ combine questions from q_eigns ----------------
\item Suppose we have $A = 
    \begin{bmatrix} 
        1 & 0 & 0\\
        0 & 2 & 0\\
        0 & 0 & 3\\
    \end{bmatrix}$
, what are the eigenvalues?

\ans{
The matrix is diagonal. By inspection, the eigenvalues are 1,2,3.
}

% Part B
\item Suppose we have $A = 
    \begin{bmatrix} 
        1 & 2\\
        0 & 3\\
    \end{bmatrix}$
. What are the eigenvalues and eigenvectors?

\ans{

    $
        \mathbf{A} - \lambda * \mathbf{I} = 
            \begin{bmatrix} 
            1 - \lambda & 2\\
            0 & 3 - \lambda
            \end{bmatrix} \\
    $    
    $    (1 - \lambda)(3 - \lambda) = 0 $\\
        
    $    \lambda = \text{1, 3} $\\
        
    
        \text{when $\lambda$ = 1} \\
        \begin{bmatrix}
            0 & 2 \\
            0 & 1
        \end{bmatrix} \rightarrow 
        \begin{bmatrix}
            0 & 1 \\
            0 & 0
        \end{bmatrix} \\
        
        $\vec{v_1}$ = \begin{bmatrix}
            1 \\
            0
        \end{bmatrix} 
        
        \text{when $\lambda = 3$} \\
        \begin{bmatrix}
            -2 & 2 \\
            0 & 0
        \end{bmatrix} \rightarrow 
        \begin{bmatrix}
            1 & -1 \\
            0 & 0
        \end{bmatrix} \\
        
        $\vec{v_2}$ = \begin{bmatrix}
            1 \\
            1
        \end{bmatrix}
       
}

% Part C
\item Suppose we have a vector $\vec{b} = 
    \begin{bmatrix} 3\\ 2 \end{bmatrix}$
. Draw this vector in the standard basis and eigenbasis.

\meta{ Review with students what a basis is and how the eigenvectors of a matrix form a basis. This part is trying to show that the eigenbasis is just another basis except its basis vectors are just not the standard ones that we are used to. 
}

\ans{
Solve $[v_1 \text{ }v_2]^\top$ $\vec{x} = \vec{b}$. You will see that $\vec{x} = \vec{v_1} + 2\vec{v_2}$. 
}

% Part D
\item If a matrix has an eigenvalue that is zero, what does that tell us about the columns of the matrix?

\ans{
Recall the definition of eigenvalue and eigenvector that $A \vec{x} = \lambda \vec{x}$ where $\vec{x}$ is a nonzero vector. If we let $\lambda$ equal to 0, the whole equation becomes $A \vec{x} = \vec{0}$. To find what is $\vec{x}$, we can think it as finding the nullspace of A. If A is linear independent, A will have a trivial nullspace, which means that only $\vec{0}$ can be the solution to $A \vec{x} = \vec{0}$. However, from the definition of eigenvalue, $\vec{x}$ should be nonzero. That's being said, if there is at least one nonzero vector existing in null space and can make $A \vec{x} = \vec{0}$, that means A must be linear independent.  
}

% Part E
% \item (PRACTICE) Suppose we have a matrix $A = 
%     \begin{bmatrix}
%         1 & 0 & 0\\
%         2 & 3 & 0\\
%         0 & 0 & 0\\
%     \end{bmatrix}$
% . Solve for the eigenvalues and eigenvectors

% \ans{
% Will be written out soon!
% }

\end{enumerate}