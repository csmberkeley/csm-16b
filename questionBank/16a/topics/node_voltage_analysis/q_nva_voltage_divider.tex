% Author: Jessica Lin
% Email: jessica.jx.lin@berkeley.edu
% CSM16A Fall 2022

% node[label={[font=\footnotesize]above:$u_1$}] {}

\qns{NVA for a Voltage Divider}

\textbf{Learning Goal:} The goal of this question is to use NVA to derive the equation of a voltage divider.

\meta{Remind students to follow passive sign convention, but that it does not matter how they choose to label their currents, as long as they are consistent.}

\begin{enumerate}

\item Use NVA to solve for $V_{AB}$ in terms of $V_s$, $R_1$, and $R_2$ in the circuit below. 

\begin{center}
\begin{circuitikz} 
\draw (0, 0)
to [V, v = $V_s$, invert] (0, 4)
to [R = $R_1$] (4, 4)
to [R = $R_2$] (4, 0)
to [short] (0, 0);

\draw (4, 4)
to [short, -o] (5, 4) node[label={[font=\footnotesize]:$A$}]{}
node[label={[font=\footnotesize]below:$+$}]{};

\draw (4, 0)
to [short, -o] (5, 0) 
node[label={[font=\footnotesize]:$-$}]{}
node[label={[font=\footnotesize]below:$B$}]{};

\node[ground] (0, -1) {};

\end{circuitikz}
\end{center}

\ans{

\textbf{Step 1:} We can label each node with a node potential, as shown in the circuit below:

\begin{center}
\begin{circuitikz} 
\draw (0, 0)
to [V, v = $V_s$, *-, invert] (0, 4)

to [R = $R_1$, *-] (4, 4)
to [R = $R_2$] (4, 0)
to [short] (0, 0);

\draw (4, 4)
to [short, *-o] (5, 4) node[label={[font=\footnotesize]:$A$}]{}
node[label={[font=\footnotesize]below:$+$}]{};

\draw (4, 0)
to [short, *-o] (5, 0) 
node[label={[font=\footnotesize]:$-$}]{}
node[label={[font=\footnotesize]below:$B$}]{};

\node[ground] (0, -1) {};

\draw (0, 4) node[label={[font=\footnotesize]:$u_1$}] {};
\draw (4, 4) node[label={[font=\footnotesize]:$u_2$}] {};
\draw (4, 0) node[label={[font=\footnotesize]below:$u_3$}] {};

\end{circuitikz}
\end{center}

\textbf{Step 2:} Then, we label each element with the current the passes through it, following passive sign convention:

\begin{center}
\begin{circuitikz} 
\draw (0, 0)
to [V, v = $V_s$, *-, i = $i_V$, invert] (0, 4)
to [R = $R_1$, i = $i_1$, v=$ $, *-] (4, 4)
to [R = $R_2$, i = $i_2$, v=$ $] (4, 0)
to [short] (0, 0);

\draw (0, 0.75) node[label=right:{$-$}]{};
\draw (0, 3.25) node[label=right:{$+$}]{};

\draw (4, 4)
to [short, *-o] (5, 4) node[label={[font=\footnotesize]:$A$}]{}
node[label={[font=\footnotesize]below:$+$}]{};

\draw (4, 0)
to [short, *-o] (5, 0) 
node[label={[font=\footnotesize]:$-$}]{}
node[label={[font=\footnotesize]below:$B$}]{};

\node[ground] (0, -1) {};

\draw (0, 4) node[label={[font=\footnotesize]:$u_1$}] {};
\draw (4, 4) node[label={[font=\footnotesize]:$u_2$}] {};
\draw (4, 0) node[label={[font=\footnotesize]below:$u_3$}] {};

\end{circuitikz}
\end{center}

\textbf{Step 3:} Write KCL equations for each junction in the circuit:
\begin{align*}
    i_V + i_1 & = 0 \\
    i_1 & = i_2 \\
    i_2 + i_V & = 0 \\
\end{align*}
\textbf{Step 4:} Write the IV relationships of each element in the circuit, using Ohm's Law for resistors:
\begin{align*}
    u_1 - u_3 & = V_s \\
    u_1 - u_2 & = i_1R_1 \\
    u_2 - u_3 & = i_2R_2 \\
    u_3 - 0 & = 0 \text{\ (ground)}
\end{align*}
\textbf{Step 5:} Simplify equations and solve. Keep in mind that our knowns are $V_s$, $R_1$, and $R_2$, and we want to determine node voltages and currents to find $V_{AB}$. Note that $V_{AB} = u_2 - u_3 = u_2 - 0 = u_2$.
\begin{align*}
    u_1 - u_3 & = u_1 - 0 = V_s \rightarrow u_1 = V_s \\
    u_1 - u_2 & = V_s - u_2 = i_1R_1 \rightarrow i_1 = \frac{V_s - u_2}{R_1} \\
    u_2 - u_3 & = u_2 - 0 = i_2R_2 \rightarrow i_2 = \frac{u_2}{R_2}
    % u_1 - u_2 & = V_s - u_2 = i_1R_1 \rightarrow u_2 = V_s - i_1R_1 \\
    % u_2 - u_3 & = u_2 - 0 = i_2R_2 \rightarrow i_2 = \frac{u_2}{R_2} \rightarrow i_1 = \frac{u_2}{R_2} \text{\ (as $i_1 = i_2$)}
\end{align*}
Now we use our KCL equations. Note that $i_1 = i_2$. We can then simplify as follows:
\begin{align*}
    i_1 & = i_2 \\
    \frac{V_s - u_2}{R_1} & = \frac{u_2}{R_2} \\
    V_s - u_2 & = u_2\frac{R_1}{R_2} \\
    % i_1 & = \frac{u_2}{R_2} \\
    % u_2 & = V_s - i_1R_1 \\
    % u_2 & = V_s - u_2\frac{R_1}{R_2} \\
    u_2 + u_2\frac{R_1}{R_2} & = V_s \\
    u_2(1 + \frac{R_1}{R_2}) & = V_s \\
    u_2(\frac{R_1 + R_2}{R_2}) & = V_s \\
    u_2 & = V_s \cdot \frac{R_2}{R_1 + R_2}
\end{align*}
\[
 \boxed{\text{Then,} $V_{AB} = u_2 = V_s \cdot \frac{R_2}{R_1 + R_2}$.}
\]
}

\item If $V_s = 6V$, $R_1 = 3 \Omega$, and $R_2 = 9 \Omega$, what is the voltage drop across $R_2$? What is the voltage drop across $R_1$?

\ans{

We can use our voltage divider equation. The voltage drop across $R_2$ is $V_R_2 = V_s \cdot \frac{R_2}{R_1 + R_2} = 6 \cdot \frac{9}{3 + 9} = 4.5V$. The voltage drop across $R_1$ is $u_1 - u_2 = V_s - V_R_2$, which is $V_R_1 = 6 - 4.5 = 1.5V$.
}

\end{enumerate}