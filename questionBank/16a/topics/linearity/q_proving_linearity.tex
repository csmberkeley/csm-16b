% Author: Dylan Supencheck
% Email: dylansupenchecka@berkeley.edu

\qns{Proving Linearity}

Soon we'll be learning how to prove things mathematically. When we prove something, we challenge ourselves to rigorously show the thing is true, given a set of facts to start with. 

Take, for example, linearity! We'll start with a set of useful facts, the definition of linearity.\\

\textbf{Linearity:} A linear function must satisfy two properties:
\begin{enumerate}
    \item[] Additivity: $f(x + y) = f(x) + f(y)$
    \item[] Homogeneity: $f(\alpha x) = \alpha f(x)$\\
\end{enumerate}

\begin{enumerate}

% Part A
\item Show that $f(x) = 5x$ is linear.

\ans{
To prove our function is linear, we must verify \textbf{additivity} and \textbf{homogeneity}.

First, to verify additivity we would like to show that $f(x + y) = f(x) + f(y)$.

$$ f(x + y) = 5(x + y) $$
$$ = 5x + 5y $$
$$ = f(x) + f(y) $$

Next, to verify homogeneity we would like to show that $f(\alpha x) = \alpha f(x)$.

$$ f(\alpha x) = 5 \alpha x $$
$$ = \alpha 5x $$
$$ = \alpha f(x) $$

Since $f(x)$ satisfies both of our properties, it must be linear!
}

% Part B
\item Show that $f(x_1, x_2) = 2x_1 + x_2$ is linear.

\ans{
To prove our function is linear, we must verify \textbf{additivity} and \textbf{homogeneity}.

First, to verify additivity we would like to show that $f(x_1 + y_1, x_2 + y_2) = f(x_1, x_2) + f(y_1, y_2)$.

$$ f(x_1 + y_1, x_2 + y_2) = 2(x_1 + y_1) + x_2 + y_2 $$
$$ = 2x_1 + 2y_1 + x_2 + y_2 $$
$$ = f(x_1, x_2) + f(y_1, y_2) $$

Next, to verify homogeneity we would like to show that $f(\alpha x_1, \alpha x_2) = \alpha f(x_1, x_2)$.

$$ f(\alpha x_1, \alpha x_2) = 2 \alpha x_1 + \alpha x_2 $$
$$ = \alpha (2x_1 + x_2) $$
$$ = \alpha f(x_1, x_2) $$

Since $f(x_1, x_2)$ satisfies both of our properties, it must be linear!
}

%Part C
\item Show that $f(x) = 2x + 1$ is \emph{not} linear.

\ans {
To prove our function is  \emph{not} linear, we just need too verify it violates at least one of \textbf{additivity} and \textbf{homogeneity}.

Let's check additivity first. We would like to show that $f(x + y) \neq f(x) + f(y)$.

$$ f(x + y) = 2(x + y) + 1 $$
$$ = 2x + 2y + 1 $$
$$ \neq 2x + 1 + 2y + 1 = f(x) + f(y) $$

Since we know $f(x)$ does not satisfy additivity, we already know that $f(x)$ is \emph{not} linear.

But what if we looked at homogeneity first? We would have wanted to show that $f(\alpha x) \neq \alpha f(x)$.

$$ f(\alpha x) = 2 \alpha x + 1 $$
$$ \neq \alpha (2x + 1) = \alpha f(x) $$

Since we know $f(x)$ does not satisfy homogeneity, we know that $f(x)$ is \emph{not} linear.\\

Wondering why this function, that would look like a line when graphed, isn't linear? Check out \notes{Note 1A: Section 1.3.3} for information on \emph{affine functions}.
}

\end{enumerate}