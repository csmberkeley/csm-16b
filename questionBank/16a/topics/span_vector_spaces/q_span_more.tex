% Author: Dun-Ming Huang
% Email: dunmingbrandonhuang@berkeley.edu
% CSM16A Fall 2022
\qns{EECS-spanned your Domain Knowledge}

\textbf{Learning Goal:} Learn about how to find the span of a set of vectors using Gaussian Elimination, and working with systems that can yield infinite solutions. \\
\meta {
    \begin{itemize}
        \item Provide students a procedure to find the span of vectors. Having ideas in mind about the general objectives and steps of mathematical operations will help students understand and operate the processes more proficiently.
        \item Graph the span on a board, and show how vectors in that span can all be found in that span. This will serve to build a great intuition when we encounter vectors that are outside spans, such as in least squares algorithm.
    \end{itemize}
    
}
That's right, we will expand your domain knowledge on spans!
\begin{ln-define}{Span}{}
    For a set of vectors $\{\vec{v_1}, \dots, \vec{v_n}\}$, its span is the set of all linear combinations of \{$\vec{v_1}, \dots, \vec{v_n}$\}. Mathematically, this is written as:
    \[
        span(\vec{v_1}, \dots, \vec{v_n})
        = \bigg\{ \sum_{i = 1}^n \alpha_i \vec{v_i}\ :\ \alpha_i \in \R \bigg\}
    \]
\end{ln-define}
And in case you are confused of what the above notation is, here's a review/preview for it:
\begin{ln-symbol}{Set Builder Notation}{}
    A set builder notation expresses a set in the format of:
    \[
        \{\text{Expression of elements } : \text{ Conditions regarding the expression of elements}\}
    \]
    The colon is sometimes substituted by a vertical bar. \\
    So the former set builder notation used for defining a span can stand for:
    \begin{quote}
        "The sum of products of scalars and each vectors in the set, with the condition that these scalars are all real".
    \end{quote}
\end{ln-symbol}

\begin{enumerate}
    \item {
         Describe the span of 
        $
            \bigg\{
                \begin{bmatrix} 0 \\ -1 \end{bmatrix},
                \begin{bmatrix} 0 \\ 1 \end{bmatrix}
            \bigg\}
        $
        
    }
    \ans {
        Let the linear combinations of this set of vector be express-able as:
        \[\alpha \begin{bmatrix} 0 \\ -1 \end{bmatrix} + \beta \begin{bmatrix} 0 \\ 1 \end{bmatrix}\]
        for real values $\alpha$ and $\beta$. \\
        In that case, the resultant linear combination will always come in the format of:
        \[(\beta - \alpha) \begin{bmatrix} 0 \\ 1 \end{bmatrix}\]
        And since $\beta$ and $\alpha$ can be any possible real value, all linear combinations of the set of vector must be some multiple of the vector $\begin{bmatrix} 0 & 1 \end{bmatrix}^T$. \\
        Therefore, the span would be:
        \[
            span(\begin{bmatrix} 0 \\ -1 \end{bmatrix}, \begin{bmatrix} 0 \\ 1 \end{bmatrix})
            = \bigg\{ k \begin{bmatrix} 0 \\ 1 \end{bmatrix} : k \in \R \bigg\}
        \]
        
    }
    
    \item {
        Describe the span of 
        $
            \bigg\{
                \begin{bmatrix} 2 \\ 0 \\ -1 \end{bmatrix},
                \begin{bmatrix} 1 \\ 0 \\ 2 \end{bmatrix}
            \bigg\}
        $
        
    }
    \ans {
        Let the linear combinations of this set of vector be express-able as:
        \[\alpha \begin{bmatrix} 2 \\ 0 \\ -1 \end{bmatrix} + \beta \begin{bmatrix} 1 \\ 0 \\ 2 \end{bmatrix}\]
        for real values $\alpha$ and $\beta$. \\
        In that case, the resultant linear combination will always come in the format of:
        \[\begin{bmatrix} 2 \alpha - \beta \\ 0 \\ 2\beta - \alpha \end{bmatrix}\]
        Let an arbitrary vector in the span be $\vec{x}$, then:
        \[
            \begin{bmatrix}
                2 & -1 \\
                0 & 0 \\
                -1 & 2
            \end{bmatrix}
            \begin{bmatrix} \alpha \\ \beta \end{bmatrix}
            = \vec{x}
        \]
        Performing some Gaussian Elimination:
        \begin{align*}
            \begin{bmatrix}
                1 & 0 \\
                0 & 0 \\
                0 & 1
            \end{bmatrix}
            \begin{bmatrix} \alpha \\ \beta \end{bmatrix}
            = \begin{bmatrix} 0.9x_1 - 0.2x_3 \\ x_2 \\ 0.2x_1 + 0.4x_3 \end{bmatrix}
        \end{align*}
        Remember that the values $x_1, x_2, x_3$ are known, which means we can find a solution for the coefficients $\alpha$ and $\beta$ as long as the second row of the system, which is prone to contradiction, does not cause a contradiction. That means in every vector of this span, $x_2 = 0$.
        Therefore, the range of all reachable vectors would be any vector with a zeroed second component. This means the span is the xz-plane.

    }
    
    \item {
         Describe the span of 
        $
            \bigg\{
                \begin{bmatrix} 2 \\ 0 \\ -1 \end{bmatrix} - \begin{bmatrix} 1 \\ 0 \\ 2 \end{bmatrix},
                \begin{bmatrix} 1 \\ 0 \\ 2 \end{bmatrix}
            \bigg\}
        $
        and its relation with $\begin{bmatrix} 0 \\ 3 \\ 0 \end{bmatrix}$.
        
    }
    \meta {
        \begin{itemize}
            \item Graph the span, as the starting meta suggests, and graph the $y$ vector outside the span for a graphical demonstration.
        \end{itemize}
    }
    \ans {
        Let us look at this in a faster approach by utilizing our solution from the previous part. \\
        Let us express:
        \[
            \vec{v_1} = \begin{bmatrix} 2 \\ 0 \\ -1 \end{bmatrix},\ 
            \vec{v_2} = \begin{bmatrix} 1 \\ 0 \\ 2 \end{bmatrix}
        \]
        Then this time we are calculating, with reference to the previous part, $span(\vec{v_1} - \vec{v_2}, \vec{v_1})$. \\
        Via a proof shown on Homework 2, Question 7 of Fall 2022 (if the proof is not shown in the semester of this question, that proof will be attached):
        \[span(\vec{v_1} - \vec{v_2}, \vec{v_2}) = span(\vec{v_1}, \vec{v_2})\]
        Therefore, the description of this span would be the same as from Q2. \\
        Finally, since the vector $\begin{bmatrix} 0 \\ 3 \\ 0 \end{bmatrix}$ does not exist in the xz-plane, it is out of the span.
        
    }
    
    \item {
        Find the span of
        $
            \bigg\{
                \begin{bmatrix} 2 \\ 0 \\ -1 \end{bmatrix},
                \begin{bmatrix} 0 \\ 3 \\ 1 \end{bmatrix}
            \bigg\}
        $
        
    }
    \ans {
        Let the linear combinations of this set of vector be express-able as:
        \[\alpha \begin{bmatrix} 2 \\ 0 \\ -1 \end{bmatrix} + \beta \begin{bmatrix} 0 \\ 3 \\ 1 \end{bmatrix}\]
        for real values $\alpha$ and $\beta$. \\
        In that case, the resultant linear combination will always come in the format of:
        \[\begin{bmatrix} 2 \alpha \\ 3 \beta \\ \beta - \alpha \end{bmatrix}\]
        Let an arbitrary vector in the span be $\vec{x}$, then:
        \[
            \begin{bmatrix}
                2 & 0 \\
                0 & 3 \\
                -1 & 1
            \end{bmatrix}
            \begin{bmatrix} \alpha \\ \beta \end{bmatrix}
            = \vec{x}
        \]
        Performing some Gaussian Elimination:
        \[
            \begin{bmatrix}
                1 & 0 \\
                0 & 1 \\
                0 & 0
            \end{bmatrix}
            \begin{bmatrix} \alpha \\ \beta \end{bmatrix}
            = \begin{bmatrix} \frac{x_1}{2} \\ \frac{x_2}{3} \\ x_3 + \frac{x_1}{2} - \frac{x_2}{3} \end{bmatrix}
        \]
        Remember that the values $x_1, x_2, x_3$ are known, which means we can find a solution for the coefficients $\alpha$ and $\beta$ as long as the third row of the system, which is prone to contradiction, does not cause a contradiction. That means in every vector of this span, $x_3 = 0$. \\
        In other words: $x_3 = \frac{x_2}{3} - \frac{x_1}{2}$. \\
        Therefore, the set of all such linear combinations, which would also be the span, is:
        \[
            span(\vec{v_1}, \vec{v_2}, \vec{v_3}) = 
            \bigg\{
                x_1 \begin{bmatrix} \frac{1}{2} \\ 0 \\ -\frac{1}{2} \end{bmatrix} +
                x_2 \begin{bmatrix} 0 \\ \frac{1}{3} \\ \frac{1}{3} \end{bmatrix} 
                : x_1, x_2 \in \R
            \bigg\}
        \]
    
    }
    
    \item {
        Find the span of
        $
            \bigg\{
                \begin{bmatrix} 1 \\ 3 \end{bmatrix},
                \begin{bmatrix} 1420 \\ 2840 \end{bmatrix},
                \begin{bmatrix} \sin (1) \\ \cos (1) \end{bmatrix}
            \bigg\}
        $
        and
        $
            \bigg\{
                \begin{bmatrix} 0 \\ 0 \\ -1 \end{bmatrix},
                \begin{bmatrix} 0 \\ 3 \\ 0 \end{bmatrix},
                \begin{bmatrix} 1 \\ 0 \\ 0 \end{bmatrix},
                \begin{bmatrix} 2 \\ 1 \\ 2 \end{bmatrix}
            \bigg\}
        $
        without using Gaussian Elimination.
        
    }
    \ans {
        For the first set of two-dimensional vectors, the first two vectors are linearly independent, already spanning $R^2$. \\
        Since the vectors of this set must all be two-dimensional, the largest range, or span, of these vectors must be $R^2$. Therefore,
        \[
            span \bigg(
                \begin{bmatrix} 1 \\ 3 \end{bmatrix},
                \begin{bmatrix} 1420 \\ 2840 \end{bmatrix},
                \begin{bmatrix} \sin(1) \\ \cos(1) \end{bmatrix}
            \bigg) = \R^2
        \]
        For the second set of three-dimensional vectors, the first three vectors are also linearly independent, already spanning $R^3$. \\
        Since the vectors of this set must all be three-dimensional, the largest range, or span, of these vectors must be $R^3$. Therefore,
        \[
            span \bigg(
                \begin{bmatrix} 0 \\ 0 \\ -1 \end{bmatrix},
                \begin{bmatrix} 0 \\ 3 \\ 0 \end{bmatrix},
                \begin{bmatrix} 1 \\ 0 \\ 0 \end{bmatrix},
                \begin{bmatrix} 2 \\ 1 \\ 2 \end{bmatrix}
            \bigg) = \R^3
        \]
        Notably, $\begin{bmatrix} 2 \\ 1 \\ 2 \end{bmatrix}$ already exists in the span of the first three vectors for this set.
        
    }
\end{enumerate}