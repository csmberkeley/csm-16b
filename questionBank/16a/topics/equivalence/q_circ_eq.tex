% \documentclass{exam}
%\usepackage[utf8]{inputenc}
%\usepackage[english]{babel}
%\usepackage[american,siunitx]{circuitikz}
%\usepackage{siunitx}

\qns{Parallel or Series Resistors}
%\author{Catherine Bumagat and Shaamer Kumar}
%\date{February 14, 2022}
% CSM16A Spring 2022

\textbf{Learning Goal: }This question will help students recognize which resistors are in series, parallel, or neither when given a circuit. They will also practice condensing resistors in equivalent networks. 

\meta{
Remind students the definition of parallel and series equivalence. 
\\Two resistors are in parallel if, and only if, they share the same two nodes on top and bottom, ie: they share the same voltage drop. Two resistors are in series if, and only if, they share one node in the same branch, ie: they share the same current through both resistors
}


%\maketitle
    \begin{center}
    
        \begin{circuitikz}
        \draw(0,0)
    	to[V_=$V$,invert] ++(0,5)
    	to[short,l=Node A,-o] ++(2,0)
     	to[short] ++(1,0)
     	to[R,l=$R_3$] ++(1,0)
     	to[short] ++(2,0)
     	to[R,l=$R_5$] ++(1,0)
     	to[short] ++(2,0)
     	to[R,l=$R_9$] ++(1,0)
     	to[short] ++(1,0)
     	to[short] ++(0,-5)
     	to[short] node[]{} ++(-3,0)
     	(8,5)to[short] ++(0,-1)
     	to[R,l=$R_6$] ++(0,-1)
     	to[short] ++(0,-1)
     	to[R,l=$R_7$] ++(0,-1)
     	to[short] ++(0,-1)
     	to[short] ++(-1,0)
     	to[R,l=$R_8$] ++(-1,0)
     	to[short] ++(-1,0)
     	(5,5)to[short] ++(0,-2)
     	to[R,l=$R_4$] ++(0,-1)
     	to[short] ++(0,-2)
     	to[short, -o] node[]{} ++(-3,0)
    	(2,5)to[short] ++(0,-1)
    	to[R,l=$R_1$] ++(0,-1)
    	to[short] ++(0,-1)
    	to[R,l=$R_2$] ++(0,-1)
    	to[short] ++(0,-1)
    	to[short, l=Node B] node[]{} ++(-2,0);
        \end{circuitikz}
        
    \end{center}
    
\begin{enumerate}    
    
    \item Which resistors in the circuit above are in series? Pick all that apply.
        \begin{enumerate}
            \item $R_1$, $R_2$
            \item $R_5$, $R_6$, $R_7$, $R_8$
            \item $R_6$, $R_7$
            \item $R_5$, $R_9$
            \item $R_5$, $R_8$
        \end{enumerate}
        
        \ans{
            \begin{oneparcheckboxes}
            \CorrectChoice $R_1$, $R_2$ 
            \CorrectChoice $R_6$, $R_7$
            \CorrectChoice $R_5$, $R_8$
            \end{oneparcheckboxes}
        }
        
    \item Which resistors in the circuit above are in parallel?  Pick all that apply. \\
        \begin{oneparcheckboxes}
         \choice $R_5$, $R_8$ 
         \CorrectChoice $R_6$ \& $R_7$, $R_9$
         \choice $R_4$, $R_5$
         \choice $R_1$, $R_2$
        \end{oneparcheckboxes} \\\\
        
        \ans{
        \begin{oneparcheckboxes}
        \CorrectChoice $R_6$ \& $R_7$, $R_9$
        \CorrectChoice $R_4$, $R_5$
        \end{oneparcheckboxes}
        }
    
    \item Find an expression for the equivalent resistance between Node A and Node B. Hint: Try and start condensing the circuit using parallel/series equivalences from the farthest point away from Node A and Node B.
        \ans{
        
        $(R_1 + R_2) \parallel [R_3 + (R_4 \parallel [R_5 + R_8 + (R_6 + R_7) \parallel R_9])]$
        }
    
    \item If our power source has a voltage difference of 5V, and $R_1$, $R_2$, $R_3$, $R_4$ are 20$\Omega$ each and the rest are 50$\Omega$, what is the current flowing through the voltage source? [Hint: use the results of your previous answer]
    
        \ans{
        
        From previous part: \\$(R_1 + R_2) \parallel [R_3 + (R_4 \parallel [R_5 + R_8 + (R_6 + R_7) \parallel R_9])]$
   \\ = 40$\Omega $ $\parallel$ [20$\Omega $ (20$\Omega $ $\parallel$ 200$\Omega $ $\parallel$ 50$\Omega $)]
   \\ = 40$\Omega $ $\parallel$ 266.666667$\Omega$
   \\ = 34.7826 $\Omega $
   \\
   \\ V = I * R
   \\ 5V = I * 34.7826 $\Omega $
   \\ I = 0.14375 A
        }
\end{enumerate}


