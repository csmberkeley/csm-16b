% Author: Anna Chou
% Email: menghuichou@berkeley.edu
% sp22 csm
% p181

\qns{Norton's Theorem}

\textbf{Learning Goal: }Practice how to use Norton's Theorem to solve complex circuits.

\meta{Make sure students know the steps and terms such as load resistance, Norton's current, and Thevenin equivalent resistance.}

Use Norton's Theorem to find $V_{AB}$.

\begin{center}
    \begin{circuitikz}
        \draw (0,0)
        to[R, l= $1\Omega$] ++(0,3)
        to[short] ++(1,0)
        to[R, l= $2\Omega$] ++(1,0)
        to[short] ++(2,0)
        to[V_=$3V$, invert] ++(1,0)
        to[short] ++(1,0)
        (6,3)to[short] ++(0,-1)
        to[R, l=$6\Omega$] ++(0,-1)
        to[short] ++(0,-1)
        to[short] ++(-6,0)
        (3,3)to[short] ++(0,-1)
        to[I, invert, l=2A] ++(0,-1)
        to[short] ++(0,-1)
        (6,3)to[short, -o] ++(3,0)
        (6,0)to[short, -o] ++(3,0)
        (9,3)to[open,v^>=$V_{AB}$] ++(0,-3)
        (9,3) node[label={[font=\footnotesize]above:A}] {}
        (9,0) node[label={[font=\footnotesize]below:B}] {}
    \end{circuitikz}
\end{center}

We will walk through the problem step by step to find out the answer:)\\\\

\begin{enumerate}
    \item Find Thevenin equivalent resistance to the circuit.
        
        \ans{
        
        Step 1: Find load resistor and remove it from the circuit. The resistor $6\Omega$ between node A and B is the load resistor. We save it for the later use, so now our circuit becomes this
        \begin{center}
            \begin{circuitikz}
                \draw (0,0)
                to[R, l= $1\Omega$] ++(0,3)
                to[short] ++(1,0)
                to[R, l= $2\Omega$] ++(1,0)
                to[short] ++(2,0)
                to[V_=$3V$, invert] ++(1,0)
                to[short] ++(1,0)
                (6,3)to[open] ++(0,-3)
                to[short] ++(-6,0)
                (3,3)to[short] ++(0,-1)
                to[I, invert, l=2A] ++(0,-1)
                to[short] ++(0,-1)
                (6,3)to[short, -o] ++(3,0)
                (6,0)to[short, -o] ++(3,0)
                (9,3)to[open,v^>=$V_{AB}$] ++(0,-3)
                (9,3) node[label={[font=\footnotesize]above:A}] {}
                (9,0) node[label={[font=\footnotesize]below:B}] {}
            \end{circuitikz}
        \end{center}
        
        Step 2: Turn off all independent sources (short voltage source and open current source) and find Thevenin equivalent resistance from looking into node A and B. 
        \begin{center}
            \begin{circuitikz}
                \draw (0,0)
                to[R, l= $1\Omega$] ++(0,3)
                to[short] ++(1,0)
                to[R, l= $2\Omega$] ++(1,0)
                to[short] ++(2,0)
                to[short] ++(1,0)
                to[short] ++(1,0)
                (6,3)to[open] ++(0,-3)
                to[short] ++(-6,0)
                (6,3)to[short, -o] ++(3,0)
                (6,0)to[short, -o] ++(3,0)
                (9,3)to[open,v^>=$V_{AB}$] ++(0,-3)
                (9,3) node[label={[font=\footnotesize]above:A}] {}
                (9,0) node[label={[font=\footnotesize]below:B}] {}
            \end{circuitikz}
        \end{center}
        
        Thevenin resistance = $1\Omega$ + $2\Omega$ = $3\Omega$.
        }
    \item Find Norton's current.
        
        \ans{
        
        To find Norton current, we short Node A and B and find $i_{sc}$ flowing from node A to B.
        \begin{center}
            \begin{circuitikz}
                \draw (0,0)
                to[R, l= $1\Omega$] ++(0,3)
                to[short] ++(1,0)
                to[R, l= $2\Omega$] ++(1,0)
                to[short] ++(2,0)
                to[V_=$3V$, invert] ++(1,0)
                to[short] ++(1,0)
                (6,3)to[short,*-] ++(0,-1)
                to[short, i_=$i_{sc}$] ++(0,-1)
                to[short] ++(0,-1)
                to[short,*-] ++(-6,0)
                (3,3)to[short] ++(0,-1)
                to[I, invert, l=2A] ++(0,-1)
                to[short] ++(0,-1)
                (6,3) node[label={[font=\footnotesize]above:A}] {}
                (6,0) node[label={[font=\footnotesize]below:B}] {}
            \end{circuitikz}
        \end{center}
        
        
        Let's use superposition to find $i_{sc}$ out.\\
        Step 1: Only turn on voltage source 3V. Find $i_{sc}^'$.
        \begin{center}
            \begin{circuitikz}
                \draw (0,0)
                to[R, l= $1\Omega$] ++(0,3)
                to[short] ++(1,0)
                to[R, l= $2\Omega$] ++(1,0)
                to[short] ++(2,0)
                to[V_=$3V$, invert] ++(1,0)
                to[short] ++(1,0)
                (6,3)to[short,*-] ++(0,-1)
                to[short, i_=$i_{sc}^'$] ++(0,-1)
                to[short] ++(0,-1)
                to[short,*-] ++(-6,0)
                % (3,3)to[short] ++(0,-1)
                % to[I, invert, l=2A] ++(0,-1)
                % to[short] ++(0,-1)
                (6,3) node[label={[font=\footnotesize]above:A}] {}
                (6,0) node[label={[font=\footnotesize]below:B}] {}
            \end{circuitikz}
        \end{center}
        
        $i_{sc}^'$ = 3V / ($2\Omega$ + $1\Omega$) = 1A.\\
        
        
        Step 2: Only turn on current source 2A. Find $i_{sc}^"$.
        \begin{center}
            \begin{circuitikz}
                \draw (0,0)
                to[R, l= $1\Omega$] ++(0,3)
                to[short] ++(1,0)
                to[R, l= $2\Omega$] ++(1,0)
                to[short] ++(2,0)
                to[short] ++(1,0)
                to[short] ++(1,0)
                (6,3)to[short,*-] ++(0,-1)
                to[short, i_=$i_{sc}^"$] ++(0,-1)
                to[short] ++(0,-1)
                to[short,*-] ++(-6,0)
                (3,3)to[short] ++(0,-1)
                to[I, invert, l=2A] ++(0,-1)
                to[short] ++(0,-1)
                (6,3) node[label={[font=\footnotesize]above:A}] {}
                (6,0) node[label={[font=\footnotesize]below:B}] {}
            \end{circuitikz}
        \end{center}
        
        $i_{sc}^"$ = 2A.\\
        
        
        Step 3: Sum up all sub-currents to get $i_{sc}$.
        
        $i_{sc}$ = $i_{sc}^'$ + $i_{sc}^"$ = 1 + 2 = 3A.
        }
    \item Assemble what you know from previous parts, construct the Norton equivalent circuit and label all important circuit elements. Don't forget the load resistor!
        
        \ans{
        
        part(a): Thevenin equivalent resistance is $3\Omega$ and load resistance is $6\Omega$. \\
        part(b): Norton's current is 3A.\\
        
        we can assemble Norton's equivalent circuit by putting everything in parallel.
        
         \begin{center}
            \begin{circuitikz}
                \draw (0,0)
                to[I, l=3A] ++(0,3)
                to[short] ++(3,0)
                to[short] ++(3,0)
                (3,3)to[short] ++(0,-1)
                to[R, l=$3\Omega$] ++(0,-1)
                to[short] ++(0,-1)
                (6,3)to[short] ++(0,-1)
                to[R, l=$6\Omega$] ++(0,-1)
                to[short] ++(0,-1)
                (0,0)to[short] ++(6,0)
                (6,3)to[short, -o] ++(3,0)
                (6,0)to[short, -o] ++(3,0)
                (9,3)to[open,v^>=$V_{AB}$] ++(0,-3)
                (9,3) node[label={[font=\footnotesize]above:A}] {}
                (9,0) node[label={[font=\footnotesize]below:B}] {}
            \end{circuitikz}
        \end{center}
        
        
        }
    
    
    \item Find $V_{AB}$ based on the Norton's circuit you just built.
        
        \ans{
        
         Since resistors are all in parallel, we can find an equivalent resistance of the circuit first and use Ohm's law to find $V_{AB}$.
         
         \begin{center}
            \begin{circuitikz}
                \draw (0,0)
                to[I, l=3A] ++(0,3)
                to[short] ++(3,0)
                (3,3)to[short] ++(0,-1)
                to[R, l=$3\Omega$ || $6\Omega$ \text{=} $2\Omega$] ++(0,-1)
                to[short] ++(0,-1)
                (0,0)to[short] ++(3,0)
                (3,3)to[short, -o] ++(3,0)
                (3,0)to[short, -o] ++(3,0)
                (6,3)to[open,v^>=$V_{AB}$] ++(0,-3)
                (6,3) node[label={[font=\footnotesize]above:A}] {}
                (6,0) node[label={[font=\footnotesize]below:B}] {}
            \end{circuitikz}
        \end{center}
         
        $V_{AB}$ = 3A * $2\Omega$ = 6V.
        }
\end{enumerate}

