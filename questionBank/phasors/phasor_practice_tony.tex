% Author: Tony Li
% Email: songli@berkeley.edu

\qns{Phasor Analysis Practice}

We have learnt how to transform a sinusoidal voltage or current input in time domain to the phasor domain, and by now you are also capable
of finding the impedances of resistors, inductors, and capacitors. Please use the following tables as reference. 
\begin{center} \begin{tabular}{|c|c|c|}
\hline
        & Time Domain                         & Phasor Domain \\ \hline
Voltage & $v(t) = V_0 \cos(\omega t + \phi_v)$ & $\widetilde{V} = V_0 e^{j\phi_v}$ \\ 
Current & $i(t) = I_0 \cos(\omega t + \phi_i)$ & $\widetilde{I} = I_0 e^{j\phi_i}$ \\
\hline
\end{tabular} \end{center}

The following table contains the impedances of normal circuit components. $C$, $R$, and $L$ are the capacitance, resistance, and inductance we have 
already known. $\omega$ is the frequency in the phasor domain, and $j$ is the imaginary number $\sqrt{-1}$.  
\begin{center} \begin{tabular}{|c|c|}
    \hline
            & Impedance   \\ \hline
    Resistor & $Z_R=R$    \\
    Capacitor & $Z_C=\frac{1}{j{\omega}C}$ \\
    Inductor & $Z_I=j{\omega}L$\\
    \hline
    \end{tabular} \end{center}

You can use the table provided below to transform one form of the given complex number to another form. In the polar form and Euler's form,
the length $r$ and angle $\theta$ are the same. $r$ in the phasor domain is equivalent to the amplitude.

\begin{table}[ht]\begin{center}\begin{adjustbox}{width=1\textwidth} \begin{tabular}{|c|c|c|c|}
        \hline
            Given Form                                & General Form  & Polar Form & Euler's Form \\ \hline
        $z=r(\cos(\theta)+j\sin(\theta))$& $a=r\cos(\theta)$ and $b=r\sin(\theta)$            & ...   & $r=r$ and $\theta=\theta$ \\ \hline
        $z=re^{j\theta}$                 & $a=r\cos(\theta)$ and $b=r\sin(\theta)$            & $r=r$ and $\theta=\theta$            & ...\\ \hline
        $z=a+bj$                         & ...      & $r=\sqrt{a^2+b^2}$  & $r=\sqrt{a^2+b^2}$ \\
                                         &          & and $\theta={\tan}^{-1}(\frac{b}{a})$ &  and $\theta={\tan}^{-1}(\frac{b}{a})$\\
        \hline
        \end{tabular} \end{adjustbox}\end{center}\end{table}

The reason why we need phasor analysis is that we can employ the full power of the complex numbers. It is challenging to 
directly obtain the sinusoidal voltages accross the components from the sinusoidal voltage inputs, because the calculation on differatiating
multiple $\sin$'s and $\cos$'s are not easy. Euler's form or the exponentials are useful because of their nice derivatives. In the time domain,
we need to solve for a system of differential equations because both voltage and current are changing through time. However, in phasor domain,
we only need to solve for a system of linear equations. 

We'll walk through the phasor analysis procedure in the circuit with a sinusoidal voltage source below:

	\begin{center}
		\begin{circuitikz}
			\draw (0,3)
			to[vsourcesin=$ $, l_=$V_s$] (0,0)
			(0,3) -- (2,3)
			to[R = $R$, v=$V_R(t)$] (4,3)
			to[short,i>= \mbox{$i(t)$}] (6,3)
			to[C = $C$, v=$V_C(t)$] (6,0)
                        to[L = $L$, v=$V_L(t)$] (0,0)
			to[short] (0,0);
		\end{circuitikz}
	\end{center}

The voltage source is given by
\begin{align}
V_s(t) = 8 \sin(\omega t + \frac{\pi}{4}),
\end{align}
with $\omega = 10^3$ rad/s, $R = 1$ $\text{k}\Omega$, $C = 1$ $\mu\text{F}$, and $L = 1-\frac{\sqrt{3}}{3}$ $\text{H}$.

We need to find the solutions for $V_C(t)$, $V_L(t)$, and $i(t)$ with the sinusoidal voltage source $V_s(t)$.

\begin{enumerate}

\qitem
\textbf{Step 1: Write independent sources(inputs) as exponentials.}

First try to convert all $\sin$'s into $\cos$'s, and then find out the phase.
\textbf{Convert $V_s(t)$ into the Euler's Form and write down its phasor representation $\widetilde{V}_s$}. \textit{Hint:} $\cos(\alpha)=\sin(\alpha+\frac{\pi}{2})$

\ws{\vspace{100px}}

\sol{
$$ V_s(t) = 8 \cos(\omega t - \frac{\pi}{4}) = \operatorname{Re}(8e^{j (-\frac{\pi}{4})} e^{j \omega t}) $$
The phase is $-\frac{\pi}{4}$, $\implies \widetilde{V}_s = 8e^{j (-\frac{\pi}{4})} $
}

\qitem \textbf{Step 2: Transform circuits elements to phasor domain impedances.}

The voltage source is represented by its phasor $\widetilde{V}_s$.
The current $i(t)$ is related to its phasor counterpart $\widetilde{I}$. 

\textbf{What are the phasor representations of $R$, $C$, and $L$?}

\ws{\vspace{80px}}

\sol{
$ Z_R = R$, $Z_C = \frac{1}{j\omega C}$, $Z_L = j{\omega}L$
}

\qitem \textbf{Step 3: Apply all circuit laws you know to the circuit and establish a system of linear equations.}

All of the circuit laws we used in the time domain can be applied in the phasor domain.
$$\text{Kirchhoff's Current Law:} \sum_{el}\widetilde{I}_{el}=0$$
$$\text{Nodal Voltage Analysis:} \sum\frac{\widetilde{V}_j-\widetilde{V}_k}{z_{jk}}=0$$
$$\text{Ohm's Law:} \widetilde{V}_{el}=\widetilde{I}_{el}z_{el}$$

In the phasor domain, we can even treat all resistors, capacitors, and inductors as loads shown below, and
we do not even care about the inductance and capacitance in the phasor domain. 

\begin{center}
        \begin{circuitikz}
                \ctikzset{resistor = european}
                \draw (0,3)
                to[vsourcesin=$ $, l_=$V_s$] (0,0)
                (0,3) -- (2,3)
                to[R = $z_R$, v=$\widetilde{V}_R$] (4,3)
                to[short,i>= \mbox{$\widetilde{I}$}] (6,3)
                to[R = $z_C$, v=$\widetilde{V}_C$] (6,0)
                to[R = $z_L$, v=$\widetilde{V}_L$] (0,0)
                to[short] (0,0);
        \end{circuitikz}
\end{center}

\textbf{Now, find out $\widetilde{V}_s$ in terms of $\widetilde{I}$, $C$, $R$, $L$, $\omega$, and $j$}.

\ws{\vspace{120px}}

\sol{
By KCL, the current through the resistor is equal to the current through the capacitor.
$$i(t) = i_{R}(t) = i_{\text{C}}(t) \Rightarrow \widetilde{I} = \widetilde{I}_R = \widetilde{I}_C = \widetilde{I}_L$$
Sum of the voltages across the resistor, capacitor, and inductors is equal to that of the source.
$$V_{s}(t) = V_{C}(t) + V_{R}(t) + V_{L}(t) \Rightarrow \widetilde{V}_s = \widetilde{V}_{C} + \widetilde{V}_{R} + \widetilde{V}_L$$
Substituting the impedances in, we get:
$$\widetilde{V}_s = \widetilde{I}(\frac{1}{j \omega C} + R + j{\omega}L)$$
}

\qitem \textbf{Step 4: Solve for unknown variables}

Just plug the values we got above into the linear equations, and \textbf{find out $\widetilde{I}$ and $\widetilde{V}_L$}. The solutions could look
complicated, but the problem aims to help you sharpen your complex number calculation skills.

\ws{\vspace{150px}}

\meta{
Please show students how to transform one complex number into another form, and normally Euler's form is the easiest form to perform multiplication and division.
}
\sol{
$$\widetilde{I} = \frac{\widetilde{V}_s}{R + \frac{1}{j \omega C} + j{\omega}L} = \frac{8 e^{-j\frac{\pi}{4}}}{10^3 \Omega + \frac{1}{j(1000)10^{-6}\text{F}} + j(1000)(1-\frac{\sqrt{3}}{3})\text{H}}$$
$$= \frac{8 e^{-j\frac{\pi}{4}}}{1000-1000\frac{\sqrt{3}}{3}j}=\frac{8 e^{-j\frac{\pi}{4}}}{\frac{2000}{3}\sqrt{3} e ^{-j\frac{\pi}{6}}}=\frac{\sqrt{3}}{250}e^{-j\frac{\pi}{12}}$$

$$V_L=\widetilde{I}{\times}z_L=\frac{\sqrt{3}}{250}e^{-j\frac{\pi}{12}}{\times}j(1000)(1-\frac{\sqrt{3}}{3})=\frac{\sqrt{3}}{250}e^{-j\frac{\pi}{12}}{\times}1000{\times}(1-\frac{\sqrt{3}}{3})e^{j\frac{\pi}{2}}=(-4+4\sqrt{3})e^{j\frac{5}{12}\pi}$$
}

\qitem \textbf{Step 5: Transform solutions back to time domain}

This step is our final step, and we can just use the phasor/time domain table above. After finishing this question, you will find that
you don't even need to have multiple differential equations to cope with this sort of quesiton. Every component has the same frequency, and that's 
why we can use phasor analysis, because we only need focus on the phases and amplitudes. 

\textbf{What is $i(t)$ and $V_{L}(t)$}? 

\ws{\vspace{80px}}

\sol{
$$|i(t)|=|\widetilde{I}|=|\frac{\sqrt{3}}{250}e^{-j\frac{\pi}{12}}|=\frac{\sqrt{3}}{250}$$
$$\angle i(t)=\angle \widetilde{I} = -\frac{\pi}{12} \Rightarrow i(t)=|i(t)|\angle i(t)=\frac{\sqrt{3}}{250} \angle -\frac{\pi}{12}$$
This one way of representing $i(t)$, or $i(t)$ can be also written as $i(t)=\frac{\sqrt{3}}{250}cos(\omega t - \frac{\pi}{12})$

Similarly, we can write $V_{L}(t)$ as:
$$V_{L}(t)=(-4+4\sqrt{3})cos(\omega t+\frac{5}{12}\pi)$$
}

% \qitem \textbf{Optional:} This question is generated by chatGPT.
%
% A circuit consists of a 50 $\Omega$ resistor, a 100 mH inductor and a 10 $\mu$F capacitor in series.
% The voltage source has a peak voltage of 100 V and a frequency of 1 kHz.
% Calculate the total impedance of the circuit, the phase angle between the voltage and current, and the current through the circuit.
% \textit{Notice:} The unit of frequency is kHz, which is equivalent to $2\pi \times 1000$ $rad/s$
%
% \sol{
% The answer is also provided by AI. Unfortunately, AI's answer is wrong, and I have to manually change it.
%
% The impedance of the circuit is given by:
% $$Z = R + j\omega L + \frac{1}{j\omega C} = 50 + j(2\pi \times 1000 \times 0.1)$$
% $$ + \frac{1}{j(2\pi \times 1000 \times 10\times 10^{-6})} = 50 + j126.6 -j15.92 \approx 50 + j110.68 \Omega$$
%
% where R is the resistance, L is the inductance, C is the capacitance, $\omega$ is the angular frequency, and j is the imaginary unit.
%
% The phase angle between the voltage and current can be found using the impedance and the voltage:
%
% $$ \theta = \tan^{-1}\left(\frac{\text{Im}(Z)}{\text{Re}(Z)}\right) = \tan^{-1}\left(\frac{110.68}{50}\right) \approx 66.8^\circ $$
%
%
% The current through the circuit can be found using Ohm's law:
%
% $$ I = \frac{V_p}{|Z|} = \frac{100}{\sqrt{50^2 + 110.68^2}} \approx 0.655 \text{A}$$
%
% To represent the current through the circuit as a phasor, we can write:
%
% $$ I = I_p \angle \theta_i = 0.655 \angle -66.8^\circ$$
%
% where $\theta_i$ is the phase angle of the current.
% }
\end{enumerate}
