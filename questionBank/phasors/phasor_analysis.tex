% Author: Taejin Hwang
% Email: taejin@berkeley.edu

\qns{Phasor Analysis}

\meta{Prereqs: An understanding of what a phasor is, and what impedances are in a circuit.}

Equipped with the phasor representation, let's see how using it can greatly simplify alternating current (AC) circuit analysis, assuming that the voltages and currents vary with time as sinusoids. 

The phasor analysis method consists of five steps.
We'll walk through it in the RC circuit below:

	\begin{center}
		\begin{circuitikz}
			\draw (0,3)
			to[vsourcesin=$ $, l_=$V_s$] (0,0)
			(0,3) -- (2,3)
			to[R = $R$] (4,3)
			to[short,i>= \mbox{$i(t)$}] (6,3)
			to[C = $C$, v=$V_\text{C}(t)$] (6,0)
			to[short] (0,0);
		\end{circuitikz}
	\end{center}

The voltage source is given by
\begin{align}
V_s(t) = 8 \cos(\omega t - \frac{\pi}{4}),
\end{align}
with $\omega = 10^3$ rad/s, $R = 1$ $\text{k}\Omega$, and $C = 1$ $\mu\text{F}$.

Our goal is to obtain a solution for $V_{\text{C}}(t)$ and $i(t)$ with the sinusoidal voltage source $V_s(t)$.

\begin{enumerate}

\qitem \textbf{Step 1: Convert $v(t)$ as a sum of complex exponentials.}

All voltages and currents with known sinusoidal functions should be expressed in the standard exponential format.
\textbf{Convert $V_s(t)$ into an exponential and write down its phasor representation $\widetilde{V}_s$.}

\ws{\vspace{100px}}

\sol{
We can extract the phasor representation of $V_s(t)$ by writing it in the form $V_s(t) = \operatorname{Re}(\widetilde{V}_s e^{j \omega t})$:

\begin{align*}
V_s(t) &= 8 \cos(\omega t - \frac{\pi}{4}) \\
	   &= \operatorname{Re}(8e^{j (-\frac{\pi}{4})} e^{j \omega t}) \\
	\implies \widetilde{V}_s &= 8e^{j (-\frac{\pi}{4})}
\end{align*}

\textit{Alternatively,} we can expand $V_s(t)$ into the sum of complex exponentials and pattern matching to find the phasor.

The two formulas below will help us convert a sinusoid into a sum of exponentials:
\begin{gather*}
\cos(\theta) = \frac{1}{2} e^{j \theta} + \frac{1}{2} e^{-j \theta} \ \ \text{and} \ \ 
\sin(\theta) = \frac{1}{2j} e^{j \theta} - \frac{1}{2j} e^{-j \theta}
\end{gather*}

Using the fact that $\cos(\theta) = \frac{1}{2} e^{j \theta} + \frac{1}{2} e^{-j\theta},$
\begin{align*}
V_s(t) &= 8 \big( \frac{1}{2} e^{j(\omega t - \frac{\pi}{4})} + \frac{1}{2} e^{-j(\omega t - \frac{\pi}{4})} \big) = \frac{1}{2} (e^{j(\omega t - \frac{\pi}{4})} + 8e^{-j(\omega t - \frac{\pi}{4})}) \\
&= \frac{1}{2} (8 e^{-j \frac{\pi}{4}} e^{j \omega t} + 8 e^{j \frac{\pi}{4}} e^{-j \omega t})
\end{align*}

Thus, the phasor is given by
\begin{align*}
\widetilde{V}_s = 8 e^{-j\frac{\pi}{4}}
\end{align*}
}

% \ans{
% \begin{align}
% v_s(t) = 12 \cos (\omega t -\frac{\pi}{4} -\frac{\pi}{2}) = 12 \cos(\omega t - \frac{3\pi}{4})
% \end{align}

% The phasor is given by
% \begin{align}
% V_s = 12 e^{-j\frac{3\pi}{4}}
% \end{align}
% }

\qitem \textbf{Step 2: Transform circuits elements to phasor domain impedances.}

The voltage source is represented by its phasor $\widetilde{V}_s$.
The current $i(t)$ is related to its phasor counterpart $\widetilde{I}$. 
% by
% \begin{align}
% i(t) = \mathfrak{Re}[I e^{j\omega t}].
% \end{align}
\textbf{What are the phasor representations of $R$ and $C$?}

\meta{Students might find it helpful to actually redraw the circuit and labeling the source as a constant phasor input and the circuit elements transformed into black box impedances.}

\ws{\vspace{50px}}

\sol{
\begin{align}
Z_R &= R\\
Z_C &= \frac{1}{j\omega C}
\end{align}
}

% \ans{
% \begin{align}
% Z_R &= R\\
% Z_C &= \frac{1}{j\omega C}
% \end{align}
% }

\qitem \textbf{Step 3: Cast KCL and/or KVL equations in phasor domain.}

Use Kirchhoff's laws to write down a loop equation that relates all phasors in Step 2.

\ws{\vspace{100px}}

\sol{
By KCL, the current through the resistor is equal to the current through the capacitor.
$$i(t) = i_{R}(t) = i_{\text{C}}(t) \rightarrow \widetilde{I} = \widetilde{I}_R = \widetilde{I}_C $$
By KVL, the sum of the voltages across the resistor and capacitor is equal to that of the source.
$$V_{s}(t) = V_{\text{C}}(t) + V_{R}(t) \rightarrow \widetilde{V}_s = \widetilde{V}_{\text{c}} + \widetilde{V}_{R}$$
Substituting the impedances in, we get:
$$\widetilde{V}_s = \widetilde{I} \frac{1}{j \omega C} + \widetilde{I} R = \widetilde{I} (R + \frac{1}{j\omega C})$$
}

\qitem \textbf{Step 4: Solve for unknown variables}

Solve the equation you derived in Step 3 for $\widetilde{I}$ and $\widetilde{V_{\text{C}}}$.
What is the polar form of $\widetilde{I}$ ($Ae^{i\theta}$, where $A$ is a positive real number)? 

\ws{\vspace{150px}}

\sol{
We first solve for the current $\widetilde{I}:$
$$\widetilde{I} = \frac{\widetilde{V}_s}{R + \frac{1}{j \omega C}}$$
We plug in for $\omega = 10^3, R = 10^3 \Omega, C = 10^{-6} \mu$F, $\widetilde{V}_s = 8 e^{-j\frac{\pi}{4}}$:

$$\widetilde{I} = \frac{8 e^{-j \frac{\pi}{4}}}{10^3 + \frac{1}{j 10^{3} 10^{-6}}} = \frac{8 e^{-j \frac{\pi}{4}}}{10^3 - 10^3 j} = \frac{8 e^{-j \frac{\pi}{4}}}{10^{3} \sqrt{2} e^{-j \frac{\pi}{4}.}} = \frac{4 \sqrt{2}}{10^{3}} \mbox{ A} = 4 \sqrt{2} \mbox{ mA}$$

In the previous line, the third to the fourth equality comes from the plotting the denominator on the complex plane and converting to polar coordinates by taking the magnitude and the phase with respect to the positive real axis.

To solve for $\widetilde{V}_{\text{C}}$ we use the voltage-current relationship of a capacitor in phasor domain (as we derived in a previous problem): $\widetilde{V} = \widetilde{I} \frac{1}{j \omega C}$.
$$\widetilde{V}_{\text{C}} = \frac{1}{10^{-3} j} \frac{4 \sqrt{2}}{10^{3}} = 4 \sqrt{2} e^{-j (\frac{\pi}{2})} \mbox{ V}$$
}

\qitem \textbf{Step 5: Transform solutions back to time domain}

To return to time domain, we apply the fundamental relation between a sinusoidal function and its phasor counterpart.
What is $i(t)$ and $V_{\text{C}}(t)$? What is the phase difference between $i(t)$ and $V_{\text{C}}(t)$? 

\ws{\vspace{100px}}

\sol{
From the previous part, we now have the phasor representations of current and voltage of the capacitor: 
\begin{align*}
\widetilde{I} &= 4 \sqrt{2} \\
\widetilde{V}_C &= 4 \sqrt{2} e^{-j (\frac{\pi}{2})}
\end{align*}

Remember that phasors encode the magnitude and phase of a sinusoid, removing the time-dependence by assuming that we are operating with a sinusoid with the given fixed frequency $\omega$. Therefore, to convert back, we can just read off the magnitude and phase of the current and voltage phasors.

\begin{align*}
\widetilde{I} = \frac{4\sqrt{2}}{10^3} = \frac{4\sqrt{2}}{10^3} e^{j (0)} &\implies A = \frac{4\sqrt{2}}{10^3} \text{, } \phi = 0 \\
&\implies i(t) = A\cos(\omega t + \phi) = \frac{4\sqrt{2}}{10^3} \cos(\omega t + 0) = 4\sqrt{2} \cos(\omega t) \mbox{ mA}
\end{align*}

\begin{align*}
\widetilde{V}_C = 4 \sqrt{2} e^{-j (\frac{\pi}{2})} &\implies A = 4\sqrt{2} \text{, } \phi = -\frac{\pi}{2} \\
&\implies V_C(t) = A\cos(\omega t + \phi) = 4\sqrt{2} \cos(\omega t - \frac{\pi}{2}) = 4\sqrt{2} \cos(\omega t - \frac{\pi}{2}) \mbox{ V}
\end{align*}

The phase difference between the two, with respect to $i(t)$ is $-\frac{\pi}{2}$.
}

% \ans{
% \begin{align}
% i(t) = \mathfrak{Re}[Ie^{j\omega t}] = \mathfrak{Re} [6e^{-j\frac{7\pi}{12}} e^{j\omega t}] = 6 \cos (\omega t - \frac{7\pi}{12})  \mbox{ mA}
% \end{align}
% \begin{align}
% v_C(t)=\mathfrak{Re}[V_Ce^{j\omega t}]=\mathfrak{Re}[6e^{-j\frac{13\pi}{12}}e^{j\omega t}]=6 \cos(\omega t -\frac{13\pi}{12}) \mbox{ V}
% \end{align}
% The phase difference between the two, with respect to $i(t)$ is $-\frac{\pi}{2}$
% }

\end{enumerate}

\meta{
Phasor analysis marks the third time we've seen the problem-solving technique of turning a hard problem into a simpler one through a change of variables/coordinates/domain. First, we used a change of variables to solve a non-homogeneous differential equation with a constant input. Then, we used a change of coordinates to turn any system of differential equations into a diagonal system of first-order homogeneous differential equations that we could solve.

The flow of the general problem-solving process looks something like: (1) convert problem into some new domain where it's easier to solve, (2) solve the problem there, (3) convert back and retrieve the answers you care about.

It's important to keep in mind that the analysis above only applies to sinusoidal inputs. Don't make the mistake of doing phasor analysis on DC inputs!
}