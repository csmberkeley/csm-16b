\qns{Introduction to Second Order Differential Equations}
\qcontributor{Jessica Fan, Olivia Huang}

Take a RLC circuit in series. We will examine the behavior of this circuit with different R, L, and C values.

%modified version of rlc_circuit.tex

\begin{center}
    \begin{circuitikz}[scale=0.8]
        \draw (0,4) 
        to [V, l= $V_{dd}$] (0,0)
        (0,4)
        to [R = $R$,v=$V_R$] (4,4)
        to [L = $L$,v=$V_L$,i=$i(t)$] (8,4)
        to [short] (10,4)
        to [C = $C$,v=$V_C$] (10,0)	
        to [short] (0,0);
    \end{circuitikz}
\end{center}

\begin{enumerate}

\qitem Using your knowledge of I-V relationships of inductors and capacitors, write out a KVL equation that represents the behavior of this circuit.

\meta{
    Current is consistent in the circuit as all components are in series.
}

\ws{\vspace{125px}}

\sol{
    $$-V_{dd} + V_L  + V_R + V_C = 0$$
    $$V_L = L\frac{dI}{dt} = LC \frac{d^2V_C}{dt^2}$$
    $$C\frac{dV_C}{dt}$$
    $$V_R = RI = RC \frac{dV_C}{dt}$$
    $$LC \frac{d^2V_C}{dt^2} + RC \frac{dV_C}{dt} + V_C = V_{dd}$$
}

\qitem What is the steady state solution of $V_C(t)$?

\sol{
    At steady state, $\frac{dV_C}{dt} = 0$, so we can set $V_{dd} = V_C(\infty)$.
}

\qitem Solve for the solution of the homogenous differential equation.

\meta{
    See https://www.youtube.com/watch?v=NW9JfMvIsxw for explanation of the general form of the solution. Also shown in Fa23 Lecture slides.
}

\ws{\vspace{125px}}

\sol{
    We can guess an exponential solution using the characteristic polynomial. We replace $\frac{dV_C}{dt}$ as r:
    $$LCr^2 + RCr + 1 = 0$$
    Dividing the equation by $\frac{1}{LC}$, we get
    $$r^2 + \frac{R}{L}r + \frac{1}{LC} = 0$$
    Solving for r,
    $$r = \frac{-\frac{R}{L} \pm \sqrt{\frac{R}{L}^2 - \frac{4}{LC}}}{2}$$
    $$V_C(t) = C_0 e^{\frac{-\frac{R}{L} + \sqrt{\frac{R}{L}^2 - \frac{4}{LC}}}{2}t} + C_1 e^{\frac{-\frac{R}{L} - \sqrt{\frac{R}{L}^2 - \frac{4}{LC}}}{2}t}$$
    Where $C_0$ and $C_1$ are determined by the initial conditions of $V_C$.
}

\qitem Let's analyze our solution. Under what conditions will we 

\begin{enumerate}

\qitem Have two real exponent terms?

\sol{
    $$\sqrt{\frac{R}{L}^2 - \frac{4}{LC}} > 0$$
    Under these conditions, our circuit will experience \textbf{overdamping}, better written as $\sqrt{\frac{R}{2L}^2 - \frac{1}{LC}} > 0$.
}

\qitem Have only one exponent term?

\sol{
    $$\sqrt{\frac{R}{L}^2 - \frac{4}{LC}} = 0$$
   Under these conditions, our circuit will experience \textbf{critical damping}.
}

\qitem Have two complex exponent terms?

\sol{
    $$\sqrt{\frac{R}{L}^2 - \frac{4}{LC}} < 0$$
    Under these conditions, our circuit will experience \textbf{underdamping}.
}

\end{enumerate}
\end{enumerate}