\qns{Filters}
\qcontributor{Son Tran}

One very common use for AC circuits is as a \emph{filter}.
Basically, a \emph{filter} is a circuit device that blocks some parts of frequencies.
The key idea behind filters is that of \emph{superposition}.
Recall, from EE16A, that in circuits consisting solely of sources and resistors, we can analyze the circuits by considering each independent source apart from the rest, then the final signal simply being the sum of all the intermediate responses from each of the sources.
\\
\\
With filters, rather than looking at the superposition of independent sources, we will look at the superposition if input signals at the same point but with different frequencies.
Let
\[
  V(\omega, \widetilde{X}) = \widetilde{X}e^{j \omega t} + \overline{\widetilde{X}}e^{-j \omega t}
.\]

Basically, $V(\omega, \widetilde{X})$ converts the signal with angular frequency $\omega$ represented by the phasor $\widetilde{X}$ from the frequency domain to the time domain.
Specifically, if we supply an input signal of the form
\[
  V_{in}(t) = \sum_i V(\omega_i, \widetilde{X_i})
.\]

to a circuit with transfer function $H(\omega)$, the output voltage will be
\[
  V_{out}(t) = \sum_i H(\omega_i)V(\omega_i, \widetilde{X_i})
.\]

The practical consequence of the above assertion is that we can treat the superposition of signals of two different frequencies as if the two signals were separate.
For instance, if we supplied an AC circuit acting as a low-pass filter with the superposition of a 60 Hz signal and a 100 Hz noise signal, the high-frequency noise would be attenuated independently of the low-frequency signal.
\\
\\
Now, we will consider various filter configurations, and study their behavior.



\begin{enumerate}



% Part(a)
\qitem Giving the following filter

\begin{center}
  \begin{circuitikz} \draw
    (0, 0) node[ground] {}
      to [sV, l_=$V_{in}$] (0, 4)
      to [R = R] (4, 4)
      to [C = C] (4, 0)
      node[ground] {}

    (4, 3) to[short, -o] (6, 3) node[anchor=west] (A) {A}

    (4, 1) to[short, -o] (6, 1) node[anchor=west] (B) {B}

    (A) to[open, l^=$V_{out}$] (B)
  ;\end{circuitikz}
\end{center}


\begin{enumerate}
  \item Write out the transfer function $H(\omega)$.
  \item Find  $H(\omega = 0)$ and $|H(\omega = 0)|$.
  \item Find  $H(\omega = \infty)$ and $|H(\omega = \infty)|$.
\end{enumerate}

\sol{

\begin{align*}
Z_R &= R \\
Z_L &= j\omega L \\
Z_C &= \frac{1}{j\omega C}
\end{align*}

}



% Part(b)
\qitem Giving the following filter

\begin{center}
  \begin{circuitikz} \draw
    (0, 0) node[ground] {}
      to [sV, l_=$v_{in}$] (0, 2)
      to [C = C] (4, 2)
      to [R = R] (4, 0)
      node[ground] {}

    (4, 2) to[short, -o] (5, 2) node[anchor=west] (A) {A}

    (4, 0) to[short, -o] (5, 0) node[anchor=west] (B) {B}

    (A) to[open, l^=$v_{out}$] (B)
  ;\end{circuitikz}
\end{center}


\begin{enumerate}
  \item Write out the transfer function $H(\omega)$.
  \item Find  $H(\omega = 0)$ and $|H(\omega = 0)|$.
  \item Find  $H(\omega = \infty)$ and $|H(\omega = \infty)|$.
\end{enumerate}

\sol{

$\widetilde{V}_C$ is a voltage divider where the output voltage is taken across the capacitor.
\begin{align*}
\widetilde{V}_C&=\frac{Z_C}{Z_R+Z_L+Z_C}\widetilde{V}_{\text{s}}, \\
H_C(\omega)&=\frac{Z_C}{Z_R+Z_L+Z_C} = \frac{\frac{1}{j\omega C}}{R+j\omega L+\frac{1}{j\omega C}}.
\end{align*}
Multiplying the numerator and denominator by $j\omega C$ gives
\[H_C(\omega)=\frac{1}{(j\omega)^2LC+j\omega RC+1}\]

}



% Part(c)
\qitem Giving the following filter

\begin{center}
  \begin{circuitikz} \draw
    (0, 2) node[ground] (lground) {}
      to [sV, l_=$v_{in}$] (0, 4)
      to [R = $R_1$] (4, 4)
      to [C = $C_1$] (4, 2)
      node[ground] (mground) {}

    (7, 3.5) node[op amp, yscale=-1] (opamp) {}
      (opamp.+) to [short] (4, 4)
      (opamp.-) -| (5.5, 2)
      (opamp.out) |- (5.5, 2)
      (opamp.out) to [C = $C_2$] (12, 3.5)
      to [R = $R_2$] (12, 0.5)
      node[ground] (rground) {}

    (12, 3.5) to[short, -o] (14, 3.5) node[anchor=west] (A) {A}

    (12, 1) to[short, -o] (14, 1) node[anchor=west] (B) {B}

    (A) to[open, l^=$v_{out}$] (B)
  ;\end{circuitikz}
\end{center}


\begin{enumerate}
  \item Write out the transfer function $H(\omega)$.
  \item Find  $H(\omega = 0)$ and $|H(\omega = 0)|$.
  \item Find  $H(\omega = \infty)$ and $|H(\omega = \infty)|$.
\end{enumerate}

\sol{

$\widetilde{V}_L$ is a voltage divider where the output voltage is taken across the inductor.
\begin{align*}
\widetilde{V}_L&=\frac{Z_L}{Z_R+Z_L+Z_C}\widetilde{V}_{\text{s}}, \\
H_L(\omega)&=\frac{Z_L}{Z_R+Z_L+Z_C} = \frac{j\omega L}{R+j\omega L+\frac{1}{j\omega C}}.
\end{align*}
Multiplying the numerator and denominator by $j\omega C$ gives
\[H_L(\omega)=\frac{(j\omega)^2LC}{(j\omega)^2LC+j\omega RC+1}\]

}



\end{enumerate}

