%author Alexander Feng
% This question arises from a natural question: DFT for diagonalization?

\qns{DFT as a basis for Diagonalization}

Looking at a DFT matrix, notice how the DFT matrix's columns could form a very nice basis.
In fact, a natural question to ask is what types of matrices can be diagonalized by the DFT?
Let's start with a motivating matrix.

\begin{enumerate}

\qitem
What's the following matrix's eigenvectors?

{\em HINT: Don't use the determinant. Try guessing and noticing the special properties of the matrix.}
\begin{align*}
  A = 
\begin{bmatrix}
1 & 4 & 3 & 2 \\
2 & 1 & 4 & 3 \\
3 & 2 & 1 & 4 \\
4 & 3 & 2 & 1 \\
\end{bmatrix}
\end{align*}

\sol{
The most obvious guess is try a vector of only ones because each row adds up to 10.
This has eigenvalue 10.
Afterwards, we can leverage the cyclic nature of the rows, and find another eigenvector in
$\begin{bmatrix}
1\\
-1\\
1\\
-1\\
\end{bmatrix}$
This vector has eigenvalue -2.
Taking a very wild guess, well try using gram schmidt to figure out the next eigenvectors.
Perhaps we'll be lucky.
What you'll find is that the other two eigenvectors are in fact the other two basis vectors for a size 4 DFT matrix!
Thus, the eigenvectors (unnormalized) are:
\begin{align*}
  \{
\begin{bmatrix}
1\\
1\\
1\\
1\\
\end{bmatrix},
\begin{bmatrix}
1\\
\omega\\
\omega^{2}\\
\omega^{3}\\
\end{bmatrix},
\begin{bmatrix}
1\\
\omega^{3}\\
\omega^{6}\\
\omega^{9}\\
\end{bmatrix},
\begin{bmatrix}
1\\
\omega^{4}\\
\omega^{8}\\
\omega^{12}\\
\end{bmatrix}
\}
\end{align*}

Note that in gram schmidt, you would apply a normalization factor to each DFT basis vector.
}

\qitem
This may seem quite interesting.
Let's try generalizing the matrix from part (a).
We define a permutation matrix as a matrix who has exactly one entry of 1 in each row.
You can think of this as an identity matrix with all of its rows scrambled around.
Write

\end{enumerate}
