% {\Large \textbf{Mechanical:}}
\qns{Complex Numbers}

A complex number, $z$, is composed of a real part and imaginary part.
If $z = a + bj$, then $re(z) = a$ (the real portion equals a), and $im(z) = b$ (the imaginary portion equals b).
Complex numbers can be expressed in two ways:

\begin{center}
Rectangular Form: $z = a + bj$ \hspace{1em} Polar Form: $z = re^{j\theta}$
\end{center}

In polar form, $r$ represents the magnitude and $\theta$ represents the angle of the complex number with respect to the origin of the complex plane.
Rectangular form makes adding and subtracting complex numbers easier; whereas, polar form makes multiplying and dividing numbers easier.
Some handy equations to switch between forms include:

\begin{center}
\begin{tabular}{ c c c }
 $tan(\theta) = \frac{b}{a}$ & $r = |z| = \sqrt{a^2 + b^2}$ \\ \\
 $sin(\theta) = \frac{b}{|z|}$ & $cos(\theta) = \frac{a}{|z|}$ \\  \\
\end{tabular}
\end{center}

\begin{enumerate}

\qitem Prove algebraically that $\frac{1}{j} = -j$.

\sol{
The key is to multiply the left-hand side of the equation by $\frac{j}{j}$: \\
$$\frac{1}{j} = \frac{1 * j}{j * j} = \frac{j}{j^2}$$
$$= \frac{j}{-1} = -j$$
}

\end{enumerate}

A complex number, $z = a + bj$ has a complex conjugate, $\overline{z} = a - bj$.
Note that the sum of a complex number and its conjugate is always real, but the difference between a complex number and its conjugate is always imaginary.

\begin{enumerate}[resume]

\qitem Use a polar graph to show that the sum of any complex number and its conjugate is always real.

\sol{

}

\qitem Recall that Euler's Formula states that $e^{j\theta} = cos(\theta) + jsin(\theta)$.
Using Euler's identity, show that $cos(\theta) = \frac{1}{2}(e^{j\theta} + e^{-j\theta})$.

\sol{

  $$e^{j\theta} = cos(\theta) + jsin(\theta)$$

  Note that $e^{j\theta}$ has the complex conjugate $e^{-j\theta}$, which means:

  $$e^{-j\theta} = cos(\theta) - jsin(\theta)$$
  $$e^{j\theta} +  e^{-j\theta} = cos(\theta) + jsin(\theta) + cos(\theta) - jsin(theta)$$
  $$e^{j\theta} +  e^{-j\theta} = 2cos(\theta)$$
  $$cos(\theta) = \frac{1}{2}(e^{j\theta} +  e^{-j\theta})$$

}

\end{enumerate}
