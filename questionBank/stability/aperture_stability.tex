% Authors: Akash Velu, Shreyas Krisnaswamy
% Emails: akashvelu@berkeley.edu, shrekris@berkeley.edu

\qns{Aperture Stability}

As an intern at Aperture Laboratories, it is your job to make sure the robots being built are stable systems.
As a reminder, if the following conditions are met the system will be stable:

\begin{itemize}

\item For discrete time systems of the form:
\begin{equation}
\vec{x}(t+1) = A\vec{x}(t) + Bu(t) + \vec{w}(t)
\end{equation}
All eigenvalues of the matrix A, $\lambda_{i},$ have magnitude $|\lambda_{i}| < 1.$

\item For continuous time systems of the form:
\begin{equation}
\ddt{}{t} \vec{x}(t) = A\vec{x}(t) + Bu(t) + \vec{w}(t)
\end{equation}
All eigenvalues of the matrix A, $\lambda_{i},$ have real part $\mathfrak{Re}(\lambda_{i})< 0.$
\end{itemize}

\begin{enumerate}

\qitem According to your boss, the first robot, GLaDOS, can be described with the following discrete time system:
\begin{equation*}
    \vec{x}(t+1) =
    \begin{bmatrix}
    \frac{3}{8} & \frac{1}{8}\\
    \frac{1}{8} & \frac{3}{8}\\
    \end{bmatrix}
    \vec{x}(t) +
    \begin{bmatrix}
    1\\
    0\\
    \end{bmatrix}
    u(t)
\end{equation*}

Is she stable?

\sol {
% \begin{align*}
% A\vec{v} &= \lambda\vec{v} \\
% A\vec{v} - \lambda\vec{v} &= 0 \\
% (A - \lambda{}I)\vec{v} &= 0 \\
% det(A - \lambda{}I) &= 0
% \end{align*}

\begin{align*}
\text{det} (A - \lambda{}I) &=
\text{det} \Big(
\begin{bmatrix}
\frac{3}{8} - \lambda{} & \frac{1}{8}\\
\frac{1}{8} & \frac{3}{8} - \lambda{}\\
\end{bmatrix} \Big) = \big(\frac{3}{8} - \lambda{}\big)\big(\frac{3}{8} - \lambda{}\big) - \big(\frac{1}{8}\big)\big(\frac{1}{8}\big) \\
&= \lambda{}^2 - \frac{3}{4}\lambda{} + \frac{1}{8} = \big(\lambda{} - \frac{1}{2}\big)\big(\lambda{} - \frac{1}{4}\big) = 0\\
\end{align*}
Therefore, we see that 
$$
\lambda{} = \frac{1}{2}, \frac{1}{4}$$

Since the system is a discrete time system, and both eigenvalues have magnitude smaller than 1, GLaDOS is stable.
}

\qitem Your boss now gives you data on the P-body robot. Is she stable?
Her motion is described with the following discrete time system:
\begin{equation*}
    \vec{x}(t + 1) =
    \begin{bmatrix}
    -2 & -1\\
    1 & -2\\
    \end{bmatrix}
    \vec{x}(t) +
    \begin{bmatrix}
    1\\
    1\\
    \end{bmatrix}
    u(t)
\end{equation*}

\sol {
  \begin{align*}
    \text{det}(A - \lambda{}I) &= \Big(
    \begin{bmatrix}
    -2 - \lambda{} & -1\\
    1 & -2 - \lambda{}\\
    \end{bmatrix} \Big) = (-2 - \lambda{})(-2 - \lambda{}) - (-1)(1) \\
    &= \lambda{}^2 + 4\lambda{} + 5 = (\lambda{} - (-2 + j))(\lambda{} - (-2 - j)) = 0 \\
  \end{align*}
  Therefore we can compute the eigenvalues as
  $$\lambda{} = -2 \pm j$$
  However, the magnitude of both of these eigenvalues are $|\lambda| = \sqrt{5} \geq 1.$ Therefore, Atlas is unstable.
}

\qitem
Now your boss gives you data on a more advanced robot, Atlas.
Is he stable?
His movements can be described with the following continuous time system:
\begin{equation*}
    \frac{d}{dt}\vec{x}(t) =
    \begin{bmatrix}
    -2 & -1\\
    1 & -2\\
    \end{bmatrix}
    \vec{x}(t) +
    \begin{bmatrix}
    1\\
    1\\
    \end{bmatrix}
    u(t)
\end{equation*}

\sol{
  This is the exact same system as the previous system but in continuous time. The eigenvalues will be the same:
  $$\lambda{} = -2 \pm j$$
  This time, since this is a continuous time system, the conditions for stability have changed. Since $\mathfrak{Re}(\lambda) < 0$ for both eigenvalues, we conclude by saying that this system is stable.
}

\qitem Lastly, your boss gives you data on the Wheatley robot. Is he stable?
His motion is described with the following discrete time system:
\begin{equation*}
    \vec{x}(t+1) =
    \begin{bmatrix}
    \frac{\sqrt{3}}{2} & -\frac{1}{2}\\
    \frac{1}{2} & \frac{\sqrt{3}}{2}\\
    \end{bmatrix}
    \vec{x}(t) +
    \begin{bmatrix}
    0\\
    0\\
    \end{bmatrix}
    u(t)
\end{equation*}

\sol {
  You can also note that A is the rotation matrix, and make the observation that the eigenvalues are on the unit circle.
  Without even doing any calculations, we know Wheatley is unstable since $|\lambda| = 1.$

  This can also be realized by computing the eigenvalues in the following manner:
  \begin{align*}
    \text{det}(A - \lambda{}I) &= \text{det} \Big(
    \begin{bmatrix}
    \frac{\sqrt{3}}{2} - \lambda{} & -\frac{1}{2}\\
    \frac{1}{2} & \frac{\sqrt{3}}{2} - \lambda{}\\
    \end{bmatrix} \Big) = (\frac{\sqrt{3}}{2} - \lambda{})^{2} + \frac{1}{4} = \lambda{}^{2} - \sqrt{3} \lambda + 1 = 0
  \end{align*}
  Using the quadratic formula, we see that the eigenvalues are:
  $$\lambda = \frac{\sqrt{3}}{2} \pm \frac{1}{2} \sqrt{(-\sqrt{3})^{2} - 4 \cdot 1} = \frac{\sqrt{3}}{2} \pm \frac{1}{2}j$$

  Intuitively, you can see that this system is unstable since it will continue rotating and will never converge to a steady state.
}
\end{enumerate}
