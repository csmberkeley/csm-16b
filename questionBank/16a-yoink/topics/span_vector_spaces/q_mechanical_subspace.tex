\qns{A Tale of Two Spaces}

\textbf{Learning Goal:} The goal of this problem is to understand subspaces, basis vectors, and dimension. 
% Please look into \notes{Note 8 Section 8.1} for more on subspaces.

\meta{
\begin{itemize}

\item Go over the definition of subspace before diving into the problem. 
\item Showcase how a subspace can have multiple bases (infinite in fact)
\item Geometrically “show” in 2D/3D how rotating the x,y, and z axes creates a different basis. Just clarify with students that orthogonality is not a requirement for a basis and that rotation is not the only dimension preserving action
\item Remind students of the distinction between vector lengths and the dimension spanned by a basis.

\end{itemize}

}
\begin{enumerate}
%\setlength\itemsep{5.5em}
	\item Consider the set $U$, which is a subset of $\R^3$, defined below. Is $U$ a subspace?
	$$U = \left\{ \begin{bmatrix}x\\0 \\x+y\end{bmatrix} \mid x,y\in \R \right\}$$
	\ans {
	In order to check whether $U$ is a subspace or not, we need to check three properties: the set contains the zero vector, is closed under vector addition, and closed under scalar multiplication. First, \textbf{$U$ must contain the zero vector of $\R^3$}, $\begin{bmatrix}0\\ 0\\ 0\end{bmatrix}$. Essentially, we want to find some $x, y\in \R$ such that $$\begin{bmatrix}x\\ 0\\x+y\end{bmatrix} = \begin{bmatrix}0\\ 0\\ 0\end{bmatrix}$$ Solving this system, we get $x = 0$ and $y = 0$. The existence of a solution for $(x, y)$ indicates that the zero vector is in the set $U$.\\
	Next, we must check for \textbf{closure under vector addition}, which means given any vectors $\vec{u_1},\vec{u_2}\in U$, then it must be true that $\vec{u_1}+\vec{u_2}\in U$. Since $\vec{u_1}$ and $\vec{u_2}$ are in $U$, we can write them as $$\vec{u_1} = \begin{bmatrix}x_1\\ 0\\ x_1 + y_1\end{bmatrix}$$ $$\vec{u_2} = \begin{bmatrix}x_2\\ 0\\ x_2 + y_2\end{bmatrix}$$Then, $$\vec{u_1} + \vec{u_2} = \begin{bmatrix}x_1 + x_2 \\ 0 \\ x_1 + y_1 + x_2 + y_2\end{bmatrix} = \begin{bmatrix}x_1 + x_2 \\ 0 \\ (x_1 + x_2) + (y_ 1 + y_2)\end{bmatrix}$$
	This resulting vector is in the form $\begin{bmatrix}x\\ 0\\ x+y\end{bmatrix}$ where $x = x_1 + x_2$ and $y = y_1 + y_2$. Therefore, it must also be in $U$, so we have confirmed that $U$ is closed under vector addition.\\
	Finally, we must check for \textbf{closure under scalar multiplication}, which means given any vector $\vec{u}\in U $ and any scalar $\alpha\in\R$, then it must be true that $\alpha\vec{u}\in U$. We can write $\vec{u}$ in the form $$\vec{u} = \begin{bmatrix}x_1\\ 0\\ x_1 + y_1\end{bmatrix}$$. Then, $$ \alpha\vec{u} =\begin{bmatrix}\alpha x_1\\ 0\\ \alpha (x_1 + y_1)\end{bmatrix} = \begin{bmatrix}\alpha x_1\\ 0\\ \alpha x_1 + \alpha y_1\end{bmatrix}$$
	This resulting vector is in the form $\begin{bmatrix}x\\ 0\\ x+y\end{bmatrix}$ where $x = \alpha x_1$ and $y = \alpha y_1$. Therefore, it must also be in $U$, so we have confirmed that $U$ is closed under scalar multiplication. Since $U$ satisfies all three properties, it is a subspace.
	}
	
	\item Find a basis for $U$. What is its dimension?
	
	\meta{Emphasize that there may be more than one possible set of basis vectors, but the dimension will always remain the same.
	}
	
	\ans{
		We can write any vector in $U$ as $$\begin{bmatrix}x\\0 \\x+y\end{bmatrix}$$ for some $x, y\in \R$. This can then be written as
		$$U = \begin{bmatrix}x\\0 \\x\end{bmatrix} + \begin{bmatrix}0\\0 \\y\end{bmatrix} = x\begin{bmatrix}1\\0\\1\end{bmatrix} + y\begin{bmatrix}0\\0\\1\end{bmatrix} $$. So a set of basis vectors is: $$\left\{ \begin{bmatrix}1\\0\\1\end{bmatrix}, \begin{bmatrix}0\\0\\1\end{bmatrix} \right\}$$.
		The dimension is then the number of basis vectors, $2$.
	}

	\item Consider the set $V$, which is a subset of $\R^3$, defined below. Is $V$ a subspace? $$V = \left\{ \begin{bmatrix}0\\z\\0\end{bmatrix} \mid z\in \R \right\}$$

	\ans{
		Yes, $V$ is a subspace. We can confirm this by following the steps from part (a):\\
		\textbf{Existence of zero vector:} When $z = 0$, the form in the definition of $V$ becomes the zero vector in $\R^3$.\\
		\textbf{Closure under addition:} Given arbitrary vectors $\vec{v_1}, \vec{v_2}\in V$, we can write them as $$\vec{v_1} = \begin{bmatrix}0\\ z_1\\ 0\end{bmatrix}$$ $$\vec{v_2} = \begin{bmatrix}0\\ z_2\\ 0\end{bmatrix}$$
		Then, $$\vec{v_1} + \vec{v_2} = \begin{bmatrix}0\\ z_1 + z_2\\ 0\end{bmatrix}$$
		which is in $V$. So $V$ is closed under addition. \\
		\textbf{Closure under scalar multiplication:} Given arbitrary vector $\vec{v_1}\in V$ and scalar $\alpha\in\R$, we can write 
		$$\alpha\vec{v_1} = \begin{bmatrix}0\\ \alpha z_1\\ 0 \end{bmatrix}$$
		which is in $V$. So $V$ is closed under scalar multiplication. Since $V$ satisfies all three properties, it is a subspace.
	}

	\item Find a basis for $V$. What is its dimension?
	
	\ans{
		We can write any vector in $V$ as $$\begin{bmatrix}0\\z \\0\end{bmatrix}$$ for some $z\in \R$. This can then be written as
		$$V = \begin{bmatrix}0\\z \\0\end{bmatrix} = z\begin{bmatrix}0\\1\\0\end{bmatrix}$$. So a set of basis vectors is: $$\left\{ \begin{bmatrix}0\\1\\0\end{bmatrix} \right\}$$.
		The dimension is the number of basis vectors, $1$.
	}

	\item Can you express the basis vector(s) you found in part 4 as a linear combination of the basis vector(s) you found in part 2? Why or why not?

	\ans{
		Let the basis vectors in part 2 be $\vec{b_1}, \vec{b_2}$ and the basis vector in part 4 be $\vec{b_3}$. To see if we can express $\vec{b_3}$ as a linear combination of  $\vec{b_1}, \vec{b_2}$, we seek scalars $m, n\in\R$ such that $m\vec{b_1} + n\vec{b_2} = \vec{b_3}$, or:
		$$m\begin{bmatrix}1\\0\\1\end{bmatrix} + n\begin{bmatrix}0\\0\\1\end{bmatrix} = \begin{bmatrix}0\\1\\0\end{bmatrix}$$
		$$\begin{bmatrix}m\\ 0\\ m+n\end{bmatrix} = \begin{bmatrix}0\\1\\0\end{bmatrix}$$
		Focusing on the second row, we have $0$ on the left hand side of the equation and $1$ on the right hand side. This is not possible, so the basis vectors found in part 2 are linearly independent to the basis vector in part 4. The only vector that can be described by both sets of basis vectors is the zero vector.
		
		Something further to think about: can you express any arbitrary vector in $V$ in terms of the basis vectors of $U$? Or vice versa?
	}
\end{enumerate}
