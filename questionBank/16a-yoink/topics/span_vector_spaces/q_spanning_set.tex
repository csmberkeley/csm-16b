\qns{Spanning Set}

\textbf{Learning Goal:} The goal of this problem is to connect Gaussian Elimination and linear (in)dependence to the concept of span. Another goal is to be comfortable with the geometric representation of span.

\meta{

\begin{itemize}
\item Before starting the problems, make sure that students are comfortable with the definition of span. 

\item Drive home the idea that knowing a vector is in a span means that the vector can be represented as a linear combination of other vectors in the span (e.g. $\vec{b}$ from part (a)).

\item It might be a good idea to explain span geometrically.  

\item An example video that may help students visualize span is linked: \underline{https://bit.ly/3aK9p9v}

\end{itemize}

}

\begin{enumerate}
 %\setlength\itemsep{6em}
\itemFor what values of $b_1$, $b_2$, $b_3$ is the following system of linear equations consistent? (``Consistent'' means there is at least one solution. Please see \notes{Note 1B: Subsection 1.2.4.2} for more details on consistency of a system.)
\begin{align*}
\mathbf{A}\vec{x}&=\vec{b}\\
    \begin{bmatrix}1 & 2\\ 2 & 1 \\ 0 & 0 \end{bmatrix}\begin{bmatrix}x_1 \\ x_2\end{bmatrix} &= \begin{bmatrix}b_1 \\ b_2 \\ b_3 \end{bmatrix}
\end{align*}

    \ans{
      For the system of linear equations to be consistent, there must exist some $\vec{x}$ such that the equality above holds. Performing matrix vector multiplication, we can rewrite the above equality as
\begin{align*}
	x_1 \begin{bmatrix} 1 \\ 2 \\ 0 \end{bmatrix} + x_2 \begin{bmatrix} 2 \\ 1 \\ 0 \end{bmatrix} = \begin{bmatrix}b_1 \\ b_2 \\ b_3 \end{bmatrix}
\end{align*}

      The LHS of the above equation is a linear combination of the columns of $\mathbf{A}$. So, the system will be consistent as long as $\vec{b}$ can be written as a linear combination of the columns of $\mathbf{A}$. In other words, $\vec{b}$ needs to be in $\text{range}(\mathbf{A})=\text{span}\left\{ \begin{bmatrix}1 \\ 2 \\ 0 \end{bmatrix}, \begin{bmatrix}2 \\ 1 \\ 0 \end{bmatrix} \right\}$. Please see \notes{Note 3: Section 3.3} for the relation between linear dependence and span.
      
\textbf{Note: Any system $\mathbf{A}\vec{x}=\vec{b}$ will be consistent if and only if, $\vec{b}\in \text{range}(\mathbf{A})$ i.e. $\vec{b}$ belongs in the span of the columns of $\mathbf{A}$. This is a very key concept. Matrix multiplication can be thought of as remapping a vector (or a set of vectors) to a new coordinate plane. The set of vectors that can be reached in the span of the columns of the matrix transformation.}

Performing Gaussian Elimination on the augmented matrix $[\mathbf{A}|\vec{b}]$, we have:
\begin{align*}
	\left[\begin{array}{cc|c}
		1 & 2 & b_1\\
		2 & 1 & b_2\\
		0 & 0 & b_3
	\end{array}\right] &\rightarrow \left[\begin{array}{cc|c}
		1 & 2 & b_1\\
		0 & -3 & b_2-2b_1\\
		0 & 0 & b_3
\end{array}\right] \mbox{using $R_2 \leftarrow R_2 - 2R_1$}\\
&\rightarrow \left[\begin{array}{cc|c}
		1 & 2 & b_1\\
		0 & 1 & (2b_1-b_2)/3\\
		0 & 0 & b_3
\end{array}\right] \mbox{using $R_2 \leftarrow -R_2/3$}\\
&\rightarrow \left[\begin{array}{cc|c}
		1 & 0 & (2b_2-b_1)/3\\
		0 & 1 & (2b_1-b_2)/3\\
		0 & 0 & b_3
\end{array}\right] \mbox{using $R_1 \leftarrow R_1-2R_2$}
\end{align*}

Looking at the last row of the row reduced matrix, that $b_3$ needs to be zero to avoid inconsistency of the system. On the other hand, solution would exist for any scalar values of $b_1$ and $b_2$.}




\itemWhat is the geometry represented by $\text{span}\left\{ \begin{bmatrix}1 \\ 2 \\ 0 \end{bmatrix}, \begin{bmatrix}2 \\ 1 \\ 0 \end{bmatrix} \right\}$ ?
    
     \meta{Some students might think $xy$-plane is equivalent to $\mathbb{R}^2$. Highlight the fact that these vectors have three components, so they $\in\mathbb{R}^3$, and cannot be $\in\mathbb{R}^2$. For purposes of explaining the answers in the next few parts, it is recommended to draw the 2 column vectors in a 3D coordinate grid. }

    \ans{
 $\text{span}\left\{ \begin{bmatrix}1 \\ 2 \\ 0 \end{bmatrix}, \begin{bmatrix}2 \\ 1 \\ 0 \end{bmatrix} \right\}$ contains any vector $\vec{v}$ that can be written as
\begin{align*}
	\vec{v} = \alpha_1 \begin{bmatrix}1 \\ 2 \\ 0 \end{bmatrix} + \alpha_2 \begin{bmatrix}2 \\ 1 \\ 0 \end{bmatrix},
\end{align*}
where $\alpha_1$ and $\alpha_2$ are scalars.
      We realize that from part (a) and part (b) that any vector whose third component is $0$ can be written in this form. Hence, the required span is the set of all vectors that can be written in the form $\begin{bmatrix}* \\ * \\ 0 \end{bmatrix}$. Geometrically, the span is the set of all vector in the $xy$-plane in $\mathbb{R}^3$.
      
    \begin{tikzpicture}[x=1cm, y=1cm, z=-0.6cm]
    % Axes
    \draw [->] (0,0,0) -- (0,0,4) node [left] {$x$};
    \draw [->] (0,0,0) -- (4,0,0) node [right] {$y$};
    \draw [->] (0,0,0) -- (0,3,0) node [left] {$z$};
    % Vectors
    \draw [->, thick] (0,0,0) -- (1,0,2);
    \draw [->, thick] (0,0,0) -- (2,0,1);
    % Ticks
        \foreach \i in {1,2}
    {
    \draw (-0.1,\i,0) -- ++ (0.2,0,0);
    \draw (\i,-0.1,0) -- ++ (0,0.2,0);
    \draw (-0.1,0,\i) -- ++ (0.2,0,0);
    }
    % Dashed lines
    \draw [loosely dashed]
        (1,0,0) -- (1,0,2) -- (0,0,2)
        (2,0,0) -- (2,0,1) -- (0,0,1)
        ;
    % Labels
     \node [below] at (1,0,2) {$\begin{bmatrix}
                                1\\2\\0
                               \end{bmatrix}$};
   \node [below] at (2,0,1) {$\begin{bmatrix}
                               2\\1\\0
                              \end{bmatrix}$};

	\end{tikzpicture}
      
   }



\itemFind out if $\vec{v_1}=\begin{bmatrix}-3 \\ 5 \\ 0 \end{bmatrix}$ is in $\text{span}\left\{ \begin{bmatrix}1 \\ 2 \\ 0 \end{bmatrix}, \begin{bmatrix}2 \\ 1 \\ 0 \end{bmatrix} \right\}$. What about $\vec{v_2}=\begin{bmatrix}-3 \\ 5 \\ 2 \end{bmatrix}$?

\meta{
\begin{itemize}
\item Given the previous answer,  the solution should be clear to students. 

\item While graphical visualization is a valid option for lower dimensions,  the mathematical approach introduced here should be followed for higher dimensions. 

\item The results contains fractions,  to show that the coefficients can be non-integers.
\end{itemize}
}

\ans{

Just at first glance,  we can immediately see that $\vec{v_2}$ is not in the span since in part (a), we determined that ${b_3}$ must be 0.
\\
\\
However,  we can use Gaussian Elimination again to find out:

\begin{align*}
	\left[\begin{array}{cc|c}
		1 & 2 & -3\\
		2 & 1 & 5\\
		0 & 0 & 0
	\end{array}\right] &\rightarrow \left[\begin{array}{cc|c}
		1 & 2 & -3\\
		0 & -3 & 11\\
		0 & 0 & 0
\end{array}\right] \mbox{using $R_2 \leftarrow R_2 - 2R_1$}\\
&\rightarrow \left[\begin{array}{cc|c}
		1 & 2 & -3\\
		0 & 1 & -11/3\\
		0 & 0 & 0
\end{array}\right] \mbox{using $R_2 \leftarrow -R_2/3$}\\
&\rightarrow \left[\begin{array}{cc|c}
		1 & 0 & 13/3\\
		0 & 1 & -11/3\\
		0 & 0 & 0
\end{array}\right] \mbox{using $R_1 \leftarrow R_1-2R_2$}
\end{align*}
So we can write that $x_1=\frac{13}{3}$ and $x_2=-\frac{11}{3}$.
\begin{align*}
	\begin{bmatrix}-3 \\ 5 \\ 0 \end{bmatrix}=\frac{13}{3}\begin{bmatrix} 1 \\ 2 \\ 0 \end{bmatrix} -\frac{11}{3}\begin{bmatrix} 2 \\ 1 \\ 0 \end{bmatrix}.
\end{align*}    
So $\vec{v_1}$ is in the span.

If we follow the same process for  $\vec{v_2}$, we are going to see that the system is inconsistent, i.e. it cannot be solved for $x_1$ and $x_2$, which means $\vec{v_2}$ cannot be in the span.
}

  

\itemReflect on your answer from part(b) and find out $\text{span}\left\{ \begin{bmatrix}1 \\ 2 \\ 0 \end{bmatrix}, \begin{bmatrix}2 \\ 1 \\ 0 \end{bmatrix}, \begin{bmatrix}-3 \\ 5 \\ 0 \end{bmatrix} \right\}$.

	\meta{This part helps establish the intuition behind linear dependence. Take some time to explain why the new vector doesn't contribute to the span.}

    \ans{
      From part (b), we found that $\begin{bmatrix}-3 \\ 5 \\ 0 \end{bmatrix}$ is in $\text{span}\left\{ \begin{bmatrix}1 \\ 2 \\ 0 \end{bmatrix}, \begin{bmatrix}2 \\ 1 \\ 0 \end{bmatrix} \right\}$, i.e. $\begin{bmatrix}-3 \\ 5 \\ 0 \end{bmatrix}$ is linearly dependent on $\begin{bmatrix}1 \\ 2 \\ 0 \end{bmatrix}$ and $\begin{bmatrix}2 \\ 1 \\ 0 \end{bmatrix}$. 
      So  any vector $\vec{v}\in\text{span}\left\{ \begin{bmatrix}1 \\ 2 \\ 0 \end{bmatrix}, \begin{bmatrix}2 \\ 1 \\ 0 \end{bmatrix}, \begin{bmatrix}-3 \\ 5 \\ 0 \end{bmatrix} \right\}$ can be written as:
 \begin{align*}
 \vec{v}&=\alpha_1\begin{bmatrix}1 \\ 2 \\ 0 \end{bmatrix}+\alpha_2\begin{bmatrix}2 \\ 1 \\ 0 \end{bmatrix}+\alpha_3\begin{bmatrix}-3 \\ 5 \\ 0 \end{bmatrix}\\
 &=\alpha_1\begin{bmatrix}1 \\ 2 \\ 0 \end{bmatrix}+\alpha_2\begin{bmatrix}2 \\ 1 \\ 0 \end{bmatrix}+\frac{13}{3}\alpha_3\begin{bmatrix}1 \\ 2 \\ 0 \end{bmatrix}-\frac{11}{3}\alpha_3\begin{bmatrix}2 \\ 1 \\ 0 \end{bmatrix}\\
&=(\alpha_1+\frac{13}{3}\alpha_3)\begin{bmatrix}1 \\ 2 \\ 0 \end{bmatrix}+(\alpha_2-\frac{11}{3}\alpha_3)\begin{bmatrix}2 \\ 1 \\ 0 \end{bmatrix}
 \end{align*}

Since $(\alpha_1+\frac{13}{3}\alpha_3)$ and $(\alpha_2-\frac{11}{3}\alpha_3)$ are both scalars in the above equation, we can say that $\vec{v}\in\text{span}\left\{ \begin{bmatrix}1 \\ 2 \\ 0 \end{bmatrix}, \begin{bmatrix}2 \\ 1 \\ 0 \end{bmatrix} \right\}$. Similarly you can show that any vector $\vec{u}\in\text{span}\left\{ \begin{bmatrix}1 \\ 2 \\ 0 \end{bmatrix}, \begin{bmatrix}2 \\ 1 \\ 0 \end{bmatrix} \right\}$ would also belong in $\text{span}\left\{ \begin{bmatrix}1 \\ 2 \\ 0 \end{bmatrix}, \begin{bmatrix}2 \\ 1 \\ 0 \end{bmatrix}, \begin{bmatrix}-3 \\ 5 \\ 0 \end{bmatrix} \right\}$. So $\text{span}\left\{ \begin{bmatrix}1 \\ 2 \\ 0 \end{bmatrix}, \begin{bmatrix}2 \\ 1 \\ 0 \end{bmatrix}, \begin{bmatrix}-3 \\ 5 \\ 0 \end{bmatrix} \right\}=\text{span}\left\{ \begin{bmatrix}1 \\ 2 \\ 0 \end{bmatrix}, \begin{bmatrix}2 \\ 1 \\ 0 \end{bmatrix} \right\}$
}

    \itemWhat is a possible choice for $\vec{v}$ that would make span$\left\{ \begin{bmatrix}1 \\ 2 \\ 0 \end{bmatrix}, \begin{bmatrix}2 \\ 1 \\ 0 \end{bmatrix}, \vec{v} \right\} = \mathbb{R}^3$ ?

    \ans{
      From part (a), we realize that any vector whose last component is 0 can be written as a linear combination of the two vectors already in the set. Hence, if we include, for example, $\begin{bmatrix} 0 \\ 0 \\ 1 \end{bmatrix}$ into the set, then we should be able to reach any vector in $\mathbb{R}^3$. Any vector whose last component is non-zero is a valid addition to the set to achieve the desired span. 
      
\textbf{Note: We need \textit{at least} $n$ linearly independent vectors $\in\mathbb{R}^n$ to span the entirety of $\mathbb{R}^n$. In other words, $\text{span}\left\{ \vec{a_1}, \vec{a_2}, ,\hdots , \vec{a_n} \right\}=\mathbb{R}^n$, for a linearly independent set of $\vec{a_1}$,$\hdots$, $\vec{a_n}$.}

}

    

\end{enumerate}

