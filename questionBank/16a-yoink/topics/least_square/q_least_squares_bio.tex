% Author: Jessica Lin
% Email: jessica.jx.lin@berkeley.edu
% CSM16A Spring 2023


\qns{Van't Hoff}

\textbf{Learning Goal:} The goal of this question is to build an understanding of least squares.



You are a molecular biologist seeking to identify the thermodynamic parameters of DNA denaturation, a process where its two strands separate. You notice that DNA has an interesting property, where when its two strands are separated, its absorbance increases. As a biologist well-versed in spectroscopy and thermodynamics, you realize that this is the key to solving your problem! As a result, you collect an abundance of data at varying temperatures, and fit it to the following equation:

\[
K = e^{-\frac{\Delta H \degree}{RT}}e^{\frac{\Delta S \degree}{R}}
\]

where $K$ is an equilibrium constant calculated from absorbance, $\Delta H \degree$ is the change in enthalpy, $\Delta S \degree$ is the change in entropy, $T$ is the temperature in Kelvin, and $R$ is a constant, $1.987$ cal/molK. You collect the following data:

\begin{center}
\begin{tabular}{c|c}
\hline
    $K$ & $T$ \\
\hline
    0.1 & 320 \\
    1.0 & 330 \\
    5.0 & 340 \\
    20 & 350 \\
    200 & 360 \\
\hline
\end{tabular}
\end{center}

\begin{enumerate}
\item Re-write the above equation such that least-squares can be used to solve for the thermodynamic parameters.


\meta{
\begin{itemize}
    \item Students should be able to see that least-squares can be applied to non-linear equations, as long as the equations can be written into a form that is linear/affine in the unknown parameters.
\end{itemize}
}

\ans{

We are looking for an equation that is linear/affine in the unknown thermodynamic parameters. As such, we can rewrite the given equation as follows:

\[
K = e^{-\frac{\Delta H \degree}{RT}}e^{\frac{\Delta S \degree}{R}} \rightarrow
K = e^{-\frac{\Delta H \degree}{RT} + \frac{\Delta S \degree}{R}} \rightarrow
\ln K = -\frac{\Delta H \degree}{RT} + \frac{\Delta S \degree}{R}
\]

So we could use 
\[
\ln K = -\frac{\Delta H \degree}{RT} + \frac{\Delta S \degree}{R}
\]
or 
\[
R \ln K = -\frac{\Delta H \degree}{T} + \Delta S \degree
\]
for least squares to solve for the thermodynamic parameters. Both of these equations will work for least-squares, but the second equation might be simpler when actually computing the least-squares solutions.

}

\item Given the data in the table above, determine $A$, $\vec{x}$, and $\vec{b}$ in $A\vec{x} = \vec{b}$ to be used for least-squares. 

\ans{

    We want to write a matrix-vector equation $A\vec{x} = \vec{b}$ given the data in the table and the linearized equation we solved for in part (a):

    \[
        R \ln K = -\frac{\Delta H \degree}{T} + \Delta S \degree
        \rightarrow
        (-\frac{1}{T}) \cdot \Delta H \degree + (1) \cdot \Delta S \degree = (R \ln K)
    \]
    
    The vector $\vec{x}$ should contain the unknowns we are trying to solve for; in this scenario, the unknowns are the thermodynamic parameters $\Delta H \degree$ and $\Delta S \degree$. The matrix $A$ should contain the coefficients multiplying the unknown parameters. The vector $\vec{b}$ should be the result of the matrix-vector multiplication.

    Every row of $A$ and every row of $\vec{b}$ corresponds to a row of data given in the table.

    \begin{center}
    \[
    \underbrace{\begin{bmatrix}
        -\dfrac{1}{320} & 1 \\[10pt]
        -\dfrac{1}{330} & 1 \\[10pt]
        -\dfrac{1}{340} & 1 \\[10pt]
        -\dfrac{1}{350} & 1 \\[10pt]
        -\dfrac{1}{360} & 1 \\[10pt]
        \end{bmatrix}
    }_{A}
    \underbrace{
    \begin{bmatrix}\Delta H \degree \\ \Delta S \degree \end{bmatrix}
    }_{\vec{x}} 
    = 
    \underbrace{
    \begin{bmatrix}
        R \ln 0.1 \\[10pt]
        R \ln 1.0 \\[10pt]
        R \ln 5.0 \\[10pt]
        R \ln 20 \\[10pt]
        R \ln 200 \\[10pt]
    \end{bmatrix}
    }_{\vec{b}}
    \]
    \end{center}

    Substituting the value of $R$ in would yield the following matrix-vector equation:

    \begin{center}
    \[
    \underbrace{\begin{bmatrix}
        -\dfrac{1}{320} & 1 \\[10pt]
        -\dfrac{1}{330} & 1 \\[10pt]
        -\dfrac{1}{340} & 1 \\[10pt]
        -\dfrac{1}{350} & 1 \\[10pt]
        -\dfrac{1}{360} & 1 \\[10pt]
        \end{bmatrix}
    }_{A}
    \underbrace{
    \begin{bmatrix}\Delta H \degree \\ \Delta S \degree \end{bmatrix}
    }_{\vec{x}} 
    = 
    \underbrace{
    \begin{bmatrix}
        1.987 \ln 0.1 \\[10pt]
        1.987 \ln 1.0 \\[10pt]
        1.987 \ln 5.0 \\[10pt]
        1.987 \ln 20 \\[10pt]
        1.987 \ln 200 \\[10pt]
    \end{bmatrix}
    }_{\vec{b}}
    \]
    \end{center}

    % (If we instead chose to use the form of the Van't Hoff equation we usually see
    % \[
    % \ln K = -\frac{\Delta H \degree}{RT} + \frac{\Delta S \degree}{R}
    % \]
    % we could equivalently write our matrix-vector equation as 
    
    % \begin{center}
    % \[
    % \underbrace{\begin{bmatrix}
    %     -\dfrac{1}{R \cdot 320} & \dfrac{1}{R} \\[10pt]
    %     -\dfrac{1}{R \cdot 330} & \dfrac{1}{R} \\[10pt]
    %     -\dfrac{1}{R \cdot 340} & \dfrac{1}{R} \\[10pt]
    %     -\dfrac{1}{R \cdot 350} & \dfrac{1}{R} \\[10pt]
    %     -\dfrac{1}{R \cdot 360} & \dfrac{1}{R} \\[10pt]
    %     \end{bmatrix}
    % }_{A}
    % \underbrace{
    % \begin{bmatrix}\Delta H \degree \\ \Delta S \degree \end{bmatrix}
    % }_{\vec{x}} 
    % = 
    % \underbrace{
    % \begin{bmatrix}
    %     \ln 0.1 \\[10pt]
    %     \ln 1.0 \\[10pt]
    %     \ln 5.0 \\[10pt]
    %     \ln 20 \\[10pt]
    %     \ln 200 \\[10pt]
    % \end{bmatrix}
    % }_{\vec{b}}
    % \]
    % \end{center}
    
    % where we divide the first matrix-vector equation by $R$ on both sides.)
    
}

\item Write out the expression in terms of $A$ and $\vec{b}$ that would allow us to approximate the parameters $\Delta H \degree$ and $\Delta S \degree$ with least-squares. Then, evaluate the expression. \textit{Hint:}  $A^TA \approx \begin{bmatrix} 4.35 \cdot 10^{-5} &  - 1.47 \cdot 10^{-2} \\ - 1.47 \cdot 10^{-2} & 5 \end{bmatrix}$

\ans{
The least squares solution for the unknown parameters is given by $\vec{x}^* = (A^{T}A)^{-1}A^{T}\vec{b}$.

Written out, we have 

\tiny \begin{align*}
\vec{x}^* & = 
(\begin{bmatrix}
    -\dfrac{1}{320} & -\dfrac{1}{330} & -\dfrac{1}{340} & -\dfrac{1}{350} & -\dfrac{1}{360} \\
    1 & 1 & 1 & 1 & 1
\end{bmatrix}
\begin{bmatrix}
        -\dfrac{1}{320} & 1 \\[10pt]
        -\dfrac{1}{330} & 1 \\[10pt]
        -\dfrac{1}{340} & 1 \\[10pt]
        -\dfrac{1}{350} & 1 \\[10pt]
        -\dfrac{1}{360} & 1 \\[10pt]
\end{bmatrix})^{-1}
\begin{bmatrix}
    -\dfrac{1}{320} & -\dfrac{1}{330} & -\dfrac{1}{340} & -\dfrac{1}{350} & -\dfrac{1}{360} \\
    1 & 1 & 1 & 1 & 1
\end{bmatrix}
\begin{bmatrix}
        1.987 \ln 0.1 \\[10pt]
        1.987 \ln 1.0 \\[10pt]
        1.987 \ln 5.0 \\[10pt]
        1.987 \ln 20 \\[10pt]
        1.987 \ln 200 \\[10pt]
\end{bmatrix} \\
\end{align*}

\small \begin{align*} 
\vec{x}^* & \approx
(\begin{bmatrix} 
4.35 \cdot 10^{-5} &  - 1.47 \cdot 10^{-2} \\ 
- 1.47 \cdot 10^{-2} & 5 \end{bmatrix})^{-1}
\begin{bmatrix}
    -\dfrac{1}{320} & -\dfrac{1}{330} & -\dfrac{1}{340} & -\dfrac{1}{350} & -\dfrac{1}{360} \\
    1 & 1 & 1 & 1 & 1
\end{bmatrix}
(1.987)
\begin{bmatrix}
        -2.30 \\[10pt]
        0 \\[10pt]
        1.61 \\[10pt]
        3.00 \\[10pt]
        5.30 \\[10pt]
\end{bmatrix} \\
& \approx 
\begin{bmatrix} 
3.55 \cdot 10^{6} &  1.04 \cdot 10^{4} \\ 
1.04 \cdot 10^{4} & 3.09 \cdot 10^{1} 
\end{bmatrix}
\begin{bmatrix}
    -\dfrac{1}{320} & -\dfrac{1}{330} & -\dfrac{1}{340} & -\dfrac{1}{350} & -\dfrac{1}{360} \\
    1 & 1 & 1 & 1 & 1
\end{bmatrix}
(1.987)
\begin{bmatrix}
        -2.30 \\[10pt]
        0 \\[10pt]
        1.61 \\[10pt]
        3.00 \\[10pt]
        5.30 \\[10pt]
\end{bmatrix} \\
& \approx 
(1.987)\begin{bmatrix} 
3.55 \cdot 10^{6} &  1.04 \cdot 10^{4} \\ 
1.04 \cdot 10^{4} & 3.09 \cdot 10^{1} 
\end{bmatrix}
\begin{bmatrix}
    -2.08 \cdot 10^{-2} \\
    7.61
\end{bmatrix} \\
& \approx
\begin{bmatrix}
    1.05 \cdot 10^{4} \\
    3.74 \cdot 10^{1}
\end{bmatrix}
\end{align*}
}

\item For a temperature $T$ of $30 \degree C$, what expression could we evaluate to approximate the $K$ value using least-squares?

\ans{

A temperature of $30 \degree C$ is $273 + 30 = 303 K$. We can use the equation we derived in part (a), which we also used in part (b): 
\[
(-\frac{1}{T}) \cdot \Delta H \degree + (1) \cdot \Delta S \degree = (R \ln K)
\]
where our thermodynamic parameters $\Delta H \degree$ and $\Delta S \degree$ are our least-squares solutions computed in part (c). We can then write:

\[
\begin{bmatrix}
    -\dfrac{1}{303} & 1
\end{bmatrix}
\vec{x}^*
= R \ln K
\]
or 
\[
K = \exp({\begin{bmatrix}
    -\dfrac{1}{R \cdot 303} & \dfrac{1}{R}
\end{bmatrix}
\vec{x}*})
\].

Evaluating this numerically using values from part (c), we have

\[
K \approx \exp({\begin{bmatrix}
    -\dfrac{1}{1.987 \cdot 303} & \dfrac{1}{1.987}
\end{bmatrix}
\begin{bmatrix}
    1.05 \cdot 10^{4} \\
    3.74 \cdot 10^{1}
\end{bmatrix}}) \approx e^{1.38} \approx 3.98
\]
}

\end{enumerate}
