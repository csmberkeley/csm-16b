% Zipeng Lin - yuslzp@berkeley.edu
% Rewritten by Dun-Ming Huang, dunmingbrandonhuang@berkeley.edu, Sp23
\newcommand{\old}[1]{}
\qns{Let the World Feel P(ythagor)ain} % Project-gorean theorem

\textbf{Learning goal:}
\begin{bindenum}
    \item Practice projection related proofs and algebraic derivations.
    \item Learn to clarify the setups of a proof before proceeding.
    \item Recognize the geometrical applications of projection.
    \item (Learn how to not feel as much pain in proofs.)
\end{bindenum}

\meta{
\begin{enumerate}
    \item This proof question is separated into quite grated subparts, so there is relatively less worry on how to teach and convey these concepts.
    \item Therefore, there are not much meta that I'd like to address here. The only advice I have is to perform this proof interactively with students, rather than individually, as students are perhaps still not familiar with what a projection is.
\end{enumerate}
}

Let us now explore an application of ``projection'' with the following setup:
\begin{quote}
    Suppose we have points $A = (a, b), B =(c, d)$ as points in the first quadrant in the 2D plane. Then, suppose $O$ is the origin $(0, 0)$, and $C$ lies at the point on $\overline{OB}$ tht is closest to $A$. \\
    What is the relationship between $\overline{OA}, \overline{OC}, \overline{AC}$?
\end{quote}

\begin{enumerate}
    \item {
        Draw a figure that represents the setup of this prompt's proof.
        
    }
    \meta {
        The bottom vector is drawn non-horizontally to present the general setup of this proof.
        
    }
    \ans {
        We may draw the setup of our proof as follows:
        \begin{center}
            \begin{tikzpicture}[>=latex]
              \draw[->,line width=0.5mm, red] (0,0) node[circ, black, label=left:$O$]{} -- (2,2) node[right]{$A$};
              \draw[->, line width=0.5mm, blue] (0,0) -- (4.5,0.3) node[right]{$B$};
              \node[circ, black, label=above:$C$] at (2.12, 0.14) {};
            \end{tikzpicture}
        \end{center}
    }

    \item {
        Express the length of $\overline{OA}$ in terms of an inner product. You can define any vector you use in this inner product definition in terms of $a, b, c, d$.
        
    }
    \ans {
        The length of $\overline{OA}$ can be written as the norm of such vector proceeding from $O = (0, 0)$ to $A = (a, b)$. \\
        Therefore, suppose we define
        \[
            \vec{OA} = \begin{bmatrix} a \\ b \end{bmatrix}
        \]
        then, the length of $\overline{OA}$ may be written as:
        \[
            \| \vec{OA} \| = \sqrt{\vec{OA}^T \vec{OA}} = \sqrt{a^2 + b^2}
        \]
    }

    \item {
        Express a vector that has the same direction and magnitude as $\overline{OC}$. \\
        \textit{Hint: $C$ is the point on some line $\overline{OB}$ that is closest to the point $A$. Does this remind you of some mathematical function we addressed in previous parts?}
        
    }
    \ans {
        The line segment $\overline{OC}$ may be treated as the vector that proceeds from $O$ to $C$, which we may write as $\vec{OC}$ as well. \\
        Notice that $C$ is the closest point on $\overline{OB}$ to $A$, which would imply that $\overline{AC} \perp \overline{OB}$.
        Therefore, $\vec{OC}$ is in fact a projection of $\vec{OA}$ onto $\vec{OB}$ (meaning that $\vec{OC}$ is the shadow of $\vec{OA}$ onto $\vec{OB}$).

        Recalling that,
        \[
            \vec{OA} = \begin{bmatrix} a \\ b \end{bmatrix},
            \vec{OB} = \begin{bmatrix} c \\ d \end{bmatrix},
        \]
        Then, the projection of $\vec{OA}$ onto $\vec{OB}$ may thus be written as:
        \begin{align*}
            \vec{OC} &= {\rm proj}_{\vec{OB}} (\vec{OA}) \\
            &= \frac{\vec{OA} \cdot \vec{OB}}{\vec{OB} \cdot \vec{OB}} \vec{OB} \\
            &= \frac{ac + bd}{c^2 + d^2} \begin{bmatrix} c \\ d \end{bmatrix}
        \end{align*}
        
    }

    \item {
        Express the length of $\overline{AC}$ in terms of $a, b, c, d$.
        
    }
    \ans {
        From the geometries of our setup, we may see that:
        \[
            \vec{CA} = \vec{OA} - \vec{OC}
        \]
        Therefore,
        \begin{align*}
            \vec{CA}
            &= \vec{OA} - \vec{OC} \\
            &= \begin{bmatrix} a \\ b \end{bmatrix} - \frac{ac + bd}{c^2 + d^2} \begin{bmatrix} c \\ d \end{bmatrix} \\
            &=
            \begin{bmatrix}
                a - c\frac{ac + bd}{c^2 + d^2} \\
                b - d\frac{ac + bd}{c^2 + d^2}
            \end{bmatrix}
        \end{align*}
        
    }

    \item {
        Using previous subparts, prove that the line segments $\overline{OA}, \overline{OC}, \overline{AC}$ follow the Pythagorean Theorem. \\
        You may write an answer with or without mathematical expressions.
        
    }
    \ans {
        Let us substitute a complicated value in the computation as:
        \[
            m = \frac{ac + bd}{c^2 + d^2}
        \]
        The square of lengths of $\overline{OC}, \overline{AC}$ may be organized as:
        \begin{align*}
            \overline{OC}^2
            &= {\| \vec{OC} \|}^2 \\
            &= {(cm)}^2 + {(dm)}^2 \\
            \overline{AC}^2
            &= {\| \vec{AC} \|}^2 \\
            &= {(a - cm)}^2 + {(b - dm)}^2 \\
            \overline{OC}^2 + \overline{AC}^2
            &= {(cm)}^2 + {(dm)}^2 + {(a - cm)}^2 + {(b - dm)}^2 \\
            &= a^2 - 2acm + b^2 - 2bdm + 2 c^2 m^2 + 2 d^2 m^2 \\
            &= a^2 + b^2 + 2(c^2 + d^2) m^2 - 2 (ac + bd) m \\
            &= a^2 + b^2 + \frac{2(c^2 + d^2) {(ac + bd)}^2}{{(c^2 + d^2)}^2} - 2 \frac{2 {(ac + bd)}^2}{c^2 + d^2} \\
            &= a^2 + b^2 + \frac{
                2 {(ac + bd)}^2 - 2 {(ac + bd)}^2
            }{c^2 + d^2} = a^2 + b^2 = \overline{OA}
        \end{align*}
    }
\end{enumerate}

\old{
Suppose we have points $A = (a, b), B =(c, d)$ as points in the first quadrant
in the 2D plane.  $O$ is the origin. Suppose projection of  $A$ to  $OB$ is
point  $C$. Prove the pythagorean theorem, which is

 \[
    AO^{2} = OC^{2} + AC^{2}
\]

and calculate the angle of $\angle AOB$

\ans{For the proof, we have $AO^{2} = (a^{2} + b^{2})$, also, we have the
    projection result has the point 

    \[
        (a, b) - \dfrac{\langle a, b\rangle \cdot \langle c, d\rangle}{\langle
        c, d\rangle \cdot \langle c, d\rangle} (c, d) =
        \langle\dfrac{ac^{2}+bcd}{c^{2} + d^{2}},
        \dfrac{bd^{2} + acd}{c^{2} + d^{2}}\rangle
    \]

    and the vector that represents $AC$  is

     \[
         \langle a, b\rangle - \langle\dfrac{ac^{2}+bcd}{c^{2} + d^{2}},
         \dfrac{bd^{2} + acd}{c^{2} + d^{2}}\rangle = \langle \dfrac{ad^{2} -
             bcd}{c^{2} +
             d^{2}}, \dfrac{bd^{2} + acd}{c^{2}+d^{2}}\rangle
    \]
    

    We just need to prove the sum of norm of those two points would be $a^{2} +
    b^{2}$, this is equivalent to proving (since all the coordinates share the
    denominator $c^{2} + d^{2}$)

    \[
        (ad^{2} - bcd)^{2} + (bc^{2} - acd)^{2} + (ac^{2} + bcd)^{2} + (bd^{2} +
        acd)^{2} = a^{2}d^{4} + b^{2}c^{2}d^{2} + b^{2}c^{4} + a^{2}c^{2}d^{2} +
        a^{2}c^{4} + b^{2}c^{2}d^{2} + b^{2}d^{4} + a^{2}c^{2}+d^{2} =
        (a^{2}+b^{2})(c^{4}+2c^{2}d^{2}+d^{4}) = (a^{2}+b^{2})(c^{2}+d^{2})^{2}
    \]

    so the sum is $a^{2}+b^{2}$
    
}


\ans{
    We have the angle is 

    \[
        \arcsin (\dfrac{\sqrt{a^{2}c^{4} +
            b^{2}d^{4}}{(c^{2}+d^{2})}\sqrt{(a^{2}+d^{2})}})
    \]
 
}
}

