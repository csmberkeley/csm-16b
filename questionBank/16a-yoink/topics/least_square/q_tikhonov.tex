% Zipeng Lin - yuslzp@berkeley.edu
\qns{Challenge question: Regularized least square}
\textbf{Learning goal: learn how to transfer a optimization problem to least
square, thus preparing for further least square topics}.

\meta{
\begin{enumerate}
    \item Make sure to assure students that problems like the last step would not appear
        on the final if they feel this is too hard. Instead, tell them this
        would help them understand least square better.
    \item Try to lead students into thinking how to reform the question into a
        least square problem by adding rows to $A$ and  $b$. Try to start from
        the problem (a) and then extend it to other elements.
    \item If the mentor has a machine learning background, explain what is
        overfit, what is
        regularization and how does adding dimension would help regularizing the
        problem.
\end{enumerate}
}



We propose what we want to solve at the end first: when solving the equation $Ax = b$ with matrix  $A \in R^{n \times n}$ and vector
$b \in R^{n}$, what we
really are finding is an  $x$ that minimizes the following equation:

 \[
     \min_{x \in R^{n}}\norm{Ax - b}_2^2
\]

and the norm expression is just 

\[
    \norm{x}_2^2 = x^{\intercal} x
\]

Now, consider the following problem with $\lambda \ge{} 0$

\[
    \min_{x \in R^n} \norm{Ax - b}^2_2 + \lambda \norm{x}^2_2
\]

that is, we want to make $\norm{Ax - b}^2_2$ as small as possible and also  $x$
itself as small as possible. Give a solution to the problem above by using least
square.

Now, use the following steps to get a sense on how an approach might possibly
work.

\begin{enumerate}[label=(\alph*)]
    \item What would happen if we add an row $\left[ 1, 0, \ldots{}, 0 \right]$ to $A$ and add a element row  $[0]$ 
        to the element  $b$? Suppose the solution $x$ has the form  $\left[ x_1
        x_2 \ldots{} x_n\right]^{\intercal}$? (Hint: consider write the
        expression out and think about the element $x_1$, what do we want this
        element to be?)
    \item What rows do we need to add to make $x_i$ to be small ($i$ is an
        integer that is in  $\left[ 1, n \right]$)?
    \item (EXTRA CHALLENGE) Treat the original problem as ``wanting to make each element in $x$ to
        be small``, and treat  $\lambda = 1$. What would the solution be?
\end{enumerate}




\ans{

    (a) We can write the product out and get 

    \[
        \min_{x \in R^{n}}(\norm{Ax - b + x_1}_2^2 = \norm{Ax - b}_2^{2} +
        \norm{x_1}^{2}_2)
    \]

    Therefore, we try to solve the least square problem while trying to make
    $x_1$ as small as possible.

    (b) Inspired by the solution in (a), we can add the row $\left[ 0, 0,
    \ldots{}, 1, 0, \ldots{} \right]$ where the $i$th element is  $1$ to  $A$
    and add  $\left[ 0 \right]$ to the end of $b$. Notice that adding the
    different rows to  $A$ and  $0$ to  $b$ are independent, so we can add as
    many rows like this as we want.
    

    (EXTRA CHALLENGE'S solution)
    Expanding what we want to minimize is
    \[
        x A^{\intercal} A x + b^{\intercal} b - XA^{\intercal}b - b^{\intercal}
        A x + \lambda x^{\intercal} x
    \]
    which could be treated as
    \[
        (X^{\intercal}) (A^{\intercal} A + \lambda I) X + \ldots{}
    \]
    which is 
\[\min_{A' x - b'}\]
with 
\[
    A' = 
\begin{bmatrix}
    A \\
    \sqrt{\lambda I}  \\
\end{bmatrix}
B' =
\begin{bmatrix}
    b \\
    0 \\
\end{bmatrix}
\]
and the solution would be 
\[
    (A^{\intercal} A + \lambda I)^{-1} A^{\intercal}b
\]
       
}



