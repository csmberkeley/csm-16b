% Author: Aditya Baradwaj
% Email: abaradwaj@berkeley.edu

\qns{Op Amps as Buffers}

\textbf{Learning Goal:} This problem helps understand the operating principle of an op-amp buffer and how it helps with loading.

\textbf{Relevant Notes:} \notes{Note 19 Section 19.7} goes different op-amp circuit topology and corresponding derivations.

\meta{Show that the loading effect becomes less pronounced when the load resistance is higher, i.e. the load is drawing less current.}

Now we will revisit a problem that you might have seen before, with our new knowledge of op-amps. We have access to a circuit inside a 'black box' as shown below, with two terminal coming out of it.

\begin{enumerate}
\itemWe need a voltage of $6 \text{V}$ power a light bulb with resistance $R_L$. Design $R_1$ and $R_2$ inside the black box so that the voltage across $R_2$ is exactly equal to this required voltage \textbf{when the bulb is not connected}; i.e. $V_{R_2} = V_{out} = 6 \text{V}$. 

\begin{center}
    \begin{circuitikz}[scale=0.8]
    \filldraw[fill=gray!40!white,draw=black] (-1,-0.5) rectangle (4,7.5);
    \draw(0,0)
	to[V_=$12 \text{V}$,invert] ++(0,7)
 	to[short] ++(3,0)
	to[R,l=$R_1$] ++(0,-4)
	to [short] ++ (2, 0)
	to [open, o-o, v^=$V_{out}$] ++ (0, -3)
	to [short] ++ (-2, 0)
	to [open] ++ (0,3)
	to[R,l=$R_2$] ++(0,-3)
	to[short] ++(-3,0)
	to[short] node[ground]{} ++(0,-1);
	
	
	
 	\draw(5,3)[dashed] 
 	-- ++(2,0); 
 	\draw(7,3)
 	to[lamp,l=$R_L$] ++(0,-3);
%  	to[R,l=$R_L$] ++(0,-3);
 	\draw(7,0)[dashed]
 	-- ++(-2,0);
	\end{circuitikz}
\end{center}


\ans{
The voltage across $R_2$ is given by 
\begin{align*}
V_{R_2}=\frac{R_2}{R_1+R_2}\times 12 \si{\volt}
\end{align*}
If we set $V_{R_2}=6 \si{\volt}$, we get 
\begin{align*}
6 \si{\volt}&=\frac{R_2}{R_1+R_2}\times 12 \si{\volt}\\
\implies R_1&=R_2
\end{align*}
For example, we can choose $R_1=R_2 =1\si{\kilo\ohm}$.
}

\itemNow let us connect the bulb $R_L$ across $R_2$. What is the voltage across $R_1$, $R_2$ and $R_L$ when the bulb is connected when $R_L=R_2$? Use the values of $R_1$ and $R_2$ from the last part. Will the light bulb turn on?
What happens if $R_L=2R_2$?

\ans{
When we connect the bulb across $R_2$, the equivalent circuit is the following:
\begin{center}
    \begin{circuitikz}[scale=0.8]

    \draw(0,0)
	to[V_=$12 \text{V}$,invert] ++(0,7)
 	to[short] ++(3,0)
	to[R,l=$R_1$] ++(0,-4)
	to [short] ++ (2, 0)
	to [open, o-o, v^=$V_{out}$] ++ (0, -3)
	to [short] ++ (-2, 0)
	to [open] ++ (0,3)
	to[R,l=$R_2 || R_L$] ++(0,-3)
	to[short] ++(-3,0)
	to[short] node[ground]{} ++(0,-1);
	\end{circuitikz}
\end{center}
The voltage $V_{out}$ given by 
\begin{align*}
V_{out}=\frac{R_2||R_L}{R_1+R_2||R_L}\times 12 \si{\volt}
\end{align*}

When $R_L=R_2$:
$$R_2||R_L=R_2||R_2=\frac{R_2}{2}.$$
Hence $V_{out}$ is:
\begin{align*}
V_{out}=\frac{R_2||R_L}{R_1+R_2||R_L}\times 12 \si{\volt}=\frac{\frac{R_2}{2}}{R_1+\frac{R_2}{2}}\times 12 \si{\volt}=\frac{\frac{R_2}{2}}{R_2+\frac{R_2}{2}}\times 12 \si{\volt}= 4 \si{\volt}< 6 \si{\volt}
\end{align*}
So the bulb will not turn on.

When $R_L=2R_2$:
$$R_2||R_L=R_2||2R_2=\frac{2R_2}{3}.$$
Hence $V_{out}$ is:
\begin{align*}
V_{out}=\frac{R_2||R_L}{R_1+R_2||R_L}\times 12 \si{\volt}=\frac{\frac{2R_2}{3}}{R_1+\frac{2R_2}{3}}\times 12 \si{\volt}=\frac{\frac{2R_2}{3}}{R_2+\frac{2R_2}{3}}\times 12 \si{\volt}= 4.8 \si{\volt} < 6 \si{\volt}
\end{align*}
Although, $V_{out}$ is now closer to $6 \si{\volt}$, the bulb will still not turn on.


}

\itemUsing your knowledge of op-amps, how could you resolve this issue of $V_{out}$ changing based on the value of $R_L$? Think about how you might use an op-amp buffer. 

\ans{
We can introduce a 'buffer' op-amp as shown below:
\begin{center}
    \begin{circuitikz}[scale=0.8]
    \filldraw[fill=gray!40!white,draw=black] (-1,-0.5) rectangle (4,7.5);
    \draw(0,0)
	to[V_=$12 \text{V}$,invert] ++(0,7)
 	to[short] ++(3,0)
	to[R,l=$R_1$] ++(0,-4)
	to[short] ++(1,0)
	to[short,l_=$ \quad V_{b} \eq 6\text{V}$,i=$I_+$] ++(1,0)
	-- ++(-2,0)
	to[R,l=$R_2$] ++(0,-3)
	to[short] ++(-3,0)
	to[short] node[ground]{} ++(0,-1);
	
	\draw(8,2.5) node[op amp, yscale=-1] (opamp) {};
	
 	\draw(5,3) 
 	-- (opamp.+);
 	
 	\draw(opamp.-)
 	to[short, i<=$I_-$] ++(0,-2)
 	-- ++(3, 0)
 	to[short] (opamp.out);
 	
 	\draw(opamp.out)
 	to[short] ++(3,0)
 	to[lamp,l=$R_L$, i=$I_L$, v=$V_{out}$] ++(0,-3)
 	to[short] node[ground]{} ++(0,-1);
	\end{circuitikz}
\end{center}

A buffer op amp effectively 'decouples' its input and output,
It does this by preventing the bulb from drawing current from the black box circuit. Remember that the input currents of an ideal op-amp:
$$I_+=I_-=0,$$
while the output current $I_L$ can adjust itself to any extent, depending on the demand placed by $V_{out}$. 


Let us use nodal analysis and the golden rules to formally solve this circuit.
Firstly, we observe that the op-amp is in negative feedback configuration. Using the golden rules, we have that $V_+ = V_- = V_b = 6V$. Also, because the feedback connection is a short, $V_{out} = V_-$. Therefore, $V_{out} = V_b = 6V$. This is exactly what we want! The voltage across $R_L$  equals $V_{out} = 6V$, which is above the required voltage.
}
\end{enumerate}