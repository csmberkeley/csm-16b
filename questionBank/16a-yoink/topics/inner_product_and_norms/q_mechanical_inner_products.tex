% Author: Jessica Lin
% Email: jessica.jx.lin@berkeley.edu
% Fall 2022

\qns{Inner Product Properties}

\textbf{Learning Goal:} The goal of this problem is to understand inner product properties.

Determine whether each of the following is a valid inner product in $\mathbb{R}^2$.
\begin{enumerate}
    \item $\vec{u}^T \begin{bmatrix} 1 & 1 \\ 1 & 1 \end{bmatrix} \vec{v}$
    
    \ans{
    \begin{enumerate} 
        \item Symmetry: We want to show that $\<\vec{u}, \vec{v}\> = \<\vec{v}, \vec{u}\>$. 
        
        \begin{align*}
            \begin{bmatrix} u_1 & u_2 \end{bmatrix} \begin{bmatrix} 1 & 1 \\ 1 & 1 \end{bmatrix} \begin{bmatrix} v_1 \\ v_2 \end{bmatrix} 
            & = \begin{bmatrix} u_1 & u_2 \end{bmatrix} \begin{bmatrix} v_1 + v_2 \\ v_1 + v_2 \end{bmatrix} 
            = u_1(v_1 + v_2) + u_2(v_1 + v_2) = v_1(u_1 + u_2) + v_2(u_1 + u_2) \\
            & = (v_1 + v_2)(u_1 + u_2) \\
            & = \begin{bmatrix} v_1 & v_2 \end{bmatrix} \begin{bmatrix} u_1 + u_2 \\ u_1 + u_2 \end{bmatrix} 
            = \begin{bmatrix} v_1 & v_2 \end{bmatrix} \begin{bmatrix} 1 & 1 \\ 1 & 1 \end{bmatrix} \begin{bmatrix} u_1 \\ u_2 \end{bmatrix} 
        \end{align*}
        
        \item Linearity: We want to show that $\<\alpha \vec{u}, \vec{v}\> = \alpha \<\vec{v}, \vec{u}\>$. 
        
        \begin{align*}
            \begin{bmatrix} \alpha u_1 & \alpha u_2 \end{bmatrix} \begin{bmatrix} 1 & 1 \\ 1 & 1 \end{bmatrix} \begin{bmatrix} v_1 \\ v_2 \end{bmatrix} 
            & = \begin{bmatrix} \alpha u_1 & \alpha u_2 \end{bmatrix} \begin{bmatrix} v_1 + v_2 \\ v_1 + v_2 \end{bmatrix} 
            = \alpha u_1(v_1 + v_2) + \alpha u_2(v_1 + v_2) \\
            & = \alpha (u_1 + u_2)(v_1 + v_2)
        \end{align*}
        
        \item Positive Definiteness: 
        
        \begin{align*}
            \begin{bmatrix} u_1 & u_2 \end{bmatrix} \begin{bmatrix} 1 & 1 \\ 1 & 1 \end{bmatrix} \begin{bmatrix} u_1 \\ u_2 \end{bmatrix} 
            = u_1(u_1 + u_2) + u_2(u_1 + u_2) = u_1^2 + 2u_1u_2 + u_2^2 = (u_1 + u_2)^2 \geq 0
        \end{align*}
        
    \end{enumerate}
    }
    
    \item $\vec{u}^T \begin{bmatrix} 1 & -1 \\ 0 & 1 \end{bmatrix} \vec{v}$
    
    \ans {
    
        Once again, we see if the properties hold.
        
        Symmetry: We want to show that $\<\vec{u}, \vec{v}\> = \<\vec{v}, \vec{u}\>$.
        
        \begin{align*}
            \begin{bmatrix} u_1 & u_2 \end{bmatrix} \begin{bmatrix} 1 & -1 \\ 0 & 1 \end{bmatrix} \begin{bmatrix} v_1 \\ v_2 \end{bmatrix} 
            & = \begin{bmatrix} u_1 & u_2 \end{bmatrix} \begin{bmatrix} v_1 - v_2 \\ v_2 \end{bmatrix} 
            = u_1(v_1 - v_2) + u_2(v_2) = u_1v_2 - u_1v_2 + u_2v_2 \\
            \begin{bmatrix} v_1 & v_2 \end{bmatrix} \begin{bmatrix} 1 & -1 \\ 0 & 1 \end{bmatrix} \begin{bmatrix} u_1 \\ u_2 \end{bmatrix} 
            & = \begin{bmatrix} v_1 & v_2 \end{bmatrix} \begin{bmatrix} u_1 - u_2 \\ u_2 \end{bmatrix} 
            = v_1(u_1 - u_2) + v_2(u_2) = u_1v_1 - u_2v_1 + u_2v_2
        \end{align*}
        
        We can see that this inner product does not satisfy the symmetry condition, so it is not a valid inner product.
        
    }
\end{enumerate}
Given vectors $\vec{u} = \begin{bmatrix} 1 \\ 2 \end{bmatrix}$ and $\vec{v} = \begin{bmatrix} 3 \\ 4 \end{bmatrix}$,
\begin{enumerate}
    \item Compute the Euclidean Norm of $\vec{u}$.
    
    \ans{
        $\|\vec{u}\| = \sqrt{\<\vec{u}, \vec{u}\>} = \sqrt{1^2 + 2^2} = \sqrt{5}$.
    }
    \item Compute the Euclidean Norm of $\vec{v}$.
    
    \ans{
        $\|\vec{v}\| = \sqrt{\<\vec{v}, \vec{v}\>} = \sqrt{3^2 + 4^2} = \sqrt{25} = 5$.
    }
    \item Find the angle between $\vec{u}$ and $\vec{v}$.
    
    \ans{
        \begin{align*}
            \cos \theta & = \frac{\<\vec{u}, \vec{v}\>}{\|\vec{u}\|\|\vec{v}\|} = \frac{1 \cdot 3 + 2 \cdot 4}{5 \sqrt{5}} \\
            \theta & = \cos^{-1}(\frac{11}{5\sqrt{5}})
        \end{align*}
    }
    \item Project $\vec{u}$ onto $\vec{v}$.
    
    \ans{
        \begin{align*}
            \text{proj}_{\vec{v}}\vec{u} & = \frac{\<\vec{u}, \vec{v}\>}{\|\vec{v}\|^2} \vec{v} = \frac{11}{25}\vec{v} = \frac{11}{25} \begin{bmatrix} 3 \\ 4 \end{bmatrix}
        \end{align*}
    }
\end{enumerate}