\qns{Intro to Gram-Schmidt}

\begin{questionmeta}
    \begin{itemize}
        \item Give students a mini-lecture on orthonormality  
        \item Try to provide some intuition and basic algorithm for doing Gram-Schmidt procedure 
    \end{itemize}
\end{questionmeta}

\textbf{Learning Goal:} The goal of this problem is to practice finding orthonormal basis vectors using Gram-Schmidt and to identify when a set of vectors is orthogonal or not using linear dependence and inner products.

Gram-Schmidt Orthonormalization is an algorithmic technique to find an orthonormal basis for a set of vectors. \\
The procedure of such algorithm is portrayed as defined in the following cell:
\begin{ln-define}{The Procedure of Gram-Schmidt Orthonormalization}{}
    Let us have a set of vectors $\{\vec{a}_1, \dots, \vec{a}_n\}$.
    \begin{bindenum}
        \item[1] $\vec{q}_1 = \frac{\vec{a}_1}{{\lVert \vec{a}_1 \rVert}}$
        \item[2] for $i$ in $\{2, \dots, n\}$:
        \item[3] \hspace{0.5cm} $\vec{z_i} = {proj}_{\{\vec{q}_1, \dots, \vec{q}_{i - 1}\}}{\vec{a}_i} = \sum_{j = 1}^{i - 1} \vec{q}_j \langle \vec{a}_i, \vec{q}_j \rangle$
        \item[4] \hspace{0.5cm} $\vec{s_i} = \vec{a}_i - \vec{z}_i$
        \item[5] \hspace{0.5cm} $\vec{q_i} = \frac{\vec{s}_i}{{\lVert \vec{s}_i \rVert}_2}$
        \item[6] return $\{\vec{q_1}, \dots, \vec{q_n}\}$
    \end{bindenum}
\end{ln-define}

\begin{enumerate}
    \item Use Gram-Schmidt to turn the basis $\left\{ \vec{a}=\begin{bmatrix} 1 \\ 1 \\ 0 \end{bmatrix}, \vec{b}=\begin{bmatrix} 1 \\ 0 \\ 1 \end{bmatrix}, \vec{c}=\begin{bmatrix} 0 \\ 1 \\ 1 \end{bmatrix} \right\}$ into an orthonormal basis for $\mathbb{R}^3$.
    
    \meta{
        \begin{itemize}
            \item Not all steps are shown in the Gram-Schmidt process. Please go over the steps in between (inner products and norms) if students need additional help.
            \item This example isn't too difficult, but is good for practicing the basics of Gram-Schmidt. Make sure to spend some time on the algorithm, so they can solve more difficult problems.
        \end{itemize}
    }
    \ans{
        We will define vectors $\vec{u_1}$, $\vec{u_2}$, $\vec{u_3}$ as follows:

        \begin{align*}
            \vec{u_1} = \begin{bmatrix} 1 \\ 1 \\ 0 \end{bmatrix}
        \end{align*}

        \begin{align*}
            \vec{u_2} &= \begin{bmatrix} 1 \\ 0 \\ 1 \end{bmatrix} - {proj}_{\vec{u_1}} (\vec{b})\\
            &= \begin{bmatrix} 1 \\ 0 \\ 1 \end{bmatrix} - \frac{\langle \vec{u_1}, \vec{b} \rangle}{\langle \vec{u_1}, \vec{u_1} \rangle}\vec{u_1} \\
            &= \begin{bmatrix} 1 \\ 0 \\ 1 \end{bmatrix} - \frac{1}{2} \begin{bmatrix} 1 \\ 1 \\ 0 \end{bmatrix} \\
            &= \begin{bmatrix} \frac{1}{2} \\ -\frac{1}{2} \\ 0 \end{bmatrix}
        \end{align*}

        \begin{align*}
            \vec{u_3} &= \begin{bmatrix} 0 \\ 1 \\ 1 \end{bmatrix} - {proj}_{\vec{\vec{u_1}}} (\vec{c}) - {proj}_{\vec{\vec{u_2}}} (\vec{c})\\
            &= \begin{bmatrix} 0 \\ 1 \\ 1 \end{bmatrix} - \frac{\langle \vec{u_1}, \vec{c} \rangle}{\langle \vec{u_1}, \vec{u_1} \rangle}\vec{u_1} - \frac{\langle \vec{u_2}, \vec{c} \rangle}{\langle \vec{u_2}, \vec{u_2} \rangle}\vec{u_2} \\
            &= \begin{bmatrix} 0 \\ 1 \\ 1 \end{bmatrix} - \frac{1}{2} \begin{bmatrix} 1 \\ 1 \\ 0 \end{bmatrix} - \frac{1}{3} \begin{bmatrix} \frac{1}{2} \\ -\frac{1}{2} \\ 0 \end{bmatrix} \\
            &= \begin{bmatrix} -\frac{2}{3} \\ \frac{2}{3} \\ \frac{2}{3} \end{bmatrix}
        \end{align*}
        
        These three vectors form an orthogonal basis. To turn them into an orthonormal basis, we divide each one by their norm.

        \begin{align*}
            \vec{e_1} &= \frac{\vec{u_1}}{\| \vec{u_1} \|} \\
            &= \frac{\vec{u_1}}{\sqrt{\langle \vec{u_1}, \vec{u_1} \rangle}} \\
            &= \frac{1}{\sqrt{2}} \begin{bmatrix} 1 \\ 1 \\ 0 \end{bmatrix}
        \end{align*}

        \begin{align*}
            \vec{e_2} &= \frac{\vec{u_2}}{\| \vec{u_2} \|} \\
            &= \frac{1}{\sqrt{\frac{1}{2}}} \begin{bmatrix} \frac{1}{2} \\ -\frac{1}{2} \\ 0 \end{bmatrix}
        \end{align*}

        \begin{align*}
            \vec{e_3} &= \frac{\vec{u_3}}{\| \vec{u_3} \|} \\
            &= \frac{1}{\sqrt{\frac{4}{3}}} \begin{bmatrix} -\frac{2}{3} \\ \frac{2}{3} \\ \frac{2}{3} \end{bmatrix}
        \end{align*}

        The vectors $\left\{\vec{e_1}, \vec{e_2}, \vec{e_3}\right\}$ form an orthonormal basis for $\mathbb{R}^3$.

    }
    \item Are the following set of vectors orthogonal? Explain why or why not.
    \meta{
        The inner product calculations are not shown with steps, so please go over them if students ask about the mechanical steps.
    }
    \begin{enumerate}
        \item $\left\{ \begin{bmatrix} 1 \\ 0 \\ -1 \end{bmatrix}, \begin{bmatrix} 1 \\ \sqrt{2} \\ 1 \end{bmatrix}, \begin{bmatrix} 1 \\ -\sqrt{2} \\ 1 \end{bmatrix}\right\}$
        \ans{
            In order to check if these vectors are orthogonal, we have to see if the inner product of every pair of vectors is 0 (This also checks if the vectors are linearly dependent or not).
            \begin{align*}
                \langle \begin{bmatrix} 1 \\ 0 \\ -1 \end{bmatrix},  \begin{bmatrix} 1 \\ \sqrt{2} \\ 1 \end{bmatrix}\rangle &= 0 \\
                \langle \begin{bmatrix} 1 \\ 0 \\ -1 \end{bmatrix},  \begin{bmatrix} 1 \\ -\sqrt{2} \\ 1 \end{bmatrix}\rangle &= 0 \\
                \langle \begin{bmatrix} 1 \\ \sqrt{2} \\ 1 \end{bmatrix},  \begin{bmatrix} 1 \\ -\sqrt{2} \\ 1 \end{bmatrix}\rangle &= 0
            \end{align*}
            Because all the vectors are linearly independent (inner products are 0), this set of vectors is orthogonal.
        }
        \item $\left\{ \begin{bmatrix} 1 \\ 2 \\ 0 \end{bmatrix}, \begin{bmatrix} 2 \\ 4 \\ 0 \end{bmatrix}, \begin{bmatrix} 3 \\ 0 \\ 1 \end{bmatrix}\right\}$
        \ans{
            We have to use the same process as the previous part to calculate the inner products of each pair of vectors.
            \begin{align*}
                \langle \begin{bmatrix} 1 \\ 2 \\ 0 \end{bmatrix},  \begin{bmatrix} 2 \\ 4 \\ 0 \end{bmatrix}\rangle = 10
            \end{align*}
            Because one pair of vectors has a non-zero inner product, we can see that they are linearly dependent, which means that this set of vectors is not orthogonal.
        }
        % \item $\left\{1, \frac{x}{\sqrt{2}}+\frac{x^2}{\sqrt{2}}, 1-\frac{x}{\sqrt{2}}\right\}$
        % \ans{
        %     Here, we can see that each vector is a polynomial, which looks a bit different than normal. To simplify this problem, we can take the coefficient of each term as a component of a vector (the coefficients of $x^0$, $x^1$, and $x^2$ each correspond to a componenet of a vector). After doing this, we can convert the problem into a set of vectors like the following: $\left\{ \begin{bmatrix} 1 \\ 0 \\ 0 \end{bmatrix}, \begin{bmatrix} 0 \\ \frac{1}{\sqrt{2}} \\ \frac{1}{\sqrt{2}} \end{bmatrix}, \begin{bmatrix} 1 \\ -\frac{1}{\sqrt{2}} \\ 0 \end{bmatrix}\right\}$. Now, to check if these vectors are orthogonal, we can see if the inner product of each pair of vectors is zero.
        %     \begin{align*}
        %         \langle \begin{bmatrix} 1 \\ 0 \\ 0 \end{bmatrix},  \begin{bmatrix} 0 \\ \frac{1}{\sqrt{2}} \\ \frac{1}{\sqrt{2}} \end{bmatrix}\rangle &= 0 \\
        %         \langle \begin{bmatrix} 1 \\ 0 \\ 0 \end{bmatrix},  \begin{bmatrix} 1 \\ -\frac{1}{\sqrt{2}} \\ 0 \end{bmatrix}\rangle &= 1
        %     \end{align*}
        %     We can see that even though the first pair of vectors has an inner product of 0, the second pair has a non-zero inner product, which means they are linearly dependent. Because of this, this set of vectors is not orthogonal.
        % }
    \end{enumerate}


\end{enumerate}

