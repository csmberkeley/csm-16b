% Author: Jessica Lin
% Email: jessica.jx.lin@berkeley.edu
% CSM16A Fall 2022

\qns{Matrix Multiplication Mania}

\textbf{Learning Goal:} The goal of this question is to 1 understand the use of matrices given a context, and 2 understand matrix-matrix, matrix-vector, and vector-vector multiplication.

\begin{enumerate}

\item Alice, Bob, and Carl went shopping at the same grocery store. They would like to know the price of each fruit they bought, but 
they unfortunately lost their receipts. Luckily, they each remember what they bought and how much they spent. 
Alice bought 2 apples and 14 tomatoes for \$9; Bob bought 4 pears, 1 apple, and 2 tomatoes for \$8; Carl bought 2 pears and 1 apple for \$4.
Set up a matrix-vector equation that would allow you to solve for the price of each fruit.

\ans{
    We define $a$ to be the price of an apple, $p$ to be the price of a pear, and $t$ to be the price of a tomato.
    We first write out the system of equations without matrices. To facilitate converting this system of equations to
    a matrix, we write each equation with all three variables (placing coefficients of 0 where appropriate) and order
    the variables in each equation in the same way.
    
    \begin{center}
        $2a + 0p + 14t = 9$
        
        $1a + 4p + 2t = 8$
        
        $1a + 2p + 0t = 4$
        
        \begin{align*}
        \begin{bmatrix} 2 & 0 & 14 \\ 1 & 4 & 2 \\ 1 & 2 & 0 \end{bmatrix} \begin{bmatrix} a \\ p \\ t \end{bmatrix} = \begin{bmatrix} 9 \\ 8 \\ 4 \end{bmatrix}
        \end{align*}
        
    \end{center}
   
}

%\vspace{0.7cm}

\item Evaluate each of the following expressions, if possible. If the expression cannot be evaluated, give a brief justification why. (\textit{Hint}: Remember to check dimensions before multiplying!)

\vspace{0.5em}

\begin{center}
\begin{tabular} {c c c c}
$A = \begin{bmatrix}
        1 & -2 & 3 \\
        2 & 4 & 6 \\
        -3 & -5 & -7
    \end{bmatrix}$
&
$B = \begin{bmatrix}
        0 & 1 \\
        2 & 4 \\
        -3 & -1 \\
    \end{bmatrix}$
&           
$\vec{c} = \begin{bmatrix}
                1 \\
                2 \\
                -3
            \end{bmatrix}$
&
$\vec{d} = \begin{bmatrix}
                -2 \\ 3 \\ -5
            \end{bmatrix}$
            
\end{tabular}
\end{center}

\vspace{0.5em}

\begin{enumerate}
    \item $A\vec{c}$
    
        \ans{
            \begin{align*}
                A\vec{c} &= \begin{bmatrix} 1 & -2 & 3 \\ 2 & 4 & 6 \\ -3 & -5 & -7 \end{bmatrix} \begin{bmatrix} 1 \\ 2 \\ -3 \end{bmatrix}
                = \begin{bmatrix} (1)(1) + (-2)(2) + (3)(-3) \\ (2)(1) + (4)(2) + (6)(-3) \\ (-3)(1) + (-5)(2) + (-7)(-3) \end{bmatrix}
                = \begin{bmatrix} -12 \\ -8 \\ 8 \end{bmatrix}
            \end{align*}
        }
    
    \begin{itemize}    
        \item {
            Compute $\vec{e} = (1) \cdot \begin{bmatrix} 1 \\ 2 \\ -3 \end{bmatrix} + 
            (2) \cdot \begin{bmatrix} -2 \\ 4 \\ -5 \end{bmatrix} + 
            (-3) \cdot \begin{bmatrix} 3 \\ 6 \\ -7 \end{bmatrix}$. How does this compare to $A\vec{c}$?
        }
        
        \meta{
            Students should understand that we can break down matrix-vector multiplication in this way:
            one way to view this is to consider 3 equations in a system of equations.
        }
        
        \ans{
            \begin{align*}
                (1) \cdot \begin{bmatrix} 1 \\ 2 \\ -3 \end{bmatrix} + 
                (2) \cdot \begin{bmatrix} -2 \\ 4 \\ -5 \end{bmatrix} + 
                (-3) \cdot \begin{bmatrix} 3 \\ 6 \\ -7 \end{bmatrix}
                = \begin{bmatrix} 1 \\ 2 \\ -3 \end{bmatrix} + 
                \begin{bmatrix} -4 \\ 8 \\ -10 \end{bmatrix} + 
                \begin{bmatrix} -9 \\ -18 \\ 21 \end{bmatrix}
                = \begin{bmatrix} -12 \\ -8 \\ 8 \end{bmatrix}
            \end{align*}
            They are the same.
        }
        
    \end{itemize}
        
    \item $\vec{c}A$
    
        \ans{$\vec{c}$ is a 3 by 1 vector, while matrix A is a 3 by 3 matrix. The dimensions do not align to allow for matrix multiplication.}
        
    \item $\vec{d}^T A$
    
        \ans{
            \begin{align*}
                \vec{d}^T A &= \begin{bmatrix} -2 & 3 & -5 \end{bmatrix} \begin{bmatrix} 1 & -2 & 3 \\ 2 & 4 & 6 \\ -3 & -5 & -7 \end{bmatrix} \\
                &= \begin{bmatrix} (-2)(1) + (3)(2) + (-5)(-3) & (-2)(-2) + (3)(4) + (-5)(-5) & (-2)(3) + (3)(6) + (-5)(-7) \end{bmatrix} \\
                &= \begin{bmatrix} 19 & 41 & 47 \end{bmatrix}
            \end{align*}
        }
        
    \begin{itemize}
        \item {
        Compute $\vec{f} = (-2) \cdot \begin{bmatrix} 1 & -2 & 3 \end{bmatrix} +
        (3) \cdot \begin{bmatrix} 2 & 4 & 6 \end{bmatrix} + 
        (-5) \cdot \begin{bmatrix} -3 & -5 & -7 \end{bmatrix}$.
        
        How does this compare to $\vec{d}^TA$?
    }
    
    \ans{
        \begin{align*}
            \vec{f} & = \begin{bmatrix} -2 & 4 & -6 \end{bmatrix} + \begin{bmatrix} 6 & 12 & 18 \end{bmatrix} + \begin{bmatrix} 15 & 25 & 35 \end{bmatrix}
            = \begin{bmatrix} 19 & 41 & 47 \end{bmatrix}
        \end{align*}
        They are the same.
    }
    \end{itemize}
        
    \item $\vec{c}\vec{d}^T$
    
        \meta{
            \begin{itemize}
                \item Though the term outer-product has not yet been introduced in 16A, mentors could consider bringing up this terminology.
                \item Mentors should make sure to compare/contrast $\vec{c}\vec{d}^T$ (this part) to $\vec{d}^T\vec{c}$ (the next part). Consider drawing attention to the dimensions of the input vectors and the dimensions of the output matrix, and how multiplying the vectors in a different order changes the product significantly.
            \end{itemize}
        }
        
        \ans{
            \begin{align*}
                \vec{c}\vec{d}^T & = \begin{bmatrix} 1 \\ 2 \\ -3 \end{bmatrix} \begin{bmatrix} -2 & 3 & -5 \end{bmatrix}
                = \begin{bmatrix} (1)(-2) & (1)(3) & (1)(-5) \\ (2)(-2) & (2)(3) & (2)(-5) \\ (-3)(-2) & (-3)(3) & (-3)(-5) \end{bmatrix}
                = \begin{bmatrix} -2 & 3 & -5 \\ -4 & 6 & -10 \\ 6 & -9 & 15 \end{bmatrix}
            \end{align*}
        }
    
    \item $\vec{d}^T\vec{c}$
    
        \meta{
            Mentors can connect this problem to the concept of inner product since we have a multiplication of two vectors.
        }

        \ans{
            \begin{align*}
                \vec{d}^T\vec{c} = \begin{bmatrix} -2 & 3 & -5 \end{bmatrix} \begin{bmatrix} 1 \\ 2 \\ -3 \end{bmatrix} 
                = \begin{bmatrix} (1)(-2) + (2)(3) + (-3)(-5) \end{bmatrix} = 19
            \end{align*}
        }

    \item $AB$
            
        \ans{
            \begin{align*}
                AB & = \begin{bmatrix} 1 & -2 & 3 \\ 2 & 4 & 6 \\ -3 & -5 & -7 \end{bmatrix} \begin{bmatrix} 0 & 1 \\ 2 & 4 \\ -3 & -1 \\ \end{bmatrix} \\ 
                & = \begin{bmatrix} (1)(0) + (-2)(2) + (3)(-3) & (1)(1) + (-2)(4) + (3)(-1) \\ (2)(0) + (4)(2) + (6)(-3) & (2)(1) + (4)(4) + (6)(-1) 
                    \\ (-3)(0) + (-5)(2) + (-7)(-3) & (-3)(1) + (-5)(4) + (-7)(-1) \end{bmatrix} \\
                & = \begin{bmatrix} -13 & -10 \\ -10 & 12 \\ 11 & -16 \end{bmatrix}
            \end{align*}
        }
    \end{enumerate}

\end{enumerate}
