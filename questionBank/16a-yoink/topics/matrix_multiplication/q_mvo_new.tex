\qns{Computations: Matrix-Vector Operations}

\textbf{Learning Goal:} The goal of this problem is to present various cases of matrix-vector operations such as addition, multiplication, and transpose.

\meta{ 

\begin{itemize}
%\item To save time,  make sure to walk through this question with the students rather than having them solve it on their own. 

\item It would be helpful to emphasize to students that for matrix vector operations, the dimensions of the matrices and vectors must match to do any type of addition,  subtraction,  or multiplication operations.

\item Matrix multiplication is not commutative.

\item In some cases, matrix-matrix multiplication may be invalid.

\end{itemize}

}

Consider the following matrices and vectors. Complete the parts below.
\begin{align*}
A = \begin{bmatrix} 2 & 4 \\ 5 & -3 \end{bmatrix} \quad B = \begin{bmatrix} 1 & 3 \\ 2 & -4  \end{bmatrix} \quad \vec{u_1} = \begin{bmatrix} 1 \\ 2 \end{bmatrix} \quad \vec{u_2} = \begin{bmatrix} 3 \\ -4 \end{bmatrix} \quad \vec{v} = \begin{bmatrix} 1 \\ 2 \\ 3 \end{bmatrix} \quad \vec{w} = \begin{bmatrix} 0 \\ -1 \\ 4 \end{bmatrix}
\end{align*}


\begin{enumerate}
 %\setlength\itemsep{5em}
    \item What is the transpose of v?

    \ans{
    \[\vec{v}^{T} = \begin{bmatrix} 1 & 2 & 3 \end{bmatrix}\]
    }

    \item What is $(\vec{v}+\vec{w})^T$? Find $\vec{v}^T+\vec{w}^T$ too. Compare the results.

    \ans{
    \[\vec{v}+\vec{w}=\begin{bmatrix} 1 \\ 2 \\ 3 \end{bmatrix} + 
    \begin{bmatrix} 0 \\ -1 \\ 4 \end{bmatrix} = 
    \begin{bmatrix} 1 + 0 \\ 2 + (-1) \\ 3 + 4 \end{bmatrix} = \begin{bmatrix} 1 \\ 1 \\ 7 \end{bmatrix}\] 
    So $(\vec{v}+\vec{w})^T=\begin{bmatrix} 1 & 1 & 7 \end{bmatrix}$.\\
    For the second part we have:
    \begin{align*}
    \vec{v}^T=\begin{bmatrix} 1 & 2 & 3 \end{bmatrix}\\
    \vec{w}^T=\begin{bmatrix} 0 & -1 & 4 \end{bmatrix}
    \end{align*}
    Hence $\vec{v}^T+ \vec{w}^T=\begin{bmatrix} 1 & 1 & 7 \end{bmatrix}$. From these results we can see that: $(\vec{v}+\vec{w})^T=\vec{v}^T+\vec{w}^T$

    }

    \item What is 2$\vec{v}$ - 4$\vec{w}$?

    \ans{
    \[2\begin{bmatrix} 1 \\ 2 \\ 3 \end{bmatrix} - 
    4\begin{bmatrix} 0 \\ -1 \\ 4 \end{bmatrix} = 
    \begin{bmatrix} 2(1) \\ 2(2) \\ 2(3) \end{bmatrix} - \begin{bmatrix} 4(0) \\ 4(-1) \\ 4(4) \end{bmatrix} = \begin{bmatrix} 2 \\ 4 \\ 6 \end{bmatrix} - 
    \begin{bmatrix} 0 \\ -4 \\ 16 \end{bmatrix} = 
    \begin{bmatrix} 2 - 0 \\ 4 + 4 \\ 6 - 16 \end{bmatrix} = \begin{bmatrix} 2 \\ 8 \\ -10 \end{bmatrix}\]
    }

    \item What is $\vec{v}^{T}\vec{w}$?

    \meta{
    Make sure that students understand the orientation of the vectors (i.e row vs column). 
    \begin{itemize}
    \item A row multiplied by a column would equal a scalar.
    \item A column multiplied by a row would equal a matrix.
    \end{itemize}
    }

    \ans{
    The dimension of $\vec{v}^{T}$ is $1\times 3$ and the dimension of $\vec{w}$ is $3\times 1$. Since the product of an $m\times n$ and an $n\times p$ matrix results in an $m\times p$ matrix, the output in this part is going to be a scalar (i.e. $1\times 1$).
    \[ \begin{bmatrix} 1 & 2 & 3 \end{bmatrix} \begin{bmatrix} 0 \\ -1 \\ 4 \end{bmatrix}  = 
    \begin{bmatrix} (1)(0) + (2)(-1) + (3)(4) \end{bmatrix} = \begin{bmatrix} 0 - 2 + 12 \end{bmatrix} = 10 \] 
    }

    \item What is $A\vec{u_1}$? What is $A\vec{u_2}$?


    \ans{
    The dimension of $\mathbf{A}$ is $2\times 2$ and the dimension of $\vec{u_1}$ is $2\times 1$. Therefore, the output in this part is going to have a dimension of $2\times 1$.
    \[\mathbf{A}\vec{u_1}=\begin{bmatrix} 2 & 4 \\ 5 & -3 \end{bmatrix} \begin{bmatrix} 1 \\ 2 \end{bmatrix} = 
    \begin{bmatrix}(2)(1) + (4)(2)\\(5)(1) +  (-3)(2)\end{bmatrix} = 
    \begin{bmatrix} 2 + 8 \\ 5 - 6 \end{bmatrix} = \begin{bmatrix} 10 \\ -1 \end{bmatrix}\]
    \[\mathbf{A}\vec{u_2}=\begin{bmatrix} 2 & 4 \\ 5 & -3 \end{bmatrix} \begin{bmatrix} 3 \\ -4 \end{bmatrix} = 
    \begin{bmatrix}(2)(3) + (4)(-4)\\(5)(3) +  (-3)(-4)\end{bmatrix} = 
    \begin{bmatrix} 6  -16 \\ 15 +12 \end{bmatrix} = \begin{bmatrix} -10 \\ 27 \end{bmatrix}\]
    }

    \item What is $\mathbf{A}\mathbf{B}$? (Do the columns of $\mathbf{A}\mathbf{B}$ look familiar?)

    \meta{
    Highlight that a matrix can be represented as a vertical stack of rows or as well as a horizontal stack of columns. We think of the first matrix (A) as a set of rows, while we think of the second matrix (B) as a set of columns. 
    }

    \ans{
        The dimension of both $\mathbf{A}$ and $\mathbf{B}$ is $2\times 2$. Therefore, the output in this part is going to have a dimension of $2\times 2$.

        \[\begin{bmatrix} 2 & 4 \\ 5 & -3 \end{bmatrix} \begin{bmatrix} 1 & 3 \\ 2 & -4  \end{bmatrix} =
        \begin{bmatrix} (2)(1) + (4)(2) & (2)(3) + (4)(-4) \\ (5)(1) + (-3)(2) & (5)(3) + (-3)(-4)\end{bmatrix} = 
        \begin{bmatrix} 2 + 8 & 6 - 16 \\ 5 - 6 & 15 + 12 \end{bmatrix} =
        \begin{bmatrix} 10 & -10 \\ -1 & 27 \end{bmatrix}
        \]
        We can observe that the columns of $AB$ are the same as the results from the last part. This is because the columns of matrix $B$ are the same as vectors $u_1$ and $u_2$, i.e.
        \begin{align*}
        \mathbf{B}=\begin{bmatrix} | & | \\ \vec{u_1} & \vec{u_2}\\| & | \end{bmatrix}\\
        \mathbf{A}\mathbf{B}=\mathbf{A}\begin{bmatrix} \vec{u_1} & \vec{u_2}\end{bmatrix}= \begin{bmatrix} \mathbf{A}\vec{u_1} & \mathbf{A}\vec{u_2}\end{bmatrix}
        \end{align*}
    }
    
    \item Find $\mathbf{B}^T$. Then express $\mathbf{B}^T$ in terms of $\vec{u_1}$ and $\vec{u_2}$? 

    \ans{
        \begin{align*}
        \mathbf{B}^T=\begin{bmatrix} 1 & 2 \\ 3 & -4 \end{bmatrix}\\
        \mathbf{B}^T=\begin{bmatrix} \vec{u_1} & \vec{u_2}\end{bmatrix}^T=\begin{bmatrix} \vec{u_1}^T \\ \vec{u_2}^T\end{bmatrix}
        \end{align*}
    }

\end{enumerate}
%\newpage