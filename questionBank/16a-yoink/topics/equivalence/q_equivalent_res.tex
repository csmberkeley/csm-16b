%% summer 2020 MT2 probeem
%% Panos Zarkos panzarkos@berkeley.edu 

\qns{Equivalence in Resistive Networks}

\textbf{Learning Goal: Students will practice finding equivalences when given circuits with resistors in series and/or parallel. } 

Please look into \notes{Note 15 Section 15.7.1 - 15.7.2} to see examples of finding circuit equivalences with resistors in series and resistors in parallel.

\meta{ 
Emphasize the idea of simplifying parts of the circuit one step at a time.
}

For all of the following networks find an expression or a numerical value for the equivalent resistance between terminals A and B.

\textit{Hint:} You can use the equivalence formulas for series and parallel combinations of resistors for all of the subparts in this question.

\begin{enumerate}
    \item {
    \newline
    \begin{minipage}{\linewidth}
        \begin{center}
        \begin{circuitikz}
        	\draw (0, 0) node[anchor=east]{A}
        				 to [R=$R_1$, o-] ++ (3, 0)
        		         to [R=$R_5$] ++ (0, -3)
        		         to [R=$R_4$, -o] ++ (-3, 0)
        		         node[anchor=east]{B};
        	\draw (3, 0) to [R=$R_2$] ++ (3, 0)
        				 to [R=$R_3$] ++ (0, -3)
        				 to [short] ++ (-3, 0);
        \end{circuitikz}
        \end{center}
    \end{minipage}
    
    }
     
%     \begin{enumerate} [(A)]
%         \item{
%    	$ R_{AB} = R_1 + R_3 || (R_2 +R_5) + R_4$
%	}  
%	
%	\item{
%    	$ R_{AB} = R_1 + R_5 || (R_2 +R_3) + R_4$
%	}            
%	
%	\item{
%    	$ R_{AB} = R_1 + R_4 || (R_2 +R_3) + R_5 $
%	}            
%	
%	\item{
%	$ R_{AB} = R_1 + R_2 || (R_3 +R_5) + R_4 $
%    	}            
%	
%	\item{
%    	$ R_{AB} = R_1 + R_3 || (R_2 +R_4) + R_5$
%	}            
%     \end {enumerate}
     


    
    \ans{
    
        \begin{center}
        \begin{circuitikz}
        	\draw (0, 0) node[anchor=east]{A}
        				 to [R=$R_1$, o-] ++ (3, 0)
        		         to [R=$R_5$, *-*] ++ (0, -3)
        		         to [R=$R_4$, -o] ++ (-3, 0)
        		         node[anchor=east]{B};
        	\draw (3, 0) to [R=$R_2$] ++ (3, 0)
        				 to [R=$R_3$] ++ (0, -3)
        				 to [short] ++ (-3, 0);
	\draw (3, 0) node[anchor=south]{C};
	\draw (3, -3) node[anchor=north]{D};
        \end{circuitikz}
        \end{center}

	The easiest way to simplify a network if unsure, is to label more nodes and try to make smaller simplifications steps. For example, after introducing terminals C and D, we can see that we can represent the equivalent resistance in between those terminals as $R_5$ in parallel with the series combination of $R_2$ and $R_3$, so $R_{CD} = R_5 || (R_2 +R_3)$. Finally, we can see that $R_{CD}$ is in series with both $R_1$ and $R_4$, as we look right from terminals A, B. This gives us the final result: $ R_{AB} = R_1 + R_5 || (R_2 +R_3) + R_4$.
    
    }
        
%    \hspace{1cm}
%    \item \ 
%    \newline
%    \begin{minipage}{\linewidth}
%        \begin{center}
%        \begin{circuitikz}
%        	\draw (0, 0) node[anchor=east]{A}
%        				 to [short, o-] ++ (1, 0)
%        				 to [C=$C_1$] ++ (0, -3)
%        				 to [C=$C_5$] ++ (0, -3)
%        				 to [short, -o] ++ (-1, 0)
%        				 node[anchor=east]{B};
%        	\draw (1, 0) to [short] ++ (3, 0)
%        				 to [C=$C_2$] ++ (0, -2)
%        				 to [C=$C_3$] ++ (0, -2)
%        				 to [C=$C_4$] ++ (0, -2)
%        				 to [short] ++ (-3, 0);
%        \end{circuitikz}
%        \end{center}
%        \end{minipage}
%%        
%%             \begin{enumerate} [(A)]
%%         \item{
%%    	$C_{AB} = (C_1 || C_5) + (C_2 || C_3 || C_4)$
%%	}  
%%	
%%	\item{
%%    	$C_{AB} = C_1 || C_5 || C_2 || C_3 || C_4$
%%	}            
%%	
%%	\item{
%%    	$C_{AB} = (C_1 || C_2) + (C_5 || C_3 || C_4)$
%%	}            
%%	
%%	\item{
%%	$C_{AB} = (C_1 || C_4) + (C_5 || C_3 || C_2)$
%%    	}            
%%	
%%	\item{
%%    	$C_{AB} = C_1 + C_5 + C_2 + C_3 + C_4$
%%	}            
%%     \end {enumerate}
%
%
%    
%    \ans{
%    Here we have two branches connected in parallel, one including capacitors $C_1$, $C_5$ (which are connected in series) and one including capacitors $C_2$, $C_3$, and $C_4$ (which are also connected in series). Hence, the correct answer is: $C_{AB} = (C_1 || C_5) + (C_2 || C_3 || C_4)$.
%
%    }
    
    \item {
    \newline
    \begin{minipage}{\linewidth}
        \begin{center}
        \begin{circuitikz}
        	\draw (0, 0) to [R=$R_1$] ++ (3, 0)
        				 to [R=$R_5$] ++ (0, -3)
        				 to [R=$R_4$] ++ (-3, 0);
        	\draw (3, 0) to [R=$R_2$] ++ (3, 0)
        				 to [R=$R_3$] ++ (0, -3)
        				 to [short] ++ (-3, 0);
        	\draw (6, 0) to [short, -o] ++ (1, 0)
        				 node[anchor=west]{A};
        	\draw (6, -3) to [short, -o] ++ (1, 0)
        				 node[anchor=west]{B};
        \end{circuitikz}
        \end{center}
    \end{minipage}
    }
        
%                     \begin{enumerate} [(A)]
%         \item{
%    	$R_{AB} = R_2 || (R_3+ R_5)$
%	}  
%	
%	\item{
%    	$R_{AB} = R_3 || (R_2+ R_5)$
%	}            
%	
%	\item{
%        $R_{AB} = R_3 || (R_2+ R_5) + R_1+R_2$
%	}            
%	
%	\item{
%	$R_{AB} = R_2 || (R_3+ R_5) + R_1 + R_2$
%    	}            
%	
%	\item{
%    	$R_{AB} = R_5 || (R_1+R_3)$
%	}            
%     \end {enumerate}

\meta {
For part (b) make sure that students understand why $R_1$ and $R_4$ don’t matter.
}
    
    \ans{
    In this question it is crucial to realize, resistors $R_1$ and $R_4$ are connected to an open-circuit so they don't contribute to the overall resistance looking left from terminals A, B. After we delete them from our circuit diagram, we can get the answer as $R_{AB} = R_3 || (R_2+ R_5)$.
    }
    
%    \item
%        \begin{center}
%        \begin{circuitikz}
%        	\draw (0, 0) node[anchor=east]{A}
%        				 to [short, o-] ++ (1, 0)
%        				 to [C=$C_1$] ++ (0, -3)
%        				 to [C=$C_2$] ++ (0, -3);
%        	\draw (0, -3) node[anchor=east]{B}
%        				 to [short, o-] ++ (1, 0);
%        	\draw (1, 0) to [short] ++ (3, 0)
%        				 to [C=$C_3$] ++ (0, -2)
%        				 to [C=$C_4$] ++ (0, -2)
%        				 to [C=$C_5$] ++ (0, -2)
%        				 to [short] ++ (-3, 0);
%        \end{circuitikz}
%        \end{center}
%        
%                     \begin{enumerate} [(A)]
%         \item{
% %   	$$
%	}  
%	
%	\item{
%    %	$$
%	}            
%	
%	\item{
%    	%$$
%	}            
%	
%	\item{
%%	$$
%    	}            
%	
%	\item{
%   % 	$$
%	}            
%     \end {enumerate}
%
%    \ans{
%    The correct answer is
%    }
    \item {
    \begin{minipage}{\linewidth}
        \begin{center}
        \begin{circuitikz}
        	\draw (0, 0) to [short] ++ (0, 6)
        				 to [short] ++ (3, 0)
        				 to [R=$5\Omega$] ++ (0, -3)
        				 to [R=$7\Omega$] ++ (0, -3)
        				 to [short] ++ (-3, 0);
            \draw (3, 3) to [short, -o] ++ (1, 0)
            			 node[anchor=west]{A};
            \draw (3, 0) to [short, -o] ++ (1, 0)
            			 node[anchor=west]{B};
        \end{circuitikz}
        \end{center}
    \end{minipage}
    }
        
%                     \begin{enumerate} [(A)]
%         \item{
%    	{$R_{AB} = 2.917\Omega$}
%	}  
%	
%	\item{
%    	$R_{AB} =  0.343\Omega$
%	}            
%	
%	\item{
%    	$R_{AB} = 0.417\Omega$
%	}            
%	
%	\item{
%	$R_{AB} = 12\Omega$
%    	}            
%	
%	\item{
%    	$R_{AB} = 0.583\Omega$
%	}            
%     \end {enumerate}

    
    \ans{
    Here we need to notice that terminal B is connected not only to the bottom of the $7\Omega$ resistor, but also to the top of the $5\Omega$ resistor. Thus, the two resistors have both terminals common, are connected in parallel and the equivalent resistance can be computed as: $R_{AB} = \frac{7\Omega \cdot 5\Omega}{7\Omega + 5\Omega} = 2.917\Omega$.
    }
    
    \item {
    \newline
    \begin{minipage}{\linewidth}
        \begin{center}
        \begin{circuitikz}
        	\draw (0, 0) to [R=$R_4$] ++ (0, -3)
        				 to [R=$R_3$] ++ (0, -3)
        				 to [short] ++ (3, 0)
        				 to [short] ++ (0, 3)
        				 to [R=$R_1$] ++ (0, 3)
        				 to [short] ++ (-3, 0);
        	\draw (0, -3) to [short] ++ (3, 0);
        	\draw (3, 0) to [short] ++ (3, 0)
        				 to [R=$R_2$] ++ (0, -3)
        				 to [R=$R_5$] ++ (0, -3)
        				 to [short] ++ (-3, 0);
        	\draw (6, 0) to [short, -o] ++ (1, 0)
        				 node[anchor=west]{A}
        				 to [open] ++ (0, -6)
        				 node[anchor=west]{B}
        				 to [short, o-] ++ (-1, 0);
        \end{circuitikz}
        \end{center}
    \end{minipage}
    }
        
%                     \begin{enumerate} [(A)]
%         \item{
%         
%	$R_{AB} = R_4 || R_1 || (R_2+R_5)$
%	}
%	\item{
%    	$R_{AB} = ((R_4 || R_1) +R_3) || (R_2+R_5)$
%	}            
%	
%	\item{
%    	$R_{AB} = ((R_4 || R_1) + R_3)|| R_2 || R_5$
%	}            
%	
%	\item{
%	$R_{AB} = R_4 || R_1 || R_2 || R_5 $
%    	}            
%	
%	\item{
%    	$R_{AB} = (R_4 + R_1) || (R_2+R_5)$
%	}            
%     \end {enumerate}


    \ans{
    Notice that here resistor $R_3$ is shorted (in other words connected in parallel with a wire). This means that it effectively can be removed from the circuit diagram, which leaves us with $R_4$ in parallel with $R_1$ and in parallel with the series combination of $R_2$ and $R_5$, hence $R_{AB} = R_4 || R_1 || (R_2+R_5)$.
    }
\end{enumerate}
 
