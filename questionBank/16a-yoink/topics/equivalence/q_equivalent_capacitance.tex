%% summer 2020 MT2 probeem
%% Panos Zarkos panzarkos@berkeley.edu 

\qns{Equivalence in Capacitive Networks}

\textbf{Learning Goal:} This objective of this problem is to practice finding equivalent capacitance for series/parallel network of capacitors.

\textbf{Relevant Notes:} \notes{Note 16} derives the equivalent capacitance formula for series/ parallel capacitors.

For all of the following networks find an expression or a numerical value for the equivalent capacitance between terminals A and B.

\meta{Note that equivalent capacitance behaves in opposite ways to equivalent resistance: series and parallel are essentially "flipped". For capacitors, the parallel operator || is used in series.

Be sure to emphasize to students that the order of the capacitors in series don’t necessarily change the overall capacitance (i.e. $C_2$ || $C_3$ || $C_4$ = $C_4$ || $C_2$ || $C_3$).

}
\begin{enumerate}

\item

    \begin{minipage}{\linewidth}
        \begin{center}
        \begin{circuitikz}
        	\draw (0, 0) node[anchor=east]{A}
        				 to [short, o-] ++ (1, 0)
        				 to [C=$C_1$] ++ (0, -3)
        				 to [C=$C_5$] ++ (0, -3)
        				 to [short, -o] ++ (-1, 0)
        				 node[anchor=east]{B};
        	\draw (1, 0) to [short] ++ (3, 0)
        				 to [C=$C_2$] ++ (0, -2)
        				 to [C=$C_3$] ++ (0, -2)
        				 to [C=$C_4$] ++ (0, -2)
        				 to [short] ++ (-3, 0);
        \end{circuitikz}
        \end{center}
        \end{minipage}
        



    
    \ans{
    Here we have two branches connected in parallel, one including capacitors $C_1$, $C_5$ (which are connected in series) and one including capacitors $C_2$, $C_3$, and $C_4$ (which are also connected in series). The equivalent capacitance of the left branch is $C_1 || C_5$, where $||$ is the parallel operator (i.e. $a||b = \frac{ab}{a+b}$). Similarly, for the right branch, the equivalent capacitance is $C_2 || C_3 || C_4$. Since the two branches are in parallel, we can sum up their equivalent capacitances:
    $$C_{AB} = (C_1 || C_5) + (C_2 || C_3 || C_4)$$

    }
    
\item
\begin{minipage}{\linewidth}
        \begin{center}
        \begin{circuitikz}
        	\draw (0, 0) node[anchor=east]{A}
        				 to [short, o-] ++ (1, 0)
        				 to [C=$C_1$] ++ (0, -3)
        				 to [C=$C_2$] ++ (0, -3);
        	\draw (0, -3) node[anchor=east]{B}
        				 to [short, o-] ++ (1, 0);
        	\draw (1, 0) to [short] ++ (3, 0)
        				 to [C=$C_3$] ++ (0, -2)
        				 to [C=$C_4$] ++ (0, -2)
        				 to [C=$C_5$] ++ (0, -2)
        				 to [short] ++ (-3, 0);
        \end{circuitikz}
        \end{center}
\end{minipage}

	\ans{
	
	Here, we have two branches connected in parallel, with the first branch containing only $C_1$. The second branch contains $C_3$, $C_4$, $C_5$, and $C_2$ in series. The equivalent capacitance of the right branch is $C_2 || C_3 || C_4 || C_5$. Then, we can sum this up with $C_1$ from the left branch to get:
	$$C_{AB} = C_1 + (C_2 || C_3 || C_4 || C_5)$$
	}	

\end{enumerate}
 
