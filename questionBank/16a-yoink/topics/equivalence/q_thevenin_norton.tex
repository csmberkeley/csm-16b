% Author: Jessica Lin
% Email: jessica.jx.lin@berkeley.edu
% CSM16A Fall 2022

\newcommand*{\equal}{=}

\qns{Equivalence}

\textbf{Learning Goal:} To build an understanding of Thevenin and Norton equivalent circuits.

\meta{With complex circuits, we often want to find ways to simplify them. Looking from the view of two terminals in a circuit, we can say that two different circuits are \textbf{equivalent} if they show the same $I-V$ relationship between the two terminals. The Thevenin and Norton equivalent circuits are two configurations to which we can simplify our circuits.

This equivalence is helpful because the I-V behavior at the terminals should be the same, and we do not need to consider a much larger, much more complex circuit each time we want to add some arbitrary load resistor.}
\begin{enumerate}

\begin{center}
\begin{circuitikz}
\draw (0, 0)
to [V, v = $V_s$, invert] (0, 3)
to [R = $R_1$] (3, 3)
to [R = $R_2$] (3, 0)
to [short] (0, 0);

\draw (3, 3)
to [short, -o] (4, 3) node[label={[font=\footnotesize]:$A$}]{}
node[label={[font=\footnotesize]below:$+$}]{};

\draw (3, 0)
to [short, -o] (4, 0) 
node[label={[font=\footnotesize]:$-$}]{}
node[label={[font=\footnotesize]below:$B$}]{};

\node[ground] (0, -1) {};

\end{circuitikz}
\hspace{2cm}
\begin{circuitikz} 
\draw (0, 0)
to [isource, i = $I_s$,] (0, 3)
to [short, -*] (2.5, 3)
to [short] (5, 3)
to [R = $R_2$] (5, 0)
to [short] (0, 0);

\draw (2.5, 3)
to [R = $R_1$] (2.5, 0);

\draw (5, 3)
to [short, -o] (6, 3) node[label={[font=\footnotesize]:$A$}]{}
node[label={[font=\footnotesize]below:$+$}]{};

\draw (5, 0)
to [short, -o] (6, 0) 
node[label={[font=\footnotesize]:$-$}]{}
node[label={[font=\footnotesize]below:$B$}]{};

\node[ground] (0, -1) {};

\end{circuitikz}
\end{center}

\item Find $V_{Th}$, $I_{No}$, $R_{Th}$, and $R_{No}$ for the voltage divider circuit.

\ans{

We begin by calculating $V_{Th}$ for this circuit. $V_{Th}$ is the open circuit voltage and is equal to $V_{AB}$. Because this circuit is a voltage divider, we know that $V_{Th} = V_{AB} = \frac{R_2}{R_1 + R_2}$.

Now, we solve for the equivalent resistance between terminals $A$ and $B$. First, zero out all independent sources: the voltage source becomes a wire, so we are left with the following circuit configuration:

\begin{center}
\begin{circuitikz}
\draw (0, 0)
to (0, 3)
to [R = $R_1$] (3, 3)
to [R = $R_2$] (3, 0)
to [short] (0, 0);

\draw (3, 3)
to [short, -o] (4, 3) node[label={[font=\footnotesize]:$A$}]{}
node[label={[font=\footnotesize]below:$+$}]{};

\draw (3, 0)
to [short, -o] (4, 0) 
node[label={[font=\footnotesize]:$-$}]{}
node[label={[font=\footnotesize]below:$B$}]{};

\draw[->] (5, 1.5) -- (6, 1.5);
\end{circuitikz}
\hspace{0.5cm}
\begin{circuitikz}
\draw (0, 0)
to [R = $R_1$] (0, 3)
to (3, 3)
to [R = $R_2$] (3, 0)
to [short] (0, 0);

\draw (3, 3)
to [short, -o] (4, 3) node[label={[font=\footnotesize]:$A$}]{}
node[label={[font=\footnotesize]below:$+$}]{};

\draw (3, 0)
to [short, -o] (4, 0) 
node[label={[font=\footnotesize]:$-$}]{}
node[label={[font=\footnotesize]below:$B$}]{};

\end{circuitikz}
\end{center}

So resistors $R_1$ and $R_2$ are in parallel, and the equivalent resistance between nodes $A$ and $B$ is $R_{eq} = \frac{R_1R_2}{R_1 + R_2}$. The equivalent resistance between the two terminals is the same as $R_{Th}$, which is equal to $R_{No}$.

Now, we can solve for $I_{No}$. We have a couple options available to us: the simplest ones are to (1) solve for the short circuit current $I_{SC}$ from terminal $A$ to $B$, or (2) compute $I_{No} = \frac{V_{Th}}{R_{No}}$. 

If we short terminals $A$ and $B$, we have the following circuit:

\begin{center}
\begin{circuitikz}
\draw (0, 0)
to [V, v = $V_s$, invert] (0, 3)
to [R = $R_1$, i = $i_{1}$] (3, 3)
to [R = $R_2$] (3, 0)
to [short] (0, 0);

\draw (3, 3)
to [short, -*] (4, 3) node[label={[font=\footnotesize]:$A$}]{}
node[label={[font=\footnotesize]right:$+$}]{};

\draw (3, 0)
to [short, -*] (4, 0) 
node[label={[font=\footnotesize]right:$-$}]{}
node[label={[font=\footnotesize]below:$B$}]{};

\draw (4, 3) to [short, i = $I_{SC}$] (4, 0);

\node[ground] (0, -1) {};

\draw[->] (5, 1.5) -- (6, 1.5);

\end{circuitikz}
\begin{circuitikz}
\draw (0, 0)
to [V, v = $V_s$, invert] (0, 3)
to [R = $R_1$, i = $i_{1}$] (3, 3)
to [short, i = $I_{SC}$] (3, 0)
to [short] (0, 0);

\draw (3, 3)
to [short, -o] (4, 3) node[label={[font=\footnotesize]:$A$}]{}
node[label={[font=\footnotesize]right:$+$}]{};

\draw (3, 0)
to [short, -o] (4, 0) 
node[label={[font=\footnotesize]right:$-$}]{}
node[label={[font=\footnotesize]below:$B$}]{};

\node[ground] (0, -1) {};

\end{circuitikz}
\end{center}

Then, we notice that the short circuit current is the same as the current through the $R_1$ resistor. We can then use Ohm's Law to determine the Norton current $I_{No} = I_{SC} = i_{1} = \frac{V_s}{R_1}$.

So we have $V_{Th} = \frac{R_2}{R_1 + R_2}V_s$, $I_{No} = \frac{V_s}{R_1}$, $R_{Th} = R_{No} = \frac{R_1R_2}{R_1 + R_2}$.
}

\item Find $V_{Th}$, $I_{No}$, $R_{Th}$, and $R_{No}$ for the current divider circuit.

\ans{
Let's begin by finding $I_{No}$ for this circuit. First, we short terminals $A$ and $B$: 

\begin{center}
\begin{circuitikz}

\draw (0, 0)
to [I, l = $I_s$,] (0, 3)
to [short] (2.5, 3)
to [short] (5, 3)
to [R = $R_2$] (5, 0)
to [short] (0, 0);

\draw (2.5, 3)
to [R = $R_1$] (2.5, 0);

\draw (5, 3)
to [short, -*] (6, 3) node[label={[font=\footnotesize]:$A$}]{}
node[label={[font=\footnotesize]right:$+$}]{};

\draw (5, 0)
to [short, -*] (6, 0) 
node[label={[font=\footnotesize]right:$-$}]{}
node[label={[font=\footnotesize]below:$B$}]{};

\draw (6, 3) to [short, i = $I_{SC}$] (6, 0);

\node[ground] (0, -1) {};

\end{circuitikz}
\end{center}

Notice that the short circuit current $I_{SC}$ is the same current as the source current $I_s$. So we have $I_{No} = I_s$.

Now, let's find the equivalent resistance between the terminals $A$ and $B$. We zero out all independent sources. The current source becomes an open circuit, and we're left with two resistors in parallel: 

\begin{center}
\begin{circuitikz}
\draw (0, 0)
to [R = $R_1$] (0, 3)
to (3, 3)
to [R = $R_2$] (3, 0)
to [short] (0, 0);

\draw (3, 3)
to [short, -o] (4, 3) node[label={[font=\footnotesize]:$A$}]{}
node[label={[font=\footnotesize]below:$+$}]{};

\draw (3, 0)
to [short, -o] (4, 0) 
node[label={[font=\footnotesize]:$-$}]{}
node[label={[font=\footnotesize]below:$B$}]{};

\end{circuitikz}
\end{center}

Hence, we know the equivalent resistance is $R_{eq} = \frac{R_1R_2}{R_1 + R_2}$, and this is the same as $R_{Th}$ and $R_{No}$.

Now, we can find $V_{Th}$. We can solve for the open circuit voltage between the two terminals, or we can use the Ohm's Law $V_{Th} = I_{No}R_{Th}$ equation to find the Thevenin voltage:  $V_{Th} = I_{No}R_{Th} = I_s\frac{R_1R_2}{R_1 + R_2}$.

So we have $V_{Th} = I_s\frac{R_1R_2}{R_1 + R_2}$, $I_{No} = I_s$, $R_{Th} = R_{No} = \frac{R_1R_2}{R_1 + R_2}$.

Alternatively, we could have realized that collapsing the $R_1$ and $R_2$ resistors in parallel would have yielded the Norton equivalent circuit configuration, and determined $V_{Th}$, $I_{No}$, $R_{Th}$, and $R_{No}$ from there.
}

\item Suppose in the voltage divider circuit, $R_1 = 3\Omega$, $R_2 = 6\Omega$, and $V_s = 9V$. Add a load resistor of $2\Omega$ between terminals $A$ and $B$ of the voltage divider circuit. Using your answer from part (a), determine the voltage through this resistor. 

\ans{

From part (a), we know the equations for the Thevenin voltage and resistance: $V_{Th} = \frac{6}{3 + 6}\cdot 9 = 6V$ and $R_{Th} = \frac{3 \cdot 6}{3 + 6} = 2\Omega$. The Thevenin equivalent circuit is then as follows:

\begin{center}
\begin{circuitikz}
\draw (0, 0)
to [V, v = $V_{Th} \equal 6V$, invert] (0, 3)
to [R = $R_{Th} \equal 2\Omega$] (3, 3)
(3, 0) to [short] (0, 0);

\draw (3, 3)
to [short, -o] (4, 3) node[label={[font=\footnotesize]:$A$}]{}
node[label={[font=\footnotesize]below:$+$}]{};

\draw (3, 0)
to [short, -o] (4, 0) 
node[label={[font=\footnotesize]:$-$}]{}
node[label={[font=\footnotesize]below:$B$}]{};

\node[ground] (0, -1) {};

\end{circuitikz}
\end{center}

We connect the $2\Omega$ load resistor between the two terminals:

\begin{center}
\begin{circuitikz}
\draw (0, 0)
to [V, v = $V_{Th} \equal 6V$, invert] (0, 3)
to [R = $R_{Th} \equal 2\Omega$] (3, 3)
to [R = $R_L \equal 2\Omega$]
(3, 0) to [short] (0, 0);

\draw (3, 3)
to [short, -o] (4, 3) node[label={[font=\footnotesize]:$A$}]{}
node[label={[font=\footnotesize]below:$+$}]{};

\draw (3, 0)
to [short, -o] (4, 0) 
node[label={[font=\footnotesize]:$-$}]{}
node[label={[font=\footnotesize]below:$B$}]{};

\node[ground] (0, -1) {};

\end{circuitikz}
\end{center}

By inspection, this is just another voltage divider circuit; hence, we can calculate the voltage across the load resistor as follows: $V_L = \frac{2}{2 + 2} \cdot 6 = 3V$.

}

\begin{center}
\begin{circuitikz}
\draw (0, 0) 
to [I, l = $I_{s1}$] (0, 3)
to [R = $R_1$] (2, 3) 
to [R = $R_2$] (2, 0)
to (0, 0);

\draw (8, 0)
to [I, l_= $I_{s2}$] (8, 3)
to [R, l_ = $R_3$] (6, 3) 
to [R = $R_4$] (6, 0)
to (8, 0);

\draw (2,3) to (6, 3)
(2, 0) to (6, 0);

\draw (4, 3)
to [short, -o] (4, 2.5) 
node[label={[font=\footnotesize]right:$+$}]{}
node[label={[font=\footnotesize]below:$A$}]{};

\draw (4, 0)
to [short, -o] (4, 0.5) node[label={[font=\footnotesize]:$B$}]{}
node[label={[font=\footnotesize]right:$-$}]{};

\draw (4, 0) node[ground]{}
;
\end{circuitikz}
\end{center}

\item Suppose you want to add a resistor of $R_L$ resistance between terminals $A$ and $B$ in the circuit above. What would the current through this resistor be?

\ans{

We can simplify this circuit down to a Thevenin or Norton equivalent circuit. Let's begin by computing the Norton current, which is the short circuit current from terminal $A$ to terminal $B$:

\begin{center}
\begin{circuitikz}
\draw (0, 0) 
to [I, l = $I_{s1}$] (0, 3)
to [R = $R_1$] (2, 3) 
to [R = $R_2$] (2, 0)
to (0, 0);

\draw (8, 0)
to [I, l_= $I_{s2}$] (8, 3)
to [R, l_ = $R_3$] (6, 3) 
to [R = $R_4$] (6, 0)
to (8, 0);

\draw (2,3) to (6, 3)
(2, 0) to (6, 0);

\draw (4, 3)
to [short, -*] (4, 2.5) 
node[label={[font=\footnotesize]right:$+$}]{}
node[label={[font=\footnotesize]left:$A$}]{};

\draw (4, 0)
to [short, -*] (4, 0.5) node[label={[font=\footnotesize]left:$B$}]{}
node[label={[font=\footnotesize]right:$-$}]{};

\draw (4, 2.5) to [short, i = $I_{SC}$] (4, 0.5);

\draw (4, 0) node[ground]{}
;
\end{circuitikz}
\end{center}

We can see that adding this wire between terminals $A$ and $B$ shorts both the $R_2$ and $R_4$ resistors. To solve for the short circuit current, we can use superposition, zero-ing out each of the independent sources:

\begin{center}
\begin{circuitikz}[scale = 0.8]
\draw (0, 0) 
to [I, l = $I_{s1}$] (0, 3)
to [R = $R_1$] (2, 3) 
to [R = $R_2$] (2, 0)
to (0, 0);

\draw (8, 0)
to [short, -o] (8, 1)
(8, 2) to [short, o-] (8, 3)
to [R, l_ = $R_3$] (6, 3) 
to [R = $R_4$] (6, 0)
to (8, 0);

\draw (2,3) to (6, 3)
(2, 0) to (6, 0);

\draw (4, 3)
to [short, -*] (4, 2.5) 
node[label={[font=\footnotesize]right:$+$}]{}
node[label={[font=\footnotesize]left:$A$}]{};

\draw (4, 0)
to [short, -*] (4, 0.5) node[label={[font=\footnotesize]left:$B$}]{}
node[label={[font=\footnotesize]right:$-$}]{};

\draw (4, 2.5) to [short, i = $I_{SC1}$] (4, 0.5);

\draw (4, 0) node[ground]{}
;
\end{circuitikz}
\hspace{0.5cm}
\begin{circuitikz}[scale = 0.8]
\draw (0, 0) 
to [short, -o] (0, 1)
(0, 2) to [short, o-] (0, 3)
to [R = $R_1$] (2, 3) 
to [R = $R_2$] (2, 0)
to (0, 0);

\draw (8, 0)
to [I, l_= $I_{s2}$] (8, 3)
to [R, l_ = $R_3$] (6, 3) 
to [R = $R_4$] (6, 0)
to (8, 0);

\draw (2,3) to (6, 3)
(2, 0) to (6, 0);

\draw (4, 3)
to [short, -*] (4, 2.5) 
node[label={[font=\footnotesize]right:$+$}]{}
node[label={[font=\footnotesize]left:$A$}]{};

\draw (4, 0)
to [short, -*] (4, 0.5) node[label={[font=\footnotesize]left:$B$}]{}
node[label={[font=\footnotesize]right:$-$}]{};

\draw (4, 2.5) to [short, i = $I_{SC2}$] (4, 0.5);

\draw (4, 0) node[ground]{}
;
\end{circuitikz}
\end{center}

%second row of circuits

\begin{center}
\begin{circuitikz}[scale = 0.8]
\draw (0, 0) 
to [I, l = $I_{s1}$] (0, 3)
to [R = $R_1$] (4, 3)
(4, 0) to (0, 0);

\draw (4, 3)
to [short, -*] (4, 2.5) 
node[label={[font=\footnotesize]right:$+$}]{}
node[label={[font=\footnotesize]left:$A$}]{};

\draw (4, 0)
to [short, -*] (4, 0.5) node[label={[font=\footnotesize]left:$B$}]{}
node[label={[font=\footnotesize]right:$-$}]{};

\draw (4, 2.5) to [short, i = $I_{SC1}$] (4, 0.5);

\draw (4, 0) node[ground]{}
;
\end{circuitikz}
\hspace{2cm}
\begin{circuitikz}[scale = 0.8]

\draw (8, 0)
to [I, l_= $I_{s2}$] (8, 3)
to [R, l_ = $R_3$] (4, 3)
(4, 0) to (8, 0);

\draw (4, 3)
to [short, -*] (4, 2.5) 
node[label={[font=\footnotesize]right:$+$}]{}
node[label={[font=\footnotesize]left:$A$}]{};

\draw (4, 0)
to [short, -*] (4, 0.5) node[label={[font=\footnotesize]left:$B$}]{}
node[label={[font=\footnotesize]right:$-$}]{};

\draw (4, 2.5) to [short, i = $I_{SC2}$] (4, 0.5);

\draw (4, 0) node[ground]{}
;
\end{circuitikz}
\end{center}

The circuit with $I_{s1}$ on contributes $I_{s1}$ current to $I_{SC}$, and the circuit with $I_{s2}$ on contributes $I_{s2}$ current to $I_{SC}$. So the $I_{SC}$ in the overall circuit is $I_{SC} = I_{s1} + I_{s2}$. This means $I_{No} = I_{s1} + I_{s2}$. 

Now, we find $R_{No}$: if we zero out both independent sources, then we have $R_1$ and $R_3$ as dangling resistors, and the equivalent resistance between terminals $A$ and $B$ is once again two resistors, $R_2$ and $R_4$ in parallel: $R_{No} = \frac{R_2R_4}{R_2 + R_4}$.

We can then draw the Norton equivalent circuit:

\begin{center}
\begin{circuitikz}
\draw (4, 0)
to [short, o-] (0, 0)
to [I, l = $I_{No}$] (0, 3)
to [short, -o] (4, 3)
(1.5, 3)
to [R = $R_{No}$] (1.5, 0);

\draw (3, 3)
to [R = $R_L$, i = $i_L$] (3, 0);

\draw (4, 3)
node[label={[font=\footnotesize]below:$+$}]{}
node[label={[font=\footnotesize]:$A$}]{};

\draw (4, 0) node[label={[font=\footnotesize]below:$B$}]{}
node[label={[font=\footnotesize]:$-$}]{};
\end{circuitikz}
\end{center}

The current through the $R_L$ resistor can then be solved using the current divider equation:

$i_{L} = \frac{R_{No}}{R_{No} + R_{L}} I_No = \frac{R_{No}}{R_{No} + R_{L}}$, where $R_{No} = \frac{R_2R_4}{R_2 + R_4}$ and $I_{No} = I_{s1} + I_{s2}$.

}


\end{enumerate}