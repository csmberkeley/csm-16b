% Author: Dun-Ming Brandon Huang
% bMail: dunmingbrandonhuang@berkeley.edu
% Question Source: Previous Exams
% Solution Source: Self

\qns{Elementary Rules of Circuit Analysis}

\begin{enumerate}
    \item\label{passive_sign_convention}{
        Which components violate passive sign convention in the following figure? Circle all that apply.
        \begin{center}
            \makebox[\linewidth]{
                \includegraphics{../q_m2_basic_circuit_analysis_figs/passive_sign_convention.PNG}
            }
        \end{center}
        
        \begin{center}
            \begin{tabular}{|c|c|c|c|c|c|c|c|}
                \hline
                $R_1$ & $R_2$ & $R_3$ & $R_4$ & $R_5$ & $I_S$ & $V_{S_1}$ & $V_{S_2}$ \\
                \hline
                $\bigcirc$ & $\bigcirc$ & $\bigcirc$ & $\bigcirc$ & $\bigcirc$ & $\bigcirc$ & $\bigcirc$ & $\bigcirc$ \\
                \hline
            \end{tabular}
        \end{center}
        \newpage
        
    }
    \meta{
        This question comes from Q3(a) of Fall 2020's Midterm 2.
        
    }
    \ans{
        Passive sign convention requires arrowmarks notating current direction be from '+' sign to '-' sign. Therefore, elements $R_3$, $R_4$, $I_S$ and $V_{S_1}$ violate passive sign conventions.
        \begin{center}
            \begin{tabular}{|c|c|c|c|c|c|c|c|}
                \hline
                $R_1$ & $R_2$ & $R_3$ & $R_4$ & $R_5$ & $I_S$ & $V_{S_1}$ & $V_{S_2}$ \\
                \hline
                $\bigcirc$ & $\bigcirc$ & $\bullet$ & $\bullet$ & $\bigcirc$ & $\bullet$ & $\bullet$ & $\bigcirc$ \\
                \hline
            \end{tabular}
        \end{center}
        
    }
    
    \item\label{KVL}{
        Write a KVL expression for the loop drawn in the following figure. Your answer should be in terms of $u_1$, $u_2$, $u_3$, $u_4$, $u_5$, $V_{S_1}$, and $V_{S_2}$, please do not add labels to the figure.
        \begin{center}
            \makebox[\linewidth]{
                \includegraphics{../q_m2_basic_circuit_analysis_figs/KVL.PNG}
            }
        \end{center}
        
    }
    \meta{
        This question comes from Q3(b) of Fall 2020's Midterm 2.
        
    }
    \ans{
        Using Kirchhoff's Voltage Law and following the provided passive sign convention along the loop drawn on the circular arrow of circuit, we can formulate the following equation:
        \[V_{S_1} - i_1R_1 - V_{S_2} + i_3R_3 = 0\]
        Then, using Ohm's Law, we also attain that:
        \begin{align*}
            u_1 - u_2 = i_1R_1 \\
            0 - u_3 = i_3R_3
        \end{align*}
        Combining the above equations, we then acquire:
        \begin{align*}
            V_{S_1} - i_1R_1 - V_{S_2} + i_3R_3
            &= V_{S_1} - (u_1 - u_2) - V_{S_2} + (0 - u_3) \\
            &= V_{S_1} - u_1 + u_2 - V_{S_2} - u_3 = 0V
        \end{align*}
        
    }
    
    \item\label{KCL}{
        Write a KCL expression at node P in terms of currents $I_S$, $i_4$, and $i_5$ as labelled in the following figure. Then, rewrite the expression in terms of $I_S$, node voltages, and resistances only.\\
        The rewritten expression should not contain $i_4$ and $i_5$. Note that P is a label for a node and is not a node voltage value.
        \begin{center}
            \makebox[\linewidth]{
                \includegraphics{../q_m2_basic_circuit_analysis_figs/KCL.PNG}
            }
        \end{center}
        
    }
    
    \meta{
        This question comes from Q3(c) of Fall 2020's Midterm 2.
        
    }
    \ans{
        At node P, while there is on current flowing into the node, the current that flows out of the node $I_{out} = I_S + i_4 + i_5$.\\
        Using Kirchhoff's Current Law, the current flowing into a node should be equal to the current flowing out of the node. Therefore:
        \begin{align*}
            I_{in}
            &= 0A \\
            &= I_S + i_4 + i_5 = I_{out}
        \end{align*}
        The above equation, however, is not in terms of node voltages. To satisfy the prompt's demand, we should find some expression that translates the terms $i_4$, $i_5$ into expressions in terms of node voltages.\\
        In this case, we can, again, use Ohm's Law to generate relationship between current and node voltages:
        \begin{align*}
            u_5 - u_4 = i_4R_4 \\
            u_5 - 0 = i_5R_5
        \end{align*}
        Now, let's rewrite the KCL equation in terms of node voltages:
        \begin{align*}
            I_S + i_4 + i_5
            &= I_S + \frac{u_5 - u_4}{R_4} + \frac{u_5 - 0}{R_5} \\
            &= I_S + \frac{u_5 - u_4}{R_4} + \frac{u_5}{R_5} = 0A
        \end{align*}
        
    }
    
    \item\label{find_node_voltage}{
        Given the node voltage $u_4=3V$ in the following figure, find the node voltage $u_2$.
        \begin{center}
            \makebox[\linewidth]{
                \includegraphics{../q_m2_basic_circuit_analysis_figs/node_voltage.PNG}
            }
        \end{center}
        
    }
    \meta{
        This question comes from Q3(d) of Fall 2020's Midterm 2
        
    }
    \ans{
        The circuit element directly between $u_2$ and $u_4$ is a $3V$ voltage source, and according to the direction this source is put:
        \[u_2 - u_4 = 3V\]
        Therefore, $u_2 = 6V$.
        
    }
    
    \item\label{amnmeter}{
        How would you connect an amnmeter to this circuit to measure current flowing through the $3\Omega$ resistor in the figure from part (d)?\\
        Recall that an amnmeter is a device that measures current, and its symbol is shown as follows.
        \begin{center}
            \makebox[\linewidth]{
                \includegraphics{../q_m2_basic_circuit_analysis_figs/amnmeter.PNG}
            }
        \end{center}
        
    }
    \meta{
        This question comes from Q3(e) of Fall 2020's Midterm 2
        
    }
    \ans{
        When an element is in series with another element, the current through both of them is equal.\\
        Therefore, to measure the current across an element, we will put an amnmeter in series with it, creating the following circuit:
        \begin{center}
            \makebox[\linewidth]{
                \includegraphics{../q_m2_basic_circuit_analysis_figs/amnmeter_sol.PNG}
            }
        \end{center}
        
    }
    
    
\end{enumerate}
