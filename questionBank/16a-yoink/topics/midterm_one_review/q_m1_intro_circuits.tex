% Author: Dun-Ming Brandon Huang
% bMail: dunmingbrandonhuang@berkeley.edu
% Question Source: Previous Exams
% Solution Source: Self

\qns{Structure and Interpretation of Introductory Circuits}

% Need to fix the circuit figures
The following questions all refer to the circuit posted below:
\begin{center}
    \input{../q_m1_intro_circuits_figs/nodes_temp.tex}
\end{center}

\begin{enumerate}
    \item\label{node_identification}{
        What is the number of nodes in the circuit, including the ground node? Label the nodes in the above diagram.
        
    }
    \meta{
        This question comes from Q1(a) of Fall 2021's Midterm 1.\\
        \begin{bindenum}
            \item Define nodes before approaching this problem. It will serve great clarifications.
        \end{bindenum}
        
    }
    \ans{
        There are four nodes, as drawn below:
        \begin{center}
            % Author: Dun-Ming Huang
% Email: dunmingbrandonhuang@berkeley.edu
% CSM16A Fall 2022
\begin{circuitikz}[american]
    \draw
        (4, 8.5) to [R, l^ = $R_2$, v = $V_{R_2}$, i = $i_2$] (4, 6.5)
        (4, 5.5) to [R, l^ = $R_3$, v = $V_{R_3}$, i = $i_3$] (4, 4)
        (3, 2.5) to [R, l^ = $R_4$, v = $V_{R_4}$, i = $i_4$] (3, 1)
        (2, 7) to [R, l^ = $R_1$, v = $V_{R_1}$, i = $i_1$] (2, 5)
        (0, 5) to [V, l_=$V_S$] (0, 3);
    \draw[blue]
        (0, 8.5) to [short] (4, 8.5)
        (2, 8.5) to [short] (2, 7)
        (0, 8.5) to [short] (0, 5);
    \draw[red]
        (4, 6.5) to [short] (4, 5.5);
    \draw[green]
        (4, 4) to [short] (4, 3) to [short] (3, 3) to [short] (3, 2.5)
        (2, 5) to [short] (2, 3) to [short] (3, 3);
    \draw[gray]
        (3, 1) to [short] (3, 0.5) to [short] (0, 0.5)
        (0, 3) to [short] (0, 0.5);
\end{circuitikz}
        \end{center}
        
    }
    
    \item\label{KCL}{
        Write out two KCL equations for the circuit, one involving $i_{S}$, another involving $i_{3}$.
        
    }
    \meta{
        This question comes from Q1(b) of Fall 2021's Midterm 1.
        
    }
    \ans{
        At the blue node (node 1), we can observe the equation
        \begin{align*}
            i_{in}
            &= i_{s} + i_{1} + i_{2} \\
            &= 0 = i_{out}
        \end{align*}
        At the green node (node 3), we can observe the equation
        \begin{align*}
            i_{in}
            &= i_{1} + i_{3} \\
            &= i_{4} = i_{out}
        \end{align*}
    }
    
\end{enumerate}