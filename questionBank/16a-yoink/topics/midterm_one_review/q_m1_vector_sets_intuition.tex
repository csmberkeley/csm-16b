% Author: Dun-Ming Brandon Huang
% bMail: dunmingbrandonhuang@berkeley.edu
% Question Source: Previous Exams
% Solution Source: Self

\qns{You're In the Matrix Now!}

You've been pranked! Your friends have stolen your things and have hidden them all over town.\\
They are communicating messages back and forth that contain coordinates representing the locations of your missing stuff.\\
Unfortunately, your friends are pretty smart, so they've used an encoding matrix to encode the vector of coordinates,\\
\[\vec{y} = A\vec{x}\]
where $\vec{y}$ is the encoded message, $\vec{x}$ is the original vector of coordinates, and $A$ is the encoding matrix.\\
But not to worry! Your 16A knowledge should help you break the code and find your stuff.\\

\begin{enumerate}
    \item\label{matrix_dimension}{
        You've successfully intercepted both the original and encoded versions of one of the locations.\\
        If the original vector, $\vec{x}$, belongs to $\R^{2}$ and the encoded vector, $\vec{y}$, belongs to $\R^{6}$, what are the dimensions of the encoding matrix, A?
        
    }
    \meta{
        This question comes from Q7(a) of Summer 2020's Midterm 1.
        
    }
    \ans{
        To express the dimensions of vectors when written in matrix expression, $\vec{x}\in\R^{2\times1}$ and $\vec{y}\in\R^{6\times1}$. Meanwhile, $\vec{y} = A\vec{x}$.\\
        To make the multiplication appropriate, there is a specific set of dimensions that matrix $A$ has to be in. A must have 2 columns such that the expression $A\vec{x}$ is defined. Meanwhile, because $\vec{y}\in\R^{6\times1}$, $A$ must have 6 rows.\\
        Summarizing the above: $A\in\R^{6\times2}$.
        
    }
    
    \item\label{system_identification}{
        What is the minimum number of pairs of original and encoded messages that you'd need to intercept in order to determine the encoding matrix?

    }
    \meta{
        This question comes from Q7(b) of Summer 2020's Midterm 1.
        
    }
    \ans{
        The amount of unknown contents we need to identify in matrix $A$ is determined by its size, and in this case we have $6\times2$ unknown values to compute. To compute all of these values, we need to construct a system of equations, which the matrix-vector multiplications will offer.\\
        Each pair of message constructs a system that takes the format of $\vec{y} = A\vec{x}$, which provides 6 equations. To identify 12 unknown values, we need a system of at least 12 equations. Therefore, since each pair provides 6 equations and we need 12 equations, we need 2 pairs of messages.\\
        It is important that these 2 pairs of messages should offer equations such that we can gain a unique set of solutions for the unknown values, but the question does not require us to concern this mathematical property.
        
    }
    
    \item\label{system_recoverability}{
        We have successfully intercepted n pairs of messages, where n is greater than or equal to the minimum number of pairs needed to determine the encoding matrix.\\
        Which of the following must be true about the n intercepted $\vec{x}$ vectors (original messages) if we want to recover the encoding matrix? Circle all that apply.
        \begin{enumerate}
            \item They must all be linearly independent.
            \item They must span $\R^{2}$.
            \item They must form a basis for $\R^{2}$.
            \item None of the other four options are correct.
            \item The $2\times n$ matrix containing these vectors as columns must have rank 2.
        \end{enumerate}
        \begin{center}
            \begin{tabular}{|c|c|c|c|c|}
                \hline
                i & ii & iii & iv & v\\
                \hline
                $\bigcirc$ & $\bigcirc$ & $\bigcirc$ & $\bigcirc$ & $\bigcirc$ \\
                \hline
            \end{tabular}
        \end{center}
        
    }
    \meta{
        This question comes from Q7(c) of Summer 2020's Midterm 1.
        
    }
    \ans{
        Let us attempt to validate each option provided by the question, having the solutions from the previous parts of this question as given information. \\
        
        Option (i): False\\
        We need 2 pairs of messages with linearly independent original messages to identify the system contents, so that when we write out the system of equations in a matrix-vector form, the columns of left side of augmented matrix are not linearly dependent.\\
        This means we could use 3, or even more pairs of messages, as long as we have 2 of these pairs with linearly independent original messages such that it works to identify the matrix content.\\
        
        Option (ii): True\\
        There must be at least 2 linearly independent original messages, and these original messages are all 2-dimensional vectors; 2 linearly independent 2-dimensional vectors will span $\R^{2}$.\\
        
        Option (iii): False\\
        If we, as described in option (i), use 2 linearly independent original messages and other linearly dependent original message(s), we will not form a basis of $\R^{2}$ because the set of original messages is not the smallest set of vectors that span $\R^{2}$ as the definition of basis claims.\\
        
        Option (iv): False\\
        Since at least 1 of the other 4 choices are true, this option cannot be true.\\
        
        Option (v): True\\
        Let this matrix be called $B$, there must at least be 2 linearly independent columns in this matrix, and this matrix's columns only have 2-dimensional vectors (because it is $2\times r$).\\
        Recall from lecture that the rank(B) must be smaller than min(number of rows, number of columns, in this case $min(2, n)$.\\
        But the solution of part (b) tells us that $n\geq2$. Therefore, while $B$ will at least have a rank of 2, it can also only have at most a rank of 2 due to its dimensions: $2\leq rank(B)\leq2$. This leads to $rank(B)=2$.\\
        \\
        Expressed in a table, the solutions would be:
        \begin{center}
            \begin{tabular}{|c|c|c|c|c|}
                \hline
                i & ii & iii & iv & v\\
                \hline
                $\bigcirc$ & $\bullet$ & $\bigcirc$ & $\bigcirc$ & $\bullet$ \\
                \hline
            \end{tabular}
        \end{center}
    }
\end{enumerate}