% Author: Dun-Ming Huang
% Email: dunmingbrandonhuang@berkeley.edu
% CSM16A Spring 2023

\qns{Give Me Subspace}

Give me some space (proofs). \\
In other words, please prove the following prompts for an arbitrary matrix $A \in \R^{m \times n}$:

\begin{enumerate}
    \item {
        The nullspace of $A$ is a subspace.
        
    }
    \ans {
        Let us remind ourselves that a subspace is a vector space with the three following qualities:
        \begin{enumerate}
            \item Contains $\vec{0}$
            \item Closure under addition: $\forall \vec{x}, \vec{y} \in S, \vec{x} + \vec{y} \in S$
            \item Closure under scalar multiplication: $\forall \vec{x} \in S, \alpha \in \R, \alpha \vec{x} \in S$
        \end{enumerate}
        Then, to prove that a vector space is a subspace, we must prove that all above qualities are satisfied by the space in prompt.
        \par
        \textbf{Proof for Quality i.} \\
        For any matrix $A$,
        \[
            A \vec{0} = \vec{0}
        \]
        Therefore, by definition,
        \[
            \vec{0} \in N(A)
        \]
        \textbf{Proof for Quality ii.} \\
        Suppose there are arbitrary vectors $\vec{x}, \vec{y} \in N(A)$, then we see that
        \begin{align*}
            A (\vec{x} + \vec{y}) &= A \vec{x} + A \vec{y} \\
            &= \vec{0} + \vec{0} = \vec{0}
        \end{align*}
        Therefore, by definition,
        \[
            \forall \vec{x}, \vec{y} \in N(A), \vec{x} + \vec{y} \in N(A)
        \]
        \textbf{Proof for Quality iii.} \\
        Suppose there is an arbitrary vector $\vec{x} \in N(A)$ and some real scalar $\alpha \in \R$, then we see that
        \begin{align*}
            A (\alpha \vec{x}) &= \alpha A \vec{x} \\
            &= \alpha \vec{0} = \vec{0}
        \end{align*}
        Therefore, by definition,
        \[
            \forall \vec{x} \in N(A), \alpha \in \R, \alpha \vec{x} \in N(A)
        \]
        
    }
    \newpage
    \item{
        The columnspace of $A$ is a subspace.
        
    }
    \ans {
        We will write the proof for this prompt according to the aforementioned logic in part (a). \\
        For convenience of notation, let us also suppose that the arbitrary matrix $A$ we perform a proof of can be expressed as:
        \[
            A = \begin{bmatrix} \vec{A_1} & \dots & \vec{A_n} \end{bmatrix}
        \]
        \par
        \textbf{Proof for Quality i.} \\
        For any matrix $A$,
        \[
            0 \vec{A_1} + \cdots + 0 \vec{A_n} = \vec{0}
        \]
        Therefore, by definition,
        \[
            \vec{0} \in Col(A)
        \]
        \textbf{Proof for Quality ii.} \\
        Suppose there are arbitrary vectors $\vec{x}, \vec{y} \in Col(A)$, then we see that
        \begin{align*}
            \vec{x} + \vec{y}
            &= (a_1 \vec{A_1} + \cdots + a_n \vec{A_n}) + (b_1 \vec{A_1} + \cdots + b_n \vec{A_n}) \\
            &= (a_1 + b_1) \vec{A_1} + \cdots + (a_n + b_n) \vec{A_n}
        \end{align*}
        Therefore, by definition,
        \[
            \forall \vec{x}, \vec{y} \in Col(A), \vec{x} + \vec{y} \in Col(A)
        \]
        \textbf{Proof for Quality iii.} \\
        Suppose there is an arbitrary vector $\vec{x} \in Col(A)$ and some real scalar $\alpha \in \R$, then we see that
        \begin{align*}
            \alpha \vec{x}
            &= \alpha (a_1 \vec{A_1} + \cdots + a_n \vec{A_n}) \\
            &= (\alpha a_1) \vec{A_1} + \cdots + (\alpha a_n) \vec{A_n} \in Col(A)
        \end{align*}
        Therefore, by definition,
        \[
            \forall \vec{x} \in Col(A), \alpha \in \R, \alpha \vec{x} \in Col(A)
        \]
    
    }
    \ans{
        Let us consider an eigenpair of matrix $A$ to be $(\lambda, \vec{v})$. \\
        Then, 
        \begin{align*}
            A^2 \vec{v} &= A A \vec{v} \\
            &= A \lambda \vec{v} = \lambda^2 \vec{v}
        \end{align*}
        Proving that $(\lambda^2, \vec{v})$ is an eigenpair of $A^2$.
        \par
        Similarly,
        \begin{align*}
            A \vec{v} &= \lambda \vec{v} \\
            \frac{1}{\lambda} \vec{v} &= A^{-1} \vec{v} \\
        \end{align*}
        Proving that $(\lambda^{-1}, \vec{v})$ is an eigenpair of $A^{-1}$ if such inverse exists.
    }
\end{enumerate}
