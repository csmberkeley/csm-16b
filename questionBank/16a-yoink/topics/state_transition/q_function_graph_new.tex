\qns{Functional Pumps}

\textbf{Learning Goal:} The goal of this problem is to present a state transition diagram and guide students to understand the meaning of a state transition matrix and its applications. Please review \notes{Note 5: Section 5.1} to understand this problem better. 

\vspace{2mm}
\meta{
Ensure that students have a clear understanding of what the state transition matrix is describing before proceeding to problems. Also explain in what scenarios this is useful. For example, what the rows and columns indicate in the matrix, what a time step is, and the difference between conservative vs. leaky systems and what that translates to for reservoirs. 

\textbf{Do NOT mention stability since students have not seen eigen-values/vectors.}
}

\vspace{2mm}
Take a look at this functional pump:

\begin{center}
\begin{tikzpicture}[->,>=stealth',shorten >=1pt,auto,node distance=4cm,
thick,main node/.style={circle,draw,font=\Large\bfseries}]
  \node[main node] (1) {$x_1$};
  \node[main node] (2) [below left of=1] {$x_2$};
  \node[main node] (3) [below right of=1] {$x_3$};

  \path
    (1) edge [loop above] node {1/2} (1)
    (1) edge [bend left] node {1/2} (2)
    (2) edge node [bend right] {1/2} (1)
    (2) edge [below right] node {1/2} (3)
    (3) edge [loop below] node {1} (3);   
\end{tikzpicture}
\end{center}

\begin{enumerate}

\item
What do the rows in a functional pump represent? What do the columns represent?

\meta{ 
Draw a general matrix that shows the transition that is happening in each element of the entire $n\times n$ matrix to emphasize the significance of rows in a functional pump.
}

\ans{

The rows in the matrix tell us how much inflow each reservoir gets from other reservoirs in the system. 

The columns in the matrix tell us how much each reservoir contributes (outflow) to other reservoirs in the system. The columns also give us information about the state of the matrix. If each column of the matrix sums to 1, then the system is conserved.

To reiterate, a system being conserved means no data is lost or gained within the system.
}

\item
Analyze the pump above and write the first column of the state transition matrix. Use the state vector:
$$
\begin{bmatrix}
x_1 \\
x_2 \\
x_3 \\
\end{bmatrix}
$$
Repeat this process for each of the reservoirs in this diagram.

\ans{
\begin{center}
$ x_1: 
\begin{bmatrix}
x_1 \rightarrow x_1 \\
x_1 \rightarrow x_2 \\
x_1 \rightarrow x_3 \\
\end{bmatrix}
$  $x_2: 
\begin{bmatrix}
x_2 \rightarrow x_1 \\
x_2 \rightarrow x_2 \\
x_2 \rightarrow x_3 \\
\end{bmatrix}
$    $x_3: 
\begin{bmatrix}
x_3 \rightarrow x_1 \\
x_3 \rightarrow x_2 \\
x_3 \rightarrow x_3 \\
\end{bmatrix}
$  
\\
$ x_1: 
\begin{bmatrix}
1/2 \\
1/2 \\
0 \\
\end{bmatrix}
$  $x_2: 
\begin{bmatrix}
1/2 \\
0 \\
1/2 \\
\end{bmatrix}
$    $x_3: 
\begin{bmatrix}
0 \\
0 \\
1 \\
\end{bmatrix}
$  
\\
\end{center}

Combining these three columns yields the state transition matrix for the above pump.
 \begin{center}
$  
\mathbf{A}=\begin{bmatrix}
x_1 \rightarrow x_1 & x_2 \rightarrow x_1 & x_3 \rightarrow x_1\\
x_1 \rightarrow x_2 & x_2 \rightarrow x_2 & x_3 \rightarrow x_2\\
x_1 \rightarrow x_3 & x_2 \rightarrow x_3 & x_3 \rightarrow x_3  \\
\end{bmatrix}
$
\\
$  
=\begin{bmatrix}
1/2 & 1/2 & 0 \\
1/2 & 0 & 0 \\
0 & 1/2 & 1 \\
\end{bmatrix}
$

\end{center}
}

\item
Is this system conserved? Why or why not? Please review \notes{Note 5: Section 5.1.4} to understand this problem better. 

\ans{
Yes; each of the columns sums up to 1. Hence, it is a conserved system. 
}

\item
Given that the initial reservoir volume, $v[0]$, is 
$\begin{bmatrix}1 \\ 1 \\ 1 \end{bmatrix}$ determine the amount of water in each of the reservoirs after turning the system on $n$ number of times. Please review \notes{Note 5: Section 5.1.7} to understand this problem better.

\ans{
Note that turning on the system means going to the next time step.
}

\begin{enumerate}
\item
Turn the system on once.

\ans{

\begin{center}
$\mathbf{A}v[0]$ = $v[1]$

$  
\begin{bmatrix}
1/2 & 1/2 & 0 \\
1/2 & 0 & 0 \\
0 & 1/2 & 1 \\
\end{bmatrix} \begin{bmatrix}
1 \\
1 \\
1 \\
\end{bmatrix}  = \begin{bmatrix}
1 \\
1/2 \\
3/2 \\
\end{bmatrix}
$  

$v[1] = \begin{bmatrix}
1 \\
1/2 \\
3/2 \\
\end{bmatrix}$
\end{center}
}

\item
Turn the system on twice.

\ans{

\begin{center}
$\mathbf{A}v[1]$ = $v[2]$
\\
$  
\begin{bmatrix}
1/2 & 1/2 & 0 \\
1/2 & 0 & 0 \\
0 & 1/2 & 1 \\
\end{bmatrix} \begin{bmatrix}
1 \\
1/2 \\
3/2 \\
\end{bmatrix}  = \begin{bmatrix}
3/4 \\
1/2 \\
7/4\\
\end{bmatrix}
$  

$
v[2] = \begin{bmatrix}
3/4 \\
1/2 \\
7/4\\
\end{bmatrix}
$
\end{center}
}

\item
What is another way to find $v[2]$ if you could only multiply one state transition matrix into the initial state once?

\meta{
Explain the significance of applying the matrix to itself as well as how finding the given matrix multiplication would be much faster than going through each timestep calculation.

}  

\ans{
\begin{align*}
\mathbf{A}v[1] &= v[2]\\
\implies \mathbf{A}(\mathbf{A}v[0])& =v[2]\\
\implies \mathbf{A}^2v[0]&=v[2]\\
\implies \mathbf{B}v[0]&=v[2],
\end{align*}
where $\mathbf{A}^2$ = $\mathbf{B}$
\begin{center}
$  
\begin{bmatrix}
1/2 & 1/2 & 0 \\
1/2 & 0 & 0 \\
0 & 1/2 & 1 \\
\end{bmatrix} \begin{bmatrix}
1/2 & 1/2 & 0 \\
1/2 & 0 & 0 \\
0 & 1/2 & 1 \\
\end{bmatrix} = \begin{bmatrix}
1/2 & 1/4 & 0 \\
1/4 & 1/4 & 0 \\
1/4 & 1/2 & 1\\
\end{bmatrix} = \mathbf{B}
$  

$\mathbf{B}v[0] = \begin{bmatrix}
1/2 & 1/4 & 0 \\
1/4 & 1/4 & 0 \\
1/4 & 1/2 & 1\\
\end{bmatrix}  \begin{bmatrix}1 \\ 1 \\ 1 \end{bmatrix} = 
\begin{bmatrix}
3/4 \\
1/2 \\
7/4 \\
\end{bmatrix} 
$
\end{center}
Similarly we can deduce that:
\begin{align*}
\mathbf{A}^nv[0]&=v[n]
\end{align*}  
}
\end{enumerate}




\end{enumerate}
