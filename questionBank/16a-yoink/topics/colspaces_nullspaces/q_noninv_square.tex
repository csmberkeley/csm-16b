%\newpage
\qns{Non-invertible Square Matrix}

\textbf{Learning Goal:} The goal of this problem is to understand loss of dimensionality in relation to nullspace.

\meta{
\begin{itemize}
\item Guide students towards using a proof by contradiction for this problem. Also spend some time explaining the format and flow of a proof by contradiction.
\item Give a reminder that a proof by contradiction is when we assume that what we want to show isn’t true, but we get something that doesn’t agree with what we already know. 
\item To help your students understand this proof, it might be helpful to go through a specific example with simple A and B matrices.

\end{itemize}

}

\begin{enumerate}

\itemFor given matrices $\mathbf{A}\in\mathbb{R}^{3\times2}$ and $\mathbf{B}\in\mathbb{R}^{2\times3}$, the products will be square matrices: $\mathbf{A}\mathbf{B}\in \mathbb{R}^{3\times3}$ and $\mathbf{B}\mathbf{A}\in \mathbb{R}^{2\times2}$. Show that $\mathbf{A}\mathbf{B}$ is not invertible.

Please look into \notes{Note 8 Section 8.3} to learn how the dimension of the output space depends on the nullspace.

Hint: \textit{A good proof strategy is to utilize what we have already proven before. Is there a way we can use the result in Question 4, "Proof on Nullspace"?}

\ans {
We \underline{know} that $\mathbf{A}\in\mathbb{R}^{3\times2}$ and $\mathbf{B}\in\mathbb{R}^{2\times3}$. We want to \underline{show} that $\mathbf{A}\mathbf{B}$ is not invertible.\\\\
\underline{Steps to get there:}
There are many ways to do this proof, and we will only show one possible method here, a proof by contradiction.\\\\
Assume, for sake of contradiction, that $\mathbf{A}\mathbf{B}$ is invertible. Since $\mathbf{A}\mathbf{B}\in \mathbb{R}^{3\times3}$, it is square, so by the theorem proved in Problem 4, $\mathbf{A}\mathbf{B}$ must have a trivial nullspace. This means that $\vec{x} = \vec{0}$ is the only solution to the equation $$\mathbf{A}\mathbf{B}\vec{x} = \vec{0}$$
Now, we are also given that $\mathbf{B}\in\mathbb{R}^{2\times3}$. $\mathbf{B}$ has 3 column vectors, each of which are $\in\R^2$. The transformation $\mathbf{B}$ results in a loss of dimensionality: the dimension of $\R^2$ is 2, so at maximum, two of the column vectors will span $\R^2$. This means that we are guaranteed that at least one of the column vectors is linearly dependent, so $\mathbf{B}$ has a nontrivial nullspace. By definition of nontrivial nullspace, there exists $\vec{x} \neq \vec{0}$ that solves the equation $\mathbf{B}\vec{x} = \vec{0}$. Call this solution $x^\prime$: $$\mathbf{B}\vec{x^\prime} = \vec{0}$$
Left-multiplying both sides of the equation by $\mathbf{A}$ gives us $$\mathbf{A}\mathbf{B}\vec{x^\prime} = \vec{0}$$
But this contradicts the result we achieved earlier in the proof: then, we found that the only solution to the equation $\mathbf{A}\mathbf{B}\vec{x} = \vec{0}$ is $\vec{x} = \vec{0}$, and now, we have shown there must exist a nonzero solution, $x^\prime$! These two statements are contradictory; therefore, there must be something wrong with our initial assumption that $\mathbf{A}\mathbf{B}$ is invertible. It follows that $\mathbf{A}\mathbf{B}$ is non-invertible.
}





\end{enumerate}

