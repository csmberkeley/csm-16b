% Zipeng Lin - yuslzp@berkeley.edu
\qns{Inverse of some matrices}
\textbf{Learning goal: learn another property of invertibility and learn
proofs}.

\meta{
\begin{enumerate}
    \item First review the definition of invertibility and how to write
        multiplication of matrix and vector as a linear combination of columns.
    \item For the proof of inverse of product, ask students not to overthink.
\end{enumerate}
}

a. Recall that if a matrix is invertible, then it has linearly independent
columns. Prove the following for equivalence in bring invertible: A matrix $A \in
R^{nxn}$ is invertible iff $\vec{v} = \vec{0}$ is the only solution to
$A\vec{v}=\vec{0}$.

\ans{
    \textbf{We need to prove for both direction}. Suppose $A$ is invertible, it
    means columns of  $A$ are linearly independent. Therefore, denote columns of
     $A$ as  $a_1, \ldots{}, a_n$ and  $v = \left[ v_1, \ldots{}, v_n
     \right]^{\intercal}$, then we have  $\sum_{i}^{n} a_i v_i = 0$ so every
     $v_i = 0$.

     On the other hand, if we have the only solution is  $0$ it means the columns are
     linearly independent. Suppose this is not true, then we have values $c_1,
     \ldots{}, c_n$ not all zero such that $\sum A_i c_i = 0$ but it would be a
     nonzero vector solution, contradicting with what we have. Therefore,  $A$ is invertible.
} 

b. Prove that if matrices $A, B \in R^{n \times n}$ are invertible, then $A
B$ is invertible. Hint: use result from 1.

\ans{
    Suppose there is an $v$ such that  $A B v = 0$. Then we have  $A (Bv) = 0$,
    so  $Bv = 0$ since  $A$ is invertible, then we have  $v = 0$ since  $B$ is
    invertible. Therefore, we have  $AB$ is invertible.
}

c. Prove that $(AB)^{-1} = B^{-1} A^{-1}$

\ans{
    We have $AB  (B^{-1} A^{-1}) = A I A^{-1} = A A^{-1} = I$
}

d. Give an counterexample for invertible matrix $A, B$ such that  $(A+B)^{-1}$ is
not $B^{-1} + A^{-1}$ 

\ans{
    Just make $B = -A$, then we have  $A+B$ is not invertible. 
}



