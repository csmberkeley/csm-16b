% Authors: Zipeng lin
% Emails: yuslzp@berkeley.edu


\qns{Trilateration}\\

\textbf{Learning objective: } learning how to solve a basic triangulation system
and view different systems geometrically.

\meta{ When going over the question, it would be helpful to draw a graph for
number 1 and number 2. After letting the students understand the Geometric
meaning of the first equation, it would be nice to give them some time drawing
out number 3 and 4}.

a). Suppose we have three points $a_1, a_2, a_3$ such that they have the
following coordinates:
 \begin{equation}
     \vec{a_1} = (0, 0)\\
     \vec{a_2} = (1 / 2, \sqrt{3} / 2 )
     \vec{a_3} = (1, 0)\\
 \end{equation}

 and we have a point $\vec{d}$ such that it has the following distances to the three
 points:

\begin{equation}
    \norm{\vec{a_1} - \vec{d}}^{2} = (1 / \sqrt{3})^{2}\\
    \norm{\vec{a_2} - \vec{d}}^{2} = (1 / \sqrt{3})^{2}\\
    \norm{\vec{a_3} - \vec{d}}^{2} = (1 / \sqrt{3})^{2}\\
\end{equation}

solve for the d point

\ans{
 We have the following

 \[
     d_1 = d_2 = d_3 = 1 / \sqrt{3} 
 \]

 and 

 \[
     \norm{a_1}^{2} = 0, \norm{a_2}^{2} = 1, and \norm{a_3}^{2} = 1
 \]
 

 Therefore, we have in the equation

 \[
\begin{bmatrix}
    2\left[ - 1 / 2, - \sqrt{3} / 2  \right]  \\
    2\left[ - 1 , 0  \right]  \\
\end{bmatrix}
\vec{x}
= 
\begin{bmatrix}
    -1 \\
    -1 \\
\end{bmatrix}
 \]

 and we get 

 \[
     \vec{x} = 
\begin{bmatrix}
    1 / 2 \\
    1 / 2\sqrt{3}  \\
\end{bmatrix}

 \] 
}

b). Interpret the geometry meaning of problem one. What does point $\vec{a_1},
\vec{a_2}, \vec{a_3}$, and what is the point $d$  representing in this figure?

\ans{
    $\vec{a_1}, \vec{a_2}, \vec{a_3}$ represents the points in a equilateral
    triangle with side length $1$, and we have  $d$ is the center of the
    triangle.
} 

c). Consider the following points

\[
    \vec{a_1} = (0, 0)
    \vec{a_2} = (1, 0)
    \vec{a_3} = (2, 0)
\]

and suppose there is a point $\vec{d}$ such that the following are the distances

 \[
     \norm{\vec{a_1}, \vec{d}} = 
     \norm{\vec{a_2}, \vec{d}} = 
     \norm{\vec{a_3}, \vec{d}} 
\]

prove that such  point $\vec{d}$ does not exist.

\ans{
    We have $d_1 = d_2$ here, so we have the following matrix to solve:

    \[
\begin{bmatrix}
    2(\vec{a_1} - \vec{a_2})^{\intercal} \\
    2(\vec{a_1} - \vec{a_3})^{\intercal} \\
\end{bmatrix}
\vec{d}=
\begin{bmatrix}
    \norm{\vec{a_1}}^{2} - \norm{\vec{a_2}}^{2} \\
    \norm{\vec{a_1}}^{2} -     \norm{\vec{a_3}}^{2} \\
\end{bmatrix}
    \]
    since we have $\norm{d_1} = \norm{d_2} = \norm{d_3}$.

    Now, however, observe the matrix, it would become

\[
 \begin{bmatrix}
    -2 & 0\\
    -4 & 0
\end{bmatrix}
\vec{d}=
\begin{bmatrix}
    -1
    -4
\end{bmatrix}  
\]

but $-1 / (-2) \neq{} (-4) / (-4)$ (since the result just depends on the first
element of $\vec{x}$), so we have there is no such point $\vec{x}$.
}


d). What is the geometric meaning of the problem c). Suppose we take a step
further: suppose the points have the following coordinate

\[
    \vec{a_1} = (0, 0)
    \vec{a_2} = (a, 0)
    \vec{a_3} = (1, 0)
\]

with $0 < a < 1$

and suppose there is a point $\vec{d}$ such that the following are the distances

 \[
     \norm{\vec{a_1}, \vec{d}} = 
     \norm{\vec{a_2}, \vec{d}} = 
     \norm{\vec{a_3}, \vec{d}} 
\]

prove that such point $\vec{d}$ does not exist. 
 
\ans{
    Look at our equation again, 
    \[
\begin{bmatrix}
    -a & 0\\
    -1 & 0
\end{bmatrix}
\vec{d}=
\begin{bmatrix}
    -a^{2}
    -1
\end{bmatrix}  
    \]    
    since we have if the ratio are the same, $a = 1$, but this breaks with the
    condition that  $0 < a < 1$
}