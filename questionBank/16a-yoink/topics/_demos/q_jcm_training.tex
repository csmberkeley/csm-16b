% write comments with the percent sign
% Author: Damanic Luck
% Email: damanicluck@berkeley.edu
% CSM16A Fall 2024

\qns{Junior Content Mentor Training Document}
This document is meant to provide a initial start on making LaTex basics on VS Code and debugging skills for part one and part two. 
Part three is optional but highly recommended for question making basics.
This document will be sent out as a .tex document, and you are expected to read through the document (including comments) in either Overleaf or VS Code.
I am creating this in VS Code, so it is preferred that it's also completed in VS Code.

As a general note, in VS Code, a line with nothing in it placed between two lines with text will result in a new paragraph.
It is good practice to format your code with one sentence PER line.
This suggestion comes from viewing .tex files on interfaces that don't wrap their sentences.
Let's say you write a really long sentence in a cell in an Excel sheet.
Depending on your settings, it will either cut off the sentence from viewing the later parts or it will create a large box to allow the text to wrap around.

%%%%%% PART ONE: BASIC BUILDING BLOCKS %%%%%%

% Commands are specified with a backwards slash
% \textit{itacilizes non-math text inside the curly braces}
% \textbf{bold fonts text inside the curly braces}

\textbf{TASK A}: Suppose you want to create a \textit{numbered} list of your favorite Sanrio or One Piece characters.
We will introduce the idea of an "environment" here.
The next line is a comment since otherwise it will fail when you compile the document.
% Often times when you want a specific format or lists, you want to use the format \text{"\begin{environment} ... \end{environment}} 
Link: https://www.overleaf.com/learn/latex/Lists

Use the above link to try out creating a list with the "itemize" and "enumerate" environment.
%%%%%%%%%%%%%%%%%%%%%%%%%%%%%%%%%%%%%%%%%%
% TASK A EXCERCISE SPACE


%%%%%%%%%%%%%%%%%%%%%%%%%%%%%%%%%%%%%%%%%%

\textbf{TASK B}: Now let's write equations and math symbols. 
Equations use dollar signs to specify math mode, otherwise certain formatting like subscripts isn't enabled.
Use a square bracket followed by a regular parenthesis for math mode as well as centering.
You'll see this in Task C.
PLEASE FOR THE LOVE OF EVERYTHING HOLY DO NOT USE DOUBLE DOLLAR SIGNS IN CENTERING MATH MODE ENVIRONMENTS.
THIS IS A HINT FOR TASK C.
Use the "equation", "align", or "align*" environment for math mode as well as centering for multiline mathmode.
Copy paste the following comment into the Task B excercise block and fix its formatting.
After you fix its formatting, also type the quadratic formula into the Task B excercise block. 
To be clear, this will compile, but its formatting isn't correct.

Hint: Ampersand will align equation lines along its the character the ampersand is next to.
%%%%%%%%%%%%%%%%%%%%%%%%%%%%%%%%%%%%%%%%%%
% TASK B EXCERCISE SPACE
$f(z_{initial}) = (\alpha_provided + i \beta)^2, squared = \alpha^2 - \beta^2 + 2i \alpha \beta$

\[\mathtext{Pythagorean Theorem: } a = \sqrt{b^2 + c2}\]

\begin{align*}
    i + 1 &= 5 + j \\
    2i + 2 = 10 + 2j \\
    3i + 3 &= 15 + 3j
\end{align*}
         
%%%%%%%%%%%%%%%%%%%%%%%%%%%%%%%%%%%%%%%%%%

%%%%%% PART SECOND: DEBUGGING %%%%%%
Debugging is honestly the least fun part of making a new question.
Your tex file can be an amazing question, but if it can't compile then it's not possible to see its output as a pdf.
Each Task will feature a different common error and it is up to you to fix it.

% the \ slash is to specify that we want the character as is and its not part of any missing formatting
\textbf{Task C}: Missing \$ inserted. 
This means that there is something that requires math mode or your equation is not properly closed by another dollar sign or environment ending.
Uncomment the code in the Task C excercise space and fix it.

%%%%%%%%%%%%%%%%%%%%%%%%%%%%%%%%%%%%%%%%%%
% TASK C EXCERCISE SPACE
% \begin{bmatrix}
% 	1 & 0 \\
% 	0 & 1
% \end{bmatrix}

% \[ vec{v} =  8\hat{i} - 6\hat{j} 

% $$ vec{v} =  8\hat{i} - 6\hat{j} $

% \frac{1}{9} = \frac{2}{18}
%%%%%%%%%%%%%%%%%%%%%%%%%%%%%%%%%%%%%%%%%%

\textbf{Task D:} Undefined control sequence.
This error comes from LaTex not understanding what command(s) you're using.
This can come from mispelling the command, the command (often custom) not existing, or wrong backslash (if you have a text editor like VS code or Overleaf you can see this).
As a word of caution, if you have brackets, curly braces, or environment, this error will show up on the line of the \textit{last} curly brace, bracket, or environment.
Fix the errors in the Task D excercise space.
You might have to google these items.

%%%%%%%%%%%%%%%%%%%%%%%%%%%%%%%%%%%%%%%%%%
% TASK D EXCERCISE SPACE
\whatIsTheCommandToMakeALinkAppear

\begin{enumerate}
    \item {
        $\Ohm$ is equal to $\unit{\Siemens}^{-1}$
    }

    \item meow meow bark bark
\end{enumerate}

{
    $x = 1$

    $y = 1$

    $z = ?$

    $z \tmes = xy = 1$
}

%%%%%%%%%%%%%%%%%%%%%%%%%%%%%%%%%%%%%%%%%%

\textbf{Task E:} There is no line here to end. 
The double back slash is used to create a new line in Overleaf. 
This is not needed in VS Code since you just need an empty line separated between the lines.
You might see this older questions that were migrated from the Spring 2023 Overleaf Repo.
In the excercise block below, remove or rearrange the double backslashes such that it will not error.
PLEASE DO NOT DO THIS.
THIS IS VERY BAD.

%%%%%%%%%%%%%%%%%%%%%%%%%%%%%%%%%%%%%%%%%%
% TASK E EXCERCISE SPACE
Weird line \\
line

Weirder line 
\\

line
%%%%%%%%%%%%%%%%%%%%%%%%%%%%%%%%%%%%%%%%%%


\textbf{Optional FYI:} If you get the error "Figure not found", chances are it means that your path to the figure was not valid. 
Due to how the make file and LaTex compilation is for the 16a repo, your path should be from the main file and not from the file itself.
Main file vs file meaning here is the source directory listed in the make file and the tex file you are editing.
Notice that the tex file you are editing is in multiple folders away from the main directory.

%%%%%% PART THREE: QUESTION MAKING %%%%%%
This is going to be more of a FYI section.

Refer to "q_template.tex" for an example. 
This file has the item description and meta switched, but don't do that.
Generally structure will follow, 

"meta --> item description --> ans"

\textbf{Meta:}
Meta should include information for mentors to follow, like how the question should be interpreted and a motivation.
Unless your question is meant for mechanical practice, it should have a learning goal for your audience.

\textbf{Item:}
The item description should be the question itself.
Be VERY careful in your wording in the question.
You shouldn't be reverse engineering an item description from your answer. 
Everything a student needs to answer a question should be given in the item description and any additional guidance needed needs to be written in the meta.

\textbf{Answer:}
Ans is the answer.
Never use the words "intuitively" or "obviously."
Even without a mentor, the solutions should be accessible to everyone and should explain the problem \textbf{accurately}.
If you don't know it, then don't write it.
Ask a \textbf{SCM} first for help, then ask the Coord if needed.