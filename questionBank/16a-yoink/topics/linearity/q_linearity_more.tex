% Author: Dun-Ming Huang
% Email: dunmingbrandonhuang@berkeley.edu
% CSM16A Fall 2022
\qns{Are you a Transformation, because I want to be real-li-near you}

Hey, stop roasting the puns, because it's time for linearity.

\textbf{Learning Goal:} In this question, you will\dots
\begin{bindenum}
    \item Learn what makes a transformation linear.
    \item Prove if a transformation is linear.
    \item Recognize if a transformation is linear or not, mostly by looking.
\end{bindenum}

\meta{
    Here are some comments:
    \begin{itemize}
        \item Treat transformations $f$ as a mathematical object, just like treating functions in Python.
    \end{itemize}
}

First, let's have a brief review on linearity:
\begin{ln-quest}{The Definition of Linearity}{}
    If a transformation $f(x)$ is linear, then it satisfies two properties:
    \begin{bindenum}
        \item \textbf{Superposition}: $\forall x, y \ \ (f(x) + f(y) = f(x + y))$
        \item \textbf{Homogeneity}: $\forall x, \alpha \ \ (f(\alpha x) = \alpha f(x))$
    \end{bindenum}
    In other words,
    \[f(\alpha x + \beta y) = \alpha f(x) + \beta f(y)\]
\end{ln-quest}
Below, decide whether the transformations in the prompts are linear or not:

\begin{enumerate}
    \item {
        $f(x, y) = 3x + 4y$ \\
        \textit{Hint: there are a couple of ways to prove this, but which one would you go with?}
        
    }
    \meta {
        Here are some comments:
        \begin{itemize}
            \item {
                This problem has two slightly different ways to solve:
                \begin{bindenum}
                    \item[1.] Prove superposition and then homogeneity, performing two smaller proofs.
                    \item[2.] Prove superposition and homogeneity in one same proof, as the previous linearity question in our contents question bank have done.
                \end{bindenum}
            }
            \item {
                \textbf{First approach is preferred} for this question. It is easier to demonstrate homogeneity and superposition in two separate proofs than in one single proof.
            }
            \item {
                While concise, \textbf{second approach can be difficult to grasp} for the amount of algebraic manipulations involve. Write out the small steps of proofs.
            }
            \item {
                Learning the \textbf{second approach is valuable}, because the proofs done by the second approach is basically a slightly larger proof that contains both a homogeneity proof and a superposition proof.
            }
        \end{itemize}
    }
    \ans {
        \textbf{Approach 1: Perform two minor derivations/proofs} \\
        Linearity is the combination of two separate properties: homogeneity and superposition. So, for a transformation to be linear, the \textbf{transformation must have the two separate properties of linearity}. \\
        Let us first prove the homogeneity of this transformation $f$:
        \begin{quote}
            Homogeneity: \\
            Prove that for arbitrary real numbers $\alpha, x, y$, $f(\alpha x, \alpha y) = \alpha f(x, y)$ .
        \end{quote}
        \begin{align*}
            f(\alpha x, \alpha y)
            &= 3 \alpha x + 4 \alpha y \\
            &= \alpha (3x + 4y)
            = \alpha f(x, y)
        \end{align*}
        Then, let's prove the superposition property of this transformation $f$:
        \begin{quote}
            Superposition: \\
            Prove that for arbitrary real numbers $x_1, x_2, y_1, y_2$, $f(x_1, y_1) + f(x_2, y_2) = f(x_1 + x_2, y_1 + y_2)$.
        \end{quote}
        \begin{align*}
            f(x_1, y_1) + f(x_2, y_2)
            &= 3x_1 + 4y_1 + 3x_2 + 4y_2 \\
            &= 3(x_1 + x_2) + 4(y_1 + y_2)
            = f(x_1 + x_2, y_1 + y_2)
        \end{align*}
        Since the transformation $f$ \textbf{has both properties of superposition and homogeneity}, $f$ also has linearity, and is \textbf{therefore a linear transformation}. \\
        \textbf{Approach 2: One proof via the shortcut expression of linearity} \\
        Linearity is the combination of two separate properties: homogeneity and superposition. So, for a transformation to be linear, the \textbf{transformation must have both homogeneity and superposition}. \\
        Translating the above sentence into mathematical terms:
        \begin{quote}
            For a transformation $f$, if:
            \[\alpha f(x_1, x_2) + \beta f(y_1, y_2) = f(\alpha x_1 + \beta y_1, \alpha x_2 + \beta y_2)\]
            then $f$ has linearity and is a linear transformation.
        \end{quote}
        Let us proceed to the derivations:
        \begin{align*}
            \alpha f(x_1, x_2) + \beta f(y_1, y_2)
            &= \alpha (3x_1 + 4x_2) + \beta (3y_1 + 4y_2) \\
            &= 3 \alpha x_1 + 3 \beta y_1 + 4 \alpha x_2 + 4 \beta y_2 \\
            &= 3(\alpha x_1 + \beta y_1) + 4(\alpha x_2 + \beta y_2)
            = f(\alpha x_1 + \beta y_1, \alpha x_2 + \beta y_2)
        \end{align*}
    }
    
    \item {
        $f(x_1, \dots, x_{5}) = \sum_{i = 1}^{5} i x_i$
        
    }
    \meta {
        Here are some comments:
        \begin{itemize}
            \item {
                Encourage students abbreviate mathematical expressions during proof, either using $\sum$ or $\dots$, instead of writing the items of a sequence out.
            }
            \item {
                Talk about how to infer the overall "shape" of a linear transformation equation; for example: the structure of a linear equation.
            }
        \end{itemize}
    }
    \ans {
        Having introduced the shortcut proof method in the previous solution, using the property:
        \[
            \alpha f(x_1, \dots, x_n) + \beta f(y_1, \dots, y_n)
            = f(\alpha x_1 + \beta y_1, \dots, \alpha x_n + \beta y_n)
        \]
        We can prove this above property for the transformation given by the prompt to show the transformation's linearity. \\
        But first, let's attempt to make the transformation's expression become easier to interpret:
        \begin{align*}
            f(x_1, \dots, x_{5}) 
            &= \sum_{i = 1}^{5} i x_i \\
            &= 1 \times x_1 + 2 \times x_2 + \dots + 4 \times x_{4} + 5 \times x_{5} \\
        \end{align*}
        Now, let us proceed with the proof:
        \begin{align*}
            \alpha f(x_1, \dots, x_n) + \beta f(y_1, \dots, y_n)
            &= \alpha (1 \times x_1 + \dots + 5 \times x_{5}) + \beta (1 \times y_1 + \dots + 5 \times y_{5}) \\
            &= 1 \alpha x_1 + 1 \beta y_1 + \dots + 5 \alpha x_{5} + 5 \beta x_{5} \\
            &= 1 (\alpha x_1 + \beta y_1) + \dots + 5 (\alpha x_{5} + \beta x_{5}) \\
            &= f(\alpha x_1 + \beta y_1, \dots, \alpha x_{5} + \beta y_{5})
        \end{align*}
        The linearity of transformation $f$ is therefore demonstrated.
    }
    
    \item {
        $f(x_1, \dots, x_{10}) = \prod_{i = 1}^{10} i x_i$.
        
    }
    \meta {
        \begin{itemize}
            \item {
                It is fine to just write off "no" on the solution, but to present a legitimate argument, a proof is still presented in the solution
            }
            \item {
                If a demonstration for proof of non-linearity is needed, feel free to turn to subproblem (d) instead of this subquestion.
            }
        \end{itemize}
    }
    \ans {
        It is much better to go with gut feeling and state that this equation represented by $f$ does not seem linear, and in fact, is a 10-degree polynomial. \\
        But, in case a proof is needed, here follows a proof: \\
        Once again, let us attempt to prove the property of linearity that:
        \[
            \alpha f(x_1, \dots, x_n) + \beta f(y_1, \dots, y_n)
            = f(\alpha x_1 + \beta y_1, \dots, \alpha x_n + \beta y_n)
        \]
        To "simplify" the expression of transformation:
        \begin{align*}
            f(x_1, \dots, x_{10})
            &= \prod_{i = 1}^{10} i x_i \\
            &= 10! (x_1 \times x_2 \times \dots \times x_9 \times x_{10})
        \end{align*}
        \begin{align*}
            \alpha f(x_1, \dots, x_{10}) + \beta f(y_1, \dots, y_{10})
            &= 10! (\alpha (x_1 \times \dots \times x_{10}) + \beta (y_1 \times \dots \times y_{10}) \\
            f(\alpha x_1 + \beta y_1, \dots, \alpha x_n + \beta y_n)
            &= 10! ((\alpha x_1 + \beta y_1) \times \dots \times (\alpha x_{10} + \beta y_{10}))
        \end{align*}
        The \textbf{equation for linearity property cannot hold for every possible combinations of inputs} ($x$ and $y$); for example, try out the case where all variables have a value of $1$. \\
        From either approach, the \textbf{transformation is not linear for every possible combination of inputs}, and therefore nonlinear.
    }
    
    \item {
        $f(x, y) = ax + by + c$, where $a, b, c \in \R$.
        
    }
    \meta {
        Alert students that \textbf{not all linear equations are linear transformations}.
        
    }
    \ans {
        While the lecture notes have explicitly listed this form of transformation to be affine rather than linear, having an assistive proof in the solutions wouldn't hurt.
        To prove that this transformation is linear, it must be that, once again:
        \[
            \alpha f(x_1, y_1) + \beta f(x_2, y_2)
            = f(\alpha x_1 + \beta x_2, \alpha y_1 + \beta y_2)
        \]
        Let's proceed with the derivations:
        \begin{align*}
            \alpha f(x_1, y_1) + \beta f(x_2, y_2)
            &= \alpha (ax_1 + by_1 + c) + \beta (ax_2 + by_2 + c) \\
            &= a (\alpha x_1 + \beta x_2) + b (\alpha y_1 + \beta y_2) + c (\alpha + \beta) \\
            f(\alpha x_1 + \beta x_2, \alpha y_1 + \beta y_2)
            &= a (\alpha x_1 + \beta x_2) + b (\alpha y_1 + \beta y_2) + c \\
            &\neq \alpha f(x_1, y_1) + \beta f(x_2, y_2)
        \end{align*}
        The transformation is not linear.
    }
    
    \item {
        $f(\vec{x}) = A \vec{x}$, where $A$ is a matrix. Assume the products of $A \vec{x}$ are defined.
        
    }
    \meta {
        \begin{itemize}
            \item {
                There has not seem to be a proof regarding matrix-vector multiplications being a linear transformation, despite it being stated and applied multiple times throughout EECS 16 series.
            }
            \item {
                Let students \textbf{remember about matrix transformations being linear} and practice \textbf{linear transformation proofs on non-scalar inputs}.
            }
        \end{itemize}
    }
    \ans {
        To prove the linearity of this transformation $f$, we must demonstrate that:
        \[f(\alpha \vec{x} + \beta \vec{y}) = \alpha f(\vec{x}) + \beta f(\vec{y})\]
        Let us proceed to the proof:
        \begin{align*}
            f(\alpha \vec{x} + \beta \vec{y})
            &= A (\alpha \vec{x} + \beta \vec{y}) \\
            &= \alpha A \vec{x} + \beta A \vec{y} \\
            &= \alpha f(\vec{x}) + \beta f(\vec{y})
        \end{align*}
        The linearity of $f$ is thus demonstrated.
    }
\end{enumerate}