% Author: Chris Duroiu
% Email: chduroiu@berkeley.edu
% CSM16A Spring 2024

\qns{Eigenvalue and Eigenvector Properties Proof}

\textbf{Learning Goal: }
\begin{bindenum}
  \item This quesiton is meant to review proofs involving eigenvalues and how to express them regarding generic matrices
  \item This question is also meant to teach students the relationships between eigenvalues/vectors and symmetric/invertible matrices
\end{bindenum}

\begin{enumerate}
  \item {
    Question Prompt: Suppose we have an invertible matrix $A \in R^{n \times n}$, that is $A=A^{T}$
  }
  \item {
    Prove that if $\lambda$ is an eigenvalue of matrix $\mathrm{A}$, then $\lambda^{n}$ is an eigenvalue of $A^{n}$ and matrix $\mathrm{A}$ and $A^{n}$ have the same eigenvectors.
  }
  \meta{
    This is a standard eigen question meant to teach students that generalizing steps is a valid form of proof.
  }
  \ans{
  Let $\bar{v}_{i}$ be some eigenvector of matrix $A^{n}$. We can write \\

  $A^{n} \bar{v}_{i}=A^{n-1} A \bar{v}_{i}$ 
  
  Let's assume that $\mathrm{A}$ and $A^{n}$ indeed have the same eigenvalues. Therefore, \\
  
  $A \bar{v}_{i}=\lambda_{i} \bar{v}_{i}$ so we can write
  
  $A^{n} \bar{v}_{i}=A^{n-1} \lambda_{i} \bar{v}_{i}$
  
  $A^{n} \bar{v}_{i}=\lambda_{i} A^{n-1} \bar{v}_{i}$ 
  
  Repeating this process, \\
  
  $A^{n} \bar{v}_{i}=\lambda_{i} A^{n-2} A \bar{v}_{i}$. 
  
  $A^{n} \bar{v}_{i}=\lambda_{i} A^{n-2} \lambda_{i} \bar{v}_{i}$
  
  $A^{n} \bar{v}_{i}=\lambda_{i}^{2} A^{n-2} \bar{v}_{i}$ 
  
  Eventually, we can reach \\
  
  $A^{n} \bar{v}_{i}=\lambda_{i}^{n} \bar{v}_{i}$, thus proving that $\lambda^{n}$ is an eigenvalue of $A^{n}$, corresponding to some shared eigenvector $\bar{v}_{i}$ 
  }

  \item {
    Now assume matrix A is invertible, that is a matrix $A^{-1}$ exists such that $AA^{-1} = I$. Prove that if $\lambda$ is an eigenvalue of matrix $\mathrm{A}$, then $\frac{1}{\lambda}$ is an eigenvalue of $A^{-1}$
  }
  \meta{
    This question is meant to relate eigen-properties to invertible matrices
  }
  \ans{
  Since $\lambda$ is an eigenvalue of matrix $A$, we can write \\

  $A \bar{v}=\lambda \bar{v}$, for the corresponding eigenvector $\bar{v}$
  
  $A^{-1} A \bar{v}=A^{-1} \bar{\lambda} \bar{v}$
  
  $\bar{v}=A^{-1} \lambda \bar{v}$
  
  $\bar{v}=\lambda A^{-1-} \bar{v}$
  
  $\frac{1}{\lambda} \bar{v}=A^{-1}\bar{v}$ 
  
  By definition, this means $\frac{1}{\lambda}$ is an eigenvalue of matrix $A^{-1}$

  }
  \item {
    Now assume matrix A is symmetric, that is $A=A^{T}$. Prove that any two distinct eigenvectors of A must be orthogonal.
  }
  \meta{
    Remember to generalize two distinct eigenvectors and relate them to their eigenvectors to show how the vectors must be orthogonal.
  }
  \ans{
  Let $\bar{v}_{1}$ and $\bar{v}_{2}$ be two distinct eigenvectors of matrix $A$ corresponding to eigenvalues $\lambda_{1}$ and $\lambda_{2}$ respectively

  We can write \\

  $A \bar{v}_{1}=\lambda_{1} \bar{v}_{1}$

  $\bar{v}_{2}^{T} A \bar{v}_{1}=\bar{v}_{2}^{T} \lambda_{1} \bar{v}_{1}$

  $\bar{v}_{2}^{T} A \bar{v}_{1}=\lambda_{1} \bar{v}_{2}^{T} \bar{v}_{1}$ 

  We can also write \\

  $\bar{v}_{2}^{T} A \bar{v}_{1}=\bar{v}_{2}^{T} A^{T} \bar{v}_{1}$, since we know $A=A^{T}$

  $\bar{v}_{2}^{T} A^{T} \bar{v}_{1}=\left(A \bar{v}_{2}\right)^{T} \bar{v}_{1}=\left(\lambda_{2} \bar{v}_{2}\right)^{T} \bar{v}_{1}=\lambda_{2} \bar{v}_{2}^{T} \bar{v}_{1}$ \\

  This implies that \\

  $\lambda_{1} \bar{v}_{2}^{T} \bar{v}_{1}=\lambda_{2} \bar{v}_{2}^{T} \bar{v}_{1}$ \\

  Since $\lambda_{1}$ and $\lambda_{2}$ are two distinct eigenvectors, we know that the quantity

  $\bar{v}_{2}^{T} \bar{v}_{1}$ must be 0 , proving $\bar{v}_{1}$ and $\bar{v}_{2}$ must be orthogonal.
  }

  \item {
    By definition, the trace of a matrix $A$, denoted Tr(A) is $\lambda_{1}+\lambda_{2}+\lambda_{3}+\ldots \lambda_{n}$ 
  Prove that $Tr(A^{2})=\lambda^{2}{ }_{1}+\lambda^{2}{ }_{2}+\lambda^{2}{ }_{3}+\lambda^{2}{ }_{n}$. Continue to assume that matrix $\mathrm{A}$ is symmetric. Hint: the trace of a product of matrices is invariant under cyclic permutations ie: Tr(XYZ) = Tr(YXZ) = Tr(ZYX) ...
  }
  \meta{
    This is a trickier question meant to introduce the concept of trace and diagonal matrices. Remember to apply the cycle permutation property of traces
  }
  \ans{
  Since $\mathrm{A}$ is symmetric, we know there exists an orthogonal matrix $\mathrm{P}$ such that \\

  $P^{T} AP = D$, where $\mathrm{D}$ is a diagonal matrix consisting of A's eigenvalues

  $P P^{T} A P=P D$

  $A P=P D$

  $A P P^{T}=P D P^{T}$

  $A=P D P^{T}$ 

  Thus, \\

  $A^{2}=\left(P D P^{T}\right)\left(P D P^{T}\right)=P D I D P^{T}=P D^{2} P^{T}$ 

  We know that $D^{2}$ is now a diagonal matrix that includes all of the squares of A's eigenvalues $Tr(A^{2}) = Tr(PD^{2}P^{T})$ \\

  As suggested by the hint, we know that the trace of a product of matrices is invariant under cyclic permutations, thus \\

  $Tr(PD^{2}P^{T})$ = $Tr(D^{2}PP^{T})$ = $Tr(D^{2}I)$ = $Tr(D^{2})$

  We know that the trace of $D^{2}$ is just \\

  $\lambda_{1}^{2}+\lambda_{2}^{2}+\ldots \lambda_{n}^{2}$ 

  Therefore we proved \\

  $Tr(A^{2})=\lambda_{1}^{2}+\lambda_{2}^{2}+\ldots \lambda_{n}^{2}$
  }
\end{enumerate}