% Author: Mihir Marathe
% Email: mihirmarathe@berkeley.edu

\qns{Eigen Calculation}

\textbf{Learning Goal:} The goal of this problem is to practice mechanically calculating eigenvalues and finding their corresponding eigenvectors.

%\textbf{Relevant Notes:} \textbf{Note 9 Sections 9.4 and 9.6} cover the process of finding eigenvalue-eigenvector pairs.

%\newcommand{\Amat}{\ensuremath{\begin{bmatrix}
%3 & 2  \\
%1 & 4
%\end{bmatrix}}}

\meta{
    This problem is supposed to be straightforward, so make sure to go over the technique used in this problem. 
    
	This problem focuses on the mechanics of solving eigenvalues and eigenvectors. Feel free to focus more on other problems if you feel your students are comfortable with this process. 
    
    Be prepared to answer why the determinant of $\mathbf{A} - \mathbf{I}\lambda$ is equivalent to 0. You can relate this process to finding the Null Space of a matrix.

	For the first part, have students understand that if a matrix only has non-zero entries on the diagonal, you can read off the eigenvalues on the diagonal.
}

\begin{enumerate}

\item{
	Solve for the eigenvalue-eigenvector pairs for the following matrices:

	\begin{align*}
		\begin{bmatrix} 
			5 & 0\\
			0 & 6
		\end{bmatrix}
	\end{align*}
}

\ans{

	To solve for eigenvalues and eigenvectors, let's go back and review the definition of eigenvectors and eigenvalues:
	
	If $\vec{x}$ and $\lambda$ are the eigenvector and eigenvalue of $\mathbf{A}$, respectively, then the following equation holds:
	
	$\mathbf{A}\vec{x} = \lambda\vec{x}$
	
	Since the (appropriately sized) identity matrix is analogous to multiplying by 1 in arithmetic, we can say:

    $\mathbf{A}\vec{x} = \lambda \mathbf{I} \vec{x}$
	
	Rearranging, we get:

	\begin{align*}
		& \mathbf{A}\vec{x} - (\lambda \mathbf{I}) \vec{x} = \vec{0} \\
		& (\mathbf{A} - \lambda \mathbf{I})\vec{x} = \vec{0}  
	\end{align*}
	
	% $\mathbf{A}\vec{x} - (\lambda \mathbf{I}) \vec{x} = \vec{0}
	% $
	% $
	% (\mathbf{A} - \lambda \mathbf{I})\vec{x} = \vec{0}
	% $
	
	What does this look like? It looks similar to solving for the nullspace of $(\mathbf{A} - \lambda \mathbf{I})$!
	
	Assuming that there is a nontrivial nullspace, that also means that $\mathbf{det}(\mathbf{A} - \lambda \mathbf{I}) = 0$!
	
	Let's solve for $\lambda$ first:
	\begin{align*}
		(\mathbf{A} - \lambda \mathbf{I}) &=  
		% - \begin{bmatrix}
		% \lambda & 0 \\
		% 0 & \lambda
		% \end{bmatrix}
		    \begin{bmatrix}
			5 - \lambda & 0 \\
			0 & 6 - \lambda
			\end{bmatrix} 	\\
		\mathbf{det}(\mathbf{A} - \lambda \mathbf{I}) &= (5 - \lambda)(6 - \lambda) - 0 \\
		&= 30 - 11\lambda + \lambda^2 \\
		&= (\lambda - 5)(\lambda - 6)
	\end{align*}

	By factoring:

	\[\lambda = 5, 6\]
	
	Let's check: We've just solved for the eigenvalues. But what about the eigenvectors? 
	
	To do that, we plug in $\lambda$ into $(\mathbf{A} - \lambda \mathbf{I})$ and solve for the nullspace!
	
	For $\lambda = 5$:
	
	\begin{align*}
		(\mathbf{A} - \lambda \mathbf{I})\vec{x} &= \vec{0} \\
		\begin{bmatrix}
			5 - \lambda & 0 \\
			0 & 6 - \lambda
			\end{bmatrix}
			\begin{bmatrix} 
			x_1 \\
			x_2
		\end{bmatrix} &= \vec{0}
	\end{align*}

	\[
		\begin{bmatrix}
			0 & 0 \\
			0 & 1
			\end{bmatrix}
			\begin{bmatrix} 
			x_1 \\
			x_2
		\end{bmatrix} = \vec{0}
	\]
	
	% Edit comment: this is incorrect
	% By row reduction:
	% \begin{align*}
	% 	\begin{bmatrix}
	% 		1 & -1 \\
	% 		0 & 0
	% 		\end{bmatrix}
	% 		\begin{bmatrix} 
	% 		x_1 \\
	% 		x_2
	% 	\end{bmatrix} &= \vec{0} \\
	% 	x_1 &= x_1 \\
	% 	x_2 &= 0 \\
	% \end{align*}  
	\[
		\begin{bmatrix} 
		x_1 \\
		x_2
		\end{bmatrix} = 
		\begin{bmatrix} 
		0 \\
		-1
		\end{bmatrix}x_2
	\]

	So the first pair is 
		\[\lambda = 5, \vec{x} =  \begin{bmatrix} 
	1 \\
	0
	\end{bmatrix}\] 

	\par

	Repeating for $\lambda = 6$, 
	\begin{align*}
		\begin{bmatrix}
			5 - \lambda & 0 \\
			0 & 6 - \lambda
			\end{bmatrix}
			\begin{bmatrix} 
			x_1 \\
			x_2
		\end{bmatrix} &= \vec{0} \\
		\begin{bmatrix}
			-1 & 0 \\
			0 & 0
			\end{bmatrix}
			\begin{bmatrix} 
			x_1 \\
			x_2
			\end{bmatrix} &= \vec{0} \\
		x_1 &= 0 \\
		x_2 &= x_2 
	\end{align*}
	We see that,
  
	\[
	\begin{bmatrix} 
	x_1 \\
	x_2
	\end{bmatrix} = 
	\begin{bmatrix} 
	0 \\
	1
	\end{bmatrix}x_2
	\]
  
	So, the second pair is

	\[\lambda = 6, \vec{x} =
	\begin{bmatrix} 
	0 \\
	1
	\end{bmatrix}\]
}

\item{
	\begin{align*}
		& \begin{bmatrix} 
			1 & 2\\
			3 & 0\\
		\end{bmatrix} \\
	\end{align*}

}

\ans{
	We can use the same approach as the previous part to solve this problem. 

	\begin{align*}
		& \mathbf{A}\vec{x} - (\lambda \mathbf{I}) \vec{x} = \vec{0} \\
		& (\mathbf{A} - \lambda \mathbf{I})\vec{x} = \vec{0}  
	\end{align*}

	\begin{align*} 
		(\mathbf{A} - \lambda \mathbf{I}) & =  - \begin{bmatrix}
		1 - \lambda & 2 \\
		3 & 0 - \lambda
		\end{bmatrix} \\
		\mathbf{det}(\mathbf{A} - \lambda \mathbf{I}) & = (1 - \lambda)(0 - \lambda) - 6 \\ 
		& = \lambda^2 - \lambda - 6 \\
		& = (\lambda - 3)(\lambda + 2)
	\end{align*}

		By factoring:
		$\lambda = 3, -2$

		
		Let's check: We've just solved for the eigenvalues. But what about the eigenvectors? 
		
		To do that, we plug in $\lambda$ into $(\mathbf{A} - \lambda \mathbf{I})$ and solve for the nullspace!
		
		For $\lambda = 3$:
		
	\begin{align*}
		(\mathbf{A} - \lambda \mathbf{I})\vec{x} &= \vec{0} \\
		\begin{bmatrix}
			1 - \lambda & 2 \\
			3 & 0 - \lambda
			\end{bmatrix}
			\begin{bmatrix} 
			x_1 \\
			x_2
		\end{bmatrix} &= \vec{0} \\
		\begin{bmatrix}
			-2 & 2 \\
			3 & -3
			\end{bmatrix}
			\begin{bmatrix} 
			x_1 \\
			x_2
		\end{bmatrix} &= \vec{0}
	\end{align*}
		
	% 	By row reduction:
	% \begin{align*}
	% 	\begin{bmatrix}
	% 		-2 & 2 \\
	% 		0 & 0
	% 		\end{bmatrix}
	% 		\begin{bmatrix} 
	% 		x_1 \\
	% 		x_2
	% 	\end{bmatrix} &= \vec{0} \\
	\begin{align*}
		x_1 &= x_2 \\ 
		x_2 &= x_2 \\
		\begin{bmatrix} 
			x_1 \\
			x_2
			\end{bmatrix} &= 
			\begin{bmatrix} 
			1 \\
			1
		\end{bmatrix}x_2
	\end{align*}
		
		So the first pair is $\lambda =3, \vec{x} = \begin{bmatrix} 
		1 \\
		1
		\end{bmatrix}$ \\
	
		Repeating for $\lambda = -2$, 
		\begin{align*}
			\begin{bmatrix}
				1 - \lambda & 2 \\
				3 & 0 - \lambda
				\end{bmatrix}
				\begin{bmatrix} 
				x_1 \\
				x_2
			\end{bmatrix} &= \vec{0} \\
			\begin{bmatrix}
				3 & 2 \\
				3 & 2
				\end{bmatrix}
				\begin{bmatrix} 
				x_1 \\
				x_2
			\end{bmatrix} &= \vec{0} \\
			\begin{bmatrix}
				3 & 2 \\
				0 & 0
				\end{bmatrix}
				\begin{bmatrix} 
				x_1 \\
				x_2
			\end{bmatrix} &= \vec{0} 
		\end{align*}

		\begin{align*}
			& 3x_1 = -2x_2 \\ 
			& x_2 = x_2
		\end{align*}
		\begin{align*}
			\begin{bmatrix} 
				x_1 \\
				x_2
				\end{bmatrix} = 
				\begin{bmatrix} 
				-2 \\
				3
				\end{bmatrix}x_2
		\end{align*}
	
		
		So, the second pair is
	
		\begin{align*}
			\lambda = -2, \vec{x} = 
			\begin{bmatrix} 
			-2 \\
			3
			\end{bmatrix}
		\end{align*}
}

\item{
	
	\begin{align*}
		& \begin{bmatrix} 
			-2 & -4 & 2\\
			-2 & 1 & 2\\
			4 & 2 & 5\\
		\end{bmatrix}\\
	\end{align*}
	
}

\ans{
	We can use the same approach as the previous part to solve this problem. 


	\begin{align*}
		& \mathbf{A}\vec{x} - (\lambda \mathbf{I}) \vec{x} = \vec{0} \\
		& (\mathbf{A} - \lambda \mathbf{I})\vec{x} = \vec{0}  
	\end{align*}

	\begin{align*} 
		(\mathbf{A} - \lambda \mathbf{I}) &= \begin{bmatrix}
		-2 - \lambda & -4 & 2\\
		-2 & 1 - \lambda & 2 \\
		4 & 2 & 5 - \lambda \\
		\end{bmatrix} \\
		\mathbf{det}(\mathbf{A} - \lambda \mathbf{I}) &= (-2 - \lambda)[(1 - \lambda)(5 - \lambda) - 2(2)] + 4[-2(5 - \lambda - 4(2))] + 2[-4 - 4(1 - \lambda)] \\
		&= -\lambda^3 + 4\lambda^2 + 27\lambda - 90\\
		&= (\lambda - 3)(\lambda + 5)(\lambda - 6)
	\end{align*}
		By factoring:
		$\lambda = 3, -5, 6$

		
		Let's check: We've just solved for the eigenvalues. But what about the eigenvectors? 
		
		To do that, we plug in $\lambda$ into $(\mathbf{A} - \lambda \mathbf{I})$ and solve for the nullspace!
		
		For $\lambda = 3$:
		
		\begin{align*}
			& (\mathbf{A} - \lambda \mathbf{I})\vec{x} = \vec{0} \\
			& \begin{bmatrix}
			-5 & -4 & 2\\
			-2 & -2 & 2\\
			4 & 2 & 2\\
			\end{bmatrix}
			\begin{bmatrix} 
			x_1 \\
			x_2\\
			x_3\\
			\end{bmatrix} = \vec{0} \\
		\end{align*}
		\begin{align*}
				-5x_1 - 4x_2 + 2x_3 &= 0 \\
			   -2x_1 - 2x_2 + 2x_3 &= 0 \\
			   \text{Set $x_1$ equal to 1}:\\
			   -2 + 3 + 2x_3 &= 0\\
		\end{align*}
		
		This gives us the following eigenvector:

		\begin{align*}
		\begin{bmatrix} 
		x_1 \\
		x_2 \\
		x_3 \\
		\end{bmatrix} =
		\begin{bmatrix} 
		2 \\
		-3 \\
		-1 \\
		\end{bmatrix}
		\end{align*}
	

		Repeating for $\lambda = -5$, 
		
		\begin{align*}
			\begin{bmatrix}
				3 & -4 & 2 \\
				-2 & 6 & 2 \\
				4 & 2 & 10 \\
				\end{bmatrix}
				\begin{bmatrix} 
				x_1 \\
				x_2 \\
				x_3 \\
			\end{bmatrix} = \vec{0}
		\end{align*}
	
		\begin{align*}
			   3x_1 - 4x_2 + 2x_3 &= 0 \\
			   -2x_1 + 6x_2 + 2x_3 &= 0 \\
			   \text{Set $x_1$ equal to 1}:\\
			   -5 + 10x_2 &= 0\\
		\end{align*}

		This gives the following eigenvector:
		$
		\begin{bmatrix} 
		x_1 \\
		x_2 \\
		x_3 \\
		\end{bmatrix} =
		\begin{bmatrix} 
		-2 \\
		1 \\
		1 \\
		\end{bmatrix}
		$

		Repeating for $\lambda = 6$, 
		\begin{align*}
			\begin{bmatrix}
				-8 & -4 & 2 \\
				-2 & -5 & 2 \\
				4 & 2 & -1 \\
				\end{bmatrix}
				\begin{bmatrix} 
				x_1 \\
				x_2 \\
				x_3 \\
			\end{bmatrix} = \vec{0}
		\end{align*}		
	
		\begin{align*}
			   -8x_1 - 4x_2 + 2x_3 &= 0 \\
			   -2x_1 - 5x_2 + 2x_3 &= 0 \\
			   \text{Set $x_1$ equal to 1:}\\
			   6 - x_2 &= 0\\
		\end{align*}

		This gives the following eigenvector:
		
		\[\begin{bmatrix} 
		x_1 \\
		x_2 \\
		x_3 \\
		\end{bmatrix} =
		\begin{bmatrix} 
		1 \\
		6 \\
		16 \\
		\end{bmatrix}\]
}




\end{enumerate}