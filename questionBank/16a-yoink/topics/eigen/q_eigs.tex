% Author: Anya Shrivastava
\qns{Eigenvalues}

\begin{enumerate}

% Part A
\item Suppose we have $A = 
    \begin{bmatrix} 
        1 & 0 & 0\\
        0 & 2 & 0\\
        0 & 0 & 3\\
    \end{bmatrix}$
, what are the eigenvalues?

\ans{
The matrix is diagonal. By inspection, the eigenvalues are 1,2,3.
}

% Part B
\item Suppose we have $A = 
    \begin{bmatrix} 
        1 & 2\\
        0 & 3\\
    \end{bmatrix}$
. What are the eigenvalues and eigenvectors?

\ans{
    $$\mathbf{A} - \lambda * \mathbf{I} = 
        \begin{bmatrix} 
        1 - \lambda & 2\\
        0 & 3 - \lambda
        \end{bmatrix}$$
            
    $$(1 - \lambda)(3 - \lambda) = 0$$
    $$\lambda = \text{1, 3}$$
    
    $$\text{when $\lambda = 1$}$$
        $$\begin{bmatrix}
            0 & 2 \\
            0 & 1
        \end{bmatrix} \rightarrow 
        \begin{bmatrix}
            0 & 1 \\
            0 & 0
        \end{bmatrix}$$ \\
        $$\vec{v_1} = \begin{bmatrix}
            1 \\
            0
        \end{bmatrix} $$

    $$\text{when $\lambda = 3$}$$
        $$\begin{bmatrix}
            -2 & 2 \\
            0 & 0
        \end{bmatrix} \rightarrow 
        \begin{bmatrix}
            1 & -1 \\
            0 & 0
        \end{bmatrix}$$ \\
        $$\vec{v_2} = \begin{bmatrix}
            1 \\
            1
        \end{bmatrix}$$

}

% Part C
\item Suppose we have a vector $\vec{b} = 
    \begin{bmatrix} 3\\ 2 \end{bmatrix}$
. Draw this vector in the standard basis and eigenbasis.

\meta{ Review with students what a basis is and how the eigenvectors of a matrix form a basis. This part is trying to show that the eigenbasis is just another basis except its basis vectors are just not the standard ones that we are used to. 
}

\ans{
Solve $[v_1 \text{ }v_2]^\top$ $\vec{x} = \vec{b}$. You will see that $\vec{x} = \vec{v_1} + 2\vec{v_2}$. 
}

% Part D
\item If a matrix has an eigenvalue that is zero, what does that tell us about the columns of the matrix?

\ans{
Recall the definition of eigenvalue and eigenvector that $A \vec{x} = \lambda \vec{x}$ where $\vec{x}$ is a nonzero vector. If we let $\lambda$ equal to 0, the whole equation becomes $A \vec{x} = \vec{0}$. To find what is $\vec{x}$, we can think it as finding the nullspace of A. If A is linear independent, A will have a trivial nullspace, which means that only $\vec{0}$ can be the solution to $A \vec{x} = \vec{0}$. However, from the definition of eigenvalue, $\vec{x}$ should be nonzero. That's being said, if there is at least one nonzero vector existing in null space and can make $A \vec{x} = \vec{0}$, that means A must be linear independent.  
}

% Part E
\item (PRACTICE) Suppose we have a matrix $A = 
    \begin{bmatrix}
        1 & 0 & 0\\
        2 & 3 & 0\\
        0 & 0 & 0\\
    \end{bmatrix}$
. Solve for the eigenvalues and eigenvectors

\ans{
Will be written out soon!
}

\end{enumerate}