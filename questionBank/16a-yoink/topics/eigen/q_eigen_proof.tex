% Zipeng Lin - yuslzp@berkeley.edu
\qns{Eigenvalues and invertibility}

\textbf{Learning goal: learn an equivalence between eigenvalues and invertibility, and how to identify a matrix's eigenvalue}.

\meta{
\begin{enumerate}
    \item First, review the equivalence of a matrix $A$ being invertible and the
        solution to $A v = \vec{0}$
    \item Second, review the relationship between determinant and eigenvalue.
    \item The direction from eigenvalue is easier to approach, but it might
        require more algebraic understanding.
\end{enumerate}
}

Think about the following problems in two perspective, one in determinant, one
in the nullspace of the matrix.

a. Prove that a matrix $A$ is invertible if  it does not have any
eigenvalue of  $0$.

\ans{
 $\Rightarrow$ Suppose a matrix $A$ is invertible. Then, if it has a eigenvalue of  $0$,
 then exists  a vector  $\vec{v} \neq{} 0$ such that  $A\vec{v} = 0 \vec{v} = \vec{0}$ ($v$ is the
 eigenvalue), which means $A$
 has nonzero vector  $\vec{v}$ in its nullspace, so we have a contradiction.
 (We prove by contradiction here) 

 $\Leftarrow$ We have the matrix does not have any nonzero vector in its
 nullspace, since if so it would have a zero eigenvalue. Therefore, the matrix
 is invertible. We can also see this from contradiction: if it has no zero
 eigenvalue and is not invertible, then it has a nontrivial nullspace, so exists
 vector $\vec{v}$ such that  $A\vec{v} = 0$, so  $0$ is an eigenvalue, and there is a
 contradiction.

 Perspective in determinant: the eigenvalue is determined by the equation $det(A
 - \lambda I) = 0$. If  $\lambda = 0$, then  $det(A) = 0$, so  $A$ is not
 invertible, the other direction is the same.
}

b. Prove that for a not invertible matrix $A$ that only has real eigenvalues, then there exists a real number
$r$ such that matrix  $A + rI$ is
invertible. Hint: what happens to the eigenvalues?
\ans{
    We have for an eigenvalue $v$ with eigenvalue  $\lambda_A$ for $A$, we have 
     \[
        A\vec{v} = \lambda_A \vec{v} \to (A + rI) \vec{v} = (\lambda_A + r)\vec{v}
    \]

    From the determinant perspective, we have

    \[
        det(A + rI - \lambda_A I ) = 0 \to det(A + (r - \lambda_A) I) = 0
    \]

    which means the eigenvalue of $A + rI$ would be  $r$ plus a eigenvalue of
    $A$. Therefore, think about the minimum eigenvalue of  $A+rI$ which
    corresponds to the minimum eigenvalue of $A$, $\lambda$. If $\lambda$ is
    positive, then we are good since  $A$ does not have a zero eigenvalue.
    Otherwise, we can add a  $r$ that is big enough so that all the eigenvalues
   of $A + rI$ are positive, so the matrix $A + r I$ is invertible. Here, we are
   reusing the result from part (a).
    
    
}
