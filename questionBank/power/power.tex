\DeclareSIUnit{\var}{VAR}
\DeclareSIUnit{\voltampere}{VA}

\qns{Power Analysis}
\qcontributor{Tony Li}


A purely resistive(non-inductive) circuit is a circuit that has inductance so small that at its typical frequency, its reactance is insignificant
compared to its resistance. Or, we can even say that the circuit does not have any inductances and capacitances. In a purely resistive circuit, the utilized voltage is consumed by overwhelming the ohmic resistance of the circuit.

\meta {
    Please explain what reactance is and its relationship with resistance to students.
}
\begin{enumerate}

\qitem Suppose that there is a purely resistive AC circuit, and the voltage source has the voltage $v(t)=V_mcos(\omega t)$ and the current through
the circuit is $i(t)=I_mcos(\omega t)$. \textbf{What is the average power of the circuit $P_{avg}$?} \textit{Definition:}$P(t)=v(t)i(t)$, and $P_{avg}=\frac{1}{T}\int_{0}^{T}P(t)dt$,
where $T$ is a period of the circuit. \textit{Hint:} $cos(2\alpha)=cos^2(\alpha)-sin^2(\alpha)$

\ws{\vspace{80px}}

\sol{
    $$P_{avg}=\frac{1}{T}\int_{0}^{T}v(t)i(t)dt=\frac{1}{T}\int_{0}^{T}V_m I_m cos^2(\omega t)dt=\frac{1}{T}\int_{0}^{T}V_m I_m \frac{cos(2 \omega t) + 1}{2}dt$$
    $$=\frac{I_m V_m}{2T}[\frac{1}{2 \omega}sin(2 \omega t) + t]_{0}^{T}= \frac{I_m V_m}{2T}T=\frac{I_m V_m}{2}$$
}
\end{enumerate}

A purely inductive circuit is a circuit that only contains inductance and not any other resistances or capacitances. Therefore, if the voltage of the voltage source
is $v(t)$ and the current throught the circuit is $i(t)$, and the inductance of the circuit is $L$, then we can say that $v(t)=L\frac{di(t)}{dt}$. The current lags behind the voltage 
by an angle of $\frac{\pi}{2}$.

\begin{enumerate}[resume]

\qitem Suppose that there is a purely inductive AC circuit, and the voltage source has the voltage $v(t)=V_mcos(\omega t)$ and the current through
the circuit is $i(t)=I_mcos(\omega t-\frac{\pi}{2})$. \textbf{What is the average power of the circuit $P_{avg}$?} \textit{Hint:} $sin(2\alpha)=2sin(\alpha)cos(\alpha)$

\ws{\vspace{80px}}

\sol{
    $$P_{avg}=\frac{1}{T}\int_{0}^{T}v(t)i(t)dt=\frac{1}{T}\int_{0}^{T}V_m I_m cos(\omega t)cos(\omega t -\frac{\pi}{2})dt=\frac{1}{T}\int_{0}^{T}V_m I_m cos(\omega t)sin(\omega t )dt$$
    $$=\frac{I_m V_m}{T}\int_{0}^{T}\frac{sin(2\omega t)}{2}dt=\frac{I_m V_m}{2T}[-\frac{1}{2 \omega}cos(2 \omega t)]_{0}^{T}= \frac{I_m V_m}{2T}(0)=0$$
}
\end{enumerate}

A purely capacitive circuit is a circuit that only contains capacitance and not any other resistances or inductances. Therefore, if the voltage of the voltage source
is $v(t)$ and the current throught the circuit is $i(t)$, and the capacitance of the circuit is $C$, then we can say that $i(t)=C\frac{dv(t)}{dt}$. The voltage lags behind the current
by an angle of $\frac{\pi}{2}$.
Consider the following circuit:

\begin{enumerate}[resume]

\qitem Suppose that there is a purely capacitive AC circuit, and the voltage source has the voltage $v(t)=V_mcos(\omega t-\frac{\pi}{2})$ and the current through
the circuit is $i(t)=I_mcos(\omega t)$. \textbf{What is the average power of the circuit $P_{avg}$?} 

\ws{\vspace{80px}}

\sol{
    $$P_{avg}=\frac{1}{T}\int_{0}^{T}i(t)v(t)dt=\frac{1}{T}\int_{0}^{T}V_m I_m cos(\omega t)cos(\omega t -\frac{\pi}{2})dt=\frac{1}{T}\int_{0}^{T}V_m I_m cos(\omega t)sin(\omega t )dt$$
    $$=\frac{I_m V_m}{T}\int_{0}^{T}\frac{sin(2\omega t)}{2}dt=\frac{I_m V_m}{2T}[-\frac{1}{2 \omega}cos(2 \omega t)]_{0}^{T}= \frac{I_m V_m}{2T}(0)=0$$
}

\end{enumerate}

Now let's deal with the general case of calculating the power of a circuit. Suppose that there is an AC circuit, and the voltage source has the voltage $v(t)=V_mcos(\omega t+{\theta}_1)$ and the current through
the circuit is $i(t)=I_mcos(\omega t + {\theta}_2)$. 

\begin{enumerate}[resume]

\qitem \textbf{Can you find $P(t)$ and try to simplify it into the form of which we can find the integral}?  \textit{Hint:} Use the Euler's Form to make the calculation easier.
$cos(\alpha)=\frac{e^{j\alpha} + e^{-j\alpha}}{2}$

\ws{\vspace{100px}}

\meta{
    Please explain to students why we choose to use Euler's form. We can not take the integral on a cosine times cosine, but we 
    can use exponential form we know to transform a cosine times cosine into a cosine plus cosine, then the integral would be much 
    easier to perform on the summation than directly on the product. 
}

\sol{
    $$P(t)=v(t)i(t)=V_mI_mcos(\omega t + {\theta}_1)cos(\omega t + {\theta}_2)$$
    $$=\frac{V_m}{2}(e^{j(\omega t + {\theta}_1)} + e^{j(-\omega t - {\theta}_1)})\cdot\frac{I_m}{2}(e^{j(\omega t + {\theta}_2)} + e^{j(-\omega t - {\theta}_2)})$$
    $$=\frac{V_mI_m}{2}(\frac{e^{j(2 \omega t+{\theta}_1+{\theta}_2)} + e^{-j(2 \omega t+{\theta}_1+{\theta}_2)}}{2} + \frac{e^{j({\theta}_1-{\theta}_2)} + e^{-j({\theta}_1-{\theta}_2)}}{2})$$
    $$=\frac{V_mI_m}{2}(cos(2 \omega t +{\theta}_1+{\theta}_2) + cos({\theta}_1-{\theta}_2))$$
}
 
\qitem \textbf{By integrating $P(t)$, prove that} $$P_{avg}=\frac{V_mI_m}{2}cos({\theta}_1-{\theta}_2)$$

\ws{\vspace{100px}}

\sol{
    $$P_{avg}=\frac{1}{T}\int_{0}^{T}P{T}dt=\frac{1}{T}\frac{V_mI_m}{2}(\int_{0}^{T}cos(2 \omega t+ {\theta}_1+{\theta}_2)dt + \int_{0}^{T}cos({\theta}_1-{\theta}_2)dt)$$
    $$=\frac{1}{T}\frac{V_mI_m}{2}[(\frac{1}{2 \omega}sin(2 \omega t + {\theta}_1 + {\theta}_2))|^{T}_{0} + Tcos({\theta}_1-{\theta}_2)]$$
    $$=\frac{1}{T}\frac{V_mI_m}{2}[0+Tcos({\theta}_1-{\theta}_2)]=\frac{V_mI_m}{2}cos({\theta}_1-{\theta}_2)$$
    Note that $sin(2 \omega T + {\theta}_1 + {\theta}_2)=sin(2 \omega 0 + {\theta}_1 + {\theta}_2)$, because two times are seperated by one period $T$.

}

\end{enumerate}
Let ${\theta}_d={\theta}_1-{\theta}_2$. Compared to the power of a purely resistive circuit without any capcitances and inductances, we see that 
because of the existance of inductances and capacitances, $P_{avg}$ is less than or equal to that of the purely resistive circuit. We call $cos({\theta}_d)$ 
the power factor. 

\meta {
    Please briefly explain what apparent power and reactive powers are.
}
Also we introduce three concepts related to the power factor: Reactive Power, Real Power, and Apparent Power, and here are several properties of the power factor you should know:
\newline 1. $\text{power factor}=\frac{\text{real power}}{\text{apparent power}}=cos({\theta}_d)$
\newline 2. $\text{apparent power}^2=\text{real power}^2+\text{reactive power}^2$
\newline 3. If the current reaches its peak value up to $\frac{\pi}{2}$ later than the voltage, the power factor is lagging, and the circuit's load is inductive.
\newline 4. If the current reaches its peak value up to $\frac{\pi}{2}$ ahead of the voltage, the power factor is leading, and the circuit's load is capacitive.
