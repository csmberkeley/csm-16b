\DeclareSIUnit{\var}{VAR}
\DeclareSIUnit{\voltampere}{VA}

\qns{Power Analysis}

Consider the following circuit:

\begin{center}
    \begin{circuitikz}
    \draw (0,3)
    to[vsourcesin=$ $, l_=$V_s$] (0,0)
    (0,3) -- (0.3,3)
    to[R = $R$, v=$V_R(t)$] (4,3)
    to[short] (6,3); 
    \draw (4,3) to[C = $C$, v=$V_C(t)$] (4,0);
    \draw (6,3) to[L = $L$, v=$V_L(t)$] (6,0);
    \draw (0,0) to[short,i<= \mbox{$i(t)$}] (6,0);
    \end{circuitikz}
\end{center}
$\omega = \SI{1000}{\radian\per\second}$, $V_s(t)=1000 \sqrt{2} \cos(1000t+\frac{\pi}{2})$, $R=\SI{1000}{\ohm}$, $C=\SI{1}{\micro\farad}$, $L=\SI{0.5}{\henry}$.
\begin{enumerate}
    \qitem First we need to transform the whole circuit from time domain to phasor domain. Please \textbf{redraw the circuit} with only
    $\widetilde{V}$, $\widetilde{I}$, and all impedances of the circuit components.
    
    \sol{
        \begin{center}
            \begin{circuitikz}
            \ctikzset{resistor = european}
            \draw (0,3)
            to[vsourcesin=$ $, l_=$\widetilde{V}$] (0,0)
            (0,3) -- (0.3,3)
            to[R = $R$, v=$\widetilde{V}_R$] (4,3)
            to[short] (6,3);
            \draw (4,3) to[R = $Z_C$, v=$\widetilde{V}_C$] (4,0);
            \draw (6,3) to[R = $Z_L$, v=$\widetilde{V}_L$] (6,0);
            \draw (0,0) to[short,i<= \mbox{$\widetilde{I}$}] (6,0);
            \end{circuitikz}
        \end{center}
    }
    \qitem \textbf{Find the phasor $\widetilde{I}$ through the circuit and the total impedance $Z_{tot}$ of the circuit.}
    \ws{\vspace{60px}}
    \sol{
        To find the phasor current $\widetilde{I}$, we need to solve for the total impedance of the circuit and use Ohm's Law in phasor form. The impedance of the circuit is given by
        \begin{align*}
            Z_{tot} &= R + \frac{1}{\frac{1}{Z_L} + \frac{1}{Z_C}} \\
            &= R + \frac{1}{\frac{1}{j \omega L} + \frac{1}{\frac{1}{j \omega C}}} \\
            &= 1000 + \frac{1}{\frac{1}{j(1000)(0.5\text{H})} + \frac{1}{\frac{1}{j(1000)10^{-6}\text{H}}}} \\
            &= 1000 + 1000j \\
            &= 1000\sqrt{2}e^{\frac{\pi}{4}j}.
        \end{align*}
        The phasor voltage can be represented in Euler's form as $\widetilde{V_s}=1000\sqrt{2}e^{\frac{\pi}{2}j}$.

        The phasor current $\widetilde{I}$ can then be found as
        \begin{align*}
            \widetilde{I} &= \frac{\widetilde{V}_s}{Z_{tot}} \\
            &= \frac{1000\sqrt{2}e^{\frac{\pi}{2}j}}{1000\sqrt{2}e^{\frac{\pi}{4}j}} \\
            &= e^{\frac{\pi}{4}j}.
        \end{align*}
        }

    \qitem \textbf{Find the real power, reactive power, and apparent power.} \textit{Hint:}
        \begin{align}
            \text{apparent power}&=\frac{1}{2} |\widetilde{V}| |\widetilde{I}| \\
            \text{reactive power}&=\frac{1}{2} |\widetilde{V}| |\widetilde{I}| \sin(\angle(Z_{tot})) \\
            \text{real power}&=\frac{1}{2} |\widetilde{V}| |\widetilde{I}| \cos(\angle(Z_{tot}))
        \end{align}
    \ws{\vspace{60px}}
    \sol{
        The real power is
        \begin{equation*}
            P = \frac{1}{2} |\widetilde{V}| |\widetilde{I}| \cos(\angle(Z_{tot})) = \SI{0.5}{\kilo\watt}
        \end{equation*}
        The reactive power is
        \begin{equation*}
            Q = \frac{1}{2} |\widetilde{V}| |\widetilde{I}| \sin(\angle(Z_{tot})) = \SI{0.5}{\kilo\var}
        \end{equation*}
        The apparent power is \(\frac{1}{2} |\widetilde{V}| |\widetilde{I}| = \SI{1.414}{\kilo\voltampere}\).
    }

    \qitem Find the power factor, and state whether it is lagging or leading. \textit{Hint:}
    $$\text{power factor}=\frac{\text{true power}}{\text{apparent power}}=\cos({\theta}_d),$$ where ${\theta}_d$ is phase difference
    betweent voltage and the current of the circuit.
    \ws{\vspace{60px}}
    \sol{
        $\text{power factor}=\cos(\text{phase difference between voltage and current})=\cos(|\angle(Z_{tot})|)=\cos(\frac{\pi}{4})=\frac{\sqrt{2}}{2}$.
        \newline Two essential properties we need to be aware of:
        \newline 1. If the current reaches its peak value up to $\frac{\pi}{2}$ later than the voltage, the power factor is lagging, and the circuit's load is inductive.
        \newline 2. If the current reaches its peak value up to $\frac{\pi}{2}$ ahead of the voltage, the power factor is leading, and the circuit's load is capacitive.
        \newline From the phasor representation to the time domain, we know that $i(t)=\cos(1000t+\frac{\pi}{4})$.
        The phase of $V_s(t)$ is $\frac{\pi}{2}$, and the phase of $i(t)$ is $\frac{\pi}{4}$.
        One way to check when the current or voltage reaches its peak value is just to check when the term inside $\cos$ becomes $2\pi$.
        In this case, when $t=\frac{\frac{7}{4}\pi}{1000}$, the current reaches its maximum. However, when $t=\frac{\frac{6}{4}\pi}{1000}$,
        the voltage reaches its maximum. The current comes later, so the power factor is lagging. 
        Alternatively, based on the value of reactive power, we know that it is positive, so the power factor is lagging.
        
    }

    \qitem Based on the previous question, \textbf{is the circuit's load resistive, inductive, or capacitive?}
    \ws{\vspace{60px}}
    \sol{
        Since the power factor is lagging, the cicuit's load is inductive.
    }
\end{enumerate}
